\documentclass[11pt, a4paper]{report}
\usepackage[utf8x]{inputenc}
\usepackage[T1]{fontenc}
\usepackage{lmodern}
\usepackage[french]{babel}
\usepackage{verse}
\usepackage{marginnote}
    
\begin{document}
\poemlines{5}

            
        \pagebreak 
    
            \begin{center} \textbf{CARMEN} \end{center}
            \subsection*{00}
      \begin{verse}
      \poemtitle{PAPYRI IERCVLANENSIS.}m Nr \\ B. M. \\ 
                     \lbrack ... \rbrack  C  \lbrack ... \rbrack  un  \lbrack ... \rbrack 
                 \\ 
                     \lbrack ... \rbrack  ael  \lbrack ... \rbrack  tia  \lbrack ... \rbrack  xim  \lbrack ... \rbrack 
                 \\ Caesaris a  \lbrack ... \rbrack  p  \lbrack ... \rbrack  hariam  \lbrack ... \rbrack 
                 \\  \lbrack ... \rbrack rt  \lbrack ... \rbrack  sille  \lbrack ... \rbrack  natocum
                         \lbrack ... \rbrack  eiiapor  \lbrack ... \rbrack 
                 \\ Quem iuvenes  \lbrack ... \rbrack  ra navos erat de  \lbrack ... \rbrack 
                    uncta \\ Bella, fide dextraque potens rerumque per usum \\ Callidus, adsiduos tractando in munere fartis. \\ Imminet opsessis Italus iam turribus hostis \\ A  \lbrack ... \rbrack  sa nec de fuiti mpetus illis \\ 
                     \lbrack ... \rbrack  s  \lbrack ... \rbrack  q  \lbrack ... \rbrack 
                 \\ 
                     \lbrack ... \rbrack  nt ipso  \lbrack ... \rbrack  re \\ 
        \pagebreak 
    \begin{center} \textbf{CARMEN} \end{center}
                     \lbrack ... \rbrack  edunt patr  \lbrack ... \rbrack  mia terris  \lbrack ... \rbrack 
                 \\ 
                     \lbrack ... \rbrack  a i  \lbrack ... \rbrack  gis, quam s  \lbrack ... \rbrack 
                    ngesta laterent \\ Cum.uper  \lbrack ... \rbrack  lius Pelusia moenia Caesar \\ 
                     \lbrack ... \rbrack  erat imperiis animos cohibere suorum: \\ ‘Quid capitis iam capta? iacent quae  \lbrack ... \rbrack 
                 \\ Subruitis ferro meca moenia? uondam er  \lbrack ... \rbrack  ostis \\ Haec mihi cum s  \lbrack ... \rbrack  a plebes quoque  \lbrack ... \rbrack 
                    victrix \\ Vindicat h  \lbrack ... \rbrack  mulam Romana tot e  \lbrack ... \rbrack  s . n .
                    m \\ 
                     \lbrack ... \rbrack  liu  \lbrack ... \rbrack 
                 \\ 
                     \lbrack ... \rbrack  im  \lbrack ... \rbrack  o  \lbrack ... \rbrack  t  \lbrack ... \rbrack 
                 \\ 
                     \lbrack ... \rbrack  t Alexandro thalamos intrare deorum \\ Di  \lbrack ... \rbrack  etiam potuisse deam vidisse . um  \lbrack ... \rbrack 
                    s \\ Actiacos, cum causa fores tu marima elli, \\ Pars etiam imperii. quae femina tanta? virorum \\ Quae series antiqua fuit? ni gloria mendax \\ Multa vetustatis nimio concedat honoris \\ 
                     \lbrack ... \rbrack  an  \lbrack ... \rbrack 
                 \\ Saepe ego, quae vestris curae sermonibus  \lbrack ... \rbrack 
                 \\ Quas igitur segnis t  \lbrack ... \rbrack  n nunc quaerere causas \\ Exsnguisque mors vitae libet? Est mihi coniunx \\ 
                     \lbrack ... \rbrack  h  \lbrack ... \rbrack  i posset Phariis subiungere
                    regnis \\ 
        \pagebreak 
    \begin{center} \textbf{PAPVRI ERCVLANENSIS.} \end{center}Qui statauit nostraeue mori pro nomine gentis. \\ Hic igitur partis animum diductus in omnis \\ Quid velit, incertum est; terris quibus aut quibus undis \\ 
                     \lbrack ... \rbrack  ctumque  \lbrack ... \rbrack  m quo noxia turba coiret \\ Praeberetque suae spectacula tristia mortis. \\ Qualis ad instantis acies cum tela parantur, \\ Signa, tubae, classesque simul terrestribus armis, \\ Est facies ea visa loci, cum saeva coirent \\ Instrumenta necis, vario congesta paratu: \\ Vudique sic illuc campo deforme coactum \\ Omne vagabatur leti genus, omne timoris. \\ 
                     \lbrack ... \rbrack  acet  \lbrack ... \rbrack  ferro tu  \lbrack ... \rbrack  is
                         \lbrack ... \rbrack  le ven  \lbrack ... \rbrack 
                 \\ Aut pendente suis cervicibus aspide mollem \\ .abitur in somnum trahiturque libidine mortis. \\ Perclit adflatu breris hunc sine morsibus anguis, \\ Volnere seu tenui pars inlita parva veneni \\ Ocius interemit; laqueis pars cogitur artis \\ Intersaeptam animam pressis effundere venis; \\ Inmersisque freto clauserunt guttura fauces. \\ Has inter strages solio descendit et inter \\ At  \lbrack ... \rbrack  alia  \lbrack ... \rbrack  nc  \lbrack ... \rbrack  a  \lbrack ... \rbrack  te \\ Sie illi inter se misero sermone fruuntur. \\ Haec regina gerit. procul hanc occulta videbat \\ Atropos inridens inter diversa agantem \\ Consilia interitus, qum iam oua fata manerent. \\ Ter fuerat revocata dies; cum parte senatus \\ Et patriae comitante suae cum milite Caesar \\ 
        \pagebreak 
    \begin{center} \textbf{CARMEN PAPYRI ERCVLANENSIS.} \end{center}CGentis Alexandri ca  \lbrack ... \rbrack  en  \lbrack ... \rbrack  ad moenia
                    venit \\ Signaque constituit. sic omnes terror in artum \\ 
                     \lbrack ... \rbrack  rere  \lbrack ... \rbrack  m portarum claustra nec urbem \\ Opsidione tamen nec corpora moenibus arcent, \\ Castraque pro muris atque arma pedestria ponunt. \\ Hos inter coetus talisque ad bella paratus \\ Vtraque sollemnis iterum revocaverat orbes: \\ Consiliis nox apta ducum, lux aptior armis. \\ 
      \end{verse}
  
            
        \pagebreak 
    
            \begin{center} \textbf{CARMINACODICIS VERGILIANI VATICANI 3867.} \end{center}
            \subsection*{0}
      \begin{verse}
      \poemtitle{OVIDII NASONIS}\poemtitle{Argmenta Aeneidis.}\poemtitle{Prefatio.}A. II 192. \\ Moyer. 862. \\ Behren. P. \\ L..IVp.161. \\ Vergilius magno quantum concessit Homero, \\ Tantum ego Vergilio ‘Naso poeta meo’. \\ Nec me praelatum cupio tibi ferre, poeta: \\ Ingenio si te subsequor, hoc satis est. \\ \begin{center} \textbf{CARMINA} \end{center}Argumenta quidem librorum prima notavi, \\ Errorem ignarus ne quis habere queat. \\ Bis quinos feci legerent quos carmine versus, \\ Aeneidos totum corpus ut esse putent. \\ Adfirmo gravitate mea, me carmine nullum \\ Livoris titulum praeposuisse tibi. \\ B. II 191. M. S60. B. IV 176. \\ \poemtitle{Aeneas primo Libyes adpellitur oris.}B. II 192. M. 862. B. IV 172. \\ Vir magnus bello, nulli pietate secundus \\ Aeneas odiis Iunonis pressus iniquae \\ Italiam quaerens Siculis erravit in undis. \\ Iactatus tandem Libyae pervenit ad oras \\ Ignarusque loci, fido comitatus Acbate \\ Indicio matris regnum cognovit Elissae, \\ Quin etiam nebula saeptus pervenit ad urbem \\ Abreptosque undis socios cum classe recepit. \\ 
        \pagebreak 
    \begin{center} \textbf{CODICIS VERGILIANI VATICANI 3867.} \end{center}Hospitioque usus Didus per cuncta benignae \\ Excidium Troiae iussus narrare parabat. \\ (cf. ad I) \\ Funera Dardaniae narrat fletusque secundo. \\ (cf. ad I) \\ Conticuere omnes. tum sic fortissimus heros \\ Fata recensebat patriae casusque suorum: \\ Fallaces Graios simulataque dona Minervae, \\ Laucontis poenam et laxantem claustra Sinonem, \\ Somnum, quo monitus acceperit Hectoris atri, \\ Iam flammas caeli, Troum patriaeque ruinas \\ Et regis Priami fatum miserabile semper \\ Impositumque patrem collo dextraque prehensum \\ Ascanium, frustra a tergo comitante Creusa; \\ Ereptam hanc fato, socios in monte receptos. \\ Tertius errores pelagi terraeque requirit. \\ Post eversa Phrygum regna ut fuga coepta moveri \\ Vtque sit in Thracen primo devectus ibique \\ \begin{center} \textbf{CARMINA} \end{center} \marginpar{[10]} Moenia condiderit Polydori caede piata, \\ Regis Ani hospitium et Phoebi responsa canentis, \\ Coeptum iter in Creten, rursus nova fata reperta, \\ Naufragus utque foret Strophadas compulsus ad undas. \\ Inde fugam et dirae narrat praecepta Celaenus, \\ Liquerit utque Helenum perceptis ordine fatis, \\ Supplicem Achaemeniden Polyphemo urgente receptum \\ Amissumque patrem Drepani. sic deinde quievit. \\ Vritur in quarto Dido flammasque fatetur. \\ At regina gravi Veneris iam carpitur igni. \\ Consulitur soror Anna; placet succumbere amori. \\ Fiunt sacra deis, onerantur numina donis. \\ Itur venatum, Veneris clam foedera iungunt. \\ Facti fama volat. monitus tum numine divum \\ Aeneas classemque fugae sociosque parabat. \\ Sensit amans Dido, precibus conata morari. \\ Postquam fata iubent nec iam datur ulla facultas, \\ Conscenditque pyram dixitque novissima verba \\ Et vitam infelix multo cum sanguine fudit. \\ 
        \pagebreak 
    \begin{center} \textbf{CODICIS VERG6ILIANI VATICANI 3867. 11} \end{center}Quintus habet ludos et classem corripit ignis. \\ Navigat Aeneas. Siculas defertur ad oras. \\ Hic manes celebrat patrios. una hospes Acestes. \\ Ludos ad tumulum faciunt, certamina ponunt. \\ Prodigium est cunctis ardens adlapsa sagitta. \\ Iris tum Beroen habitu mentita senili \\ Incendit naves, subitus quas vindicat imber. \\ In somnis pater Anchises, quae bella gerenda \\ Quoque duce ad Manes possit descendere, monstrat. \\ Transcribit matres urbi populumque volentem \\ Et placidum Aeneas Palinurum quaerit in undis. \\ Quaeruntur sexto Manes et Tartara Ditis. \\ Cumas deinde venit. fert hinc responsa Sibyllae. \\ Misenum sepelit; mons servat nomen humati. \\ Ramum etiam divum placato numine portat. \\ At vates longaeva una descendit Avernum. \\ 
        \pagebreak 
    \begin{center} \textbf{CARMNA} \end{center} \marginpar{[10]} Agnoscit Palinurum et ibi solatur Elissam \\ Deiphobumque videt lacerum crudeliter ora. \\ Vmbrarum poenas audit narrante Sibylla. \\ Convenit Anchisen penitus convalle virenti \\ Agnoscitque suam prolem monstrante parente. \\ Haec ubi percepit, graditur classemque revisit. \\ Septimus Aeneam reddit fatalibus arvis. \\ Hic quoque Caietam sepelit, tum deinde profectus \\ Laurentum venit hanc verbis cognoscit Iuli \\ Fatalem terram: ‘mensis en vescimur’ inquit. \\ Centum oratores pacem veniamque petentes \\ Ad regem mittit Latii tum forte Latiuum, \\ Qui cum pace etiam natae conubia pactus. \\ Hos furia Allecto Iunonis dissipat ira: \\ Concurrunt dictis, quamvis pia fata repugnent. \\ Belli causa fuit violatus vulnere cervus. \\ Tum gentes sociae arma parant, fremit arma iuventus. \\ 
        \pagebreak 
    \begin{center} \textbf{CODICIS VERGILIANI VATICANI 3867. 13} \end{center}Praeparat octavo bellum  \lbrack et \rbrack  quos mittat in hostis. \\ Dat belli signum Laurenti Turnus ab arce, \\ Mititur et magni Venulus Diomedis ad urbem, \\ Qui petat auxilium et doceat, quae causa petendi. \\ Aeneas divum monitu adit Arcada regem, \\ Euandrum Arcadia profugum nova regna tenentem. \\ Accipit auxilium: huic natum et socia agmina iungit \\ Euander. Pallas fatis comes ibat iniquis. \\ Iamque habilis bello et maternis laetus in armis \\ Fataque fortunasque ducum casusque suorum \\ Soctitus clipeo divina intentus in arte est. \\ Nonus habet pugnas nec adest dux ipse tumultu. \\ Atque ea diversa penitus dum parte geruntur, \\ Iunonis monitu Turnus festinat in hostem. \\ Teucrorum naves Rutulis iaculantibus ignem \\ Nympharum in speciem divino numine versae. \\ Euryali et Nisi coeptis fuit exitus impar. \\ Pugnatur: castra Aeneadae vallumque tuentur. \\ Audacem Remulum dat lcto pulcher lulus. \\ 
        \pagebreak 
    \begin{center} \textbf{CARMINA} \end{center} \marginpar{[14]} Fit via vi. Turnus itian et Pandaron altum \\ Deicit et totis victor dat funera castris, \\ Iamque fatigatus recipit se in castra suorum. \\ Occidit Aeneae decimo Mezentius ira. \\ Concilium divis hominum de rebus habetur. \\ Interea Rutuli portis circum omnibus instant. \\ Advenit Aeneas multis cum milibus heros. \\ Mars vocat et totis in pugnam viribus itur. \\ Interimit Pallanta potens in proelia Turnus \\ Caedunturque duces, cadit et sine nomine vulgus. \\ Subtrahitur pugnae lunonis numine Turnus. \\ †Aeneas perstat Mezenti caede piata† \\ Et Lausum invicta perimit per vulnera dextra. \\ Mox ultor nati Mezentius occidit ipse. \\ Vndecimo victa est non aequo Marte Camilla. \\ Constituit Marti spoliorum ex hoste tropaeum \\ Exanimumque patri natum Pallanta remittit. \\ 
        \pagebreak 
    \begin{center} \textbf{CODICIS VERGILIANI VATICANI 3867. 15} \end{center}Iura sepulturae tribuit tempusque Latinis. \\ Euander patrios adfectns edit in urbe. \\ Corpora caesa virum passim digesta cremantur. \\ Legati referunt, Diomeden arma negasse. \\ Drances et Turnus leges aequante Latino \\ Concurrunt dictis. Aeneas imminet urbi. \\ Pugnatur. vincunt Troes. cadit icta Camilla. \\ Deinde duces castris, donec cessere, minantur. \\ Duodecimo Turnus divinis occidit armis. \\ Turnus iam fractis adverso Marte Latinis \\ Semet in arma parat pacem cupiente Latino. \\ Foedus percutitur, passuros omnia victos. \\ Hoc Turni Iuturna soror confundit et ambos \\ In pugnam populos agit ementita Camertem. \\ Aenean volucri tardatum membra sagitta \\ Anxia pro nato servavit cura parentis. \\ Vrbs capitur. vitam laqueo sibi fnit Amata. \\ Aeneas Turnum campo congressus utrimque \\ Circumfusa acie vita spoliavit et armis. \\ 
        \pagebreak 
    \begin{center} \textbf{CARMINA} \end{center} \marginpar{[16]} 
      \end{verse}
  
            \subsection*{2}
      \begin{verse}
      \poemtitle{OVIDII NAS0NIS}B. II 19, 14. \\ M. 836, 1 4. \\ \poemtitle{Argumenta Bucolicon et Georgicon}Qualis bucolicis, quantus tellure domanda, \\ Vitibus arboribusque, satis pecorique apibusque \\ Aeneadum fuerit vates, tetrasticha dicent. \\ Contineat quae quisque liber, lege carmina nostra. \\ B. II 193,1 4. \\ M. 863, 1 4. \\ \poemtitle{Bucolica.}B. IV 173. \\ Tityrus agresti modulatus carmine ruris, \\ Formosum per quod Corydon dilexit Alexin, \\ Silenumque seuem sertisque meroque ligavit, \\ Pastorumque melos facili deduxit avena. \\ \poemtitle{ \lbrack Georgica \rbrack }B. II 189,5 . \\ M. 836, q. \\ B. IV 444. \\ Quid faciat laetas segetes, quae sidera servet \\ Agricola, ut facilem terram proscindat aratro, \\ 
        \pagebreak 
    \begin{center} \textbf{CODCIS VERGILANI VATICANM 3867. 17} \end{center}Semina quae iacienda, modos cultusque locorum, \\ Et docuit messes magno cum fenore reddi. \\ Hactenus arvorum cultus et sidera caeli, \\ Pampineas canit inde comas collisque virentis \\ Descriptasque locis vites et dona Lyaei \\ Atque oleae ramos, pomorum et condere fetus. \\ Teque Pales et te, pastor memorande per orbem, \\ Et pecorum cultus et gramine pascua laeta, \\ Quis habitent armenta locis stabulentur et agni, \\ Omnia divino monstravit carmine vates. \\ Protinus aerii mellis redolentia regna, \\ Hyblaeas et apes, alveorum et cerea texta, \\ Quique apti flores, examina quaeque legenda, \\ Indicat humentisque favos, caelestia dona. \\ 
      \end{verse}
  
            \subsection*{4}
      \begin{verse}
      
        \pagebreak 
    \begin{center} \textbf{CARMEN} \end{center}\poemtitle{CODICIS VERGILIANI MEDICEI 39, 29.}
                    Turcius Rufius Apronianus Asterius uc. et inl. ex comite domest.
                    protect. ex com. priu. largit. ex praef. urbi patricius et consu ordi. legi
                    et distincxi codicem fratris Macharii c. non mei fiducia set eius cuius et
                    ad omnia sum devotus arbitrio XI lal. Mai. Romac.  \\ B. II 187. M. 281. B V 110. \\ Distincxi emendans gratum mihi munus amici. \\ Suscipiens operi sedulus incuui. \\ Quisque legis, reegas felix parcasque bens, \\ Siqua minus vacuus praeteriit animus \\ Tempore, quo penaces circo subiuncximus atque \\ Scenam euripo extulimus subitam, \\ Vt ludos currusque simul variumque ferarum \\ Certamen iunctim Roma teneret oans. \\ 
        \pagebreak 
    \begin{center} \textbf{CARMEN CODICIS VERGILIANI MDICEI 39, 29. 19} \end{center}Ternum quippe ‘sofos’ merui, terna agmina vulgi \\ Per caveas plausus concinuere meos. \\ In quaestum famae census iactura cucurrit, \\ Nam laudis fructum talia damna ferunt. \\ Sic tot consumptas servant spectacula gazas, \\ Festorumque trium permanet una dies, \\ Asteriumque suum vivax transmittit in aevum, \\ Qui partas trabeis tam bene donat opes. \\ 
        \pagebreak 
    \begin{center} \textbf{CARMEN} \end{center}CODICIS PARI SINI 8084. \\ B. M. \\ \poemtitle{ \lbrack Contra paganos \rbrack }B. III 287. \\ Dicite, qui colitis lucos antrumque Sibyllae \\ fol. 156 \\ Idaeumque nemus, Capitolia celsa Tonantis, \\ Plladium Priamique Lares Vestaeque sacellum \\ Incestosque deos, nuptam cum fratre sororem, \\ Inmitem puerum, Veneris monumenta nefandae, \\ Purpurea quos sola facit praetexta sacratos, \\ Quis numquam verum Phoebi cortina locuta est, \\ Etruscus ludit semper quos vanus aruspex: \\ luppiter hic vester, Ledae superatus amore, \\ Fingeret ut cycnum, voluit canescere pluma? \\ Perditus ad Danaen flueret subito aureus imber? \\ Per freta Parthenopes
                    taurus mugiret adulter? \\ Haec si monstra placent nulla sacrata †pudica, \\ Pellitur arma Iovis fugiens regnator Olympi? \\ Et quisquam supplex veneratur templa tyranni, \\ Cum patrem videat nato cogente fugatum? \\ 
        \pagebreak 
    \begin{center} \textbf{CARMEN CODICIS PARISNI 808.} \end{center}\begin{center} \textbf{0 1} \end{center}Postremum, regitur fato si Iuppiter ipse, \\ Quid prodest miseris perituras fundere voces? \\ Plangitur in templis iuvenis formonsus Adonis, \\ Nuda Venus deflet, gaudet Mavortius heros, \\ luppiter in medium nescit finire querellas, \\ urgantesque deos stimulat Bellona flagello. \\ Convenit his ducibus, proceres, sperare salutem \\ Sacratis? Vestras liceat conponere lites. \\ Dicite: praefectus vester quid profuit urbi, \\ Cu lovis ad solium raptor trabeatus obisset, \\ Cum poenas scelerum tracta vix morte rependat? \\ Mensibus iste tribus, totam qui concitus urbem \\ Lustravit, metas tandem pervenit ad aevi! \\ 0 Quae fuit haec rabies animi? quae insania metis? \\ Nempe lovis vestram posset turbare quietem! \\ Quis tibi iustitium incussit, pulcerrima Roma? \\ Ad saga confugerent, populus quae non habet olim? \\ Sed fuit in terris nullus sacratior illo, \\ Quem Numa Pompilius, e multis primus aruspex, \\ Edocuit vano ritu pecudumque cruore \\ Polluere (insanum) bustis putentibus aras. \\ Non ipse est, †vinum patriae qui prodidit olim, \\ Antiquasque domus, turres ac tecta priorum \\ Subvertens urbi vellet cum inferre ruinam, \\ 
        \pagebreak 
    \begin{center} \textbf{CARMN} \end{center} \marginpar{[00]} Ornaret lauro postes, convivia daret, \\ Pollutos panes infectos ture vaporo \\ fol. 157 Poneret, in risum quaerens quos dederet morti, \\ Collaribus subito circumdare membra suevit, \\ Fraude nova semper miseros profanare paratus? \\ Sacratus vester urbi quid praestitit, oro? \\ Qui hierium docuit sub terra quaerere solem, \\ Cum sibi forte pirum fossor de rure dolasset \\ Diceretque esse deum comitem Bacchique magistrum, \\ Sarapidis cultor, Etruscis semper amicus: \\ Fundere qui incautis studuit concepta venena, \\ Mille nocendi vias, totidem cum quaereret artes: \\ Perdere quos voluit, percussit, luridus anguis, \\ Contra deum verum frustra bellare paratus, \\ Qui tacitus semper lugeret tempora pacis \\ Nec proprium interius posset vulgare dolorem. \\ u. 7 \\ Quis tibi, taurobolus, vestem mutare suasit, \\ 62: \\ Inflatus dives, subito mendicus ut esses, \\ B. I 7. \\ M. 605. \\ Obsitus et pannis, modica stipe factus epaeta, \\ Sub terram missus, pollutus sanguine tauri, \\ Sorddns, infectus? vestes servare cruentas, \\ Vivere cum speras viginti mundus in annos? \\ Ambieras censor meliorum caedere vitam, \\ Hinc tua confisus possent quod facta latere, \\ Cum canibus Megales semper circumdatus esses \\ Quem lasciva cohors (monstrum) comitaret ovantem. \\ 
        \pagebreak 
    \begin{center} \textbf{CODCIS PARISINI 80s4.} \end{center}Sexaginta senex annis duravit, efebus, \\ Saturni cultor, Bellonae semper amicus, \\ Qui cunctis Faunosque deos persuaserat esse \\ Egeriae Nymplae comites Satyrosque Panasque, \\ Nympharum Bacchique comes Triviaeque sacerdos; \\ Quem lustrare choros ac molles sumere thyrsos \\ Cymbalaque inbuerat quatere Berecyntia mater, \\ Quis Galatea potens iussit Iove prosata summo, \\ Iudicio Paridis pulcrum sortita decorem. \\ acrato nulli liceat servare pudorem, \\ Frangere cum vocem soleant Megalensibus actis. \\ Christicolas multos voluit sic perdere demens, \\ Qui vellent sine lege mori; donaret honores \\ Oblitosque sui caperet quos daemonis arte, \\ Muneribus cupiens quorundam frangere mentes \\ Aut alios facere prava mercede profanos \\ Mittereque inferias miseros sub Tartara secum. \\ Solvere qui . . voluit pia foedera leges. \\ .eucadium fecit fundos curaret Afrorum; \\ ol. 158 \\ Perdere Marcianum, sibi proconsul ut esset. \\ Quid tibi diva Paphi custos, quid pronuba luno \\ Saturnusque senex potuit praestare sacrato? \\ Quid tibi Neptuni promisit fuscina, demens? \\ 9 Reddere quas potuit sortes Tritonia virgo? \\ Dic mihi, Sarapidis templum cur nocte petebas? \\ Quid tibi Mercurius fallax promisit euti? \\ 
        \pagebreak 
    \begin{center} \textbf{CARMEN} \end{center} \marginpar{[24]} Quid prodest coluisse Lares lanumque bifrontem? \\ Quid tibi Terra potens, mater formonsa deorum, \\ Quid tibi sacrato placuit latrator Anubis? \\ Quid miseranda Ceres mater, Proserpina subter, \\ Quid tibi Vulcanus claudus, pede debilis uno? \\ Quis te plangentem non risit, calvus ad aras \\ Sistriferam Fariam supplex cum forte rogares? \\ ..1sisCumqueOsirimmiserumlugens.. \\ Quaereret, inventum rursum quem perdere posset, \\ Post lacrimas ramum fractum portares olivae? \\ Vidimus argento facto inga ferre leones, \\ Lignea cum traherent iuncti stridentia plaustra, \\ Dextra laevaque istum argentea frena tenere, \\ Egregios proceres currum servare Cybellae, \\ Quem traheret conducta manus Megalensibus actis, \\ M. 606. \\ Arboris excisae truncum portare per urbem, \\ Attin castratum subito praedicere Solem. \\ Artibus heu magicis procerum dum quaeris honores, \\ Sic, miserande, iaces parvo donatus sepulcro. \\ Sola tamen gaudet meretrix te consule Flora, \\ Ludorum turpis genetrix Venerisque magistra, \\ Conposuit templum nuper cui Symmachus heres. \\ Omnia quae in templis positus tot monstra colebas, \\ Ipsa mola et manibus coniunx altaria supplex \\ Dum cumulat donis votaque in limine templi \\ Solvere dis deabusque parat superisque minatur, \\ 
        \pagebreak 
    \begin{center} \textbf{0.} \end{center}\begin{center} \textbf{CODICIS PARISNI 808.} \end{center}Carminibus magicis cupiens Acheronta movere, \\ Praecipitem inferias miserum sub Tartara misit. \\ Desine post hydropem talem deflere maritum, \\ De love qui Latio voluit sperare salutem! \\ 
      \end{verse}
  
            \subsection*{5}
      \begin{verse}
      
        \pagebreak 
    \begin{center} \textbf{CARMINAC0 D I CI S V 0 S SI A N I 0. 9.} \end{center}\poemtitle{Precatio Terrae Matris.}B. M. \\ B. I 138. \\ \poemtitle{Carmen sic dices:}Dea sancta Tellus, rerum naturae parens, \\ Quae cuucta generas et regeneras in dies, \\ Quod sola praestas  \lbrack tuam \rbrack  tutelam gentibus, \\ Caeli ac maris diva arbitra rerumque omnium, \\ Per quam silet natura et somnos capit, \\ Itemque lucem reparas et noctem fugas; \\ Tu Ditis umbras tegis et inmensum chaos \\ Ventosque et imbres tempestive contines \\ 
        \pagebreak 
    \begin{center} \textbf{CAIRMINA CODICIS VOSSIANI . 9.} \end{center} \marginpar{[27]} .t, cum libet, dimittis et misces freta \\ Fugasque solem et procellas concitas \\ Itemque, cum vis, bilarem promittis diem; \\ Alimeuta vitae tribuis perpetua fide \\ Et, cum recesserit anima, in te refugiemus: \\ Ita, quidquid tribuis, in te cuncta recidunt. \\ Merito vocaris Magna tu Mater deum, \\ Pietate quia vicisti divum numina. \\ Tu  \lbrack es \rbrack  illa vere gentium et divum parens, \\ Sine qua nil maturatur nec nasci potest. \\ Tu es magna, tuque divum regina  \lbrack ac \rbrack  dea. \\ Te, diva, adoro tuumque ego numen invoco, \\ Facilisque praestes hoc mihi, quod te roge, \\ Referamque, diva, gratias merito tibi. \\  \lbrack Me \rbrack  rite exaudi, quaeso, et fave coeptis meis. \\ Hoc quod peto a te, diva, mihi praesta volens: \\ Herbas, quascumque generat maiestas tua, \\ Salutis causa tribuis cunctis gentibus: \\ Inc  \lbrack nunc mihi permittas medicinam tuam. \\ Veniat me  \lbrack ... \rbrack  cum tuis virtutibus. \\ Quidquid ex his fecero, habeat eventum bonum; \\ Cuique easdem dedero quique easdem a me acceperint, \\ Sanos eosdem praestes  \lbrack ... \rbrack  \\  \lbrack ... \rbrack  nunc, diva, postulo ut hoc mihi \\ Miestas praestet  \lbrack tua \rbrack , quod te supplex rogo. \\ 
      \end{verse}
  
            \subsection*{6}
      \begin{verse}
      
        \pagebreak 
    \begin{center} \textbf{CARMINA CODICIS VOSSIANI Q. 9.} \end{center} \marginpar{[28]} \poemtitle{Precatio omnium herbarum.}Nunc vos, potentis omnes herbas, deprecor \\ Maiestatemque vestram, quas Tellus parens \\ Generavit atque gentibus cunctis dedit. \\ Medicinam sanitatis in vos contulit, \\ Vt omni generi humano utilissimum \\ Auxilium sitis. hoc supplex posco  \lbrack et \rbrack  precor: \\ Vos huc adeste vestris cum virtutibus; \\ Qui vos creavit, ipse permisit mihi, \\ Vt colligam. faveatis hoc etiam  \lbrack mihi \rbrack , \\ Cui tradita est medicina, quantum vestraque \\ Virtus potest, praestate medicinam bonamt, \\ Causam salutis. gratiam, precor, mihi \\ Praestetis per tutelam vestram, ut omnibus \\ Virtutibus, de vobis quidquid fecero, \\ Cuique homini dedero  \lbrack quique id a me acceperit \rbrack , \\ Effectum habeat celerrimum et eventus bonos. \\  \lbrack Praestetis etiam \rbrack , semper ut liceat mihi \\ 
        \pagebreak 
    \begin{center} \textbf{CARMINA CODICIS VOSSIANI . 9.} \end{center} \marginpar{[29]} Favente maiestate vestra †vos colligere, \\ Ponamque vobis fruges et agam gratias \\ Per nomen eius, qui vos iussit nascier. \\ 
        \pagebreak 
    \begin{center} \textbf{CARME} \end{center}
      \end{verse}
  
            \begin{center} \textbf{C 0 D I C I S MA I IIN G E N S I S.} \end{center}
            \subsection*{6a}
      \begin{verse}
      Lux mundi laeta, deus, haec tibi celeri cursV, \\ Alme potens, scribsi, soli famulatus et unI, \\ Vt te, vita, fruar teque casto inveniam cultV \\ Rectaque per te ad te ducente te gradiar viA. \\ Excelse cernis deus, quae me plurima cingunT \\ Nota et ignota tuis male nata eania satiS. \\ Tu sed mihbi certa salus spesque unica vitaE. \\ Inmeritum licet lucis facias adtingere limeN. \\ Verba nam tua valida imis me tollunt AvernI; \\ Sola haec misero mihi te, vitam, dabunt servul0. \\ 
      \end{verse}
  
            
        \pagebreak 
    
            \begin{center} \textbf{CARMNACODICIS PARISINI 10318} \end{center}
            
        \pagebreak 
    
            \begin{center} \textbf{EPIGRAMMATON LIBRI} \end{center}
            \subsection*{7}
      \begin{verse}
      B. I 171. \\ M. 1608. \\ B. IV 191. \\ Ipse manu patiens inmensa volumina versat \\ cd. \\ Adtollitque globos. sonuerunt omnia plausu. \\ Tunc Cererem corruptam undis emittit ab alto. \\ Septem ingens gyros, septena volumina traxit, \\ Lubrica convolvens et torrida semper ab igni. \\ At rubicunda Ceres oleo perfusa nitescit. \\ Scitillae absistunt, opere omnis semita fervet. \\ 
        \pagebreak 
    \begin{center} \textbf{CARMINA} \end{center} \marginpar{[34]} Fervet opus redoletque, volat vapor ater ad auras. \\ Instant ardentes veribusque trementia figunt, \\ Conclamant rapiuntque focis onerantque canistris. \\ Vndique conveniunt pueri innuptaeque puellae. \\ 
      \end{verse}
  
            \subsection*{8}
      \begin{verse}
      B. III 81. \\ M. I613. \\ \poemtitle{De alea}B. IV 192. \\ Artis opisque  \lbrack tuae \rbrack , tua si mihi certa voluntas, \\ Expediam dictis donum exitiale Minervae. \\ Tu vatem, tu, diva, mone. nunc ipsa vocat res \\ Et furiis agitatus amor; protentus in octo \\ Ipse dies agitat festos pro nomine tanto. \\ Effera vis animi numeros et nomina fecit. \\ Ossa minutatim fundo volvuntur in imo. \\ Mille nocendi artes. varium et mutabile semper \\ Artificis scelus, atque inprovida pectora turbat. \\ Per varios casus levium spectacula rerum \\ Intenti ludo exercent rapiuntque ruuntque \\ Incerti, quo fata ferant, atque aere sonoro \\ Insanire libet: duris dolor ossibus ardet. \\ Omnibus extemplo magnum dat ferre talentum, \\ Qui vocat; adrectae mentes stupefactaque corda \\ Vota metu duplicant; tantae est victoria curae. \\ Ergo ubi delapsae, mixto premit agmine turba \\ Consilium quaerens; subitus tremor occupat artus. . ? \\ Tunc certare odiis. multos alterna revisens \\ Lusit et in solido rursus Fortuna locavit: \\ 
        \pagebreak 
    \begin{center} \textbf{CODICIS SALMASIANI.} \end{center} \marginpar{[35]} Aut doluit miserans inopem aut invidit habenti. \\ Multa viri nequiquam inter se vulnera iactant \\ Et tenues rumpunt tunicas, caecique furore \\ Hincmetuuntcupiuntque,dolentgaudentque. \lbrack quidultra? \\ Vidi oculos ante ipse meos me voce vocantem, \\ Contulimusque manus:  \lbrack experto \rbrack  credite, quantus \\ Corde dolorl quid non mortalia pectora cogis? \\ Monstrum horrendum ingens  \lbrack tot \rbrack  sese vertit in ora. \\ Tu potes unanimes armare in proelia fratres: \\ Aere renidenti de vita et sanguine certant. \\ Tunc duo Trinacrii iuvenes noctesque diesque \\ Intenti ludo exercent fulgentiaque aera \\ Accipiunt redduntque; remittunt omnia fatis. \\ Conveniunt, quibus ipsa procul discordibus armis \\ Fundit humi facilem victum iustissima tellus, \\ Aut qui divitiis soli incubuere repertis: \\ Considunt transtris nati melioribns annis; \\ Multi praeterea, quos fama obscura recondit, \\ Stant circum. \\ Tunc vero ad vocem celeres miserum inter amorem \\  \lbrack Experiuntur \rbrack  et in medium quaesita reponunt; \\ Pro se quisque viri summa nituntur opum vi. \\ Nec mora, missus adest fati sortisque futurae. \\ Scinditur incertum studia in contraria vulgus. \\ p. \\ Hinc atque hinc ardent animi: vox omnibus una est. \\ 
      \end{verse}
  
            \subsection*{}
      \begin{verse}
      
        \pagebreak 
    \begin{center} \textbf{CARMINA} \end{center} \marginpar{[36]} Et quamvis socium certent superare priorem, \\ Ima petunt: veris facilis datur exitus umbris. \\ Praecipites pariterque ruunt, non deficit alter. \\ Vos, o Calliope. precor, adspirate canenti, \\ Quae loca quive habeant homines, ubi sistere detur. \\ In summo collem, qui plurimus; alter ab illo \\ Est locus, quem iuxta sequitur, quo deinde sub ipso \\ Hic locus est partis semper sublimis: at illum \\ Quinque tenent ebuli bacis minioque rubentem. \\ Terna tibi haec primum fundo volvuntur in imo. \\ Sunt alii, quos ipse via sibi repperit usus. \\ Triginta magnos adversosque orbibus orbes \\ Eloquar (an sileam?), levium spectacula rerum; \\ Mores et studia et populos et praelia victis \\ Expediam, sed summa sequar fastigia rerum. \\ Primus habet; capit ante locum fremituque secundo \\ Prima tenet;  \lbrack ardent \rbrack  animi risuque soluto \\ Voce vocant. \\ Tunc vero in curas animo diducitur omnis, \\ Quem petis, obtutuque haeret defixus in uno \\ Atque animum nunc huc celerem nunc dividit illuc, \\ Spemque metumque  \lbrack inter \rbrack , secumque ita corde volutat. \\ Vt primum discussae umbrae et lux reddita menti, \\ Sortitus fortunam oculis sic incipit ore: \\ 
        \pagebreak 
    \begin{center} \textbf{CODICIS SALMASIANI.} \end{center} \marginpar{[4]} ‘Quae nunc deinde mora est? veniam, quocumque vocaris; \\ Quin age, si qua animo virtus, et consere dextram: \\ Efficiam, posthac ne quemquam voce lacessas.’ \\ p.4 \\ Dixit et e curru magna ter voce vocavit, \\ Terque quaterque simul vox ingeminata remugit. \\ Tunc variae comitum facies et pallor in ore; \\ Nunc victi tristes, nil magnae laudis egentes \\ Deponunt animos et inania murmura miscent. \\ Quondam etiam victis redit in praecordia virtus \\ Victoresque cadunt, quoniam fors omnia versat. \\ Hic victor superans reddi sibi poscit honorem \\ Talia vociferans: ‘da, non indebita posco. \\ Quin age, si quid habes, quo me decet usque teneri’ \\ Clamat. \\ Tunc vero victus socios simul increpat omnis, \\ Nomine quemque vocans. illi obstipuere silentes; \\ Non ipsi inter se sortem miserantur iniquam, \\ Sed graviter vario noctem sermone requirunt. \\ Visceribus miserorum atque inter pocula laeti \\ Cantantes laetique animos convivia curant. \\ Ecce autem elapsus, genitor quem miserat urgens, \\ Vnus natorum longo post tempore venit. \\ Hos aditus, iamque hos aditus, omnemque pererrat \\ Vndique circuitum, aditumque per avia quaerit. \\ Verum ubi nulla datur dextra exsuperare potestas, \\  Constitit in digitos et toto vertice supra \\ 
        \pagebreak 
     \marginpar{[38]} \begin{center} \textbf{CARMINA} \end{center}Obnixus latis umeris et †pectore duro \\ Et super incumbens, furiis accensus et ira \\ Talia voce refert: \\ ‘Quo, moriture, ruis? quae te dementia ducit? \\ p. 5 \\ Non vires alias conversaque numina sentis? \\ Cede locis!’ \\ Talia fatus erat pressoque obmutuit ore. \\ Illa autem, cui fata parent, et luppiter hostis \\ Deserit  \lbrack inceptum \rbrack ; conversa \lbrack que \rbrack  numina sentit. \\ Postquam illum vita victor spoliavit et auro, \\ Tunc vero ardentes oculi atque adtractus ab alto \\ Spiritus interdum gemitu; furor iraque mentem \\ Praecipitant, maestis late loca questibus implet, \\ Multa gemens ignominiam plagasque superbi \\ Victoris, caput  \lbrack et \rbrack  glauco velatus amictu \\ Ardua tecta petit rursusque ad limina nota \\ Victus abit guttisque umectat grandibus ora. \\ 
      \end{verse}
  
            \subsection*{9}
      \begin{verse}
      B. I. 146. \\ M. 669. \\ \poemtitle{Narcissus}B. IV 197. \\ Candida per silvam primaevo flore iuventus \\ Adsidue veniebat: ibi haec caelestia dona \\ Et fontes sacros insigni laude ferebat \\ 
        \pagebreak 
    \begin{center} \textbf{CODCIS SALMASIANI.} \end{center} \marginpar{[39]} Insignis facie longumque bibebat amorem, \\ Intentos volvens oculos, securus amorum. \\ Dum stupet atque animum pictura pascit inani, \\ Expleri mentem nequit ardescitque tuendo \\ Egregium forma iuvenem, quem nympha crearat: \\ Sic oculos, sic ille manus, sic ora ferebat. \\ His amor unus erat, dorso dum pendet iniquo, \\ Oblitusve sui est et membra decora iuventae \\ Miratur rerumque ignarus imagine gaudet. \\ Ilicet ignis edax secreti ad fluminis undas \\ p.6 \\ Ipsius in vultu vana spe lusit amantem, \\ Et praeceps animi collo dare brachia circum \\ Ter conatus erat nec, quid speraret, habebat. \\ 
      \end{verse}
  
            \subsection*{10}
      \begin{verse}
      \poemtitle{MAVORTI}\poemtitle{Iuicium Paridis}B. I 147. \\ M. 282. \\ B. IV 198. \\ Pictus acu tunicas et barbara tegmina crurum \\ Forte recensebat numerum sub tegmine fagi. \\ Horrescit visu subito et memorabile numen \\ Aut videt aut vidisse putat. ‘quo tenditis’ inquit \\ ‘Caelicolae magni? pacemne huc fertis an arma?’ \\ Ad quem tum luno supplex his vocibus usa est: \\ ‘O lux Dardaniae, Troianae gloria gentis, \\ Quam dives pecoris, nivei quam lactis abundans, \\ †Et proprio fuerint distentae lacte capellae \\ Vbera, nec metas rerum nec tempora pono. \\ Haec tibi semper erunt, hic inter flumina nota \\ 
        \pagebreak 
    \begin{center} \textbf{CARMINA} \end{center} \marginpar{[40]} Sponte sua sandyx pascentis vestiet agnos. \\ Praeterea sceptrum dabitur, Troiane, quod optas.’ \\ Talibus orabat Huno. Tritonia Pallas \\ Orsa loqui, nimbo efulgens et Gorgone saeva: \\ Disce, puer, virtutem ex me verumque laborem, \\ Militiam et grave Martis opus; sit pectore in isto \\ Vulnera dirigere et calamos armare veneno.’ \\ Has inter voces, media inter talia verba \\ Sic contra est ingressa Venus, male numen amicum, \\ Nuda genu, nudos cervix cui lactea crines \\ p. 7 \\ Corripit in nodum; rosea cervice refulsit \\ Et vera incessu patuit dea. ille repente \\ Obstipuit subitaque animum dulcedine movit \\ Et mentem Venus ipsa dedit. decus enitet ore \\ Exultatque animis et se cupit ante videri. \\ ‘Sic tua Cyrneas fugiant examina taxos, \\ Sic cytiso pastae distendant ubera vaccae: \\ Formosi pecoris custos, formosior ipse, \\ Aspice nos tantum, Lacedaemoniosque hymenaeos \\ Coniugio iungam stabili propriamque dicabo \\ Reginam thalamis Phrygio servire marito’ \\ Ille deae donis ac tanto laetus honore \\ Vltro animos tollit dictis ac talia fatur: \\ ‘Iam iam nulla mora est neque me sententia vertit. \\ Do quod vis: licet arma mihi mortemque minetur, \\ Me tamen urit amor. vcniam quocumque vocaris; \\ Tu modo promissis maneas.’ ea verba locutus \\ Vendidit hic auro patriam. dux femina facti. \\ Nec mora, continuo penetrat Lacedaemona pastor \\ 
        \pagebreak 
    \begin{center} \textbf{CODICIS SALMASIANI.} \end{center} \marginpar{[41]} Ledaeamque Helenam Troianas vexit ad urbes, \\ Et si fata deum, si mens non laeva fuisset, \\ 
      \end{verse}
  
            \subsection*{11}
      \begin{verse}
      B. I 170. \\ M. 1607. \\ \poemtitle{Hippodamia}B. IV 199. \\ Pandite nunc Helicona, deae, nunc pectore firmo \\ Este duces, o si qua via est, et pronuba Iuno; \\ Pallida Tisiphone, fecundum concute pectus! \\ Non hic Atridae et scelus exitiale Lacaenae: \\ Hic crudelis amor. nunc illas promite vires, \\ Maius opus moveo: quaesitas sanguine dotes \\ p.8 \\ Et scelerum poenas inconcessosque hymenaeos. \\ Vrbs antiqua fuit: fama est obscurior annis. \\ Quid memorem infandas caedes et facta tyranni? \\ Ausi omnes inmane nefas irasque minasque. \\ Quis tam crudelis optavit sumere poenas? \\ Hic, qui forte velit currus agitare volantis, \\ Invitat pretiis animos et praemia ponit. \\ Tormenti genus incertum de patre ferebat. \\ Fama malum; incautum dementia cepit amantem \\ Horresco referens: rapido contendere cursu \\ Conposuit legesque dedit populosque propinquos \\ Infelix habuit thalamus. ruit omnis in urbem \\ Magnanimum heroum primaevo flore iuventus. \\ Vndique conveniunt et virginitatis amore \\ 
        \pagebreak 
    \begin{center} \textbf{CARMINA} \end{center} \marginpar{[42]} Contendunt petere,dubii seu vivere credant \\  \lbrack Sive extrema pati miseri, quibus ultimus esset \rbrack  \\ Ille dies, vitamque volunt pro laude pacisci. \\ Post, ubi confecti cursus, circensibus actis \\ Supplicia expendunt iuvenes ante ora parentum; \\ Linquebant dulces animas et corpora patrum. \\ Pro molli viola, pro purpureo narcisso \\ Ora virum tristi pendebant pallida tabo \\ Vestibulum ante ipsum saevique in limine regis \\ Terribiles visu formae inposuere coronas. \\ Quin ipsae obstipuere domus noctesque diesque; \\ Vmbrae ibant tenues, odium crudele tyranni \\ Saepe queri et longas in fletum ducere voces. \\ O virgo infelix, iam fas est parcere genti! \\ Pestis et ira deum crudeli funere virgo, \\ p. \\ Quam cum sanguineo sequiur Bellona flagello. \\ Tempore iam ex illo nil magnae laudis egentes \\ Deponunt animos scelerata excedere terra. \\ Ecce inter sanctos ignis, dum sacra morantur, \\ Et iuxta genitorem adstat lasciva puella, \\ Cui pater et coniunx, si qua fors adiuvet ausum, \\ Ora puer prima signans intonsa iuventa, \\ Pictus acu chlamydem et barbara tegmina crurum \\ Venit. amor fidens animi  \lbrack atque \rbrack  in utrumque paratus. \\  \lbrack lnclusum ut buxo aut Oricia terebintho \rbrack  \\ Lucet ebur, tantum egregio decus enitet ore. \\ Postquam introgressi et coram data copia fandi, \\ 
        \pagebreak 
    \begin{center} \textbf{CODICIS SALMASINI.} \end{center} \marginpar{[43]} Rex prior adgreditur dictis atque increpat ultro: \\ ‘Quo, moriture, ruis? quae te dementia cepit? \\ Aut quisnam ignarum conubia nostra petentem \\ Iussit adire domos? quidve hinc petis’ inquit \\ ‘Poenarum exhaustum satis est miseretque pudetque: \\ Pone animos et pulsus abi; miserere tuorum! \\ Non fugis hinc praeceps, dum praecipitare potestas? \\ Sunt aliae innuptae: thalamis ne crede paratis! \\ Ne pete conubiis natam: dabis, inprobe, poenas.’ \\ Ad quae subridens paucis ita reddidit heros: \\ ‘Hostis amare, quid increpitas mortemque minaris? \\ Ne tantos mihi finge metus tam fortibus ausis. \\ Nec mortem horremus, nec nos via fallit euntis: \\ Quo res cumque cadent, nec me sententia vertit. \\ Audentes Fortuna iuvat. stat, quidquid acerbi est, \\ Morte pati; quo fata trahunt retrahuntque, sequamur \\ Talia dicentem iam dudum aversa tuetur \\ Causa mali tanti multos servata per annos. \\ Qualis gemma micat, fulvum quae dividit aurum, \\ Inter utramque viam talem se laeta ferebat, \\ Ac veluti Pariusve lapis circumdatur auro \\ Arte nova, talis virgo dabat ore colores \\ Insignis facie, oculos deiecta decoros. \\ Vritur infelix, subitoque accensa furore \\ Stare loco nescit. quis enim modus adsit amori \\ Null Venus, nulli quondam flexere hymenaei; \\ 
        \pagebreak 
    \begin{center} \textbf{CODICIS SALMASIANI.} \end{center} \marginpar{[45]} Graecus erat, fama multis memoratus in oris, \\ Nec visu facilis nec dictu afabilis ulli. \\ Vndique visendi studio turbante tumultu \\ Conveniunt, quibus aut odium crudele tyranni \\ Aut metus acer erat puerique parentibus orbi \\ Et trepidae matres et lamentabile regnum. \\ Flent maesti mussantque patres, hic cara sororum \\ Pectora maerentum, quibus est fortuna peracta. \\ Hos inter motus stat ductis sortibus urna. \\ Tunc loca sorte legunt. extemplo arrectus uterque \\ Stat sonipes ac frena ferox spumantia mandit. \\ Nec mora: continuo vasto certamine tendunt \\ Custodes lecti atque arrectis auribus adstant \\ Orantes veniam; certatur limine in ipso. \\ Atque ea diversa penitus dum parte geruntur, \\ Discessere omnes medii, signoque repente, \\ Qua data porta, ruunt. sic densis ictibus heros p. 1? \\ Stridore ingenti atque oculis vigilantibus exit, \\ Incumbens umero; sonitu quatit ungula campum. \\ Dant animos plagae, pronique in verbera pendent \\ Pro se quisque viri; tunc caeco pulvere campus \\ Conditur in tenebras, qua proxima meta viarum, \\ Et longum superant flexu caecique furore. \\ Illi inter sese de vita et sanguine certant. \\ Regina e speculis miro properabat amore \\ Omnia tuta timens, quoniam fors omnia versat. \\ Audit equos, audit strepitus, timet omnia secum \\ 
        \pagebreak 
     \marginpar{[46]} \begin{center} \textbf{CARMINA} \end{center}Praescia venturi. sed spes incerta futuri. \\ Et proni dant lora: volat vi fervidus axis, \\ Liquitur, in medioque ardentem deserit ictu. \\ Carpit enim vires et, haec ut cera liquescit, \\ Excoquitur vitium, tum nititur acer et instat. \\ Vertitur interea et scelus expendisse merentem \\ Matres atque viri voces ad sidera iactant. \\ Dum trahitur curruque haeret resupinus inani, \\ Radit iter laevum interior subitoque priorem \\ Praeterit et super haec inimico pectore fatur: \\ ‘Istic nunc, metuende, iace vetitosque hymenaeos \\ Sume, pater frustraque animis elate superbis. \\ En qui nostra sibi tot iam labentibus annis \\ Servabat senior, nostrasne evadere demens \\ Sperasti te posse manus circensibus actis? \\ Hic tibi mortis erant metae: submitte furorem, \\ Qui iuvenum tibi semper erat! speravimus ista \\ Et tandem laeti sociorum ulciscimur umbras.’ \\ p. 13 \\ Dixit et e curru saltum dedit ocius arvis. \\ Excipiunt plausu; caelum tonat omne tumultu. \\ Ipse etiam eximiae laudis cum virgine victor \\ Ibat ovans umeroque Pelops insignis eburno. \\ Tunc vero exarsit iuveni dolor ossibus ingens. \\ Olli (sensit enim simulata mente locutam) \\ Nec latuere doli, caput horum et causa malorum; \\ Tunc quassans caput haec effundit pectore dicta: \\ ‘Me (adsum qui feci) merui, nec dcprecor’ inquit, \\ ‘Spargite  \lbrack me \rbrack  in fluctus. en haec promissa fides est? \\ 
        \pagebreak 
    \begin{center} \textbf{CODICIS SALMASIANI.} \end{center} \marginpar{[47]} I nunc, ingratis offer te, inrise, periclis. \\ His etiam struxi manibus, deceptus amore. \\ Nusquam tuta fides; varium et mutabile semper \\ Femina.’ sic fatus liquidas proiecit in undas \\ Aeternam moriens famam, quae maxima semper \\ Dicitur aeternumque tenet per saecula nomen. \\ 
      \end{verse}
  
            \subsection*{12}
      \begin{verse}
      B. I 45. \\ M. 581. \\ \poemtitle{Hercules et Antaeus}B. IV 205. \\ Litus harenosum  \lbrack ad \rbrack  Libyae caelestis origo \\ Alcides aderat, terrae omnipotentis alumnum \\ Caede nova quaerens et ineluctabile fatum. \\ Protinus Antaeum vasta se mole moventem \\ Occupat, ille suae contra non inmemor artis \\ Concidit atque novae rediere in praelia vires. \\ Adrepta tellure semel vim crescere victis \\ Non tulit Alcides et terra sublevat ipsum. \\ Namque manus inter conantem et plurima frustra \\ Corripit in nodum nisuque inmotus eodem \\ Auxilium solitum eripuit, corpusque per ingens \\ Non iam mater alit Tellus viresque ministrat. \\ Verum ubi nulla datur dextra adtrectare potestas, \\ lllum exspirantem magnum Iovis incrementum \\ Excutit effunditque solo. ruit ille volutus \\ Ad terram, non sponte fluens, vitaque recessit. \\ 
        \pagebreak 
    \begin{center} \textbf{CARMINA} \end{center}
      \end{verse}
  
            \subsection*{13}
      \begin{verse}
      B. I 168. \\ M. 689. \\ \poemtitle{Progne et Philomela}B. IV 206. \\ Aspice ut insignis vacua atria lustrat hirundo! \\ Vere novo maestis late loca questibus implet; \\ Victum infelicem maerens Philomela sub umbra \\ Adsiduo resonat cantu miserabile carmen. \\ Causa mali tanti coniunx, thalamique cruenti \\ Virginis os; notumque, furens quid femina possit. \\ Hic crudelis amor: crudelis tu quoque, mater; \\ Infelix puer, atque odium crudele tyranni. \\ Progeniem parvam curaeque iraeque coquebant, \\ Threicio regi cum iam securus amorum \\ Coniugis infandae inter deserta ferarum \\ Fas omne abrumpit, pariterque loquentis ab ore \\ Decidit exanimis vox ipsa  \lbrack et frigida lingua: \\ Haut impune quidem dementia cepit amantem. \\ Pectore in adverso saevi monumenta doloris \\ Fertque refertque soror, crimenque  \lbrack et \rbrack  facta tyranni \\ Sanguis ait. Solidae postquam data copia fandi, \\ (Vulnera siccabat circum praecordia) ‘Sanguis, \\ Accipe’  \lbrack ait \rbrack  ‘vocem’, ac saevo sic pectore fatur: \\ ‘leu miserande puer, nunc te fata impia tangunt!’ \\ Roalis inter mensas genitoris et ora \\ p.15 \\ Polluit ore dapes, quidquid solamen humandi est. \\ Dum genitor nati morsu depascitur artus, \\ Et soror et coniunx petierunt aethera pinnis. \\ 
      \end{verse}
  
            \subsection*{14}
      \begin{verse}
      
        \pagebreak 
    \begin{center} \textbf{CODICIS SALMSIANI.} \end{center} \marginpar{[49]} B. I 14. \\ M. 575. \\ \poemtitle{Europa}B. IV 27. \\ Vulneris inpatiens hominum rerumque repertor \\ Et faciem tauro propior descendit ad undas. \\ Europam nivei solatur amore iuvenci. \\ Dulcibus illa quidem inlecebris in litore sicco \\ Luserat, insignis facie, candore nivali. \\ Saucius at quadrupes saltus ingressus apertos \\ Forte fuit iuxta, superi regnator Olympi, \\ Obtulerat qui se ignotum venientibus ultro \\ Virginibus Tyriis aurata fronte iuvencum. \\ At circum late comites per litora passim \\ DiHugiunt visu exsangues taurumque relinquunt, \\ Sola (novum dictu) contra stetit ora iuvenci \\ Ante Iovem: nam te voluit rex magnus Olympi. \\ Hunc Phoenissa tenet vasta se mole moventem \\ Purpureosque iacit flores omnemque pererrat, \\ Ille autem spissa iacuit revolutus harena. \\ Inponit regina manum patiensque pericli \\ Mollibus intexens ornabat cornua sertis. \\ Hunc ubi contiguum summo tenus attigit ore \\ Et super incumbens sertis et fronde coronat \\ Iam iam nulla mora est animum labefactus amore \\ Accepit venientem ac mollibus extulit nndis. \\ Illi (sensit enim tuus, o clarissime, frater) \\ Subsidunt undae, straverunt aequora venti. \\ Nunc pelagi nymphae crinem de more solutae \\ 
        \pagebreak 
     \marginpar{[50]} \begin{center} \textbf{CARMINA} \end{center}Suspensum et pariter comitique onerique timentem \\ 
                     \lbrack ... \rbrack 
                 \\ Egregia interea summa sublimis ab unda \\ p. 16 \\ Prona petit maria et pelago decurrit aperto. \\ Tunc laeva taurum cornu tenet inscia culpae \\ Obliquatque sinus in ventum auramque patentem: \\ Ille, manum patiens, miro properabat amore, \\ Et ductus cornu rex omnipotentis Olympi \\ Insuetum per iter tacitis subremigat undis \\ Perfidus, alta petens abducta virgine praedo. \\ 
      \end{verse}
  
            \subsection*{15}
      \begin{verse}
      B. I 172. \\ M. 1609. \\ \poemtitle{Alcesta}B. IV 208. \\ Egregium forma iuvenem pactosque hymenaeos \\ Incipiam et prima repetens ab origine pergam, \\ Si qua fides, animum si veris inplet Apollo. \\ Iam gravior Pelias multis memoratus in oris \\ Rex erat et tantas servabat filia sedes. \\ Illam omnis tectis primaevo flore iuventus \\ Ardebat, sed res animos incognita turbat. \\ Iura dabat legesque viris, sub rupe leonem \\ Aut spumantis apri cursum qui foedere certo \\ Et premere et laxas sciret dare iussus habenas. \\ Iamque aderat Phoebo ante alios dilectus amore; \\ Ipse inter primos caput obiectare periclis \\ Obtulerat, fidens animi fretusque iuventa. \\ Ergo iussa parat, multis comitantibus armis. \\ Itur in antiquam silvam, stabula alta ferarum, \\ 
        \pagebreak 
    \begin{center} \textbf{CODICIS SALMASIANI.} \end{center} \marginpar{[51]} Atque hic exsultans animis patiensque pericli \\ Optat aprum ant fulvum descendere monte leonem. \\ Tunc breviter super aspectans sic voce prccatur: \\ ‘Sancte deum, summi custos Soractis Apollo, \\ p. 17 \\ Quem primi colimus, tua si mihi certa voluntas, \\ Ibo animis contra nec me labor iste gravabit’ \\ Nec mora nec requies; orauti et multa precanti \\ Aethere se mittit auditque vocatus Apollo \\ Et iuveni ante oculos his se cum vocibus offert: \\ ‘Incipe si quid habes, si tantum pectore robur \\ Concipis et si adeo dotalis regia cordi est: \\ Mecum erit iste labor, mitte hanc de pectore curam’ \\ Per silvas tum saevus aper cum murmure montis, \\ Tum demum movet arma leo vastoque sub antro \\ Asper acerba tuens vasta se mole ferebat, \\ Excutiens cervice toros; ea frena furenti \\ Concuntit et stimulos sub pectore vertit Apollo . \\ 
                     \lbrack ... \rbrack 
                 \\ Dat iuveni et tenuis  \lbrack fugit \rbrack  ceu fumus in auras: \\ Ille autem inpavidus et munere victor amici \\ Emicat in currum et manibus molitur habenas. \\ Vt ventum ad sedes, reddi sibi poscit honorem; \\ Adiungi generum miro properabat amore. \\ Tum sic mortalis referebat pectore voces: \\ Non haec humanis opibus, non arte magistra \\ (Accipio agnoscoque libens) tibi ducitur uxor, \\ Omnis ut tecum meritis pro talibus annos \\ 
        \pagebreak 
    \begin{center} \textbf{CARMNA} \end{center}Exigat et possit parvos educere natos’ \\ Haec ubi dicta dedit, solio se tollit ab alto \\ Iam senior mediisque parant convivia tectis. \\ Iuterea magnum sol circumvolvitur annum, \\ Parcarumque dies et vis inimica propinquat \\ Egregium forma iuvenem iam morte sub aegra: \\ Iamque dies infanda aderat et tempora Parcae \\ Debita conplerant crudeli morte sodalis. \\ Vt primum fari potuit crinitus Apollo, \\ Multa gemens casuque animum concussus amici, \\ Ipsius ante oculos sic fatis ora resolvit: \\ ‘Disce tuum, ne me incuses, volventibus annis \\ Advenisse diem; nam lux inimica propinquat’ \\ Hacc ubi deflevit, caeli cui sidera parent, \\ Tunc sic pauca refert fatis adductus iniquis: \\ ‘Phoebe, tot incassum fusos patiere labores? \\ Nil nostri miserere? mori me denique cogis? \\ Eripe me his, invicte, malis; miserere tuorum, \\ Si qua fata sinant, et eris mibi magnus Apollo’ \\ Tlibus oranti sic ore effatus amico est: \\ ‘Desine fata deum flecti sperare precando. \\ Sed cape dicta memor, duri solacia casus. \\ Obiectare animam quemquam aut opponere morti \\ Fas et iura sinunt; prohibent nam cetera Parcae. \\ Audiat haec genitor: patet atri ianua Ditis; \\ Hactenus indulsisse vacat’ sic fatus Apollo \\ Mortalis visus medio sermone reliquit. \\ Tunc vero ancipiti mentem formidine pressus \\ Obstipuit, cui fata parent, quem poscat Apollo; \\ Ire ad conspectum cari genitoris et ora \\ 
        \pagebreak 
    \begin{center} \textbf{CODICIS SALMASIANI.} \end{center} \marginpar{[53]} Cogitur et supplex animum temptare precando; \\ Multaque praeterea longaevo dicta parenti \\ Cum fletu precibusque tulit, ne vertere secum \\ Cuncta pater fatoque urgenti incumbere vellet; \\ Ecce iterum stimulat, sed nullis ille movetur \\ p. 19 \\ Fletibus aut voces ullas tractabilis audit. \\ Tunc genitor natum dictis affatur amicis: \\ Non, ut rere, meas effugit nuntius auris; \\ Infelix causas nequiquam nectis inanes. \\ Hoc uno responso animum delusit Apollo. \\ Stat sua cuique dies, lacrimae volvuntur inanes. \\ Vutere sorte tua: patet atri ianua Ditis.’ \\ Talia perstabat memorans fixusque manebat. \\ Egregia interea coniunx in limine primo \\ Agnovit longec gemitus (praesaga mali mens), \\ Tunc sic pauca refert: ‘Quid, o pulcherrime coniuns, \\ Tare, age, quid venias? quae causa indigna serenos \\ Foedavit vultus? quae te fortuna fatigat? \\ Quaecumque est fortuna, mea est.’ et talia fata \\ Demisit lacrimas factoque hic fine quievit. \\ Ille autem gemitus imo de pectore ducens \\ Talia voce refert: ‘Quid me alta silentia cogis \\ Rumpere et obductum verbis vulgare dolorem? \\ Eloquar an sileam? luctum ne quaere tuorum; \\ Vixi et, quem dederat cursum fortuna, peregi. \\ Iamque dies nisi fallor adest; crinitus Apollo \\ Hos mihi praedixit luctus, pro nomine tanto \\ Obiectare animam seu certae occumbere morti’ \\ At regina gravi iamdudum saucia cura, \\ 
        \pagebreak 
    \begin{center} \textbf{CARMINA} \end{center} \marginpar{[54]} Tristior et lacrimis et pallida morte futura, \\ Deficit ingenti luctu (miserabile visu) \\ Atque illum, talis iactantem pectore curas, \\ Talibus affata est dictis seque obtulit ultro \\ Decrevitque mori: ‘breve et inreparabile tempus p. . \\ Omnibus est vitae neque habet fortuna regressus: \\ Sed moriamur’, ait, ‘nihil est, quod dicta retractent \\ Concordes stabili fatorum numine Parcae, \\ Si fratrem Pollux alterna morte redemit. \\ Est hic, est animus, lucis contemptor et istum \\ Qui vita bene credat emi: nova condere fata \\ Nec morte horremus; sub terras ibit imago, \\ Si te fata vocant; in mne mora non erit ulla.’ \\ Ergo aderat promissa dies lacrimansque gemensque \\ Debita conplerat pesti devota futurae. \\ Testatur moritura deos stratisque relictis \\ Incubuitque toro dixitque novissima verba: \\ ‘O dulcis coniunx, dum fata deusque sinebant, \\ Fortunati ambo, scirent si ignoscere manes: \\ Te propter alia ex aliis in fata vocamur. \\ His lacrimis vitam damus et miserescimus ultro, \\ Quod te per superos et conscia numina veri, \\ Per conubia nostra, per inceptos hymenaeos \\ Adiuro et repetens iterumque iterumque monebo. \\ O lulcis coniunx, castum servare cubile \\ Sis emor; extremum hoc munus morientis habeto, \\ Si bene quid de te merui, lectumque iugalem \\ Natis parce tuis. Sic, sic iuvat ire sub umbras. \\ Hanc siue me spem ferre tui, audentior ibo. \\ Iussa mori feror ingenti circumdata nocte. \\ 
        \pagebreak 
    \begin{center} \textbf{CODICIS SALMASIANI.} \end{center} \marginpar{[55]} Haec sunt, quae nostra liceat te voce moneri. \\ I decus i nostrum, melioribus utere fatis.’ \\ Iaec effata silet, pallor simul occupat ora. \\ Nam quia nec fato ingeminat iam frigida cumba, \\ Sed misera ante diem, matrum de more locuta, \\ Multa patri mandata dabat, solatia luctus: \\ Interea dulces pendent  \lbrack (circum \rbrack  oscula nati; \\ Illa manu moriens umeros dextramque tenebat \\ Amborum et vultum. lacrimis ingressus obortis \\ ‘0 dolor atque decus magnum, sanctissima coniunx, \\ Tu lacrimis evicta meis, per sidera iuro, \\ Per superos, haerent infixi pectore vultus \\ Verbaque; per caeli iucundum lumen et auras, \\ Dum memor ipse mei, dum spiritus hos regit artus \\ Oblitus fatorum, † \rbrack  manet alta mente repostum; \\ Quisquis honos tumuli, quidquid solamen humandi est, \\ Servati facimus. semper celebrabere donis, \\ Et cum frigida mors anima seduxerit artus, \\ Ipse tibi ad tua templa feram sollemnia dona, \\ Cui tantum de te licuit. neque enim ipsa feretur \\ Fama levis tantive abolescet gratia facti. \\ Funeris heu tibi causa fui! quas dicere grates, \\ Quasve referre parem fati sortisque futurae? \\ Aeternam moriens famam tam certa tulisti, \\ Contra ego vivendo vici mea fata superstes \\ Morte tua vivens.’ media inter talia verba \\ ‘Non lacrimis hoc tempus eget’ Cyllenia proles, \\ 
        \pagebreak 
    \begin{center} \textbf{CARMNA} \end{center} \marginpar{[56]} ‘Adceleremus’ ait; ‘nos flendo ducimus horas.’ \\ Regina ut tectis venientem conspicit hostem, \\ Agnoscit lacrimans sua nunc promissa reposci; \\ Tempus’, ait, ‘deus, ecce deus’ cui talia fanti \\ Dilapsus color atque in ventos vita recessit. \\ 
      \end{verse}
  
            \subsection*{16}
      \begin{verse}
      \poemtitle{ \lbrack MAVORTI \rbrack }B. M. \\ \poemtitle{De ecclesia}B. IV 214. \\ Tectum augustum, ingens, centum snblime columnis, \\ Religione patrum laetum et venerabile templum \\ Hoc dedit esse suum superi regnator Olympi. \\ Nam deus omnipotens, qui res hominumque deumque \\ Aeternis regit imperiis, ‘quo tenditis’? inquit, \\ ‘Hic domus est vobis; haec ara tuebitur omnis. \\ Hic matres puerique simul mixtaeque puellae \\ Sacra canunt pariterque oculos ad sidera tollunt. \\ Hic exaudiri voces, hic vota precesque; \\ Noctes atque dies ferit aurea sidera clamor.’ \\ Postquam prima quies et facta silentia tectis, \\ Incipit effari divino ex ore sacerdos: \\ ‘Accipite haec animis laetasque advertite mentes, \\ Matres atque viri, pueri innuptaeque puellae. \\ Discite iustitiam moniti et spes discite vestras. \\ Haut incerta cano: deus aethere missus ab alto \\ Ipsius a solio regis, via prima salutis, \\ 
        \pagebreak 
    \begin{center} \textbf{CODICIS SALMASANI.} \end{center} \marginpar{[57]} Quem nobis partu sub luminis edidit oras \\ Virginis os habitumque gerens, mirabile dictu. \\ Ore  \lbrack dei \rbrack  afflata est spiritu propiore paritque. \\ Sic nova progenies caelo descendit ab alto. \\ Ast ubi iam firmata virum te prodidit aetas, \\ Negavere deum miseri, quibus ultimus esset \\ Ille dies, †quando furentes ac dira canentes \\ Insontem magno ad regem clamore trahebant. \\ Ille nihil (namque ipse volens), seque obtulit ultro, \\ Hoc ipsum ut strueret, vatum praedicta priorum \\ Prodere iussa dei, telluris operta subire. \\ p.23 \\ Primus ibi ante omnes, sceptrum qui forte gerebat, \\ Sustulit ablutas lymphis ad sidera palmas \\ Hoc dicens: ‘equidem in iusto nil tale repertumst; \\ Nec fas. o miseri, quae tanta insania, cives? \\  \lbrack At \rbrack  me nulla dies tantis neque fortibus ausis \\ Addiderit socium. vestra’ inquit ‘munera vobis: \\ Vos animam hanc potius quocumque absumite leto.’ \\ Tum magis atque magis magnis furoribus acti \\ Clamores simul horrendos ad sidera tollunt \\ Et magis atque magis poenas cum sanguine poscunt. \\ Has inter voces medio in flagrante tumultu \\ 
        \pagebreak 
    \begin{center} \textbf{CARMINA} \end{center} \marginpar{[58]} Arboris obnixus trunco (tibi, magne, tropaeum, \\ Omnipotens genitor!) palmas utrasque tetendit \\ Teque vocans multo vitam cum sanguine fudit. \\ Et tamen interea tua nati maxima cura \\ Non tulit hanc speciem; graviter commotus et alto \\ Dat clarum e caelo signum. na tempore in illo \\ Sol medium caeli conscenderat igneus orbem; \\ Eripiuut subito nubes caelumque diemque \\ Et nox atra polum bigis subvecta tenebat. \\ Tres tenuit caeli spatium non amplius horas: \\ Tunc repetens iterum sol clara in luce refulsit. \\ Nona fugae melior, rebus iam rite peractis. \\ Inde datum molitur iter. iamque arva tenebat \\ Scrupca, tuta lacu nigro nemorumque tenebris. \\ Vt statim ad fauces venit grave olentis Averni, p. \\ Tum demum horrisono stridentes cardine portae \\ Panduntur vastae solidoque adamante columnae; \\ Sponte sua umbrosae penitus patuere cavernae. \\ Ingreditur linquens antrum; tum maxima turba, \\ Vt videre deum fulgentiaque ora per umbras, \\ Ingenti trepidare metu. nec plnra moratus \\ Haec ait et dictis maerentia pectora mulcet: \\ ‘Ne trepidate, meae animaeque umbraeque pateruae; \\ Vobis parta quies. genitor mihi talia namque \\ Dicta dedit, divosque haec limina teudere adegit.’ \\ 
        \pagebreak 
    \begin{center} \textbf{CODICIS SALMASIANI.} \end{center} \marginpar{[59]} Haec fatus animas,quae per iuga longa sedebant, \\ Deturbat, antro miserans submittit aperto, \\ Et dicto parens supera ad convexa revexit. \\ Interea magnam subito vulgata per urbem \\ Fama volat, illum expirantem sedibus imis \\ Iam revocare gradum superasque evadere ad auras. \\ Obstipuere animis alii; set sanguinis auctor \\ Se causam clamat crimenque caputque malorum, \\ Et nodum informis leti trabe nectit ab alta \\ Proque suis meritis superis concessit ab oris. \\ Nec minus interea se matutinus agebat \\ Ad socios, quibus in mediis sic deinde locntus \\ Extulit os sacrum claraque in luce refulsit, \\ Omnia longaevo similis, cunctisque repente \\  \lbrack Improvisus ait \rbrack  ‘ \lbrack ... \rbrack  \\ Ire iterum in lacrimas? coram, que quaeritis, adsum. \\ En perfecta mei cari praecepta parentis. \\ Quare agite, o socii, tantarum in munera laudum \\ Ite’, ait; ‘egregias animas natique patrisque \\ Sermonum memores pluviali spargite lympba. \\ Ipse, ubi tempus erit, omnis in fonte lavabo.’ \\ Dixit et in caelum paribus se sustulit alis \\ Conditus in nubem; hinc regia tecta subivit \\ Dona ferens victor, cari genitoris et ora. \\ Oscula libavit dextramque amplexus inhaesit. \\ Huius in adventum cernes a sedibus imis \\ †Eruere summas arces et moenia verti \\ 
        \pagebreak 
    \begin{center} \textbf{CARMNA} \end{center} \marginpar{[60]} Atque omnem ornatum flamma crepitante cremari. \\ Tunc autem innumerae gentes populique frequentes \\ Terrentur visu subito. rex omnibus idem \\ Iura dabit populis pariter subigetque fateri, \\ Quae quis aput superos furto laetatus inani \\ 
                     \lbrack ... \rbrack 
                 \\ 
                     \lbrack ... \rbrack 
                 \\ 
                     \lbrack ... \rbrack 
                 \\ ‘Sed vos, o lecti, ferro pro nomine tanto, \\ Quod superest, moriamur et in media arma ruamus. \\ Sanguine quaerendi reditus animamque litando.’ \\ Haec ubi pro meritis, finem dedit ore precandi. \\ Succedunt alii graves aetate ministri. \\ Pars in frusta secant onerantque altaria donis; \\ Tum demum pueri et pavidae longo ordine matres \\ Stant circum. \\ Quos ubi confertos munus circumtulit omnes, \\ Sic prior adgreditur mensas atque incipit ipse; \\ Et postquam primus summo tenus adtigit ore, \\ Accipiunt proceres pariterque antistites omnes \\ Et pueri rudes; sequitur tunc cetera pubes. \\ Protinus ad reditum quisquis ac tecta domorum \\ Tendimus, et laetum semper celebramus honorem. \\ 
        \pagebreak 
    \begin{center} \textbf{CODICIS SALMASIANI.} \end{center} \marginpar{[61]}  \lbrack 6e Cumque Mavortio clamaretur ‘Maro iunior!’, ad prae \\ sen hoc recitavit \\ Ne, quaeso, ne me ad talis inpellite pugnas! \\ Namque erit ille mihi semper deus, ille magister. \\ Nam memini (neque enim ignari sumus ante malorum): \\ Formonsum pastor Phoebum superare canendo \\ Dum cupit et cantu vocat in certamina divos, \\ Membra deo victus ramo frondente pependit. \\ 
      \end{verse}
  
            \subsection*{17}
      \begin{verse}
      \poemtitle{ \lbrack HOSIDII GETAE \rbrack }B. I. 178. \\ M. 235. \\ \poemtitle{Medea  \lbrack tragoedia \rbrack }B IV 219. \\ Esto nunc Sol testis et haec mihi Terra. precanti \\ Et Dirae ultrices et tu, Saturnia luno! \\ Ad te confugio, nam te dare iura loquuntur. \\ Conubiis. si quid pietas antiqua labores \\ Respicit humanos, nostro succurre labori, \\ Alma Venus! quicnmque oculis haec aspicis aequis, \\ Accipite haec meritumque malis advertite numen! \\ Quid primum deserta querar? conubia nostra \\ 
        \pagebreak 
    \begin{center} \textbf{CARMINA} \end{center} \marginpar{[62]} Reppulit et sparsos fraterna caede penates. \\ Quid Syrtes aut Scylla mihi, quid vasta Charybdis \\ Profuerit mediosque Iugam tenuisse per hostis? \\ Inprobe amor, quid non mortalia  \lbrack pectora \rbrack  cogis? \\ Iussa aliena pati iterumque revolvere casus, \\ p.V7 \\ Ire iterum in lacrimas. sed nullis ille movetur \\ Fletibus, infixum stridit sub pectore vulnus. \\ Extinctus pudor atque inmitis rupla tyranni \\ Foedera et oblitus famae melioris amantis \\ Oblitusve sui est: lacrimae volvuntur inanes. \\ Nusquam tuta fides, vana spe lusit amantem \\ Crudelis. quid, si non arva aliena domosque \\ Ignotas peteret, pro virinitate reponit? \\ Heu pietas, heu prisca tlides! captiva videbo \\ Reginam thalamo cunctantem ostroque superbo \\ Iaut inpune quidem, si quid mea carmina possunt!Rerum cui summa potestas, \\ Precibus si flecteris ullis \\ Et si pietate meremur, \\ Nostro snccurre labori. \\ Et tu Saturnia Iuno, \\ Cui vincla iugalia curae, \\ Oculis haec aspicis aequis? \\ Nemorum Latonia custos \\ Triviis ululata per urbes, \\ Sic nos in sceptra reponis? \\ Quid, o pulcherrime coniunx, \\ 
        \pagebreak 
    \begin{center} \textbf{CODICIS SALMASIANI.} \end{center} \marginpar{[63]} Potuisti linquere solam, \\ Per tot discrimina rerum \\ Nequiquam erepte periclis? \\ Manet aa mente repostum, \\ p. 28 \\ Quam forti pectore et armis \\  \lbrack Medioque ex hoste recepit \rbrack  \\ Quaesitas sanguine dotes. \\ Felix, heu nimium felix, \\ Dum fata deusque sinebant! \\ Nescis heu perdita necdum, \\ Quae te dementia cepit \\ Caput obiectare periclis. \\ Iaec nos suprema manebant, \\ †Hoc ignes araeque parabant? \\ Nostram nunc accipe mentem: \\ Vaginaque eripe ferrum \\ Ferroque averte dolorem! \\ 
                    Creon
                    Medea
                 \\ Femina, quae nostris erras in finibus hostis, \\ Flecte viam velis; neque enim nescimus et urbem \\ Et genus invisum et non innoxia verba; \\ Hostilis facies  \lbrack ne \rbrack  occurrat et omina turbet. \\ Nullae hic insidiae nec tanta superbia victis, \\ Non ea vis animo, nec sic ad praelia veni. \\ Non ut rere meas effugit nuntius auris, \\ Vnde genus ducis varium et mutabile semper. \\ Tu potes unanimes armare in praelia fratres \\ 
        \pagebreak 
     \marginpar{[64]} \begin{center} \textbf{CARMINA} \end{center}Funereasque inferre faces et cingere flamma, \\ Pacem orare manu et vertere sidera retro \\ Atque odiis versare domos. tibi nomina mille, \\ Mille nocendi artes  \rbrack  fecundaque poenis \\ Viscera  \rbrack  \rbrack  notumque, furens quid femina possit. \\ Cede locis pelagoque volans da vela patenti! \\ Rex, genus egregium, liceat te voce moneri. \\ Pauca tibi e multis, quoniam est oblata facultas, \\ Dicam equidem, licet arma mihi mortemque mineris. \\ Ne pete conubiis natam: meminisse iuvabit; \\ Dissice conpositam pacem; miserere tuorum! \\ Ne tantos mihi finge metus neve omine tanto \\ Prosequere: causas nequiquam nectis inanes. \\ Stat sua cuique dies. non ipsi excindere ferro \\ Caelicolae valeant, fati quod lege tenetur. \\ Nec mea iam mutata loco sententia cedit.Non equidem invideo genero dignisque hymenaeis, \\ Non iam coniugium antiquum, quod prodidit, oro: \\ Tempus inane peto: liceat subducere classem; \\ Extremam hanc oro veniam. succurre relictae, \\ Dum pclago desaevit hiems, miserere parentis, \\ O genitor! et nos aliquod nomenque decusque \\ Gessimus \lbrack et \rbrack  scis ipse neque est te fallere quidquam. \\ Nunc victi tristes (quoniam fors omnia versat) \rbrack  \\ Submissi petimus terram litusque rogamus \\ Innocuum, neque te ullius violentia vincat. \\ Quid causas petis  \rbrack  in me exitiumque meorum? \\ Cr. \\ 
        \pagebreak 
    \begin{center} \textbf{CODICIS SALMASIAN.} \end{center} \marginpar{[65]} Quidquid id est, timeo vatum praedicta priorum. \\ Eia age, rumpe moras; quo me decet usque teneri? \\ . Quem sequimur? quove ire iubes? ubi ponere sedes? \\ Ire ad conspectum cari genitoris et ora, \\ Dum curae ambiguae, dum spes incerta futuri. \\ Nunc scio quid sitamor. hospitio prohibemur arenae, \\ Nec spes ulla fugae, nulla hinc exire potestas, \\ Quassataeque rates, geminique sub ubere nati, \\ Et glacialis hiems aquilonibus asperat undas. \\ Si te nulla movet tantae pietatis imago, \\ Indulge hospitio noctem non amplius unam! \\ Hanc sine me spem ferre tui; audentior ibo. \\ Desine iam tandem: tota quod mente petisti, \\ Largior et repetens iterumque iterumque monebo. \\ Si te his adtigerit terris Aurora morantem, \\ Vnum pro multis dabitur caput. \\ Vox deintus. Chorus \\ O digno coniuncta viro, dotabere virgo. \\ Ferte facis propere, thalamo deducere adorti. \\ Ore favete omnes et cingite tempora ramis. \\ Velamus fronde per urbem \\ Votisque incendimus aras. \\ Heu corda oblita tuorum \\ Vatum praedictn priorum, \\ Fati sortisque futurae! \\ Spe multum captus inani \\ Mactat de more bidentis \\ Phoeboque patrique lLyaeo, \\ Cui vincla iugalia curae, \\ 
        \pagebreak 
     \marginpar{[66]} \begin{center} \textbf{CARMINA} \end{center}Cumulatque altaria donis. \\ Tremere omnia visa repente, \\ Fibrae apparere minaces, \\ Vox reddita fertur ad aures: \\ ‘Thalamis neu crede paratis, \\ Funus crudele videbis.’ \\ p. VI \\ Carpebant membra quietem, \\ Animalia somnus habebat, \\ Ferali carmine bubo \\ In fletum ducere voces: \\ Tristis denuntiat iras. \\ Quae tanta insania, cives, \\ Velati tempora ramis? \\ Thalamo deducere adorti \\ Quaeso miserescite regis! \\ Recubans sub tegmine fagi \\ Divino carmine pastor \\ Vocat in certamiua divos: \\ Ramo frondente pependit. \\ Quae te dementia cepit, \\ Saxi de vertice pastor, \\ Divina Palladis arte \\ Phoebum superare canendo? \\ Raptim secat aethera pinnis \\ Fugiens Minoia regna \\ Ausus se credere caelo \\ Vitamque relinquit in auras. \\ Demens videt agmina Pentheus, \\ Incensas pectore matres; \\ Vocat agmina saeva sororum: \\ 
        \pagebreak 
    \begin{center} \textbf{CODICIS SA.MASIANI.} \end{center} \marginpar{[67]} Caput a cervice revulsum, \\ Iuvenem sparsere per agros. \\ Medea. Nutrix \\ En quid ago? vulgi quae vox pervenit ad aures? \\ Obstipui magnoque irarum fluctuat aestu \\ p. 39 \\ Dnrus amor; taedet caeli convexa tueri. \\ Quae potui infelix! quae memet in omnia verti, \\ Cui pecudum fibrae, caeli cui sidera parent, \\ Heu furiis incensa feror! stat gratia facti. \\ Illum ego per flammas et mille sequentia tela, \\ Per varios casus, per tot discrimina rerum \\ Eripui leto. fateor, arma impia sumpsi. \\ Sed quid ego haec autem nequiquam ingrata revolvo? \\ Quid loquor aut ubi sum? ictum iam foedus et omnes \\ Conpositae leges. credo, mea vulnera restant. \\ Non hoc ista sibi tempus spectacula poscit, \\ Sed cape dicta memor, duri solacia casus. \\ Sensibus hic imis nostram nunc accipe mentem: \\ Heu fuge crudelis terras, fuge litus avarum! \\ Cara mihi nutrix, claudit nos obice pontus, \\ Deest iam terra fugae; rerum pars altera adempta est. \\ Hac gener atque socer patriaque excedere suadet. \\ Tu ne cede malis, sed contra audentior ito; \\ Et quo quemque modo fugias \lbrack que \rbrack  ferasque laborem, \\ Tu modo posce deos veniam, tu munera supplex \\ Tende petens pacem causasque innecte morandi \\ Carminibus: forsan miseros meliora sequentur. \\ Nunc oblita mihi tot carmina. \rbrack  vox faucibus haesit; \\ 
        \pagebreak 
    \begin{center} \textbf{CARMINA} \end{center} \marginpar{[68]} Mens inmota manet et caeco carpitur igni. \\ Carmina vel caelo possunt deducere lunam, \\ Sistere aquam fluviis, deducere montibus ornos. \\ Has herbas atque haec ponto mihi lecta venena \\ Ipse dedit; nihil ilte deos, nuil carmina curat. \\ Quid struis? aut qua spe inimica in gente moraris? \\ Aut pugnam aut aliquid iam dudum invadere magnum; \\ Seu versare dolos, seu certae occumbere morti. \\ Iason. Satelles. Mede \\ Quod votis optastis, adest: timor omnis abesto. \\ Hic domus, haec patria est, nullum marisaequoraran \\ Solvite corde metum tandem tellure potiti  \lbrack dum. \\ Per varios casus. bene gestis corpora rebus \\ Procurate, viri; iuvat indulgere choreis. \\ Vnde tremor terris? qua vi maria alta tumescunt? \\ Quid tantum Oceano properent se tingere soles? \\ Nescio quod certum est: in nubem cogitur aer. \\ Aspice convexo nutantem pondere mundum, \\ Et fratris radiis obnoxia surgere luna. \\ Media fert tristis sucos, infecta venenis, \\ Quo thalamum eripiat atque ossibus implicet ignem. \\ Fare age quid venias iam istinc et conprime gressum. \\ Ad te confugio, precibusque inflectere nostris. \\ O dulcis coniunx, non haec sine numine divum \\ Eveniunt. || tanta meae si te ceperunt taedia laudis, \\ 
        \pagebreak 
    \begin{center} \textbf{CODICIS SALMASIANI.} \end{center} \marginpar{[69]} Hos cape factorum comites, his moenia quaere! \\ Non fugis hinc praeceps, dum praecipitare potestas \\ Iam propiore deo? nescis, heu perdita, nescis, \\ Nec quae te circumstent deinde pericula, cernis! \\ Hanc quoque deserimus sedem. tibi ducitur uxor. \\ Cui, pater et coniun, quondam tua dicta relinquor? \\ Et sedet hoc animo, dotalis regia cordi est \\ Externique iterum thalami. \\ Mene fugis? hoc sum terraque marique secuta? \\ Hic labor extremus, longarum haec meta viarum, \\ Hi nostri reditus expectatique triumphi? \\ Quid tua sancta fides? iterum crudelia retro \\ Fata vocant. tantis nequiquam erepte periclis, \\ Mene fugis? per ego has lacrimas, per si quis amatae \\ Tangit honos animum, et mensas, quas advena adisti, \\ Per connbia nostra, per inceptos hymenaeos \\ Te precor:  \lbrack o \rbrack  miserere animi non digna ferentis. \\ Namque aliud quid sit, quod iam inplorare queamus? \\ Ipse mihi nuper Libycis tu testis in undis: \\ Tum rauca adsiduo longe sale saxa sonabant; \\ Ionioque mari tantis surgentibus undis, \\ Luctantis ventos tempestatesque sonoras \\ Conpressi et rabiem tantam caeliqne marisque. \\ Vnius in miseri exitium proque omnibus unum \\ Obieci caput, id sperans fore munus amanti. \\ Sed quid ego ambages et iussa exorsa revolvo? \\ Nil super imperio moveor; speravimus ista \\ 
        \pagebreak 
    \begin{center} \textbf{CARMINA} \end{center} \marginpar{[70]} Tempore, quo primum fortes ad aratra iuvencos \\ Obieci et tauros spirantis naribus ignem, \\ Seminibusque satis inmanis dentibus hydri \\ Exiluit legio et campo stetit agmen aperto, \\ Telorum seges et iaculis increvit acutis. \\ Ferrea progenies duris caput extulit arvis. \\ Illi inter sese magna vi vulnera miscent, \\ Confixique suis telis et pectora duro \\ Transfossi ligno animasque in vulnera ponunt. \\ Auro ingens coluber servabat in arbore ramos \\ Nec visu facilis nec dictu affabilis ulli. \\ Ille manum patiens inmania terga resolvit. \\ Vt me conspexit flammantia lumina torquens, \\ Cervicem inflexam posuit somnosque petivit. \\ Si te nulla movet tantarum gloria rerum, \\ Sin absumpta salus nec habet fortuna regressum, \\ Si nulla est regio, miseris quam det tua coniunx, \\ I decus i nostrum, faciat te prole parentem \\ Egregia interea coniunx melioribus, opto, \\ Auspiciis!  \lbrack  possem  \rbrack  hinc asportare Creusam: \\ Spero equidem mediis, si quid pia numina possunt, \\ Supplicia hausurum scopulis; dabis, inprobe, poenas, \\ Quod minime reris, rebus iam rite paratis. \\ Desine meque tuis incendere teque querellis. \\ 
        \pagebreak 
    \begin{center} \textbf{CODICIS SALMASIANI.} \end{center} \marginpar{[71]} Nam mihi parta quies, uullum maris aequor arandum, \\ Nec veni, nisi fata locum sedemque dedissent. \\ Heu tot incassum fusos patiere labores, \\ Nec venit in mentem fumans sub vomere taurus, \\ Iam gravior Pelias et aena undantia flammis \\ Squamosusque draco et quaesitae sanguine dotes? \\ In regnis hoc ausa tuis \\ p. 36 \\ laec loca non tauri spirantes naribus ignem, \\  \lbrack Invertere satis inmanis dentibus hydri \rbrack ; \\ Nec galea densisque virum seges horruit hastis, \\ Nec vim tela ferunt: mitte hanc de pectore curam. \\ Nam quis te, iuvenum confldentissime, nostras \\ Iussit adire domos? pelagine erroribus actus, \\ An fratris miseri letum ut crudele videres? \\ Sive errore viae seu tempestatibus acti \\  \lbrack Auguriis agimur divom; feror exul in altum \rbrack . \\ Quis deus in fraudem, quae  \lbrack te \rbrack  demeutia cepit \\ Commaculare manns, fraterna caede penates? \\ Aut ego tela dedi aut vitam committere ventis \\ Hortati sumus?  \lbrack aut quae dura potentia nostra? \\ Nil nostri miserere, nihil mea carmina curas; \\ Efficiam, posthac ne quemquam voce lacessas. \\ Nec dulcis natos, Veneris nec praemia noris. \\ Quid causas petis || et inrita iurgia iactas? \\ 
        \pagebreak 
    \begin{center} \textbf{CARMNA} \end{center}Iamque vale, melior quoniam pars acta diei est. \\ Vtere sorte tua, susceptum perfice munus. \\ Nunc iter ad regem nobis. quod te adloquor, hoc est. \\ Num fletu ingemuit nostro aut miseratus amantem  \lbrack est \rbrack  ? \\ Et dubitamus adhuc? lacrimantem et multa volentem \\ Dicere deseruit rapidusque in tecta recessit. \\ Quid labor aut benefacta iuvant? mea tristia fata \\ Fessa iacent. ubi nunc nobis deus ille magister \\ Et Fnriis agitatus amor et conscia virtus? \\ Nam quid dissimulo aut quaeme ad maiora reservo? . 3 \\ Stat casus renovare omnis, dare lintea retro, \\ Rursus et  \lbrack est \rbrack  casus abies visura marinos. \\ Te sine, frater, erit. quod si mea numina non sunt, \\ Flectere si nequeo superos, Acheronta movebo! \\ Chorus \\ Dictis exarsit in iras \\ Insani Martis amore, \\ Poenorum qualis in arvis \\ Venantum septa corona \\ Fulva cervice leaena; \\ Qualis mala gramina pastus \\ Tractu se colligit anguis. \\ Tumidum quem bruma tegebat: \\ Caput altum in proelia tollit, \\ Linguis micat ore trisulcis; \\  \lbrack Qualis  \lbrack ... \rbrack  \\ Furiis agitatus Orestes \\ 
        \pagebreak 
    \begin{center} \textbf{CODICIS SALMASIANI.} \end{center}Armatam facibus matrem; \\ Ardens agit aequore toto \\ Patriasque obtruncat ad aras; \\ Triviis ululata per urbem \\ Qualis trieterica Baccho \\ Inter deserta ferarum, \\ Palla subcincta cruenta, \\ Vocat agmina saeva sororum; \\ Qualis philomela sub umbra \\ Pectus signata cruentum \\ Late loca questibus implet, \\ Maerens miserabile carmen, \\ p. 38 \\ Cantu solata laborem; \\  \lbrack Qualis miserabilis Orpheus \rbrack  \\ Graviter pro coniuge saevit \\ Deserti ad Strymonis undam: \\ Te solo in litore secum \\ Anima fugiente vocabat, \\ Scirent si ignoscere Manes. \\ Nuntius. Creon \\ Quoferor? unde abii?  \lbrack rumpit \rbrack  pavor,ossaque et artus \\ Perfudit toto proruptus corpore sudor. \\ Genua labant,  \lbrack gelidus \rbrack  oculos stupor urget inertis, \\ Arrectaeque horrore comae et vox faucibus haesit. \\ Quo res summa loco? unde haec tam clara repente \\ Tempestas sine more furit? maria omnia caelo \\ Miscuit, ingeminant abruptis nubibus ignes. \\ 
        \pagebreak 
    \begin{center} \textbf{CARMMNA} \end{center} \marginpar{[4]} Fare age || mihique haec edissere vera roganti. \\ Aedibus in mediis quaeque ipse miserrima vidi \\ Horresco referens. palla subcincta cruenta \\ In medioque focos nocturnas inchoat aras \\ Intenditque locum sertis et fronde coronat \\ Funerea, crinem vittis innexa cruentis, \\ Vnum exuta pedem vinclis, in veste recincta, \\ Spargens humida mella soporiferumque papaver. \\ Sparserat et latices simulatos fontis Averni, \\ Sanguineam volvens aciem, manibnsque cruentis \\ Pro molli viola casiaque crocoque rubenti \\ Vrit odoratam nocturno in lumine cedrum \\ Scillamque elleborosque gravis et sulfura viva, \\ Obscuris vera involvens lacrimisque coactis \\ Voce vocans lecaten; et non memorabile numen \\ Ferro accincta vocat. \\ Haec effata silet, oculis micat acribus ignem, \\ Expectans, quae signa ferant, ignara futuri. \\ Eripiunt subito nubes caelumque diemque, \\ Et tremefacta solo tellus; micat ignibus aether. \\ Continuo auditae voces vagitus et ingens; \\ Visus adesse pedum sonitus et saeva sonare \\ Verbera;  \lbrack tum \rbrack  visaeque canes ululare per umbras \\ Adventante dea, refluit \lbrack que \rbrack  exterritus amnis \\ Et pavidae matres pressere ad pectora natos. \\ 
        \pagebreak 
    \begin{center} \textbf{CODICIS SALMASIANI.} \end{center}Exhinc Gorgoneis Allecto infecta venenis \\ Exurgitque facem adtollens atque intonat ore: \\ ‘Respice ad haec; adsum dirarum ab sede sororum, \\ Rella manu letumque gero.’ \\ Talia cernentem tandem sic orsa vicissim: \\ ‘Venisti tandem. mecum partire laborem, \\ Tu dea, tu praesens animis inlabere nostris. \\ Dissice conpositam pacem, sere crimina belli \\ (Namque potes), colui vestros si semper honores.’ \\ Talibus Allecto dictis exarsit in iram \\ Horrendum stridens rabidoque haec addidit ore: \\ ‘O germana mihi, mitte hanc de pectore curam. \\  \lbrack Et \rbrack  nunc, sibellare paras etluctu miscere hymenaeos \\ Funereasque inferre faces et cingere flamma, \\ Quidquid in arte mea possum, meminisse necesse est, \\ Quantum ignes animaeque valent: absiste precando.’ \\ Dixerat; adtollit stridentis anguibus alas, \\ p. 40 \\ Ardentis dare visa faces, supera ardua linquens. \\ Ila dolos operi flammisque sequacibus iras \\ Iungebat, duplicem gemmis auroque coronam \\ Consertam squamis serpentum: flamma volantem \\ Inplicat involvitque domum caligine caeca, \\ Prospectum eripiens oculis; mihi frigidus horror \\ Membra quatit gelidusque coit formidine sanguis: \\ lnprovisum aspris veluti qui sentibus anguem \\ 
        \pagebreak 
    \begin{center} \textbf{CARMINA} \end{center} \marginpar{[76]} Aut videt aut vidisse putat, metuensque pericli \\ Incipit eflari, nec vox aut verba sequuntur. \\ Idque audire sat est, quo me decet usque teneri. \\ Vadite et haec regi memores mandata referte. \\ Nutrix. Medea \\ Hoc habet. haec melior magnis data victima divis. \\ Talia coniugia et talis celebrent hymenaeos! \\ Tu secreta pyram natorum maxima nutrix \\ Erige,  \lbrack  tuque ipsa pia tege tempora vitta, \\ Verbenasque adole pinguis nigrumque bitumen. \\ Sacra lovi Stygio, quae rite incepta paravi, \\ Perficere est animus finemque inponere curis. \\ Discessere omnes medii spatinmque dedere. \\ Medea. Filii. Vmbra Apsyrti \\ Heu stirpem invisam et fatis contraria nostris! \\ Huc ades, o formose puer. qui spiritus illi! \\ Sic oculos, sic ille manus, sic ora ferebat! \\ Perfidus et cuperem ipse parens spectator adesset. \\ Parce pias scelerare manus! aut quo tibi nostri \\ Pulsus amor? || si iuris materni cura remordet, \\ Natis parce tuis et nos rape in omnia tecum; \\ Quo res cumque cadunt, unum et commune periclum. \\ Aspice nos. adsum dirarum ab sede sororum:  \\ Infelix simulacrum, || laniatum corpore toto. \\ 
        \pagebreak 
    \begin{center} \textbf{CODICIS SALMASNI.} \end{center}Quid dubitas? audendum dextra, nunc ipsa vocat res. \\ Auctor ego audendi. fecundum concute pectus. \\ Si concessa peto, si poenas ore reposco, \\ Nullum in caede nefas;  \lbrack et \rbrack  amor non talia curat. \\ Hostis amare, quid increpitas mea tristia fata? \\ Suggere tela mihi finemque inpone labori. \\ Sanguine quaerendi reditus. \\ Nec te noster amor pietas nec mitigat ulla, \\ Nec venit in mentem natorum sanguine matrem \\ Conmaculare manus? nostri tibi cura recessit \\ Et matri praereptus amor? \\ Crimen amor vestrum spretaeque iniuria formae \\ His mersere malis. fratrem ne desere frater. \\ Poenarum exhaustum satis est, via facta per hostis, \\ Et genus invisum dextra sub Tartara misi. \\ Iamiam nulla mora est currus agitare volantis. \\ Iason. Nuntius. Medea ex alto \\ Ei mihi, quid tanto turbantur moenia luctu? \\ Quaecumque estfortuna,mea est; quid denique restat? \\ Dic age, namque mihi fallax haut ante repertus. \\ En perfecta tibi promissa coniugis arte \\ Munera! || ingentem luctum ne quaere tuorum. \\ Sed si tantus amor menti, si tanta cupido est, \\ Expediam dictis et te tua fata docebo. \\ Conspectu in medio cum dona inponeret aris \\ (A virgo infelix!) oculos deiecta decoros, \\ 
        \pagebreak 
    \begin{center} \textbf{CARMINA} \end{center} \marginpar{[78]} Vndique conveniunt per limina laeta frequentes \\ Matres atque viri cumulantque altaria donis. \\ Religione patrum biforem dat tibia cantum, \\ Cum subito dictuque oritur mirabile monstrum. \\ Ecce levis summo descendit corpore pestis, \\ Incipit ac totis Vulcanum spargere tectis, \\ Regalisque accensa comas, accensa coronam \\ Membra sequebantur, artus sacer ignis edebat. \\ Diffugiunt comites et quae sibi quisque timebat \\ Tecta metu petiere, et sic ubi concava furtim \\ Saxa petunt, furit inmissis Vulcanus habenis. \\ Nec vires herbarum infusaque flumina prosunt, \\ Quaesitaeque nocent artes, miserabile dictu! \\ †Illa autem per populos aditumque per avia quaerit, \\ Arte nova speculata locum, paribusque revinxit \\ Serpentum spiris ventosasque addidit alas, \\ Ense levis nudo, perfusos sanguine currus. \\ Quo sequar?aut quidiam misero mihi denique restat? \\ Me me, adsum qui feci, || in me omnia tela \\ Conicite, hanc animam quocumque absumite leto! \\ Funeris heu tibi causa fui; dux femina facti! \\ Huc geminas nunc flecte acies et conde sepulcro \\ Corpora natorum, cape dona extrema tuorum. \\ Ettumulum faciteet tumulosuperaddite carmen: p.43 \\ ‘Saevus amor docuit natorum sanguine matrem \\ Conmaculare manus, luctu miscere hymenaeos \\ 
        \pagebreak 
    \begin{center} \textbf{CODCIS SALMASAN.} \end{center} \marginpar{[79]} Et super aetherias errare licentius auras.’ \\ Ias. Crudelis mater, tanton me crimine dignum \\ Duxisti et patrios foedasti funere vultus? \\ Arma, viri, ferte arma! || date tela, ascendite muros! \\ Quo moriture ruis? thalamos ne desere pactos! \\ Iortator scelerum, nostram nunc accipe mentem. \\ Sive animis sive arte vales,  \lbrack si pectore robur \\ Concipis, \rbrack  et si adeo dotalis regia cordi est; \\  \lbrack Quae nunc deinde mora est \rbrack  nostrasne evadere demens \\ Sperasti te posse manus? opta ardua pinnis \\ Astra sequi clausumque cava te condere terra \\ Et famam extingui veterum sic posse malorum. \\ Haec via sola fuit, haec nos suprema manebat \\ Exitiis positura modum. \\ Sat fatis Venerique datum est. feror exul in altum, \\ Germanum fugiens et non felicia tela, \\ Vltra anni solisque vias. quid denique restat? \\ Et longum, formose, vale, et quisquis amores \\ Aut metuet dulces aut experietur amaros. \\ 
      \end{verse}
  
            \subsection*{18}
      \begin{verse}
      \poemtitle{Epithalamium Tridi a}B. ILuxor.84. \\ M. 382. \\ \poemtitle{LVXORIO}B. IV 237. \\ \poemtitle{uiro clarissimo  \lbrack et \rbrack  speetabili dictum centone.}Sol, qui terrarum flammis opera omnia lustrat, \\ Etulit os sacrum caelo tenebrasque resolvit. \\ 
        \pagebreak 
    \begin{center} \textbf{CARMNA} \end{center} \marginpar{[80]} Laetitia ludisque viae plausuque fremebant, \\ p. 44 \\ At Venus aetherios inter dea candida nimbos \\ Aurea subnectens exertae cingula mammae, \\ Dona ferens, pacem aeternam pactosque hymenaeos \\ Atque omnem ornatum, Capitolia celsa tenebat, \\ Punica regna videns, Tyrios et Agenoris urbem. \\ Hinc atque hinc glomerantur Oreades et bona Iuno; \\ Incedunt pariter pariterque ad limina tendunt. \\ Tectum augustum, ingens, centum sublime columnis, \\ Hae sacris sedes epulis, atque ordine longo \\ Perpetuis soliti patres considere mensis. \\ Vna omnes, magna iuvennm stipante caterva, \\ Deveniunt faciemque deae vestemque reponunt. \\ Dant signum, fulsere ignes et conscius aether \\ Conubiis, mediisque parant convivia tectis. \\ Fit strepitus tectis vocemque per ampla volntant \\ Atria, ubi adsuetis biforem dat tibia cantum. \\ At tuba terribilem sonitum procul aere canoro \\ Increpuit mollitque animos et temperat iras. \\ It clamor caelo, cithara crinitus lopas \\ Obloquitur numeris septem discrimina vocum, \\ Iamque eadem digitis, iam pectine pulsat eburno. \\ Nec non et Tyrii per limina laeta frequentes \\ Convenere, toris iussi discumbere pictis. \\ Tunc Venus aligerum dictis affatur Amorem: \\ ‘Nate, meae vires, mea magna potentia solus, \\ Huc geminas nunc flecte acies, illam aspice contra, \\ Quae vocat insignis facie viridique iuventa, \\ Iam matura viro, iam plenis nubilis annis, \\ p. 45 \\ 
        \pagebreak 
    \begin{center} \textbf{CODICIS SALMASIANI.} \end{center} \marginpar{[81]} Cui genus a proavis ingens clarumque paternae \\ Nomen inest virtutis et nota maior imago. \\ Hoc opus, hic labor est: thalamos ne desere pactos! \\ Credo equidem, nova mi facies inopinave surgit. \\ Nonne vides, quantum egregio decus enitet ore? \\ Os humerosque deo similis, cui lactea colla \\ Auro innectuntur, crines nodantur in aurum, \\ Aurea purpuream subnectit fibula vestem; \\ Qualis gemma micat, qualis Nereia Doto \\ Et Galatea secant spumantem pectore pontum. \\ Cura mihi comitumque foret nunc una mearum! \\ Hanc ego nunc ignaram huius quodcumque pericli est, \\ Cum tacet omnis ager, noctem non amplius unam \\ Conubio iungam stabili propriamque dicabo. \\ Hlic lymenaeus erit monumentum et pignus amoris. \\ Incipe si qua animo virtus, et consere dextram, \\ Occnltum inspires ignem paribusque regamus \\ Auspiciis: liceat Frido servire marito, \\ Cui natam egregio genero dignisque hymenaeis \\ Dat pater et pacem hanc aeterno foedere iungit.’ \\ Paret Amor dictis carae genetricis et alas \\ Exuit et gressu gaudens sic ore locutus: \\ ‘Mecum erit iste labor; si quid mea numina possunt, \\ Cum dabit amplexus atque oscula dulcia figet \\ Inmiscentque manus manibus pugnamque lacessunt, \\ Nusquam abero, solitam flammam (datur hora quieti) \\ Desuper infundam et, tua si mihi certa voluntas, \\ 
        \pagebreak 
    \begin{center} \textbf{CARMINA} \end{center} \marginpar{[2]} Omnia praecepi atque animo mecum ante peregi. \\ Sentietl’ atque animum praesenti piguore firmat. \\ Illa autem (neque enim fuga iam super ulla pericli est) \\ Cogitur et supplex animos summittere amori, \\ Spemque dedit dubiae menti solvitque pudorem. \\ Illum turbat amor; ramum qui veste latebat \\ Eripit a femine et flagranti fervidus infert. \\ It cruor inque humeros cervix conlapsa recumbit. \\ His demum exactis geminam dabit llia prolem, \\ Laeta deum partu, centum conplexa nepotes. \\ 
      \end{verse}
  
            \subsection*{19}
      \begin{verse}
      B. M. \\ \poemtitle{Praefatio}B. IV 241. \\ 
        \pagebreak 
    \begin{center} \textbf{CODICIS SALMASIANI.} \end{center} \marginpar{[83]} 
        \pagebreak 
    \begin{center} \textbf{CARMINA} \end{center} \marginpar{[84]} 
      \end{verse}
  
            \subsection*{20}
      \begin{verse}
      \poemtitle{OCTAVIANI}\poemtitle{viri inlustris annorum XVI, filii Crescentini viri magnifici.}B. M. — \\ B. IV 244. \\ \poemtitle{Sunt vero versus CLXXII.}Candida sidereo fulgebat marmore Cypris, \\ Nec cinctam reddit nobilis arte lapis, \\ Mystica secreti dirumpens claustra pudoris \\ Cum urtica  \lbrack e \rbrack  gremio prosilit aetherio. \\ Proles heu niveis nutritur pessima membris \\ Gratum iamque locum protegit herba ferox. \\ Sed recte factum. celantur fervida membra, \\ Cultibus ut lateat tecta libido malis. \\ Mulciber an Martem metuens hoc sponte peregit, \\ Horreat ut Mavors dulcia adulteria? \\ Sordet pulcra Venus, temnuntur Cypridis artus! \\ Quid placeat nobis, si Venus ipsa piget? \\ 
        \pagebreak 
    \begin{center} \textbf{CODICIS SALMASIAN1.} \end{center} \marginpar{[85]} 
      \end{verse}
  
            \subsection*{21}
      \begin{verse}
      B. M. \\ B. IV 244. \\ \poemtitle{‘sacrlegus capite puniatur’. De templo | Neptuni . aurum periit.
                    interposito tempore piscator piscem aureum posuit et titulo inseribsitc ‘De
                    tuo tbi ep tune’. Reus fit sacrilegii. contra dicit.  \lbrack convincitur \rbrack .} \lbrack Prooemium \rbrack  \\ ‘Vnde redit fulgor templis? quis inania nuper \\ Tantis Salsipotis distendit limina donis? \\ Ecce abiit damnum: splendescunt tecta metallis \\ Marmora et antiquus cepit laquearia fulgor’ \\ Pone animos laetos, quisquis testantia furtum \\ Dona vides. titulis votum quod lucet opimis, \\ Gaudendum fuerat, nisi munus pauperis esset. \\ Heu scelus et magnis nequiquam prodiga rebus \\ Mens humilis! miseros semper quam maxima produnt! \\ Sordidus et nigrae dudum vagus accola harenae \\ Nunc aurum piscator habet gaudetque metallis: \\ Nec satis est: donat templis, per limina figit \\ Et titulo confessus ovat. consurgite in iram, \\ Quis caelum, quis templa placent! modo limine in omni \\ Supplex, maiorum portans munuscula mensis \\ Vel tenuem spectabat opcm; nunc ditior illis \\ Quos coluit, meliorque deo est; quod perdidit ille, \\ Hic donat. prorsus magna est iniuria Nerei. \\ Dignus non fuerat titulis, nisi perderet aurum? \\ 
        \pagebreak 
    \begin{center} \textbf{CARMINA} \end{center} \marginpar{[86]} Non  \lbrack tantum \rbrack  facinus caeso est auctore piandum? \\ Multa patent, sed pluralatent. scelusundique densum est. \\ Tollere rem templis furor est temploque vicissim \\ Rem furti donare nefas. pro dira nocentum \\ Consilia in scaevis! quae mens excogitet istud: \\ Res auferre sacras et consecrare rapinas? \\ Sit similis vindicta malo (nunc ipsa pudori est \\ Vox mea, ne magnos laedat magis ultio divos. \\ Audiet haec populus nosque hoc narrabimus; ergo \\ Quod factum est (meminisse nefas), referetur in urbe: \\ Elusus custos raptumque altaribus aurum, \\ Mens audax, scelus†hoc, manus inproba, perditus ardor, \\ Antistes victus, penetralia prodita, numen \\ Contemtum, templum pauper, piscator abundans. \\ Vos, o caelicolae, vestrum nunc invoco numen, \\ Sit mihi fas reticenda loqui, dum proditur iste. \\  \lbrack Narratio \rbrack  \\ Natus ut, ignotum est. neque enim de limine celso \\ Piscandi doctus ducit genus. inprobus ergo, \\ Cum tantas terris dederit labor inclitus artes, \\ Non Chalybum massas recoquit, non doctior aeris \\ Ducit molle latus fulvumve intentus in aurum \\ Multiplici gemmas radiantes lumine vestit, \\ Non ager in voto est illi fortesque iuvenci, \\ Non inlex fenus, non classica, non pia Musa, \\ Sed spretis divum rebus placet omnibus istud: \\ 
        \pagebreak 
    \begin{center} \textbf{CODCIS SALMASIANI.} \end{center} \marginpar{[87]} FTraus, dolus et furtum pelagi. conponitur ergo \\ Saeta nocens, fallax calamus et perfidus amus, \\ Principium sceleris: iam tunc, iam perfidus iste \\ Neptunum spoliare parans petit alta profundi \\ Nereos et vitreo resupinos marmore campos. \\ Illic sollicite  \lbrack per \rbrack  saxa madentia curas \\ p. 50 \\ Disponens imoque trahens animalia fundo \\ Serus furtivum referebat munus ad urbem. \\ †Sed palam sane viderunt moenia, saepe \\ Dum relevant populos vario commercia pisce. \\ Cernere erat genus omne maris, conpleret ut urbem: \\ Hinc scarus, hinc varius, hinc purpura, polypus inde, \\ Hinc murena ardens, illinc aurata coruscans \\ Et cancer mordax, tergo et russante locusta, \\ Thynnus, salpa, pager, lupus, ostrea, sepia, mullus \\ Et quidquid captum faciebat copia vile. \\ Proderat hoc illi tantum ad conpendia vitae, \\ Nec dabat ars alind; quamvis praedives adesset \\ Mercatus populi, tamen hinc manus ista nocentis \\ Vix erat aere gravis, nedum copiosior auro. \\ Laudatus sane, quantum spectabat ad artem, \\ Et stulte multis dictus ‘Neptuius heros’; \\ Hlinc etiam adsiduus templo, dum solus ad aras, \\ Solus ad altare est precibusque insistere cultor \\ Creditur et placidos pelagi sibi poscere fluctus, \\ 
        \pagebreak 
     \marginpar{[88]} \begin{center} \textbf{CARMMNA} \end{center}Aurum (pro facinus), veterum donaria, priscum \\ Obsequium, antiquum munus, videt, arripit, aufert. \\ Quis populigemitus, quis tunc concursus in urbe! \\ Quis fuit ille dies, miseri cum pendere poenas \\ Custodes iussi fuso de sanguine crimen \\ Ignotum insontes luerent facinusque negarent! \\ Heu male magnorum semper sub nomine tali \\ Velamen scelerum. vilis persona: quis ergo, \\ Despiciens hominem, tantum quis crederet umquam \\ Pauperis esse nefas? volitat cum funere dives \\ p. 51 \\ Multorum. Nec scire potest sua crimina solus. \\ Hoc rursus magna statuere primordia rerum, \\ Quod cito tam prodit crimen, quam concipit ardor. \\ Mens hominum facinus sine fine admitteret ullo, \\ Si posset celare diu: cultoris honore \\ Sacrilegus lucet, mauibusque ablata nocentis \\ Post spatium produnt crimen redeuntia dona. \\ Excessus \\ Huc huc tergemino letalia fulmina telo, \\ Iuppiter undarum, valido, Neptune, tridenti \\ Concutiens maria alta iace pontoque verendus \\ Litoreas transcende moras! stet turbidus axis \\ Nubibus et ephyris fundo revolutus ab imo \\ Gurges inexpletum feriat vada marmore cano! \\ Piscator scaevus meritum confundit utrimque; \\ Stat post furta pius, templis tua munera reddens, \\ Et post dona reus. pro vili, summa potestas, \\ 
        \pagebreak 
    \begin{center} \textbf{CODCIS SALMASIANI.} \end{center} \marginpar{[89]} Bis tibi calcato facta est iniuria caelo: \\ Cum tua sacrilegus raperet donaria templo, \\ Contemptus fueras; iam nunc obnoxius esse \\ Coepisti, ablatum post \lbrack quam \rbrack  tibi reddidit aurum. \\ ‘rs inquit‘studiumquededitmihi,non scelus, aurum’. \\ Verum est? Eoos etenim mercator adisti \\ Et repetis patriam longo post tempore dives? \\ Scilicet his manibus viduatos cernimus esse \\ Ture Arabas, Persen gemmis et vellere Seres, \\ Dente Indos, ferro Chalybes et murice Poenos? \\ Non pudet hanc artem, scelerum,  \lbrack te \rbrack  dicere, princeps? \\ Remus, cumba, fretum, gurges, notus, ancora, lembus, p. \\ Barca, amus, puex, conchae, vada, litus, harena, \\ Contus, seta, salum, calamus, †notae, retia, suber: \\ Hic labor, haec ars est, hinc fulvum colligis aurum! \\ Mercator madidus, parvae stipis actor, ad aurum \\ Vt venias,  \lbrack id \rbrack  scire velim; quem quando patronus \\ Maximus antiquo donarit tegmine vestis, \\ Mensibus ignorant maria intermissa clientem. \\ ‘Quis me’ inquit ‘tantum facinus committere vidit?’ \\ Hoc bene habet, haec vox mihi  \lbrack iam confessio pura est. \\  \lbrack Probatio \rbrack  \\ Nunc ergo incipiam crimen sic pandere verbis, \\ Vt visum te, scaeve, putes. ergo omnis ob istud \\ 
        \pagebreak 
    \begin{center} \textbf{CARMINA} \end{center} \marginpar{[90]} Huc ades,  \lbrack o \rbrack  iudex, facinus. signantia rebus \\ Argumenta feram, magno quae septa vigore \\ Interdum visus fallunt et crimina produnt. \\ Omne equidem furtum, dirus quod concipit ardor, \\ His nisi nunc fallor rebus constare necesse est, \\ An locus admittat facinus conplerier an non, \\ An valeat persona nefas committere tantum. \\ Singula si excutimus, casurum est crimen in istum. \\ Ergo, ut distinctum est, videamus ab ordine primo, \\ An locus admittat facinus conplerier an non. \\ Templum est, unde istud sublatum dicimus aurum. \\ Maxima res: venerandus honos, custodia nulla, \\ Quod mane inpactis foribus vix vespere nigro \\ Stridula cardinibus claudit antica retortis. \\ Hoc patet adsidue, patet omnibus, utpote quisque \\ Insistit precibus, nec fas est claudere postes. \\ Ingressos nullos servat custodia, nulla \\ p. 53 \\ Egressos, licet  \lbrack et \rbrack  semper discurrere ad aras, \\ Omnibus et simulacra modis contingere miris; \\ Dona etiam veterum populorum, insignia regum, \\ Et laudare licet cunctis et tangere fas est. \\ Ianitor hinc longe est primoque in limine custos. \\ lpse etiam interdum penitus discedit ab aris \\ Antistes metuitque precantibus arbiter esse. \\ Hinc facilis causa scelerum facilisque malorum. \\ Nullus custodit templum, quia creditur aras \\ Caelicolum servare timor. patet omnibus ergo. \\ 
        \pagebreak 
    \begin{center} \textbf{CODICIS SALMASIAN.} \end{center} \marginpar{[91]} Exemplum \\ Sic Phrygiae spes sola perit, dum milite lecto \\ Palladii numen servantibus undique Teucris \\ Ingressus templum furtim non creditur hostis, \\ Et licet liacus flammam Vestamque regentem \\ Ioc metuens Priamus muris vallasset et armis, \\ Dum tamen ingressos fas qui sint poscere non est, \\ Invisum e templis antistes fugit Vlixen. \\ Non mirum est ergo, quod nos sic perfidus iste \\ Decepit, templis numquam suspectus et aris, \\ Sicut Pergameas caesis custodibus aras \\ Audax, ut numen raperet, penetravit Achivus. \\  \lbrack Probatio \rbrack  \\ Nunc quoniam cunctis manifestum cernimus esse \\ Ad causam scelerum templum patuisse rapinis, \\ Quod sequitur, certo tractandum examine rerum est: \\ An valeat persona nefas committere tantum. \\ Quid metuat pauper (neque enim est iam dives habendus, \\ Et cum dona ferat; quamvis maria alta peragret \\ Perditus et templis furtivum congreget aurum, \\ Pauper erit, cui nullus honos, cui gratia nulla, \\ Non clarus genitor, non noto semine mater)? \\ Scilicet horrescit, prisco sine nomine avorum \\ Ne cadat in fasces, miser undique, solus ubique! \\ An non hoc genus est, cuius de examine multo \\ Quisquis honoratos respexit forte potentes \\ Ob meritum fulgere viros, mox improbus, audax, \\ Fortunam incusans et tetro lividus ore \\ 
        \pagebreak 
     \marginpar{[92]} \begin{center} \textbf{CARMINA} \end{center}Pauperiem monstrat superis ac pectore laevo \\ Dira quiritatus fundit convitia caelo? \\ Pauperis omne nefas: facile scelus aptus ad omne, \\ In pretium pronus, despectu numinis audax, \\ Vilis, iners, scaevus, turpis, temerarius, ardens, \\ Perditus, abiectus, maledictus, sordidus, amens. \\ An non sunt isti, quorum de nomine multi, \\ Ducere concessis dum nolunt artibus aevum, \\ Caedibus infamant silvas et crimine cauto \\ Insidias tendunt domibus gregibusque rapinas? \\ In quibus haut ulla est caro de sanguine cura; \\ Pactas temporibus vendunt in proelia mortes. \\ An vobis mirum est, furtum quod fecerit ille, \\ Sanguinis et vitae pretium cui extinguit honorem? \\ Nunc age, si veris tractavimus undique causis, \\ Pauperis esse nefas, quidquid peccatur in orbe, \\ Quod superest, positis iam rebus ab ordine primo \\ An vindex sceleris sit raptor  \lbrack et ipse \rbrack , videndum est. \\ Neptuni e templo votivum perdimus aurum. \\ Heu male cum diris altaria iuncta metallis! \\ p. 55 \\ Qui primus templis aurum dedit omine diro, \\ Is causa scelerum primus fuit. omne paratas \\ In facinus mentes hominum succendier auro \\ Non scierat? rectis semper contraria rebus \\ Fulva metallorum est rabies; haec proelia miscet, \\ Haec castos vendit thalamos, haec polluit aras. \\ Mille nocendi artes. volumus si visere priscos, \\ 
        \pagebreak 
     \marginpar{[93]} \begin{center} \textbf{CODICIS SALMASANI.} \end{center}Dicite, quod facinus commissum non sit ob aurum. \\ Auro ardet Glauce, Danae corrumpitur auro, \\ Auro emitur Pluton, Phlegethon transcenditur auro, \\ Proditur Ampbiaraus atque Hector venditur auro; \\ HIoc Medea maga est, serpens vigil, exul Iason, \\ Hoe Mida ieiunus, Paris ultus, †naufragus lelles, \\ Hoc †sapiens Furia, Venus invida, Iuno cruenta, \\ Hippomenes cursu velox, hoc tarda Atalante est. \\ Aurum quod nigris Pactolus miscet harenis, \\ Quod condit tellus, tristis quod celat Avernus, \\ Quod ferrum intundit, liquidus quod conficit ignis, \\ Quod furor exposcit demens, quod praelia †saeva, \\ Quod raptum queritur coluber, quod Punica virgo \\ Amissum plangit, Tyria damnandus in aula \\ Pygmalion caeso quod perdit fraude Sichaeo, \\ Quod tutum nec templa tenent nec pauperis ardor. \\  \lbrack Refutation \rbrack  \\ ‘Qui raperet, donum templis non redderet’ inquit. \\ Sentio, quas nobis subrepto praeparet auro \\ Callidus ambages: templorum abscondere furem \\ Cultoris temptat donis et divite censu \\ Pauperiem foedam scelerum causamque malorum \\ Excusat largus; nos autem insistimus inde. \\ Hoc ideo factum est, ut crimen frangere possis. \\ 
        \pagebreak 
    \begin{center} \textbf{CARMINA} \end{center} \marginpar{[94]} Hinc etiam est illud, docto quod concipis astu, \\ Squamigerum in piscem raptum vertatur ut aurum, \\ Vt titulum inscribas ‘Tibi nunc, salis alme profundi, \\ Quod dedimus, Neptune,tuum est’. pulcreomnia, pulcre \\ Dissimulas; sed vera patent. iam frangere votum est \\ Hoc quoque, quod longo meditatum tempore profers, \\ Argumentum ingens: ‘templis non redderet aurum, \\ Qui tulerat.’ macte, scelerum doctissime rhetor, \\ Verborum auxiliis subverso crimine rerum! \\ Rteddere te donum deus inpulit, inpulit ardor, \\ Inpulit et scelerum mens conscia, conpulit index \\ Furtorum semper timor anxius atraque mentis \\ Tristities pallensque metus resecansque medullas \\ Post causam raptus trepidis penitudo secunda. \\ Haec scaevos vexant. non sunt, mihi credite, non sunt \\ Eumenides dirae, fallax quas fabula narrat \\ Cocyti in gremio rapidi, Phlegethontis ad ignes \\ Tartarei, incinctas facibus, serpente, flagellis: \\ Sed metus et facinus, sed mens est conscia pravi. \\ Ni fallor, victum est, magno quod protulit astu. \\ Sed superest pars magna mihi de crimine vero: \\ ‘Qui raperet, totum templis non redderet’ ergo \\ Hoc quoque sic vincam, verum fatearis ut ipse. \\ Sustuleras templis; partiris, perfide, furtum; \\ Non totum reddis: superavit copia mentem. \\ 
        \pagebreak 
    \begin{center} \textbf{CODICIS SALMASIANI.} \end{center} \marginpar{[95]} Nunc quoniam manifesta fides gradibusque malorum \\ Hinc illinc lucent conlatis crimina rebus, \\ Officium invadam, valeant ut cernere cuncti, \\ Piscandi doctis semper nil nequius esse. \\ Hic taceam, audaces ducit quod pallida semper \\ In scelus omne fames, secretaque litora cogunt. \\ Hoc loquor: infaustis levior cum scanditur alnus, \\ Quid faciant remo celeri lemboque volantes, \\ Excussum ventis pelagus cum litora frangit: \\ Naufragium expectant. sidit cum rapta sub undas \\ Puppis, submersi fundo scrutantur harenas. \\ At cum lassatus portum vix navita vidit, \\ Furta parant missosque secant in litora funes. \\ O scelerum auctores, tetro et cum crimine ponti \\ Cladum participes et tempestatis amici! \\ Haec quoque si excutitur, quam magni criminis ars est! \\ Non scelus est, unco piscem quod fallitis amo, \\ Quod placidas subter lina intertexitis undas? \\ Piscibus adsuetis fallaces tendere morsus \\ Neptuni pulcrum visum est non parcere templis. p. i6 \\ Epilogus \\ Iam satis haec. factis  \lbrack mea \rbrack  vox inpensa nefandis \\ Piscantis facinus cecinit versuque coegit \\ Aurum, templa, nefas, titulos, epigrammata, munus. \\ Supplicium restat scelerum, quod reddere debet \\ Iudex, horrendo tollens tortore securim. \\ 
        \pagebreak 
    \begin{center} \textbf{CARMINA} \end{center} \marginpar{[96]} Dicite, quos ius est examina figere causis, \\ Dicite iam poenas mandatas legibus almis. \\ Vos quoque, quis ferro mortales caedere fas est, \\ Cum iam damnati iugulos ac colla petetis, \\ Ne campis patriaeque loco nec caedite inxta. \\ Deprecor. ad nigras ducatur vinctus harenas, \\ Vltima despumans pelagus qua litora lambit; \\ Hic iaceat medius ponto terrisque nefandus, \\ Et cum sollicitum ventis mare tollitur alte, \\ Destruat unda rogum rapiantque animalia corpus. \\ Hic tamen expositis tumulos conponite membris \\ Et titulum facite et versu hoc includite carmen: \\ ‘Piscibus hic vixit, deprensus piscibus bic est, \\ Piscibus occubuit. spes crimen poena sub uno est.’ \\ 
      \end{verse}
  
            \subsection*{22}
      \begin{verse}
      00 \\ B. III 260. \\ M. 1005. \\ \poemtitle{Epithalamium}B. IV 256. \\ Ite, verecundo coniungite foedera lecto \\ Atque Cupidineos discite ferre iocos. \\ Concordesque tegat cum maiestate benigna \\ Quae regit Hdalium, quae Cnidon alma regit. \\ Alliget amplexus ‘tenerorum mater Amorum’ \\ Constituat \lbrack que \rbrack  patres ut cito reddat avos. \\ 
        \pagebreak 
    \begin{center} \textbf{CODICIS SALMASANI.} \end{center} \marginpar{[97]} 
      \end{verse}
  
            \subsection*{23}
      \begin{verse}
      B. III 278. \\ M. 1013. \\ \poemtitle{Verba amatoris ad pictorem}Pinge, precor, pictor, tali caudore puellam, \\ p. 59 \\ Qualem pinxit amor, qualem meus ignis anhelat. \\ Nil pingendo neges; tegat omnia Serica vestis, \\ Quae totum prodat teuui velamine corpus. \\ Te quoque pulset amor, crucient pigmenta medullas; \\ Si bonus es pictor, miseri suspiria pinge. \\ 
      \end{verse}
  
            \subsection*{24}
      \begin{verse}
      B. III 279. \\ M. 1014. \\ \poemtitle{Amans amanti}B. IV 256. \\ Dic quid agis, formosa Venus, si nescis amanti \\ Ferre vicem? perit omne decus, dum deperit aetas. \\ Marcent post rorem violae, rosa perdit odorem, \\ Lilia post vernum posito candore liquesunt. \\ Haec metuas exempla precor, et semper amanti \\ Redde vicem, quia semper amat, qui semper amatur. \\ 
      \end{verse}
  
            \subsection*{25}
      \begin{verse}
      B. III 280. \\ M. 1015. \\ \poemtitle{Rescriptum}B. IV 237. \\ Non redit in florem, sed munus perdit amantis, \\ Quidquid vile iacet. dulce est, quodcumque negatur. \\ Nam si formosa facili penetratur amore, \\ FTacit adulterium, sed munus perdit amantis. \\ 
        \pagebreak 
     \marginpar{[98]} \begin{center} \textbf{CARMNA} \end{center}
      \end{verse}
  
            \subsection*{26}
      \begin{verse}
      \poemtitle{MARTIALIS}B. III 59. \\ M. 280. \\ \poemtitle{De habitatione ruris}B. IV 116. \\ Rure morans quid agam, respondeo pauca, rogatus. \\ Mane deos oro; famulos, post arva reviso \\ Partitusque meis iustos indico labores. \\ Deinde lego Phoebumque cio Musamque lacesso. \\ Hinc oleo corpus fingo mollique palaestra \\ Stringo libens. animo gaudens et fenore liber \\ Prandeo, poto, cano, ludo, lavo, cene, quiesco. \\ Dum parvus lychnus modicum consumit olivi, \\ Haec dat nocturnis elucubrata Camenis. \\ 
        \pagebreak 
    \begin{center} \textbf{CODICIS SALMASIANI.} \end{center} \marginpar{[99]} 
      \end{verse}
  
            \subsection*{27}
      \begin{verse}
      07 \\ B. I 166. \\ M. 687. \\ \poemtitle{De Progne et Philomela}B. IV 257. \\ Da sensus mihi, Phoebe, precor; nam poena puellae \\ Non habet exemplum. tristis post funera linguae \\ Sanguis inest pingitque cruor tormenta pudoris. \\ 
      \end{verse}
  
            \subsection*{28}
      \begin{verse}
      Vr \\ \poemtitle{LINDINI}B. III 94. \\ M. 541. \\ B. IV 257. \\ \poemtitle{De aetate}Vitam vivere si cupis beatam \\ Et votis Lachesis dabit senectam, \\ Annos ludere te decem decebit, \\ Viginti studiis dabis severis, \\ Triginta pete litium tribuna, \\ Quadraginta stilo polita dicas, \\ Quinquaginta velim diserta scribas, \\ Sexaginta tuo satis fruaris, \\ Septuaginta velis venire mortem, \\ Octoginta senes caveto morbos, \\ Nonaginta time labante sensu, \\ Centum nec puer unus adloquetur. \\ 
        \pagebreak 
    \begin{center} \textbf{CARMINA} \end{center} \marginpar{[100]} 
      \end{verse}
  
            \subsection*{29}
      \begin{verse}
      \poemtitle{AVITI}B. III 259. \\ M. 259. \\ B. IV 258. \\ \poemtitle{Adlocutio sponsalis}Linea constricto de pectore vincula solve \\ Et domino te crede tuo. ne candida laedas \\ Vnguibus ora, vide, vel ne contacta repugnes. \\ Est in nocte timor, non est in nocte periclum. \\ Nec volo contendas: vinces, cum vicerit ille. \\ 
      \end{verse}
  
            \subsection*{30}
      \begin{verse}
      B. V 137. \\ M. 1075. \\ \poemtitle{De somnio ebriosi}B. V 258. \\ Phoebus me in somnis vetuit potare Lyaeum. \\ Pareo praeceptis: tunc bibo, dum vigilo. \\ 
      \end{verse}
  
            \subsection*{31}
      \begin{verse}
      B. III 87. \\ M. 935. \\ \poemtitle{De uvis}B. IV 258. \\ Vindicat ipsa suos, quos pertulit, uva labores; \\ Quae pede dum premitur, subtrabhit ipsa pedem. \\ 
      \end{verse}
  
            \subsection*{32}
      \begin{verse}
      B. I 24. \\ M. 584. \\ B. IV 258. \\ \poemtitle{De Libero patre}Orgia lassato quotiens solvuntur Iaccho, \\ p. 061 \\ Sic deus uda mero ponere membra solet. \\ 
        \pagebreak 
    \begin{center} \textbf{CODCIS SALMASIANI.} \end{center} \marginpar{[101]} 
      \end{verse}
  
            \subsection*{33}
      \begin{verse}
      B. I 76. \\ M. 621. \\ \poemtitle{De Luna et Musis}B. IV 259. \\ Phoeba sedens gremio cum fert †germana reclinem \\ Languidulos pueri respiciens oculos, \\ Vos Heliconiadae lentum submittite carmen: \\ Carmine somnus adest, carmine somnus abest. \\ 
      \end{verse}
  
            \subsection*{34}
      \begin{verse}
      B. I 67. \\ M. 611. \\ \poemtitle{De statua Veneris}B. IV 259. \\ In gremio Veneris quoddam genus herba virescit. \\ Sensit dura silex, quo foco exaestuet ignis. \\ 
      \end{verse}
  
            \subsection*{35}
      \begin{verse}
      B. V 167. \\ M. 1097. \\ \poemtitle{De vipera}B. IV 259. \\ Accensa in Venerem serpens genitalibus auris \\ Sic coit ut perimat, sic parit ut pereat. \\ Hi sunt affectus, baec oscula digna venenis, \\ †Coniugioque Venus semper amore nocens. \\ 
      \end{verse}
  
            \subsection*{36}
      \begin{verse}
      B. III 30. \\ M. 893. \\ \poemtitle{De balneis}B. IV 259. \\ Exultent Apono Veneti, Campaia Bais, \\ Graecia Thermopolis: his ego balneolis. \\ 
        \pagebreak 
     \marginpar{[102]} \begin{center} \textbf{CARMINA} \end{center}
      \end{verse}
  
            \subsection*{37}
      \begin{verse}
      0 4 \\ B. Lu. init. \\ M. 296. \\ \poemtitle{De titutlo Luxorii cum versibus}Priscos, Luxori, certum est te vincere vates; \\ Carmen namque tuum duplex Victoria gestat. \\ 
      \end{verse}
  
            \subsection*{38}
      \begin{verse}
      B . \\ M. 929. \\ \poemtitle{De fortuitis casibus}B. IV 260. \\ Omnia casus agit. fatum consulta sequuntur. \\ Cedamus fatis: omnia casus agit. \\ 
      \end{verse}
  
            \subsection*{39}
      \begin{verse}
      B. I 143. \\ M. 666. \\ \poemtitle{De Narcisso}B. IV 260. \\ Dum putat esse parem vitreis Narcissus in undis, \\ Solus amore perit, dum putat esse parem. \\ 
      \end{verse}
  
            \subsection*{40}
      \begin{verse}
      B. I 120. \\ M. 651. \\ \poemtitle{De iudicio Paridis}B. IV 260. \\ Iudicium Paridis provexit couiuge Troiam, \\ Decepit Troiam iudicium Paridis. \\ p. 62 \\ 
      \end{verse}
  
            \subsection*{41}
      \begin{verse}
      B. I 117. \\ M. 648. \\ \poemtitle{De equis Diomedis}B. IV 260. \\ Vim Diomedis equi monstrabant hospite caeso; \\ Hospite fregerunt vim Diomedis equi. \\ 
      \end{verse}
  
            
        \pagebreak 
    
            \begin{center} \textbf{CODICIS SALMASIANM.} \end{center}
             \marginpar{[103]} 
            \subsection*{42}
      \begin{verse}
      \poemtitle{De Polyxena}Deest. \\ 
      \end{verse}
  
            \subsection*{43}
      \begin{verse}
      nB. I 92. \\ M. 633. \\ \poemtitle{ \lbrack De Deidamia \rbrack }B. IV 261. \\ Credita virgo parem blande decepit Achilles, \\ Implevitque piam credita virgo parem. \\ 
      \end{verse}
  
            \subsection*{44}
      \begin{verse}
      B. I 132. \\ M. 661. \\ \poemtitle{De Orete et Clytaemestra}B. IV 261 \\ Pro pietate nefas matris purgavit Orestes; \\ Incurrit magnum pro pietate nefas. \\ 
      \end{verse}
  
            \subsection*{45}
      \begin{verse}
      B. I 134. \\ M. 663. \\ \poemtitle{De Pentheo et Agave}B. IV 261. \\ Fert miseranda caput, domino quod monstret, Agave. \\ Solum, quod doleat, fert miseranda caput. \\ 
      \end{verse}
  
            \subsection*{46}
      \begin{verse}
      B. I 116. \\ M. 647. \\ \poemtitle{De Turno et Pallante}. IV 261. \\ Turnus honore ruit fusi Pallantis in hostem; \\ Pallantis fusi Turnus honore ruit. \\ 
      \end{verse}
  
            
        \pagebreak 
    
             \marginpar{[104]} 
            \begin{center} \textbf{CARMINA} \end{center}
            \subsection*{47}
      \begin{verse}
      B. I 127. \\ M. 656. \\ \poemtitle{De Iasone et Medea}B. IV 261. \\ Coniugis arte decus patriae reduxit Iason; \\ Amisit patriae coniugis arte decus. \\ 
      \end{verse}
  
            \subsection*{48}
      \begin{verse}
      B. I 152. \\ M. 673. \\ e ero et Leandro \\ B. IV 261. \\ Fecit amore viam iuvenis crudele per aequor; \\ Praedurae morti fecit amore viam. \\ 
      \end{verse}
  
            \subsection*{49}
      \begin{verse}
      B. I 118. \\ M. 649. \\ \poemtitle{De Euryalo}B. IV 232. \\ Enicus Euryalus meruit solacia matris; \\ Ereptus matri est unicus Euryalus. \\ 
      \end{verse}
  
            \subsection*{50}
      \begin{verse}
      B. I 156. \\ M. 677. \\ \poemtitle{De Iyacintho}B. IV 262. \\ Sanguine flos genitus fraudem testatur Vlixis; \\ Servat erile decus sanguine flos genitus. \\ 
      \end{verse}
  
            \subsection*{51}
      \begin{verse}
      p. 63 \\ B. I 115. \\ M. 646. \\ \poemtitle{De Pallante}B. IV 262. \\ Quae dedit, ipsa tulit virtus Pallanta dolendum; \\ Prima dies, bello quae dedit, ipsa tulit. \\ 
        \pagebreak 
    \begin{center} \textbf{CODICIS SALMASIANI.} \end{center} \marginpar{[105]} 
      \end{verse}
  
            \subsection*{52}
      \begin{verse}
      B. I 129. \\ M. 658. \\ \poemtitle{Ee Creonte et Medea}B. IV 252. \\ Mens tibi dira, Creon, vel cum Medea fugatur, \\ Vel cum busta negas: mens tibi dira, Creon. \\ 
      \end{verse}
  
            \subsection*{53}
      \begin{verse}
      B. I 160. \\ M. 682. \\ \poemtitle{De Deidamia}B. IV 262. \\ Deidamia virum qua coepit nocte mereri, \\ Perdidit hac dulcem Deidamia virum. \\ 
      \end{verse}
  
            \subsection*{54}
      \begin{verse}
      B. I 133. \\ M. 662. \\ \poemtitle{De Theseo}B. IV 262. \\ Thesea magnanimum non cepit ianua Ditis; \\ Decepit coniunx Thesea magnanimum. \\ 
      \end{verse}
  
            \subsection*{55}
      \begin{verse}
      B. I 40. \\ M. 596. \\ \poemtitle{De Iunone et Hercule}B. IV 26. \\ Viribus Herculeis dum noxia facta requirit, \\ Iuno dedit laudem viribus Herculeis. \\ 
      \end{verse}
  
            \subsection*{56}
      \begin{verse}
      B I 68. \\ M 612. \\ \poemtitle{De Venere}B. IV 263. \\ Vritur igne suo fumantibus Aetna cavernis. \\ Pendet amore Venus: uritur igne suo. \\ 
        \pagebreak 
    \begin{center} \textbf{CARMINA} \end{center} \marginpar{[106]} 
      \end{verse}
  
            \subsection*{57}
      \begin{verse}
      B. I 94. \\ M. 634. \\ \poemtitle{De tractu Heetoris}B. IV 263. \\ Funere turbat equos necdum satiatus Achilles, \\ Hector et exanimis funere turbat equos. \\ 
      \end{verse}
  
            \subsection*{58}
      \begin{verse}
      B. I 122. \\ M. 691. \\ \poemtitle{De Aegypto et Danao}B. IV 263. \\ Perfida nox Danai dirarum caede sororum; \\ Mitis ypermestrae perfida nox Danai. \\ 
      \end{verse}
  
            \subsection*{59}
      \begin{verse}
      B I 10. \\ . 566. \\ \poemtitle{De oygno et Leda}B. IV 263. \\ Carmine dulcis olor dum virginis otia mulcet, \\ Texit furta Hovis carmine dulcis olor. \\ 
      \end{verse}
  
            \subsection*{60}
      \begin{verse}
      B I 158. \\ M. 680. \\ \poemtitle{De Calypso et Didone}B. IV 263. \\ Inputat aegra toris vim per deserta Calypso; \\ Vim Dido incensis inputat aegra toris. \\ p. 64 \\ 
      \end{verse}
  
            \subsection*{61}
      \begin{verse}
      B. I 69. \\ M. 613. \\ \poemtitle{De Adone etc Venere}B. IV 264. \\ Pingitur ora Venus, ne se contemnat Adonis; \\ t roget Armipotens, pingitur ora Venus. \\ 
        \pagebreak 
    \begin{center} \textbf{CODICIS SALMASIANI.} \end{center} \marginpar{[107]} 
      \end{verse}
  
            \subsection*{62}
      \begin{verse}
      B. I 46. \\ M. 599. \\ \poemtitle{De Castore et Polluee}B. IV 264. \\ Ordine mortis eunt alternae ad munera vitae, \\ Inque diem fratres ordine mortis eunt. \\ 
      \end{verse}
  
            \subsection*{63}
      \begin{verse}
      B. I 95. \\ M. 636. \\ \poemtitle{De Dolone et Aohille}B. IV 264. \\ Praemia magna Dolon currus dum poscit Acbillis, \\ Prodidit ipse cadens praemia magna Dolon. \\ 
      \end{verse}
  
            \subsection*{64}
      \begin{verse}
      B. I 167. \\ M. 688. \\ \poemtitle{De Progne et Philomela}B. IV 264. \\ Sanguine muta probat facinus Philomela sorori, \\ Vimque vice linguae sanguine muta probat. \\ 
      \end{verse}
  
            \subsection*{65}
      \begin{verse}
      B. I 162. \\ M. 684. \\ \poemtitle{De Lemniadibus}B. IV 264 \\ FTunera Lemniadum nescit veneranda Thoantis; \\ Sola tamen sensit funera Lemniadum. \\ 
      \end{verse}
  
            \subsection*{66}
      \begin{verse}
      B. I 109. \\ M. 643. \\ \poemtitle{De lauco et  \lbrack Archemoro \rbrack }B. IV 265. \\ Squameus anguis erat, qui reddidit aethera Glauco; \\ Qui tulit Archemoro, squameus anguis erat. \\ 
        \pagebreak 
     \marginpar{[108]} \begin{center} \textbf{CARMINA} \end{center}B. I 163. \\ M. 685. \\ 
      \end{verse}
  
            \subsection*{}
      \begin{verse}
      \poemtitle{De Pelope}B. IV 265. \\ Fervidus axe Pelops contemnit iura tyranni, \\ Fata ligat soceri fervidus axe Pelops. \\ 
      \end{verse}
  
            \subsection*{68}
      \begin{verse}
      B. I 70. \\ . 614. \\ \poemtitle{De Adone}B. IV 265. \\ Vulnera saevus aper laesae spectanda Dianae, \\ Flenda dedit Veneri vulnera saevus aper. \\ 
      \end{verse}
  
            \subsection*{69}
      \begin{verse}
      B. I 151. \\ M. 672. \\ \poemtitle{De Iyla et Hereule}B. IV 265. \\ Raptus aquator lylas: Nympharum gaudia crescunt. \\ Herculis ira tumet: raptus aquator Hylas. \\ 
      \end{verse}
  
            \subsection*{70}
      \begin{verse}
      B. \\ . \\ \poemtitle{De incesto partu}B. IV 265. \\ Prodita prole parens partus enixa biformes; \\ Facta paterna luit prodita prole parens. \\ B. I 164. \\ M. 686. \\ \poemtitle{De Caphareo monte}B. IV 265. \\ Relliquias Danaum dira premit arte Caphareus; \\ Albula Tuscus amat relliquias Danaum. \\ 
        \pagebreak 
    \begin{center} \textbf{CODICIS SALMASIAN.} \end{center} \marginpar{[109]} B. I 8. \\ M. 564. \\ 
      \end{verse}
  
            \subsection*{}
      \begin{verse}
      \poemtitle{De anymede  \lbrack et Xermaphroditol . c}Captus amante puer aquila moderante pependit; \\ Infamavit aquas captus amante puer. \\ B. I 150. \\ M. 671. \\ 
      \end{verse}
  
            \subsection*{}
      \begin{verse}
      \poemtitle{De Pyramo et Thisbe}B. IV 266. \\ Pallia nota fovet lacrimis decepta Themisto; \\ Pyramus heu lacrimis pallia nota fovet. \\ 
      \end{verse}
  
            \subsection*{74}
      \begin{verse}
      B. I 130. \\ M. 659. \\ \poemtitle{De Iocstae et Oedipo}B. IV 266. \\ Dirum Iocasta nefas, vel cum venit effera coniunx, \\ Vel cum fit mater: dirum locasta nefas. \\ B. I 153. \\ M. 674. \\ \poemtitle{De ippolto et Phaedra}. IV 266. \\ Vincere falsa pudor poterat. sed castus et insons \\ Erubuit Phaedrae vincere falsa pudor. \\ 
      \end{verse}
  
            \subsection*{76}
      \begin{verse}
      B. I 96. \\ M. 637. \\ †De tumulo Aehlllis \\ B. IV 266. \\ Iurgia conflat amor, ut blandius urat amantes: \\ Ad cumulum fidei iurgia conflat amor. \\ 
        \pagebreak 
    \begin{center} \textbf{CARMINA} \end{center} \marginpar{[110]} 
      \end{verse}
  
            \subsection*{77}
      \begin{verse}
      B. I 119. \\ M. 650. \\ \poemtitle{De Niso etc Euryalo}B. IV 266. \\ Nomen amicitiae magna pietate colendum est; \\ Maxima pars vitae est nomen amicitiae. \\ 
      \end{verse}
  
            \subsection*{78}
      \begin{verse}
      B. III 283. \\ M. 537. \\ Item unde upr \\ B. IV 267. \\ Mens, ubi amaris, ama: rarum est agnoscere amicos, \\ Rarum invenire  \lbrack est \rbrack  Mens, ubi amaris, ama. \\ 
      \end{verse}
  
            \subsection*{79}
      \begin{verse}
      B . \\ M. 577. \\ \poemtitle{De Apolline}B. IV 267. \\ Gratia magna tibi, Paean, qui pectora conples; \\ Lector, si faveas, gratia magna tibi. \\ 
      \end{verse}
  
            \subsection*{80}
      \begin{verse}
      B. Lux. init. \\ M. 297. \\ Epitaphion \\ p. 66 \\ B. IV 267. \\ Nil mihi mors faciet: pro me monumenta relinquo. \\ Tu modo vive, liber: nil mihi mors faciet. \\ 
        \pagebreak 
    \begin{center} \textbf{CODICIS SALMASIANI.} \end{center}\begin{center} \textbf{1 4 1} \end{center}
      \end{verse}
  
            \subsection*{81}
      \begin{verse}
      \poemtitle{PORYIRII}B. III 89. \\ M. 236. \\ V er u s an a c y e i ci \\ \poemtitle{B. IV 238}landitias fera Mors Veneris persensit amando, \\ Permisit solitae nec Styga tristitiae. \\ Tristitiae Styga nec solitae permisit, amando \\ Persensit Veneris Mors fera blanditias. \\ Omnipotens pater huic semper concessit amori, \\ Fecit nec requiem tot sibi fulminibus. \\ Fulminibus sibi tot requiem nec fecit amori, \\ Concessit semper huic pater omnipotens. \\ Purpureus tibi flos vultum nou pingit, Iacche, \\ Monstrat nec mitem frons nova laetitiam. \\ Laetitiam nova frons mitem nec monstrat, lacche, \\ Pingit non vultum flos tibi purpureus. \\ Occubuit minor hic, fractis et viribus astu \\ Torpuit oppressus Amphitryoniades. \\ Amphitryoniades oppressus torpuit astu \\ Viribus et fractis hic minor occubuit. \\ Incaluit iubar hoc externis iguibus ardens \\ Fortius; ardorem Sol sibi congeminat. \\ Congeminat sibi Sol ardorem; fortius ardens \\ Ignibus externis hoc iubar incaluit. \\ 
        \pagebreak 
    \begin{center} \textbf{1 10} \end{center}\begin{center} \textbf{CARMINA} \end{center}Deposita face Nox quaesivit lumina Pboebes, \\ Vulnere sed blandus hanc tenet Endymion. \\ . Endymion tenet hanc blandus sed vulnere; Phoebes \\ Lumina quaesivit Nox ace deposita. \\ Armipotens deus hoc suspirat pondere, vulnus \\ Ferrea nec rabies aut furor exsuperat. \\ p. 6 \\ Exsuperat furor aut rabies nec ferrea vulnus, \\ Pondere suspirat hoc deus armipotens. \\ npatiens Venus est, silvas dum lustrat Adonis, \\ Carpit si Martem, iam cui conveniat. \\ Conveniat cui iam, Martem si carpit, Adonis \\ Lustrat dum silvas, est Venus inpatiens. \\ 
      \end{verse}
  
            \subsection*{82}
      \begin{verse}
      B. III 79. \\ M. 17. \\ \poemtitle{De tabula}B. IV 269. \\ Has acies bello similes cano, quas Palamedes \\ Constituit. casu vario paribusque periclis \\ Inscius ac sollers sistunt se; namque superbis \\ Vana supervacue crescunt mendacia buccis \\ Et se sollertes punctis fallentibus inflant. \\ En proceres Fridi asseclae lususque magister. \\ Assidue similes mittuntur semper in †imo, \\ Vtque deus †dedit, se praefert inscius artis. \\ 
        \pagebreak 
    \begin{center} \textbf{CODICIS SALMASIANI.} \end{center}Quid labor ingeniumque iuvat? cur pallidus extat \\ Atque iners dubitat? securus ludat amator, \\ Nummos quisquis habet: discedat lividus hostis \\ Se doctum iactans. quodsi reppertor adesset \\ Princeps ac sollers, victum se saepe vocasset. \\ Adversis punctis doctum se nemo fatetur. \\ Vulnera plus crescunt punctis, quam bella sagittis. \\ 
      \end{verse}
  
            \subsection*{83}
      \begin{verse}
       \lbrack Eplstula. Dtdo Aeneae \rbrack  \\ B. I 173. \\ M. 1610. \\ Praefatlo \\ B. IV 271. \\ V \\ XII Sic tua semper ames, quisquis pia vota requiris, \\ Nostra libenter habe; quid carminis otia ludant, \\ Cerne bonus mentisque fidem probus indue iudex. \\ Dulce sonat, quod cantat amor; cui grata voluptas \\ Esse potest, modicum dignetur amare poetam. \\ p. 68 \\ Carmen \\ Debuit ingrato nullam dictare salutem \\ Laesus amor. sed nulla iuvant convitia flentem; \\ Si modo flere vacet. nam me magis, inprobe, mortis \\ Fata vocant. Troiane nocens, haec dona remittis? \\ Quamvis saepe gravi conponam carmine fletus, \\ Plus habet ipse dolor, nec conplent verba dolorem, \\ Quem sensus patientis habet. vertenda requiro, \\ 
        \pagebreak 
    \begin{center} \textbf{CARMINA} \end{center}\begin{center} \textbf{M} \end{center}Quae maledicta dedi, miseris circumdata fatis! \\ Pendet amore dolor, cassum dolor auget amorem. \\ Dum studet iratas calamus celerare querellas, \\ Continuit dolor ipse manuum, nec plura loquentem \\ Passus amor mentisque vias et verba ligavit. \\ A, quotiens revocata manus, dubiumque pependit! \\ Quid factura fuit trepidanti pollice! dextram \\ Torpor et ira ligat, dum dura vocabula format, \\ Et minus explicitam condemnat littera vocem; \\ Torsit iter male tractus apex dubiaque remissus \\ Mente pudor, dum verba notat, dum nomina mandat \\ Flamma nocens, iram tardans; penitusque cucurrit \\ Sopitus per membra calor diroque medullas \\ lgne vorat. nullus confessam culpet amantem. \\ Conubium nunc crimen erat? male credula votis \\ Cuncta dedi (nec mira fides) sub lege mariti, \\ Cunius et ipsa fui; numquam nec conscia reddeut \\ Vota fidem, si talis erit non digna marito. \\ Hlanc reddis, Troiane, vicem? meus ista meretur \\ Affectus? non ille torus, non conscia lecti \\ Sacramentatenent?totumprocrimineperdo,p \\ Quidquid amore dedi. fatis licet, iprobe, tendas \\ Aemula regna meis: nibil est quod, perfide, iactes; \\ Fraude perit, non sorte, fides. Sed regna petebas \\ 
        \pagebreak 
    \begin{center} \textbf{COD7CIS SALMASIANI.} \end{center}\begin{center} \textbf{1 1 7} \end{center} \marginpar{[110]} Debita, nec mecum poteras coniungere sortem? \\ Si datur ire, placet. nam quod fugis, unde recursus, \\ Vota ‘nocentis’ habes; nihil est, quod dura querellis \\ Verba fidemque voco; quisquis mea vulnera deflet, \\ Invidiam fecisse neget. trahit omuia casus. \\ Dum sortem natura rapit, ‘sua taedia solus \\ Fallere nescit amor’ reparatum Cynthia format \\ Lucis honore iubar curvatis cornibus †arcus, \\ Quod de fratre rubet: cessurus lege sorori \\ Consumit sua iura dies. sic continet orbem \\ Dum recipit natura vicem. ‘sua taedia solus \\ Fallere nescit amor’ mersum pallentibus umbris \\ Circumdat nox atra diem fruiturque tenebris \\ Lege poli, peraguntque micantia sidera cursus . . . \\ Navifragi tacet unda salis nec murmurat auster \\ Nec flexum quatit aura nemus. ‘sua taedia solus \\ FTallere nescit amorLramis male garrula pendens \\ Iam philomela tacet damuo male victa pudoris, \\ Amplexuque fovens querulos sub culmine nidos \\ Pensat amore nefas, miserasque alitura querellas \\ Nocte premit, quod luce dolet. ‘sua taedia solus \\ Fallere nescit amor, nunc iam bene iunctus amantes \\ Ardor alit thalamique fidem sua pignera conplent. \\ C0 Coninnx laeta viro, felix uxore maritus. \\ 
        \pagebreak 
    \begin{center} \textbf{CARMINA} \end{center} \marginpar{[116]} Vota recenset amor secretaque dulcia; somnus \\ Concordat cum nocte torum. ‘sua taedia solus \\ Fallere nescit amor’ fecundo semine rerum \\ Mutat terra vices et alumni temporis auras \\ Laeta vocat; spisso revirescit gramine campus \\ Et vitreas ligat herba comas nec fallit aristas \\ Proventu meliore dies. ‘sua taedia solus \\ Fallere nescit amor’ fessus iuga solvit arator \\ Et noctem per vota capit: reparare labores \\ Novit grata quies, nec cessat reddere vires \\ Infusus per membra sopor rurisque ministram \\ Ruricolis dat semper opem. ‘sua taedia solus \\ Fallere nescit amor.’ reparant sua litora ponti \\ Successu post damna suo, perituraque ludunt \\ Incrementa maris dubii, regit aequora fluctus \\ Lege sua vicibusque suis quod deperit auget. \\ Officiis natura vacat. ‘sua taedia solus \\ Fallere nescit amor’  \lbrack Discussis imbribus atra \\ Cum requievit hiems, \rbrack  gemmatis roscida verni \\ Rident prata rosis et floribus arva tumescunt. \\ Pictus ager sub flore latet, dat fronde coronas \\ ‘suataediasolusLascivisnaturarsis: \\ Fallere nescit amor’ nec grata silentia noctis \\ Nec somni pia dona placent, nec munera lucis \\ 
        \pagebreak 
    \begin{center} \textbf{CODICIS SALMSMANI.} \end{center}\begin{center} \textbf{q} \end{center}Carpit et indutias fugientis non capit anni: \\ Sed sua victus amor tantummodo vulnera pascit \\ Inter mille dolos totidemque pericula fraudis. \\ Vota queror: vellem tacitis peritura querellis \\ Flere domo, vellem  \lbrack iam \rbrack  tabida fundere fletus. \\ Sed negat ipse dolor, quod iam pudor ante negavit; \\ Scribere iussit amor miseram me, cuius honestam p. 1 \\ Fecit culpa fidem. poteram dispergere ponto \\ Membra manusque tuas miseramque tumentibus undis \\ Praecipitare diem, poteram cresceuntis Iuli \\ Rumpere fata manu parvumque resolvere corpus \\ Morte gravi mersumque in viscera figere ferrum \\ Vel dare membra feris; sed nostro pectore pulsum \\ Cessit amore nefas et honesta pericula passus \\ Corda ligavit amor. quis tantum in hospite vellet \\ Hoc andere nefas? quis vota nocentis habere? \\ Nullus amor sub †laude latet. ‘Cui digna rependes, \\ Si mihi dura paras?’ miserandae fata Creusae \\ Lamentis gemituque trahens infanda peregi \\ Vota deis durumque nefas sortemque malorum \\ Te narrante tuli; gemitus mentisque dolorem \\ Et lacrimas prior ipsa dedi. ‘cui digna rependes, \\ Si mihi dura paras?’ dulcis mea colla fovebat \\ Ascanius miserumque puer figebat amorem, \\ Cui modo nostra fides amissam reddere matrem \\ 
        \pagebreak 
     \marginpar{[118]} \begin{center} \textbf{CARMNA} \end{center}Dum cupit, boc verum mentito pignore nomen \\ Format amor gemitusque graves atque oscula figit \\ Confessus pietate dolor. ‘cui digna rependes, \\ Si mihi dura paras?’ nostri modo litoris hospes \\ Nudus et exul eras, dispersa classe per undas \\ Naufragus; ut taceam clades quascumque videbas \\ Inpendisse ibi: licet haec tibi cuncta fuissent, \\ legna tamen Carhago dedit. ‘cui digna rependes, \\ Si mibi dura paras?’ nihil est, quod dura reposcam: \\ Nequiquam donasse velim! quae perdere possem, \\ Numquam damna voco. vel hoc mihi, perfide, redde, \\ Quod sibi debet amor, si nil pia facta merentur. \\ Esse deos natura docet; non esse timendos, \\ erum facta probant. quid enim non credere possum? \\ Tutus fraude manes et nos pietate perimus! \\ Inprobe dure nocens crudelis perfide fallax \\ Officiis ingrate meis! quid verba minantur? \\ Non odit, qui vota dolet, nec digna rependit, \\ Quidquid †lexa gcmit. tibi nempe remissus habetur \\ Lege pudoris amor! qui tauta dedisse recusem: \\ Sceptra domum Tyrios regnum Carthaginis arces \\ Et quidquid regnantis erat? de coniuge, fallax, \\ Non de iure queror, meritum si non habet ardor: \\ Sed quod hospes eras, nec te magis esse nocentem \\ Quam miserum, Troiane, puto, qui digna repellis, \\ Dum non digna cupis: nondum bene siccus ad aequor \\ 
        \pagebreak 
    \begin{center} \textbf{CODICIS SALMASIAN.} \end{center} \marginpar{[119]} Curris et extremas modo naufragus arripis undas. \\ Tutior esse times et honesta pericula poscis. \\ Cum mala vota cupis, solus tibi dura profecto \\ Damna paras. fugis, ecce fugis nostrosque penates \\ Deseris et miseram linquis Carthaginis aulam, \\ Quae tibi regna dedit, sacro diademate crines \\ Cinxit et augustam gemmato sidere frontem \\ Conplevit nostrumque tibi commisit amorem. \\ Nil puto maius habes et adhuc sine coniuge regnas, \\ Aeneas ingrate meus. regat ira dolenti \\ Consilium! sed praestat amor. mea vulnera vellem \\ Fletibus augeri, sed iam discrimine mortis \\ Victa feror. neque enim tantus de funere luctus, \\ Quantus erat de fratre. licet simul inprobus exul \\ Et malus hospes eras et ubique timendus haberis: \\ Vive tamen nostrumque nefas post fata memento! \\ 
      \end{verse}
  
            \subsection*{84}
      \begin{verse}
      B. IM 2s8. \\ M. 1020. \\ D e ro s i \\ B. IV 278. \\ XII  \lbrack l quales ego mane rosas procedere vidi! \\ Nascebantur adhuc neque erat par omnibus aetas. \\ Prima papillatos ducebat  \lbrack tecta \rbrack  corymbos, \\ Altera puniceos apices umbone levabat, \\ Tertia iam totum calathi patefecerat orbem, \\ 
        \pagebreak 
     \marginpar{[120]} \begin{center} \textbf{CARMNA} \end{center}Quarta simul nituit nudato germine floris. \\ Dum levat una caput dumque explicat altera nodum, \\ Ac dum virgineus pudor exsinuatur amictu, \\ Ne pereant, lege mane rosas:  \lbrack cito \rbrack  virgo senescit. \\ 
      \end{verse}
  
            \subsection*{85}
      \begin{verse}
      \poemtitle{EIV SDEM}B. III 289. \\ M. 1021. \\ \poemtitle{De rosa}B. IV 278. \\ Aut hoc risit Amor aut hoc de pectine traxit \\ Purpureis Aurora comis, aut sentibus haesit \\ Cypris et bic spinis insedit sanguis acutis! \\ 
      \end{verse}
  
            \subsection*{86}
      \begin{verse}
      \poemtitle{EIV SDEM}B. III 290. \\ M. 1022. \\ \poemtitle{De rosis}B. IV 278. \\ Hortus erat Veneris, roseis circumdatus herbis, \\ Gratus ager dominae, quem qui vidisset amaret. \\ Dum puer hic passim properat decerpere flores \\ Et velare comas, spina violavit acuta \\ Marmoreos digitos: mox ut dolor adtigit artus \\ Sanguineamque manum, tinctus sua lumina gutta \\ Pervenit ad matrem frendens defertque querellas: \\ ‘Vnde rosae, mater, coeperunt esse nocentes? \\ 
        \pagebreak 
    \begin{center} \textbf{CODICIS SALMAMAN1.} \end{center} \marginpar{[0]} Vnde tui flores pugnant latentibus armis? \\ Bella gerunt mecum. floris color et cruor unum est’ \\ 
      \end{verse}
  
            \subsection*{87}
      \begin{verse}
      B. III 291. \\ M. 221. \\ \poemtitle{L. 0R I}p. 74 \\ B. IV 279. \\ Venerunt aliquando rosae. pro veris amoeni \\ Ingenium! una dies ostendit spicula florum, \\ Altera pramidas nodo maiore tumentes, \\ Tertia iam calathos; totum lux quarta peregit \\ Floris opus. pereunt hodie, nisi mane legantur. \\ 
      \end{verse}
  
            \subsection*{88}
      \begin{verse}
      B. I 73. \\ M. 616. \\ \poemtitle{De Musi}B. IV 27. \\ Clio saecla retro memorat sermone soluto. \\ Euterpae geminis loquitur cava tibia ventis. \\ Voce Thalia cluens soccis dea comica gaudet. \\ Melpomene reboans tragicis fervescit iambis. \\ Aurea Terpsichorae totam lyra personat aethram. \\ FTila premens digitis Erato modulamina fingit. \\ lectitur in faciles variosque Polymnia motus. \\ Vranie numeris scrutatur sidera mundi. \\ Calliope doctis dat laurea serta poetis. \\ 
        \pagebreak 
    \begin{center} \textbf{CARMIN} \end{center} \marginpar{[100]} B. I 131. \\ M. 660. \\ 
      \end{verse}
  
            \subsection*{}
      \begin{verse}
      \poemtitle{EIVSDEM}B. IV 230. \\ Stat duplex uullo conpletus corpore Chiron. \\ B. Lux. iit. \\ M. 298. \\ Praefatio \\ B. IV 2e1. \\  \lbrack XIVq Parvula quod lusit, sensit quood iunior aetas, \\ Quod sale Pierio garrula lingua sonat, \\ Hoc opus inclusit. tu, lector, corde perito \\ Omnia perpendens delige quod placeat. \\ 
      \end{verse}
  
            \subsection*{}
      \begin{verse}
      \poemtitle{De velo ecelesiae}B. M. \\ 91a \\ B. IV 281. \\ Omnia quae poscis domiuum, si credis, habebis. \\ Quae bona vota pelunt, recipit alma fides. \\ B. M. \\ e. De ohristiano infante mortuo \\ Nobilis adque insons occasu inpubes acerbo \\ Decessit, lacrimas omnibus incutiens. \\ Sed quia regna patent semper caelestia iustis \\ 2. 75 \\ Atque animus caelos inmaculatus adit, \\ 
        \pagebreak 
    \begin{center} \textbf{CODICIS SALMASIANI.} \end{center} \marginpar{[123]} Damnantes fletus casum laudemus ephebi, \\ Qui sine peccato raptus ad astra viget. \\ Felix morte sua est, celeri quem funere constat \\ Non liquisse patrem, sed placuisse deo. \\ 
      \end{verse}
  
            \subsection*{93}
      \begin{verse}
      B. M. \\ \poemtitle{De iudlcio alomonis}B. IV 281. \\ Inventa est ferro pietas prolemque necando \\ Conservat mater, contemto pignore victrix. \\ 
      \end{verse}
  
            \subsection*{94}
      \begin{verse}
      B. V 1883. \\ M. 1118. \\ \poemtitle{De cereo}B. IV 282. \\ Lenta paludigenam vestivit cera papyrum, \\ Lumini ut accenso dent alimenta simul. \\ 
      \end{verse}
  
            \subsection*{95}
      \begin{verse}
      B. V 1e9. \\ M. 1112. \\ Aliter \\ B. IV 282. \\ Vt devota piis clarescant limina templis, \\ Niliacam texit cerea lamna budam. \\ Congrua votiferae submittit pabula flammae, \\ Quae castis apibus praebuit ante domus. \\ 
      \end{verse}
  
            \subsection*{96}
      \begin{verse}
      B. II268. \\ M. 854. \\ \poemtitle{De maistro ludi nelegenti .}Indoctus teneram suscepit cauculo pubem, \\ Quam cogat primas discere litterulas. \\ 
        \pagebreak 
    \begin{center} \textbf{CARMNA} \end{center}\begin{center} \textbf{10 4} \end{center}Sed cum discipulos nullo terrore coercet \\ Et ferulis culpas tollere cessat iners, \\ Proiectis pueri tabulis Floralia ludunt. \\ Iam nomen ludi rite magister habet. \\ 
      \end{verse}
  
            \subsection*{97}
      \begin{verse}
      B. I 123. \\ M. 53. \\ \poemtitle{De lellerophonte}B. IV 282. \\ Bellerophon superans incendia dira Chimaerae \\ Victor Gorgoneo nubila tangit equo. \\ 
      \end{verse}
  
            \subsection*{98}
      \begin{verse}
      B. V 165. \\ M. 1095. \\ \poemtitle{De Chimaera}B. IV 2s3. \\ Ore leo tergoque caper postremaque serpens \\ Bellua tergemino mittit ab ore faces. \\ p. 76 \\ 
      \end{verse}
  
            \subsection*{99}
      \begin{verse}
      B. I 110. \\ M. 644. \\ \poemtitle{De Lauconte}B. IV 23. \\ Laucontem gemini distendunt nexibus angues \\ Cumque suis genitis sors habet una patrem. \\ Quod sacra iligni violarit terga caballi, \\ linc lacerasse ferunt saeva venena virum. \\ Quid sperare datur superum iam numine laeso, \\ Cum sic irasci ligneus audet ecus? \\ 
        \pagebreak 
    \begin{center} \textbf{CODCIS SALMASIANI.} \end{center} \marginpar{[0]} 
      \end{verse}
  
            \subsection*{100}
      \begin{verse}
      De templo Veneris. quod ad muros ,. \\ M. 615. \\  \lbrack extruendos dirutum est \rbrack  \\ B. IV 283. \\ Caeduntur rastris veteris miracula templi \\ Inque usum belli tecta sacrata ruunt. \\ Nam qua delectis volvuntur saxa catervis, \\ Hac sunt murorum mox relocanda minis. \\ Pilati Mavors conpendia cepit Amoris: \\ Per muros quaerit iam sua templa Venus! \\ 
      \end{verse}
  
            \subsection*{101}
      \begin{verse}
      B. III 183. \\ M. 958. \\ \poemtitle{De baterna}B. IV 283. \\ Aurea matronas claudit basterna pudicas, \\ Quae radians patulum gestat utrumque latus. \\ Hlanc geminus portat duplici sub robore burdo, \\ Provehit et modico pendula saepta gradu. \\ Provisum est cante, ne per loca publica pergens \\ Fucetur visis casta marita viris. \\ 
      \end{verse}
  
            \subsection*{102}
      \begin{verse}
      B. I 128. \\ M. 657. \\ \poemtitle{De Medea cum fliis suis}B. IV 284. \\ Opprimit insontes infidi causa parentis, \\ Iasonis et nati crimina morte luunt. \\ 
        \pagebreak 
     \marginpar{[102]} \begin{center} \textbf{CARMINA} \end{center}Sed quamvis mater vivo viduata marito \\ Coniugis in poenam pignora cara metat, \\ Sacra tamen pietas insanae mitigat ausus: \\ Hunc furiata premit, hunc miserata levat. \\ 
      \end{verse}
  
            \subsection*{103}
      \begin{verse}
      B. III 15. \\ M. 960. \\ I i s De homine ui per se molebat \\ \poemtitle{. 77}Cum possis parvo sumptu conducere asellum, \\ Qui soleat teretes volvere rite molas, \\ Cur nummi cupidus sic te contemnis, amice, \\ Vt cupias duro subdere colla iugo? \\ Linque precor gyros! poteris pistore ministro \\ Candentis quadrae munus habere sedens. \\ Per te namque terens Cererem patiere labores, \\ Quos quaerens natam pertulit ipsa Ceres. \\ 
      \end{verse}
  
            \subsection*{104}
      \begin{verse}
      B. V 168. \\ M. 1098. \\ \poemtitle{De formica}B. IV 285. \\ Verrit tetra boum gratos formica labores \\ Et caveis fruges turba nigella locat, \\ Quae, licet exiguo videatur pectore, sollers \\ Colligit hibernae commoda grana fami. \\ Ianc iuste famulam nigri iam dixeris Orci, \\ Quam color et factum conposuit domino. \\ 
        \pagebreak 
    \begin{center} \textbf{CODICIS SALMASIAN.} \end{center} \marginpar{[127]} Namque ut Plutonis rapta est Proserpina curru, \\ Sic formicarum verritur ore Ceres. \\ 
      \end{verse}
  
            \subsection*{105}
      \begin{verse}
      B. I 161. \\ M. 683. \\ D Iecuba \\ B. IV 285. \\ Prole viro regnoque carens Priameia coniunx \\ Dura sorte venit sub iuga nunc lthaci. \\ Quae cupiens tantos lacrimis aequare dolores \\ Perpetuo planctu transit in ora canis. \\ Quid valeat, variis monstrat Fortuna figuris: \\ Post regnum in vico saucia latrat anus. \\ 
      \end{verse}
  
            \subsection*{106}
      \begin{verse}
      B. V 152. \\ M. 106. \\ \poemtitle{De ansere}B. IV 285. \\ Aedibus in nuostris volitans argenteus anser \\ Dulcisono strepitu colla canora levat. \\ Ales grata bono duplici; nam fercula mensae \\ Conplet et adservat nocte silente domum. \\ \poemtitle{. 78}Solus Tarpeia canibus in rupe quietis \\ Eripuit CGallis Romula tecta vigil. \\ 
      \end{verse}
  
            \subsection*{107}
      \begin{verse}
      B. V 173. \\ M 1103. \\ Do sepia \\ B. IV 286. \\ Femineo geminum designat nomine sexum \\ Et candens piceum sepia claudit onus. \\ Vuilior nullus piscis per caerula oberrat, \\ Cui pretium capto debuit esse duplum. \\ 
        \pagebreak 
    \begin{center} \textbf{CARMNA} \end{center} \marginpar{[128]} Praestat carne cibos, apicum dat felle figuras, \\ Atque brevi specie usum ad utrumque facit. \\ Hanc potius doctos adsumere convenit escam, \\ Quae sapit in morsu et probat articulos. \\ 
      \end{verse}
  
            \subsection*{108}
      \begin{verse}
      B. II 175. \\ M. 2. \\ \poemtitle{De eunueho}B. IV 286. \\ Quem natura marem dederat, fit femina ferro; \\ Nam teneri pubes viribus exuitur. \\ Hinc iuvenem cernis tanto sub robore mollem, \\ Et dubii pulcher corporis errat homo. \\ Coniugibus cautis placita est monstrosa voluptas; \\ Fidus enim est custos, qui sine teste datur. \\ 
      \end{verse}
  
            \subsection*{109}
      \begin{verse}
      B. III 176. \\ M. 953. \\ Aliter \\ B. IV 286. \\ Incertum ex certo sexum fert pube recisa, \\ Quem tenerum secuit mercis avara manus. \\ Namque ita femineo eunuchus crure movetur, \\ Vt dubites quid si, vir  \lbrack magis \rbrack  an mulier. \\ Omnem grammaticam castrator sustulit artem, \\ Qui docuit neutri esse hominem generis. \\ 
      \end{verse}
  
            \subsection*{110}
      \begin{verse}
      B. III 31. \\ M. 894. \\ \poemtitle{De balneis}B. IV 287. \\ Hic ubi Baiarum surrexit grata voluptas \\ Et rudibus splendens molibus extat opus, \\ 
        \pagebreak 
    \begin{center} \textbf{CODICIS SALMASIANI.} \end{center} \marginpar{[129]} Rura prius, nullum domino praestantia quaestum, \\ Nullaque tecta tulit glebula frugis inops. \\ Haec nunc Bellator multo sublimis honore \\ Vestivit cameris balnea pulcra locans. \\ Prospera facta viri naturae munera mutant, \\ Cum salsum salubri litus abundat aqua. \\ Alpheum fama est dulcem per Tethyos arva \\ Currere nec laedi gurgitibus pelagi. \\ Dant simile exemplum nostri miracula fontis: \\ Vicinum patitur nec sapit unda salum. \\ 1 1 \\ B. III 178. \\ M. 954. \\ 
      \end{verse}
  
            \subsection*{}
      \begin{verse}
      \poemtitle{De pantomimo}B. IV 287. \\ Mascula femineo derivans pectora flexu \\ Atque aptans lentum sexum ad utrumque latus \\ Ingressus scenam populum saltator adorat; \\ Sollerti spondet prodere verba manu. \\ Nam cum grata chorus difundit cantica dulcis, \\ Quae resonat cantor, motibus ipse probat. \\ Pugnat ludit amat bacchatur vertitur adstat, \\ Inlustrat verum, cuncta decore replet. \\ Tot linguae quot membra viro. mirabilis ars est, \\ Quae facit articulos ore silente loqui. \\ 
        \pagebreak 
    \begin{center} \textbf{CARMNA} \end{center} \marginpar{[130]} 1 10 \\ B. lII 179. \\ M. 956. \\ 
      \end{verse}
  
            \subsection*{}
      \begin{verse}
      \poemtitle{De Tunambulo}B. IV 288. \\ Stuppea suppositis tenduntur vincula lignis, \\ Quae fido ascendit docta iuventa gradu. \\ Quae super aerius protendit crura viator \\ Vixque avibus facili tramite currit homo. \\ Brachia distendens gressum per inane gubernat, \\ Ne lapsu facili planta rudente cadat. \\ Daedalus adstruitur terras mutasse volatu \\ Et medium pinnis persecuisse diem. \\ p. 80 \\ Praesenti exemplo firmatur fabula mendax: \\ Ecce hominis cursus funis et aura ferunt. \\ B 1 1 \\ B. III 151. \\ M. 255. \\ 
      \end{verse}
  
            \subsection*{}
      \begin{verse}
      \poemtitle{De citharoedo}B. IV 288. \\ Musica contingens subtili stamina pulsu \\ Ingreditur, vulgi auribus ut placeat. \\ Stat tactu cantuque potens, cui brachia linguae \\ Concordant sensu conciliata pari. \\ Nam sic aequali ambo moderamine librat \\ Atque oris socias temperat arte manus, \\ 
        \pagebreak 
    \begin{center} \textbf{CODICIS SALMASANI.} \end{center}Vt dubium tibi sit gemina dulcedine capte, \\ Vox utrumne canat an lyra sola sonet. \\ 
      \end{verse}
  
            \subsection*{114}
      \begin{verse}
      B. III 12. \\ M. 957. \\ Aliter \\ B. IV 289. \\ Doctus Apollineo disponere carmina plectro \\ Gaudet multifidam pectore ferre chelyn, \\ Quam mox linguato decurrens pollice cogit, \\ Humanum ut possit chorda canora loqui. \\ Amphion cithara Thebarum moenia saepsit, \\ Aurita ad muros currere saxa docens. \\ Nec minus hac valuit reparator coniugis Orpheus, \\ lmpia cum flexit Tartara dulcis amor. \\ Ars laudanda nimis, cuius moderamine sacro \\ Vnum ex diversis vox digitique canunt! \\ 
      \end{verse}
  
            \subsection*{115}
      \begin{verse}
      B. III 184. \\ M. 959. \\ \poemtitle{De pyrrioha}B. IV 289. \\ In spatio Veneris simulantur proelia Martis, \\ Cum sese adversum sexus uterque venit. \\ Femineam maribus nam confert pyrricha classem \\ Et velut in morem militis arma movet. \\ Quae tamen haut ullo chalybis sunt tecta rigore, \\ Sed solum reddunt buxea tela sonum. \\ Sic alterna petunt iaculis clipeisque teguntur, \\ Nec sibi congressi vir nocet aut mulier. \\ 
        \pagebreak 
     \marginpar{[10]} \begin{center} \textbf{CARMINA} \end{center}Lusus habet pugnam, sed dant certamina pacem; \\ Nam remeare iubent organa blanda pares. \\ 
      \end{verse}
  
            \subsection*{116}
      \begin{verse}
      IB. V 66. \\ M. 1035. \\ aus temporum quattuor \\ B. IV 290. \\ Carpit blanda suis ver almum dona rosetis. \\ Torrida collectis exultat frugibus aestas. \\ Indicat autumnum redimito palmite vertex. \\ Frigore pallet hiems designans alite tempus. \\ V 84. \\ 4 \\ M 1050. \\ aus omnium mensuum \\ B. IV 290. \\ Fulget honorilico indutus mensis amictu, \\ Signans lomuleis tempora consulibus. \\ Rustica Bacchigenis intentans arma novellis \\ Ilic meruit Februi nomen habere dei. \\ Martius in campis ludens simulacra duelli \\ Ducit Cinypbii lactea dona gregis. \\ Sacra Dioneae refereins sollemnia matris \\ Lascivis crotalis plaudit Aprilis ovans. \\ Maius Atlantis natae dicatus honori  \lbrack est \rbrack , \\ Expolit et pulcris florea serta rosis. \\ 
        \pagebreak 
    \begin{center} \textbf{0q} \end{center}\begin{center} \textbf{CODICIS SALMASIAN.} \end{center}Ornat sanguineis aestiva prandia moris \\ Iunins: huic nomen fausta iuventa dedit. \\ Quintilis mensis Cereali germine gaudet; \\ Iulius a magno Caesare nomen habet. \\ Augustum penitus torret Phaethontius ardor, \\ Cum recreant fessum gillo flabella melo. \\ 
      \end{verse}
  
            \subsection*{}
      \begin{verse}
      \poemtitle{. 82}Aequalis Librae September digerit horas, \\ Cum botruis captum rure ferens leporem. \\ Conterit October lascivis calcibus uvas \\ Et spumant pleno dulcia musta lacu. \\ Arva November arans fecundo vomere vertit, \\ Cum teretes sentit pinguis oliva molas. \\ Pigra suum cunctis commendat bruma Decembrem, \\ Cum sollers famulis tessera iungit eros. \\ 
      \end{verse}
  
            \subsection*{118}
      \begin{verse}
      B. I 90. \\ M. 1 4. \\ \poemtitle{De Thetide}B. IV 291. \\ Cauta quidem genetrix, noceant ne vulnuera nato, \\ Confirmat Stygio fonte puerperium. \\ Sed quia fas nulli est humanam vincere sortem, \\ In membris tincti dant sibi fata locum. \\ 
        \pagebreak 
    \begin{center} \textbf{CARMINA} \end{center} \marginpar{[134]} 
      \end{verse}
  
            \subsection*{119}
      \begin{verse}
      B. III 42. \\ M. 897. \\ \poemtitle{De balneis}B. IV 58. \\ Aspice fulgentis tectis et gurgite Baias, \\ Dant quibus haut parvum pictor et unda decus. \\ Namque gerunt pulcras splendentia culmina formas, \\ Blandaque perspicuo fonte fluenta cadunt. \\ Gaudia qui gemino gestit decerpere fructu \\ Et vita novit praetereunte frui, \\ Hic lavet: hic corpus reparans mentemque relaxans \\ Lumina picturis, membra fovebit aquis. \\ 
      \end{verse}
  
            \subsection*{120}
      \begin{verse}
      B. III 43. \\ M. 9. \\ Aliter \\ B. IV 298. \\ austa novum domini condens Fortuna lavacrum \\ fnvitat fessos huc properare viae. \\ Eaude operis fundi capiet sua gaudia praesul, \\ Ospes dulciflua dum recreatur aqua. \\ Condentis monstrant versus primordia nomen \\ Auctoremque facit littera prima legi. \\ 
      \end{verse}
  
            \subsection*{}
      \begin{verse}
      \poemtitle{. 83}Eustrent pontivagi Cumani litoris antra: \\ Indigenae placeant plus mihi deliciae. \\ 
      \end{verse}
  
            \subsection*{121}
      \begin{verse}
      B. III 44. \\ M. 900. \\ Aliter \\ B. IV 299. \\ Quisquis Cumani lustravit litoris antra \\ Atque hospes calidis saepe natavit aquis, \\ 
        \pagebreak 
    \begin{center} \textbf{CODCIS SALMASANI.} \end{center} \marginpar{[135]} Hic lavet, insani vitans discrimina ponti; \\ Baiarum superant balnea nostra decus. \\ 100 \\ B. III 45. \\ M. 901. \\ Aliter \\ B. IV 299. \\ Flammea perspicuis coeunt bic lumina lymphis \\ Dantque novum mixti Phoebnus et unda diem. \\ Denique succedit nostris lux tanta lavacris, \\ Vt cernas nudos erubuisse sibi. \\ 
      \end{verse}
  
            \subsection*{123}
      \begin{verse}
      B. III 46. \\ M. 902. \\ Aliter \\ B. IV 299. \\ Infundit nostris Titan sua lumina Bais \\ Inclusumque tenet splendida cella iubar. \\ Subiectis caleant aliorum balnea flammis: \\ Haec reddi poterunt sole vapora suo. \\ 
      \end{verse}
  
            \subsection*{124}
      \begin{verse}
      B. III 39. \\ M. 895. \\ \poemtitle{De thermis}B. IV 299. \\ Delectat variis infundere corpora lymphis \\ Et mutare maris saepe fluenta libet. \\ Nam ne consuetae pariant fastidia thermae, \\ Hinc iuvat alterno tingere membra lacu. \\ 
        \pagebreak 
     \marginpar{[136]} \begin{center} \textbf{CARMNA} \end{center}
      \end{verse}
  
            \subsection*{125}
      \begin{verse}
      Aliter \\ Deest \\ 
      \end{verse}
  
            \subsection*{126}
      \begin{verse}
      B. III 40. \\ M. 896. \\ .  \lbrack De liotheca in ticlinium mutte \rbrack  \\ Tecta novem Phoebi nuper dicata Camenis \\ Nunc retinet Bacchus et sua tecta vocat. \\ Namque ubi tot veterum manserunt scripta virorum, \\ Hic potat laete dulcia vina Cypris. \\ \poemtitle{. 85}Cognato semper lustrantur numine sedes: p.uacu ess’ \\ Quas coluit Phoebus, has colit et Bromius. \\ B. III 168. \\ M. 947. \\ 
      \end{verse}
  
            \subsection*{}
      \begin{verse}
      \poemtitle{De lenone uxoris suae}B. IV 300. \\ Graecule, consueta lenandi callidus arte, \\ Coepisti adductor coniugis esse tuae, \\ Et, quem forte procax penitus conroserat uxor, \\ Consueras propria praecipitare domo. \\ Sed praetensa catus derisit retia quidam, \\ Conversa statuens sorte manere domo: \\ Nam semel admissus  \lbrack tenuit mox omnia felix \rbrack  \\ Teque tuis miserum depulit e laribus. \\ 
        \pagebreak 
    \begin{center} \textbf{CODICIS SALMASANI.} \end{center}Solus vera probas iucundi verba poetae: \\ ‘Dum iugulas hircum, factus es ipse caper.’ \\ 
      \end{verse}
  
            \subsection*{128}
      \begin{verse}
      B. III 169. \\ M. 948. \\ Ad lenonem eomitiacum \\ B. IV 300. \\ Militiae cultus et nigri tegmina panni \\ Cur magis eoptes, dissere, leno, mihi. \\ Exiguosne tibi praebebat cellula quaestus? \\ Non gravis adducta virgine saccus erat? \\ An nescis populi pastum sibi tollere paucos, \\ Vnde miser fisco paupere miles eget? \\ Effuge vitandos, si qua potes arte, labores, \\ Vt valeas tenso vivere, leno, pede. \\ Nam si formonsas redeas lenare puellas \\ Et dederit quaestus cotidiana Venus, \\ Non iam miles eris humilis, sed divite nummo \\ Fies militiae mox utriusque comes. \\ 
      \end{verse}
  
            \subsection*{129}
      \begin{verse}
      B. III 170. \\ M. 949. \\ \poemtitle{De Martio cinaedo}p. 86 \\ B. IV 301. \\ Quid prodest Martis nomen de nomine ductum, \\ Pruriat infami cum tibi clune Venus? \\ Sors fuerat melior, Cypridos si nomen haberes \\ Et natura daret Martia membra tibi. \\ 
        \pagebreak 
     \marginpar{[138]} \begin{center} \textbf{CARMINA} \end{center}Nunc utroque carens, ignoti fabula sexus, \\ Femina cum non sis, vir tamen esse nequis. \\ r \\ 
      \end{verse}
  
            \subsection*{130}
      \begin{verse}
      B. III 171. \\ M. 950. \\ \poemtitle{De Caballina meretriee}B. IV 301. \\ Caballina furens amanda nulli \\ Excussis modo calcibus fremebat; \\ Quae quamvis facie micet rubenti \\ Et vibret Parium nitens colorem, \\ Hirsutis tamen est petenda mulis, \\ Qui possint pariles citare iunctas. \\ 0 \\ B. III 167. \\ M. 946. \\ 
      \end{verse}
  
            \subsection*{}
      \begin{verse}
      \poemtitle{De Arucitano vate}B. IV 302. \\ Praecisae silicis cautibus edite, \\ Silvestri iuvenis durior arbuto, \\ Trunco cum stupeas horridior, cupis \\ Formare  \lbrack e propriis carmina versibus \\ Et metri variis ludere legibus? \\ Sed quis te docilem iudicet artium, \\ Quas natura dedit cordis acutior? \\ Solus ligna dolans fortibus asceis \\ Et duri resecans robora pectoris \\ Vatem te poterat reddere ligneum, \\ 
        \pagebreak 
    \begin{center} \textbf{CODICIS SALMASIANI.} \end{center} \marginpar{[139]} Qui vaccam trabibus lusit adulteris \\ V4 \\ qui struxit ecum fraudis Achaicae. \\ 00 \\ B. V 150. \\ M. 260. \\ 
      \end{verse}
  
            \subsection*{}
      \begin{verse}
      \poemtitle{De capone phasianacio}B. IV 302. \\ Candida Phoebeo praefulgunt ora rubore, \\ p. 7 \\ Crista riget radiis, ignea barba micat. \\ Alae colla comae pectus femur inguina cauda \\ Paestanis lucent floridiora rosis. \\ Flammea sic rutilum distinguit pinna colorem, \\ Vt vibrare putes plumea membra faces. \\ 
      \end{verse}
  
            \subsection*{133}
      \begin{verse}
      B. III 208. \\ M. 980. \\ Do malis Matiani \\ B. IV 303. \\ Haec poterant celeres pretio tardare puellas, \\ Haec fuerant Veneri iudice danda Phryge. \\ Nam sic ingenuo flvescunt mala colore, \\ Vt superent auro vera metalla suo. \\ 
      \end{verse}
  
            \subsection*{134}
      \begin{verse}
      B. III 209. \\ M. 981. \\ Aliter. aus. \\ B. IV 303. \\ IHis constat Veneri praclatae gratia formae, \\ Haec moriente sacrum perdidit angue nemus. \\ 
        \pagebreak 
    \begin{center} \textbf{CARMNA} \end{center} \marginpar{[140]} 
      \end{verse}
  
            \subsection*{135}
      \begin{verse}
      B. 1I1 210. \\ M. 4982. \\ Aliter. Vituperatio \\ B. I1V 03. \\ His contempta deum tenuit Discordia mensam, \\ Prodidit atque urbem his Briseida suam. \\ 
      \end{verse}
  
            \subsection*{136}
      \begin{verse}
      B. V 130. \\ M. 107. \\ \poemtitle{De glone}B. 1V 309 \\ Cillo vomit gelidum vastis singultibus amnem, \\ Vis aliena cui frigoris addit opem. \\ Nam tepidum laticem curamus claudere testa, \\ Vt mersa imbrigenis unda nivescat aquis. \\ IB. V 198. \\ M. 1127. \\ 
      \end{verse}
  
            \subsection*{}
      \begin{verse}
      \poemtitle{De theo}B. IV 304. \\ lnguine suspensam gestas  \lbrack sub ventre \rbrack  lagoenam, \\ Quae tibi fit turgens amphora flante Noto. \\ Vectigal poteras figulorum reddere fisco, \\ Quorum tam tereti ramice vincis opus. \\ p. 88 \\ 
      \end{verse}
  
            \subsection*{138}
      \begin{verse}
      B. V 199. \\ M. 1128. \\ Aliter \\ B. IV 304. \\ Moles tanta tibi pendet sub ventre syringis, \\ Vt te non dubitem dicere bicipitem. \\ 
        \pagebreak 
     \marginpar{[141]} \begin{center} \textbf{CODICIS SALMASIANI.} \end{center}Nam te si addictum mittat sententia campo, \\ Vispillo ignoret, quod secet ense caput. \\ 
      \end{verse}
  
            \subsection*{1393}
      \begin{verse}
      B. I 7. \\ M. 562. \\ \poemtitle{De Iove in pluteo}B. IV 304. \\ Flexilis obliquo sinuatur circulus orbe \\ Inclusumque gerit machina sacra Iovem. \\ Vana sub aspectum duxit mendacia fictor: \\ Orbis rectorem quis probat orbe tegi? \\ 
      \end{verse}
  
            \subsection*{140}
      \begin{verse}
      Aliter \\ Deest \\ 
      \end{verse}
  
            \subsection*{141}
      \begin{verse}
      B. I 9. \\ M. 565. \\ \poemtitle{ \lbrack De Iove et Loda \rbrack }B. IV 304. \\ Cygneas Genitor gestans post fulmina pinnas \\ Dulcia difundit carmina virginibus, \\ Quem retinens Leda, prenso cum gaudet olore, \\ Amissa agnovit virginitate Iovem. \\ 
      \end{verse}
  
            \subsection*{142}
      \begin{verse}
      \poemtitle{B. I 11.}M. 567. \\ \poemtitle{De ovo Ledae}B. IV 305. \\ Ledaei partus ovo monstrantur aperto, \\ In cygnum verso a Iove quod genuit. \\ 
        \pagebreak 
     \marginpar{[142]} \begin{center} \textbf{CARMINA} \end{center}Vna tribus genetrix, sed sors diversa creatis: \\ Sidera pars faciet, pars fera bella Phrygum. \\ 
      \end{verse}
  
            \subsection*{143}
      \begin{verse}
      B. I 12. \\ M. 568. \\ \poemtitle{De Europa}B. IV 35. \\ Terga bovis credens Europa ascendit alumni \\ Inseditque Iovi non revisura patrem. \\ Fraude suos Genitor celat vel conplet amores: \\ Nam deus in tauri corpore praedo latet. \\ 
      \end{verse}
  
            \subsection*{144}
      \begin{verse}
      B. I 13. \\ M. 569. \\ Allter \\ p. 59 \\ B. IV 305. \\ Mentitus taurum Europam luppiter aufert, \\ Virgineos ardens pandere fraude sinus. \\ Iumano tandem veniam donemus amori, \\ Si tibi, summe deum, dulcia furta placent. \\ 
      \end{verse}
  
            \subsection*{145}
      \begin{verse}
      B. I 140. \\ M. 243. \\ \poemtitle{De Narcisso}B. IV 305. \\ Invenit proprios mediis in fontibus ignes \\ Et sua deceptum urit imago virum. \\ 
      \end{verse}
  
            \subsection*{146}
      \begin{verse}
      B. I 144. \\ M. 667. \\ Aliter \\ B. IV 306. \\ Ardet amore sui lagrans Narcissus in undis \\ Cum modo perspicua se speculatur aqua. \\ 
        \pagebreak 
    \begin{center} \textbf{CODICIS SALMASEANI.} \end{center} \marginpar{[143]} 4 \\ B. I 145. \\ M. 668. \\ Aliter \\ B. IV 306. \\ Suspirat propriae Narcissus gaudia formae, \\ Quem scrutata suis vultibus unda domat. \\ 
      \end{verse}
  
            \subsection*{148}
      \begin{verse}
      B. 1s. \\ M. 961. \\ pro ceoncubitu \\ B. IV 306. \\ Causidicus pauper mcdia sub nocte lucubrans \\ Cornipedis voluit terga fricare suae. \\ Sed cum corpus equae dextra famulante titillat, \\ Invasit iuvenem prodigiosa Venus. \\ Nam qua longa solet quadruvia carpere sessor, \\ Subducens durae peudula crura viae, \\ Hanc fovet amplexu molli cunuumque caballae \\ Adterit adsiduo pene fututor hebes. \\ Concubitus Cressa legitur quaesisse iuvenci, \\ Quam gravis ira deae iussit amare pecus. \\ Par crimen flamae nostris fors intulit anuis: \\ Passiphae tauro, Filager arsit equa. \\ p. 0 \\ 
      \end{verse}
  
            \subsection*{149}
      \begin{verse}
      B. III 187. \\ M. 362. \\ AHiter \\ B. IV 306. \\ Defensor probe tristium reorum, \\ Cuius voce sacrum tonat tribunal \\ 
        \pagebreak 
    \begin{center} \textbf{CARMINA} \end{center} \marginpar{[144]} Et palmas capiunt lares Vitenses: \\ Cur post athla fori togaeque pompam \\ Gaudens monstrifero calere luxu, \\ Vectricis propriae furens in usus \\ Fessae cornipedis fricas hiatum, \\ Verso et munere dignitatis optas \\ Admissarius esse quam patronus? \\ Expellas animo nimis, rogamus, \\ Mores inlicite libidinantes. \\ Horrendum vitium est in advocato, \\ Orando solitum movere caulas \\ Subantis pecudis tenere ambas. \\ 
      \end{verse}
  
            \subsection*{150}
      \begin{verse}
      B. III 80. \\ M. 98. \\ \poemtitle{De tabula piet}B. IV 307. \\ Hunc, quem nigra gerit tabella, vultum, \\ Clarum linea quem brevis notavit, \\ Mox pictor varios domans colores \\ Callenti nimium peritns arte \\ Formavit similem, probante vero \\ Ludentem propriis fidem figuris, \\ Vt, quoscumque manu repingat artus, \\ Credas corporeos habere sensus. \\ 
        \pagebreak 
    \begin{center} \textbf{CODICIS SALMASIANI.} \end{center} \marginpar{[145]} 
      \end{verse}
  
            \subsection*{151}
      \begin{verse}
      B. I 85. \\ M. 626. \\ \poemtitle{De alatea}B. IV 307. \\ Defugiens pontum silvas Galatea peragrat, \\ Custodem ut pecorum cernere possit Acim. \\ Nam teneros gressus infigit sentibus ardens \\ p. 91 \\ Nec tamen alta pedum vulnera sentit amor. \\ Ipsa Cupidineae laedunt tormenta pharetrae, \\ Cuius et in mediis flamma suburit aquis. \\ 
      \end{verse}
  
            \subsection*{152}
      \begin{verse}
      B. I 86. \\ M. 627. \\ Aliter. De Galatea ln vase \\ B. IV 308. \\ Fulget et in patinis ludens pulcherrima Nais, \\ Prandentum inflammans ora decore suo. \\ Congrua non tardus difundat iura minister, \\ Et lateat positis tecta libido cibis. \\ 
      \end{verse}
  
            \subsection*{153}
      \begin{verse}
      B. I 87. \\ M. 630. \\ Aliter \\ B. IV 308. \\ Ludere sueta vadis privato nympha natatu \\ Exornat mensas membra venusta movens. \\ Comtas nolo dapes; vacuum mihi pone boletar: \\ Quod placet aspiciam, renuo quod saturat! \\ 
      \end{verse}
  
            \subsection*{154}
      \begin{verse}
      B. I 88. \\ M. 631. \\ Aliter \\ B. IV 308. \\ Inmediogeneratasalonuncartemagistra \\ hicquoquenudanato.Perveniadmensam. \\ 
        \pagebreak 
     \marginpar{[146]} \begin{center} \textbf{CARMINA} \end{center}Si prandere cupis, differ spectare figuram, \\ Ne tibi ieiuno lumina tentet amor. \\ Quae sim ne dubites: ludens sine nomine nympha \\ Quod Galatea vocer, lactea massa probat. \\ 
      \end{verse}
  
            \subsection*{155}
      \begin{verse}
      B. II 21. \\ M. 712. \\ \poemtitle{De scaevola}B. IV 309. \\ Lictorem pro rege necans nunc Mucius ultro \\ Scrifico propriam concremat igne manum. \\ Miratur Porsenna virum poenamque relaxans \\ Maxima cum obsessis foedera victor init. \\ Plus flammis patriae confert, quam iuverat armis \\ Vna domans bellum funere dextra suo. \\ 
      \end{verse}
  
            \subsection*{156}
      \begin{verse}
      B. III 172 \\ M. 951. \\ o. De viro quem mulier caedebat \\ \poemtitle{. 92}Cum te Barbati referas de sanguine cretum, \\ Vt tibi cognatus sit Varitinna ferox, \\ Cur tua femineo caeduntur tergora socco \\ Infamique manu barbula vulsa cadit? \\ Desine iam tibimet auctores fingere fortes \\ Vimque tuis membris stirpis inesse ferae. \\ Illa Salautensi magis est de stirpe creata, \\ Audet quae proprium sternere calce virum. \\ 
        \pagebreak 
    \begin{center} \textbf{CODICIS SALMASIANI.} \end{center} \marginpar{[147]} 
      \end{verse}
  
            \subsection*{157}
      \begin{verse}
      B. III 277. \\ M. 1012. \\ \poemtitle{De die frigido}B. IV 309. \\ Sint tibi deliciae, sint ditis prandia mensae, \\ Munera post Bacchi sit tibi pulcra Venus, \\ Vincere nec libeat villosa veste rigorem, \\ Sed iungat calidum fervida virgo latus. \\ 
      \end{verse}
  
            \subsection*{158}
      \begin{verse}
      B. II 200q. \\ M. 529 . \\ \poemtitle{De imagine vergilii}B. IV 310. \\ Subduxit morti vivax pictura Maronem \\ Et quem Parca tnlit, reddit imago virum. \\ Lucis damna nibil tanto valuere poetae, \\ Quem praesentat honos carminis et plutei. \\ 
      \end{verse}
  
            \subsection*{159}
      \begin{verse}
      B. III 1s8. \\ M. 963. \\ \poemtitle{De diseipulo mediei}B. IV 31. \\ Discipulum medicus quidam suscepit adultum, \\ Traderet ut inveni dogma salutiferum. \\ Qui primo, ut inssum nosset tolerare magistri, \\ Pnblica selliferum per loca duxit ecum. \\ Artis prolixae breviavit tempora doctor: \\ Incepto puerum reddidit ‘xop. \\ 
        \pagebreak 
     \marginpar{[160]} B. e venatore qui eum aprum excepit \\ ys serpentem calcvit inprudens \\ B. IV 158. \\ Sus, iuvenis, serpens casum venere sub unum: \\ lic fremit, ille gemit, sibilat hic moriens. \\ 
      \end{verse}
  
            \subsection*{161}
      \begin{verse}
      \poemtitle{EIVSDEM}B. I 93. \\ M. 635 \\ In Aehillem \\ B. IV 159. \\ Inprobe distractor, pretium si poscere nosses, \\ Non traheres quod pondus erat . . \\ 
        \pagebreak 
    \begin{center} \textbf{CODICIS SALMASIANI.} \end{center} \marginpar{[149]} 
      \end{verse}
  
            \subsection*{162}
      \begin{verse}
      B. M . \\ \poemtitle{De TProia}B. IV 159. \\ Desine, Troia, tuos animo deflere labores: \\ Romam capta creas; merito tua postuma regnat. \\ 
      \end{verse}
  
            \subsection*{163}
      \begin{verse}
      B. I 121. \\ M. 652. \\ \poemtitle{De iudieio aridis}B. IV 159. \\ Conubii bellique deas superavit amorum, \\ Cum pastor pulcram iudicat esse Cyprin. \\ 
      \end{verse}
  
            \subsection*{164}
      \begin{verse}
       \lbrack Alie \rbrack  \\ B. M. B. b. \\ Verticis et thalami pignus sublime Tonantis \\ Exuperat Paridis laude probata Venus. \\ 
      \end{verse}
  
            \subsection*{165}
      \begin{verse}
       \lbrack Aliter \rbrack  \\ B. M. B. b. \\ Extat causa mali, malum cessisse Dionae: \\ Corruerunt Graia Pergama pulsa manu. \\ 
      \end{verse}
  
            \subsection*{166}
      \begin{verse}
       \lbrack Aliter \rbrack  \\ B. M. B. lb. \\ Dat Veneri malum formae pro munere pastor; \\ Cum lunone dolens victa Minerva redit. \\ 
        \pagebreak 
     \marginpar{[150]} \begin{center} \textbf{CARMNA} \end{center}
      \end{verse}
  
            \subsection*{167}
      \begin{verse}
      B. I 155. \\ M. 675. \\ \poemtitle{De yacntho}B. IV 310. \\ Discrimen vitae, ludit dum forte, lyacinthus \\ Incurrit, disco tempora fissa gerens. \\ Non potuit Phoebus fato subducere amatum, \\ Sed cruor extincti florea rura replet. \\ 
      \end{verse}
  
            \subsection*{168}
      \begin{verse}
      \poemtitle{B. .}M. 676. \\  \lbrack Aiter \rbrack  \\ B. ib. \\ Dispersit remeans ludentis tempora discus \\ Et dira pulcher morte Hyacinthus obit. \\ Gratia magna tamen solatur morte peremtum: \\ Semper Apollineus flore resurgit amor. \\ 
      \end{verse}
  
            \subsection*{169}
      \begin{verse}
      B. V 134. \\ M. 1114. \\ \poemtitle{De citro}B. IV 311. \\ Septa micant spinis felicis munera mali: \\ Permulcet citri aureus ora tumor. \\ p. 94 \\ Hippomenes tali vicit certamina malo; \\ Talia poma nemus protulit Hesperidum. \\ 
      \end{verse}
  
            \subsection*{170}
      \begin{verse}
      B. V 185. \\ \poemtitle{M. 1115}Aliter \\ B. IV 311. \\ Stat similis auro citri mirabilis arbos, \\ Omnibus autumni anteferenda bonis. \\ 
        \pagebreak 
    \begin{center} \textbf{CODCIS SALMASIAN.} \end{center} \marginpar{[151]} Haec ornant mensas, haec praestant poma medellam, \\ Cum quatit incurvos tussis anhela senes. \\ B. V 188. \\ M. 1116. \\ Aliter \\ B. IV 311. \\ Omne genus mali dignum est adsurgere citro, \\ Vis cui multa subest corticis et medii. \\ Vnum quaeque suum referunt pomuscula sucum: \\ Ternus ab hoc semper carpitur ore sapor \\ B. I 159. \\ M. 681. \\ 
      \end{verse}
  
            \subsection*{}
      \begin{verse}
      \poemtitle{De Daphne}B. IV 311. \\ Frondibus et membris servavit dextera sollers, \\ Congruus ut sculptis posset inesse color. \\ Dant mirum iunctae ars et pictura decorem, \\ Ostendit varius cum duo signa lapis. \\ 
      \end{verse}
  
            \subsection*{173}
      \begin{verse}
      B. I 138. \\ M. 665. \\ \poemtitle{ \lbrack De Mara \rbrack }B. IV 312. \\ Aerio victus dependet Marsya ramo \\ Nativusque probat pectora tensa rubor. \\ Docta manus varios lapidem limavit in artus; \\ Arboris atque hominis fulget ab arte fldes. \\ 
        \pagebreak 
    \begin{center} \textbf{CARMNA} \end{center} \marginpar{[152]} 
      \end{verse}
  
            \subsection*{174}
      \begin{verse}
      B. I 125. \\ M. 654. \\ \poemtitle{De Philoetet}B. IV 312. \\ Prodentem ducibus Tirynthia tela Pelasgis \\ Laesa Philoctetam vulnere planta domat. \\ Docta manus vivos duxit de marmore sensus; \\ Sentit adhuc poenam, tristis et in lapide. \\ 
      \end{verse}
  
            \subsection*{175}
      \begin{verse}
      B. III 49. \\ M. 905. \\ \poemtitle{De balnel}B. IV 312. \\ Vna salus homini est calidum captare lavacrum, \\ Ne tepidus reddat morbida membra vapor. \\ 
      \end{verse}
  
            \subsection*{176}
      \begin{verse}
      , yys. De ansere, qut intra se eapit eopiam \\ M. 1087. \\ prandii \\ B. IV 312. \\ Eminet impletus pullorum carnibus anser \\ Et varias mensae turgidus ambit opes. \\ Inguinibus nam portat olus ventrisque soluti \\ Truditur e medio esitiata nitens. \\ Fulcit utrumque latus turdus cum turture pinguis \\ †Multaque peruniferum corpore pandit opus. \\ Quis non credat ecum Graiam celasse phalangem, \\ Si parvus tantas anser habet latebras? \\ Plura saginato conclusit fercula capso \\ Aucupia et pulpas ducere docta manus. \\ 
        \pagebreak 
    \begin{center} \textbf{CODICIS SALMASIANI.} \end{center} \marginpar{[153]} Intus inest quodcumque placet; crescitque voluptas, \\ Cum scisso multas pectore prodit opes. \\ Cedat Cecropii lascivans bucula fabri, \\ Qua consuerat amor claudere Passiphaen; \\ Cedat et ille, dolo sollers quem struxit Epeos, \\ Qui gravidus bellis Pergama solvit ecus. \\ Maiorem in parvo haec monstrat fabrica technam; \\ Vna capit totas anseris arca dapes. \\ B. I 101. \\ M. 639. \\ 
      \end{verse}
  
            \subsection*{}
      \begin{verse}
      \poemtitle{De Pyrrho}B. IV 313. \\ Placat busta patris iugulata virgine Pyrrhus \\ Dilectasque litat Manibus inferias. \\ Sors nova nympbigenae: votum post fata meretur; \\ Quam pepigit thalamis, hanc habet in tumulis. \\ B. III 47. \\ M. 903. \\ 
      \end{verse}
  
            \subsection*{}
      \begin{verse}
      \poemtitle{De balnels ceuiusdam puperis . i}Vita opibus tenuis, sed parvo in cespite sollers \\ Fundavit gemino munere delicias. \\ Nam nova  \lbrack in \rbrack  angusto erexit balnea campo, \\ p. 6 \\ Edulibusque virens fetibus hortus olet. \\ Quae natura negat, confert industria parvis: \\ Vix sunt divitibus, quae bona pauper habet. \\ 
        \pagebreak 
     \marginpar{[154]} \begin{center} \textbf{CARMINA} \end{center}
      \end{verse}
  
            \subsection*{179}
      \begin{verse}
      B. III 48. \\ M. 904. \\ Aiter \\ B. IV 314. \\ Parvula succinctis ornavit iugera Bais \\ Vrbanos callens fundere Vita locos. \\ lic quoque pomiferum coniunxit sedulus hortum, \\ Qui vario auctorem gramine dives alat. \\ Rus gratum domino duplici iam munere constat: \\ Hinc capitur victus, sumitur inde salus. \\ 
      \end{verse}
  
            \subsection*{180}
      \begin{verse}
      B. V 166. \\ M. 1096. \\ \poemtitle{De sphinga}B. IV 31. \\ Ales virgo leo crevit de sanguine Lai, \\ Thebano nascens et peritura malo. \\ Haec fecit thalamos Hdipum conscendere matris, \\ Vt prolem incestam mutua dextra necet. \\ 
      \end{verse}
  
            \subsection*{181}
      \begin{verse}
      \poemtitle{De catto qui comedens picam}B. V 162. \\ M. 1093. \\ mnortuus est \\ B. IV 314. \\ Mordaces morsu solitus consumere mures \\ Invisum et domibus perdere dente genus \\ Cattus in obscuro cepit pro sorice picam \\ Multiloquumque vorax sorbuit ore caput. \\ 
        \pagebreak 
    \begin{center} \textbf{CODICIS SALMASIANI.} \end{center} \marginpar{[155]} Poena tamen praesens praedonem plectit edacem, \\ Nam claudunt rabidam cornea labra gulam. \\ Faucibus obsessis vitalis semita cessit \\ Et satur escali vulnere raptor obit. \\ Non habet exemplum volucris vindicta peremptae: \\ lostem pica suum mortua discruciat. \\ 
      \end{verse}
  
            \subsection*{182}
      \begin{verse}
      B. IIM 163. \\ M. 949. \\ \poemtitle{De Aeptio}B. IV 315. \\ Ex oriente die noctis processit alumnus. \\ Sub radiis Phoebi solus habet tenebras. \\ p. 97 \\ Corvus carbo cinis concordant cuncta colori. \\ Quod legeris nomen, convenit: Aethiopis. \\ 
      \end{verse}
  
            \subsection*{183}
      \begin{verse}
      B. III 163. \\ M. 942. \\ Aliter \\ B. IV 315. \\ Faex Garamantarum nostrum processit ad axen \\ Et piceo gaudet corpore verna niger, \\ 
        \pagebreak 
    \poemtitle{CARMINA}Quem nisi vox hominem labris emissa sonaret, \\ Terreret visu horrida larva viros. \\ Dira, Hadrumeta, tuum rapiant sibi Tartara monstrum; \\ Custodem hunc Ditis debet habere domus. \\ 
      \end{verse}
  
            \subsection*{184}
      \begin{verse}
      B. I 99. \\ M. 185. \\ \poemtitle{De elepho}B. IV 316. \\ Telephus, excellens Alcidis pignus et Augae, \\ Externae sortis bella inopina tulit. \\ Nam Grai Troiam peterent cum mille carinis \\ Tangeret et classis litus adacta suum, \\ Occurrens Danais forti dum pugnat Achilli, \\ Scyria pugnanti perculit hasta femur. \\ Pro cuius cura consultus dixit Apollo, \\ Hostica quod salubrem cuspis haberet opem. \\ Mox precibus flexi Pelidae robore raso, \\ Iniecto membris pulvere plaga fugit. \\ Monstrant fata viri vario miracula casu: \\ Vnde datum est vulnus, contigit inde salus. \\ 
      \end{verse}
  
            \subsection*{185}
      \begin{verse}
      B. V 187. \\ M. 1117. \\ \poemtitle{De cicindelo}B. IV 318. \\ Igniculus tenuis pingui fulcitur olivo, \\ Vt frangat tenebras luminis igne sui. \\ 
        \pagebreak 
    \begin{center} \textbf{CODICIS SALMASIANI.} \end{center} \marginpar{[157]} 
      \end{verse}
  
            \subsection*{186}
      \begin{verse}
      B. V 169. \\ M. 1099. \\ \poemtitle{De easpris}B. IV 317. \\ Lenaeos latices hircorum tergora gestant, \\ Fitque caper romio, fuerat qui victima, carcer. \\ 
      \end{verse}
  
            \subsection*{187}
      \begin{verse}
      B. V 170. \\ M. 1100. \\ Allter \\ p. 98 \\ B. IV 317. \\ Barbati pecoris vulgus per rura vagatur, \\ Pampineum gaudens laedere dente nemus. \\ Omnibus hinc aris Baccho caper hostia fertur \\ Puniturque sacro munere culpa gregis. \\ 
      \end{verse}
  
            \subsection*{188}
      \begin{verse}
      B. V 171. \\ M. 1101. \\ Aliter \\ B. IV 317. \\ Sanguine Cinypbio placantur templa Lyaei, \\ Caprigenum cui fit victima iusta pecus. \\ Sed licet ulcisci nequeant animalia divos \\ Nec possit reddi talio numinibus, \\ Est vindicta capris: cassatur nomine Liber, \\ Cum hircino adversus clauditur utre deus. \\ 
      \end{verse}
  
            \subsection*{189}
      \begin{verse}
      B. I 107. \\ M. 641. \\ \poemtitle{De Memnone}B. IV 317. \\ Filius Aurorae, Pboebi nascentis alumnus, \\ Producit gentis milia tetra suae. \\ 
        \pagebreak 
     \marginpar{[158]} \begin{center} \textbf{CARMINA} \end{center}Succurrens fessis fausto non omine Teucris \\ Pergit Pelidae protinus ense mori. \\ am tunc monstratur, maneat qui Pergama casus, \\ Cum nigrum Priamus suscipit anxilium. \\ B. V 195. \\ 
      \end{verse}
  
            \subsection*{190}
      \begin{verse}
      M. 1124. \\ \poemtitle{De Bumbulo}B. IV 318. \\ Nominis et formae pariter ludibria gestans \\ Conventus nostros, Sumbule parvus, adis. \\ Sed ratio est: extas longis Pygmaeus in armis, \\ Ne te deprensum e 2gina voret. \\ Nec frustra ostendis proprio placuisse parenti, \\ Qnod tnrpis nomen sumpseris heniochi. \\ Ille habnit doctas circi prostrare puellas; \\ Te duce lascivae nocte fricantur anus. \\ B. V 196. \\ 
      \end{verse}
  
            \subsection*{191}
      \begin{verse}
      M. 1125. \\ Aliter \\ B. IV 318. \\ Dum sis patris heres teneas et, Bumbule, censum \\ Vtile nec tibi sit, pietas si laesa probetur, \\ Das operam proprio auctori adversus haberi. \\ Discordat multum contra suscepta voluntas; \\ Dilexit genitor prasinum, te russeus intrat. \\ B. III 76. \\ 
      \end{verse}
  
            \subsection*{192}
      \begin{verse}
      M. 74. \\ \poemtitle{De tabula}B. IV 318. \\ Discolor ancipiti sub iactu cauculus astat \\ Decertantque simul candidus atque rubens. \\ 
        \pagebreak 
    \begin{center} \textbf{CODICIS SALMASIANI.} \end{center} \marginpar{[159]} Qui quamvis parili scriptorum tramite currant, \\ Is capiet palmam, quem bona fata iuvant. \\ 
      \end{verse}
  
            \subsection*{193}
      \begin{verse}
      B. III . \\ M. 9. \\ Aliter \\ B. IV 318. \\ In parte alveoli pyrgus velut urna resedit, \\ Qui vomit internis tesserulas gradibus, \\ Sub quarum iactu discordans cauculus exit \\ Certantesque fovet sors variata duos. \\ Hic proprium faciunt  \lbrack ars \rbrack  et fortuna periclum: \\ Haec cavet adversis casibus, illa favet. \\ Conposita est tabulae nunc talis formula belli, \\ Cuius missa facit tessera principium. \\ Ludentes vario exercent proelia talo, \\ Russeus an nitidns praemia sorte ferat. \\ Pascitur a multis avide damnosa voluptas, \\ Ne foedet gliscens otia segnities. \\ Hoc opus inventor nimie Palamedes amavit \\ Et parili excellens Mucius ingenio. \\ 
      \end{verse}
  
            \subsection*{194}
      \begin{verse}
      B. III 78. \\ M. 931. \\ Aliter \\ B. IV 319. \\ Indica materies blandum certamen amicis \\ Ofert, sed belli fert simulacra tamen. \\ Namque acie aequali concurrit russeus albo, \\ Vt gravibus damnisse domet alteruter. \\ p. 100 \\ 
        \pagebreak 
    \begin{center} \textbf{CARMINA} \end{center} \marginpar{[160]} Contorquet varios alternans tessera missus \\ Fataque ludentum collis et ima probant. \\ Pax ac pugna simul ludo iunguntur in unum, \\ Cum victi spoliis victor amicus ovat. \\ 
      \end{verse}
  
            \subsection*{195}
      \begin{verse}
      B. V 145. \\ M. 1081. \\ \poemtitle{De elephanto}B. IV 320. \\ lorrida cornuto procedit belua rostro, \\ Quam dives nostris Hndia misit oris. \\ Sed licet inmani pugnet proboscide barrus \\ Spondeat et saevis dentibus interitum, \\ Fert tamen et domitus residentis iussa magistri, \\ Quoque velit monitor, cogitur ire ferus. \\ Vis humana potest rabiem mutare ferinam: \\ Ecce hominem parvum belua magna timet. \\ 
      \end{verse}
  
            \subsection*{196}
      \begin{verse}
      IB V 144. \\ M. 1080. \\ Aliter \\ B. IV 320. \\ Monstrorum princeps elepbans proboscide saevus \\ Horret mole nigra, dente micat niveo. \\ Sed vario fugienda malo cum belua gliscat, \\ Est tamen expertis mors pretiosa feri. \\ 
        \pagebreak 
    \begin{center} \textbf{CODICIS SALMASIANI.} \end{center} \marginpar{[161]} Nam quae conspicimus montani roboris ossa, \\ Iumanis veniunt usibus apta satis. \\ Consulibus sceptrum, mensis decus, arma tablistis, \\ Discolor et tabulae calculus inde datur. \\ Haec est humanae semper mutatio sortis: \\ Fit moriens ludus, qui fuit ante pavor. \\ 
      \end{verse}
  
            \subsection*{197}
      \begin{verse}
      B. III 15. \\ M. 891. \\ \poemtitle{De eireensibus}B. IV 320. \\ Circus imago poli, formam cui docta vetustas \\ Condidit et numeros limitis aetherei. \\ Nam duodenigenas ostendunt ostia menses \\ Quaeque ineat cursim aureus astra iubar. \\ p. 01 \\ Tempora cornipedes referunt, elemeata colores; \\ Auriga, ut Phoebus, quattuor aptat equos. \\ Cardinibus propriis includunt septa quadrigas, \\ Ianus vexillum quas iubet ire levans. \\ Ast ubi panduntur funduntque repagula currus \\ Vnus et ante omnes cogitat ire prius, \\ Metarum tendunt circumdare cursibus orbes; \\ Namque axes gemini ortum obitumque docent. \\ Iamque his Euripus quasi magnum interiacet aequor, \\ Et medius centri summus obliscus adest. \\ 
        \pagebreak 
     \marginpar{[162]} \begin{center} \textbf{CARMNA} \end{center}Septem etiam gyri claudunt certamina palmae, \\ Quot caelum stringunt cingula sorte pari. \\ Lunae biga datur semper Solique quadriga, \\ Castoribus simpli rite dicantur equi. \\ Divinis constant nostra spectacula rebus, \\ Gratia magna quibus crevit honore deum. \\ 198 . \\ 
      \end{verse}
  
            \subsection*{}
      \begin{verse}
      \poemtitle{XV1}Verba Aohillis in parthenone, eum \\ M. 695. \\ tubam Diomedis audisset \\ B. IV 322. \\ Vana velut cautae surgens formido parenti \\ Femineos iuvenem iussit me sumere cultus \\ Et celare virum falso per tegmina sexu. \\ Vnde ego lanificas temptavi inglorius artes \\ Virgineisque toris carpenda ad vellera inhaesi. \\ Nunc autem proprium servans natura vigorem \\ Conpellit fluxos umeris depellere amictus. \\ Dum nos bella vocant rauco clangore tubarum, \\ Mens quatitur saevusque movet praecordia Mavors. \\ Rumpe moras omnes, fervens in praelia virtus, \\ Nee fuerit latuisse meum! si classica temno, \\ p. 102 \\ Dilexi latebras, neque culpa est ulla parentis. \\ Nam cum respicio quo sim de semine cretus \\ Semiferique animo recolo praecepta magistri, \\ Nec puerum decuit muliebri pectora peplo \\ Induere et teneram gressu simulare puellam. \\ 
        \pagebreak 
    \begin{center} \textbf{CODICIS SALMASIANI.} \end{center} \marginpar{[163]} Nunc igitur crescens annis sapientior aetas \\ Devovit Marti tenuit quae corda Cupido, \\ Belligerumque deum cum mens tum membra secuntur. \\ Stamina linquentes currant ad spicula palmae \\ Terrificumque caput praefixa casside mitram \\ Pellat, tecta gravi decorentur tempora ferro, \\ Arma tegant nostrum potius quam suppara corpus. \\ A telis ad tela decet transire iuventam \\ Atque hostes gladio quam lanas fundere fuso. \\ Tenuia loricae cedant multicia forti \\ Splendeat et raptus proiecto pectine mucro; \\ Contemtis radiis onerentur brachia pilis, \\ Clausa diu thalamo reddamus pectora campo. \\ Praesumit certam virtus sibi conscia palmam \\ Ac dubios gaudet perferre interrita casns. \\ Nil metuit, qui magna cupit. constantia mentis \\ Fata domat, nec iam potis est Fortuna nocere \\ Securio mortis, cui non sunt bella timori. \\ Fortibus una viris parilisque per omnia sors est, \\ Aut palmae aut leti pugnando adquirere laudem. \\ Cernere iam videor, quanta mercede cruoris \\ Constabit raptus Paridi crimenque iacenti, \\ Obvia cristatus cum sparserit agmina vertex \\ 0 Et solo aspectu claudentur Pergama nostro; \\ 
      \end{verse}
  
            \subsection*{}
      \begin{verse}
      \poemtitle{. 103}Cumque novus visos bellator fudero Teucros, \\ 
        \pagebreak 
    \begin{center} \textbf{CARMINA} \end{center} \marginpar{[164]} Multa trahet Xanthi Troiana cadavera grges \\ Maioremque feret cultorum sanguine cursum. \\ Sed mihi quis referat: ‘Tu, quem praesaga creatrix \\ Subducens fatis alieni schemate sexus \\ Ad Lycomedeos fecit transire penates \\ Depositumque suum maluit committere blandis \\ Virginibus, ne, te rapiat si Martius ardor, \\ Orbatam crucies inviso funere matrem: \\ Vadis in arma ferox thalamum natumque relinquens \\ Nec venit in mentem, quantum mereatur amorem, \\ Quae te prima virum conlato pectore fecit? \\ Is ad bella libens, ubi quaeritur alea mortis, \\ Nec spondet certam tristis Bellona salutem!’ \\ Sed Danais comes esse placet sociumque pericli \\ Pro famae titulis meliori adiungere causae. \\ ‘ScilicetutconiunxviduoreddaturA1ridi, \\ Procumbat vilis Teucrorum victima Achilles?’ \\ Aufer, iners monitor, turpis fomenta medellae \\ Meque sine proprio sectari praelia sensu. \\ Non ita me genitor praeclarus nomine Peleus \\ Aut dilecta Iovi fudit Thetis alma sub auras, \\ Degener ut lateam primaevo in flore iuventae \\ Maior et ignavo tantum mihi torpeat aetas \\ Abstrudamque toris iam debita pectora castris, \\ 
        \pagebreak 
     \marginpar{[165]} \begin{center} \textbf{CODICIS SALMASIAN1.} \end{center}Cum manus Argivum ultricia iuret in arma. \\ Denique cum promptum ruat in certamina vulgus, \\ Solus ego in cunctis infami carcere clausus \\ Subducar pugnae? quanto tolerabo pudore \\ p. 104 \\ Me non ferre pedem, quo fert Thersites, in omni \\ Parte miser, forma brevior menteque fugaci? \\ Absit ab ingenio ac viribus Aeacidarum, \\ Vt dubitem pro laude mori metuamque supremum \\ Quem dat Parca diem. mihi nam lux amplior illa est, \\ Quae virtute cluit, quae nescit claustra sepulcri. \\ Namque hominis semper meritorum lege perenni, \\ Quam breviat fatum, propagat gloria vitam. \\ Ergo animus fidens in Dardana saeviat arma \\ Nec mibi iam gemini dilectus pigneris obstet: \\ Deidamiam Pyrrhumque meos nunc Scyros habebit \\ Visuros nostrum reditum celebremque triumphum. \\ Me pudor hortatur rapere in certamina gressus; \\ Ferre potest, quaecumque †labans successibus aetas \\ Exigit. observans matris praecepta verendae \\ Induxi molles habitus velut edita virgo; \\ Lusimus et tactis modulantes carmina chordis. \\ Virtuti adsurgat, fuerat quaecumque, voluptas; \\ Succedat ferrum citharae. quod nutrit amores, \\ Depensum est Veneri; reddamus cetera Marti. \\ 
        \pagebreak 
     \marginpar{[166]} \begin{center} \textbf{CARMINA} \end{center}\poemtitle{XVI)}
      \end{verse}
  
            \subsection*{199}
      \begin{verse}
      \poemtitle{VES2AE}B. M. \\ B. IV 326. \\ Iudieium cocei et pistoris iudiee Vulcano \\ Ter ternae, varias docte quae traditis artes, \\ Linquite Pierios colles et scribite mecum. \\ Ille ego Vespa precor, cui divae saepe dedistis \\ Per multas urbes populo spectante favorem. \\ Scribere maius opus et dulcia carmina quaero, \\ Nec mel erit solum: aliquid quoque iuris habebit. \\ Contendit pistor, cocus est contrarius illi, \\ p. 105 \\ His est Vulcanus iudex, qui novit utrosque. \\ Ad causam pistor procedit primus agendam, \\ Canitiem capiti toto praebente farina: \\ ‘Numina per Cereris iuro, per Apollinis arcus! \\ Miror enim (fateor), et iam vix credere possum, \\ Quod cocus iste mihi sit respondere paratus, \\ Quisve sit utilior, audet contendere mecum, \\ De cuius manibus semper fit pane satullus. \\ Sunt testes anni faustae lanique lxalendae, \\ Quique meum studium per Saturnalia norunt; \\ Quorum epulas semper rerum commendo paratu. \\ Sis memor, o Saturne, tuis quod praesto diebus, \\ Et me prae studio trepidum tu numine firma. \\ 
        \pagebreak 
    \begin{center} \textbf{CODICIS SALMASIANI.} \end{center} \marginpar{[167]} Aurea coeperunt sub te quoque saecula farre. \\ Denique si Cereris non tu pia dona dedisses, \\ Roderet adsidue cocus iste sub ilice glandes. \\ Nempe opus est cunctis panis, quem nemo recusat; \\ Quo sine quas possunt mortales ponere cenas? \\ Qui vires tribuit, qui primum poscitur, hic est, \\ Quem serit agricola, quem maximus educat aether. \\ Hunc pater Aeneas Troianis vexit ab oris, \\ Nil sine quo tua iura valent, ingrate, coquina. \\ Provocor ut dicam: mihi panem tu, coce, temptas, \\ Quem docuit notus Cerealis fingere panes \\ Vrbe Placentinus, cunctas qui tradidit artes. \\ Pythagoras populo nescis quae suaserit olim? \\ Mandere ne vellent mixto cum sanguine carnes. \\ ‘Si iugulatis oves, quid erit quod vestiat?’ inquit, \\ ‘Mactentur vituli: nec erit iam vomeris usus \\ p. 106 \\ Nec segetum fecunda dabit sua munera tellus’ \\ Set temere facio, si te, coce, conparo nobis, \\ Cum possim, numen quodcumque potest superorum. \\ luppiter ipse tonat: tono, cum molo, sic ego pistor. \\ Mars subigit bello mnltas cum sanguine gentes: \\ Pistor ego macto flavas sine sanguine messes. \\ Tympana habet Cybele: sunt et mihi tympana cribri; \\ Thyrsitenens Satyros: facio et saturos ego plures; \\ Hllumr praecedunt Panes: facio mihi panes. \\ 
        \pagebreak 
    \begin{center} \textbf{CARMINA} \end{center} \marginpar{[168]} Quidque etiam manibus nostris non dulce paratur? \\ Nos facimns populo studiose coptoplacentas, \\ Nos adipata damus, nos grata canopica vobis, \\ Crustula nos lano; sponsae mustacia mitto. \\ Noverunt omnes pistorum dulcia facta: \\ Noverunt multi crudelia facta cocorum. \\ Tu facis in tenebris miserum prandere Thyestem; \\ Nescius ut Tereus cenet, facis, inprobe, natum; \\ Tu facis, in lucis ut cantet tristis aedon \\ Maestaque sub tecto sua murmuret acta chelidon. \\ Talia si numquam feci nec talia suasi, \\ Ordine primus ero, dignus quem palma sequatur’ \\ Conticuit pistor. coepit cocus ordine fari, \\ Ora niger studio, faciem mutante favilla: \\ ‘Si verbis pistor damnavit iura cocorum, \\ lli ne credas aliquid, quia fingere novit, \\ Qui semper multis dicit se vendere fuhum, \\ Stat qui sub saxo quasi Sisyphus atque laborat, \\ Denique qui tantum de melle et polline fingit \\ Has quas iactat opes. nobis quae copia, dicam. \\ Silva feras tribuit, pisces mare et aura volucres, \\ Dat vinum Bromius, Pallas mihi praestat olivam, \\ Datque sues Calydona et saepe ego condio dammas, \\ Saepe etiam perdix iacet et Iunonius ales, \\ Gemmatam pinnis solitus producere caudam.. \\ 
        \pagebreak 
    \begin{center} \textbf{CODICIS SALMASIANI.} \end{center} \marginpar{[169]} Certe quem extollit, quem laudat saepius ille, \\ Ille tuus panis sine nobis, crede, placere \\ Solus non poterit, nec si sit melleus ipse. \\ Quis me non laudet sternentem pisce patellas, \\ Cum positus madeat deceptus ab aequore rombus? \\ Sed similem superis ego me magis esse docebo. \\ Est Bromio Pentheus: est et mihi de bove Pentheus. \\ Vritur Alcides flammis: conburor ad ollas. \\ Sicut Neptuno, fervent in caccabo fluctus. \\ Novit Apollo suas studiose tangere chordas: \\ Et mihi per digitos texuntur quam bene chordae! \\ Exseco sic gallos, quasi A Berecyntia Gallos. \\ Partes quisque suas tollit, qui cenat aput me. \\ Vngellam Oedippi, sycotum pono Promethei, \\ Pentheo pono caput, ficatum do Tityoni; \\ Solus aqualiculum reddi sibi Tantalus orat. \\ Cervinam Actaeon tollit, Meleager aprinam, \\ Agninam Pelias, taurinam lingulus Aiax. \\ Orpheu, tu tollis chordas; Leaudre, lacertos; \\ Me sterilem Niobe, linguam Philomela rogant me. \\ 
        \pagebreak 
    \begin{center} \textbf{CARMINA} \end{center} \marginpar{[170]} Pluma Pbiloctetae servit, rogat Icarus alas; \\ Bubula Passiphae, Europe bubula poscit. \\ Auratam Danaae, cygnum bene condio Ledae. \\ Iam finem pugnae faciat sententia nobis.’ \\ p. 10 \\ Vtque cocus pressit vocem, sic Mulciber infit: \\ ‘Es, coce, suavis homo; dulcis sed tu queque, pistor. \\ Aequales dimitto deus, qui vos bene novi. \\ Consentite (probis sine rixa vivere dulce est), \\ Ne frigus faciam, si me subduxero vobis’ \\ \poemtitle{ \lbrack XVII}P erv i giiu m V en eri \\ B. M. \\ Snt vero versus XI \\ B. I V 292. \\ Crasametquinumquamamavitquiqueamavitcras amet! \\ Ver novum, ver iam canorum; vere natus orbis est, \\ 
        \pagebreak 
    \begin{center} \textbf{CODICIS SALMASIANI.} \end{center}Vere concordant amores, vere nubunt alites, \\ Et nemus comam resolvit de maritis imbribus. \\ Cras amorum copulatrix inter umbras arborum \\ Inplicat casas virentis de flagello myrteo, \\ Cras Dione iura dicit fulta sublimi throno. \\ 8 Crasametquinumquamamavitquiqueamavit crasamet! \\ psa gemmis purpurantem pingit annum floridis, \\ psa surgentes papillas de Favoni spiritn \\ Vrget in nodos patentes; ipsa roris lucidi, \\ Noctis aura quem relinquit, spargit umentis aquas. \\ Gutta praeceps orbe parvo sustinet casus suos \\ 1 Et micant lacrimae trementes de caduco pondere. \\ En pudorem florulentae prodiderunt purpurae! \\ Vmor ille, quem serenis astra rorant noctibus, \\ Mane virgineas papillas solvit †umenti peplo. \\ Ipsa iussit, mane totae virgines nubant rosae: \\ Facta Cypridis de cruore deque Amoris osculis \\ Deque gemmis deque flammis deque solis purpuris \\ Cras ruborem, qui latebat veste tectus ignea, \\ 
        \pagebreak 
     \marginpar{[172]} \begin{center} \textbf{CARMINA} \end{center}Vnico marita voto non pudebit solvere. \\ Crasametqui numquam amavitquiqueamavitcrasamet! \\ Ipsa Nymphas diva luco iussit ire myrteo: \\ ‘Ite, Nymphae, posuit arma, feriatus est Amor.’ \\ lt puer comes puellis; nec tamen credi potest, \\ Esse Amorem feriatum, si sagittas exuit. \\ ‘Iussus est inermis ire, nudus ire iussus est, \\ Neu quid arcu neu sagitta neu quid igne laederet.’ \\ Sed tamen, Nymphae, cavete, quod Cupido pulcher est: \\ Totus est in armis idem quando nudus est Amor. \\ Cras ametquinumquamamavitquiqueamavitcrasamet! \\ Conpari Venus pudore mittit ad te virgines: \\ p. 10 \\ ‘Vna res est quam rogamus: cede, virgo Delia, \\ Vt nemus sit incruentum de ferinis stragibus. \\ Ipsa vellet te rogare, si pudicam flecteret, \\ lpsa vellet ut venires, si deceret virginem. \\ Iam tribus choros videres feriantis noctibus \\ Congreges inter catervas ire per saltus tuos \\ Floreas inter coronas, myrteas inter casas. \\ Nec Ceres nec Bacchus absunt nec poetarum deus. \\ 
        \pagebreak 
    \begin{center} \textbf{CODICIS SALMASAN1.} \end{center} \marginpar{[173]} Detinenda tota nox est, perviglanda canticis. \\ Regnet in silvis Dione; tu recede, Delia’ \\ Cras amet qui numquamamavitquiqueamavitcrasamet! \\ Iussit Hyblaeis tribunal stare diva floribus: \\ Praeses ipsa iura dicet, adsidebunt Gratiae. \\ Hybla, totos funde flores, quidquid annus adtulit! \\ Iybla, florum subde vestem, quantus Ennae campus est, \\ Vt recentibus virentes ducat umbras floribus! \\ Ruris hic erunt puellae vel puellae montium \\ Quaeque silvas quaeque lucos quaeque fontes incolunt. \\ Iussit omnes adsidere pueri mater alitis, \\ Iussit et nudo puellas nil Amori credere. \\ Cras amet qui numquamamavitquiqueamavitcrasamet! \\ Cras erit, cum primus Aether copulavit nuptias. \\ Vt pater totum crearet vernis annum nubibus, \\ In sinum maritus imber fluxit almae coniugis, \\ Vnde fetus mixtus omnis aleret magno corpore. \\ Tunc cruore de superno spumeo pontus globo \\ Caerulas inter catervas inter et bipedes equos \\ Fecit undantem Dionen †de maritis imbribus. \\ 
        \pagebreak 
    
      \end{verse}
  
            \subsection*{}
      \begin{verse}
      \poemtitle{CARMNA}Crasamet qui numquamamavitquiqueamavitcrasamet! \\ Ipsa venas atque mentem permeanti spiritu \\ 63 \\ Intus occultis gubernat procreatrix viribus, \\ n. 177 \\ Perque caelum perque terras perque pontum subditum \\ Pervium sui tenorem seminali tramite \\ Inbuit iussitque mundum nosse nascendi vias. \\ Crasamet quinumquam amavitquiqueamavitcrasamet! \\ Ipsa Troianos penates in Latinos transtulit, \\ Ipsa Laurentem puellam coniugem nato dedit, \\ Moxque Marti de sacello dat pudicam virginem; \\ omuleas ipsa fecit cum Sabinis nuptias, \\ Vnde Ramnes et Quirites, unde prolem posterum, \\ Romuli gentem, crearet et nepotem Caesarem. \\ Cras amet qui numquam amavitquiqueamavitcrasamet! \\ Rura fecundat voluptas, rura Venerem sentiunt; \\ Ipse Amor puer Dionae rure natus dicitur. \\ Iunc ager cum parturiret, ipsa suscepit sinu, \\ psa florum delicatis educavit osculis. \\ Crasametqui numquamamavitquiqueamavitcrasamet! s0 \\ Ecce iam subter genestas explicant tauri latus, \\ Quisque tutus quo tenetur coniugali foedere. \\ Subter umbras cum maritis ecce balantum greges \\ 
        \pagebreak 
    \begin{center} \textbf{CODICIS SALMASIANI.} \end{center} \marginpar{[175]} Et canoras non tacere diva iussit alites. \\ Iam loquaces ore rauco stagna cgni perstrepunt, \\ Adsonat Terei puella subter umbram populi, \\ Vt putes motus amoris ore dici musico \\ Et neges queri sororem de marito barbaro. \\ Illa cantat, nos tacemus. quando ver venit meum? \\ Quando faciam †uti cbhelidon, ut tacere desinam? p. \\ Perdidi Musam tacendo nec me Phoebus respicit. \\ Sic Amyclas cum tacerent perdidit silentium. \\ Crasametquinumquamamavitquiqueamavitcrasamet! \\ 
      \end{verse}
  
            \subsection*{201}
      \begin{verse}
      \poemtitle{III)}B. I 91. \\ M. 632. \\ e hetide \\ B. IV 331. \\ Pande manum, genetrix; totus tingatur Achilles! \\ Tu facies natum mortis habere locum. \\ 
      \end{verse}
  
            \subsection*{202}
      \begin{verse}
      B. III 52. \\ M. 928. \\ \poemtitle{De lueo amoeno}B. IV 331. \\ Hlic, Cytherea, tuo poteras cum Marte iacere: \\ Vulcanus prohibetur aquis, Sol pellitur umbris. \\ 
        \pagebreak 
     \marginpar{[176]} \begin{center} \textbf{CARMNA} \end{center}
      \end{verse}
  
            \subsection*{203}
      \begin{verse}
      B. III 27. \\ M. 383. \\ LVX \\ B. IV 331. \\ In Anclas; in salutatorium domini regis \\ lildirici regis fulget mirabile factum \\ Arte opere ingenio divitiis pretio. \\ Hinc radios sol ipse capit, quos huc dare possit: \\ Altera marmoribus creditur esse dies. \\ Hic sine nube solum, †nix iuncta et sparsa putatur; \\ Dum steterint, credas mergere posse pedes. \\ 
      \end{verse}
  
            \subsection*{204}
      \begin{verse}
      B. V 201. \\ M. 1133. \\ \poemtitle{De Servando medioo .}B. IV 332. \\ Servandum spurcum medicum nostrumque medeurum, \\ Qui se Tartareo missum de carcere finxit, \\ Auctoritate tumens Orci, cui corpora mittit \\ Inperitus iners, haustu terrae repetendus! \\ Cum staret in medio, mox illi voce superba \\ Burdonum ductor (paleas nam forte gerebat) \\ ‘Servande infamis, Servande abule, pestis’ \\ Aibat, ‘Servande canis, servande catenis, \\  \lbrack O Servande meis semper servande flagellis, \\ 
        \pagebreak 
    \begin{center} \textbf{CODICIS SALMASIAN1.} \end{center} \marginpar{[177]} Servande in parte misera nabras tanos aesis \\ Vitivalas valmam vitiduis tanda vitritam \\ Capia feis’: gibatus enim transire volebat. \\ 
      \end{verse}
  
            \subsection*{205}
      \begin{verse}
      B. V 205. \\ M. 1134. \\ \poemtitle{De eastellano}B. IV 332. \\ Castellanesorexcluacae,pressuratuorump \\ Horrida, caeca fames! cenum tibi bullit in ore, . \\ Putredo et cancer trivit dentesque malasque. \\ Accipe, nariputens oris latrina, Filippe: \\ Lividus in rubro color est tibi, cepula, vultu, \\ Obscenitas frontem mortis sulcavit aratro, \\ Fistula sunt oculi, polypus de naribus horret \\ Et Manes patitur putldo sub coniuge coniunx, \\ Ac viduam simul . facit fortuna suorum. \\ Qua te cumque moves, os culum porrigis ultro; \\ Nam turpe est fetore gravi, si forte loquaris; \\ Si taceas, fissis secessum naribus efflas. \\ Lasanus es plenus, †guttur fusorius est tibi grandis. \\ 
        \pagebreak 
     \marginpar{[178]} \begin{center} \textbf{CARMINA} \end{center}
      \end{verse}
  
            \subsection*{206}
      \begin{verse}
      B. V 200. \\ M. 1129. \\ \poemtitle{De Perpetuo}B. IV 333. \\ Parce, rapa, epulis vehemens, Perpetue, glutto \\ Mensarumque vorax, omnis cui †cicula cessit, \\ Dulci qui maragas furto subducis ofellas. \\ 
      \end{verse}
  
            \subsection*{207}
      \begin{verse}
      B. V 202. \\ M. 1I131. \\ \poemtitle{De Cresituro}B. IV 333. \\ Cresciture, ferox ne quid tibi dorsa flagellis \\ Conscindat coniunx, iunctis tu pedibus astas. \\ 
      \end{verse}
  
            \subsection*{208}
      \begin{verse}
      B. V 201. \\ M. 1130. \\ \poemtitle{De Tautano}B. IV 333. \\ Tautane, infamem nulla quem coniuge captum \\ Spiritus inmundus subito praecordia torsit, \\ Vendere mancipium pulcrum commune theatro \\ 
      \end{verse}
  
            \subsection*{209}
      \begin{verse}
      B. V 203. \\ M. 1132. \\ \poemtitle{De Abcare servo dominloo}B. IV 354. \\ legius est Abcar servus, †palus hispidus ursus \\ Rana nanus strobilus palmus zeloceca cylindrus. \\ 
        \pagebreak 
    \begin{center} \textbf{CODCIS SALMASIANI.} \end{center} \marginpar{[179]} Piperis exigui formam vix corpore conples. \\ Pulicis e corio vestit te gunna profusa. \\ Ad maratros dabitur grandis formica caballus. \\ p. 114 \\ Pulveris ut pilula brevis es, ut glomus hic erras. \\ Ast ubi dormieris, caveat castissima coniunx, \\ Erres ne subito mistus sub nocte tomento. \\ Lendis forma tibi, statu non transilis ova, \\ Aequalis pipcri, par est tibi forma cumini, \\ Fasciarii pondus, longi pars summa telonis. \\ Sic sepiam pelagus crispanti vertice gestat, \\ Sic niger inpellit pilulas de stercore inzar \\ Araneosque leves digitis pendens in stamine bulla. \\ 
      \end{verse}
  
            \subsection*{210}
      \begin{verse}
      \poemtitle{ELICIS}vlrt clrlsslmi \\ B. III 3. \\ M. 291. \\ \poemtitle{De thermis Alianarum}B. IV 3E1. \\ lic ubi conspicuis radiant nunc signa metallis \\ Et nitido clarum marmore fulget opus, \\ Arida pulvereo squalebat cespite tellus \\ Litoreique soli vilis arena fuit. \\ Pulcra sed inmenso qui duxit culmina caelo, \\ Ostendens pronis currere saxa iugis, \\ Publica rex populis Thrasamundus gaudia vovit, \\ Prospera continuans numine saecla suo. \\ 
        \pagebreak 
    \begin{center} \textbf{CARMNA} \end{center} \marginpar{[180]} Paruit imperiis mutato lympha sapore \\ Et dulcis fontes proluit unda novos. \\ Expavit subitas Vulcanus surgere thermas \\ Et trepida flammas subdidit ipse manu. \\ 01 1 \\ 
      \end{verse}
  
            \subsection*{}
      \begin{verse}
      \poemtitle{EIVSDEM}B. III 35. \\ M. 22. \\ Aiter \\ B. IV 335. \\ Nobilis insultat Baiarum fabrica thermis \\ Et duplicat radios fontibus aucta dies. \\ Hoc uno rex fecit opus Thrasamundus in anno, \\ Inclita dans populis munera temporibus. \\ Hic senibus florens virtus renovatur anhelis, \\ Hic fessos artus viva lavacra fovent. \\ p. 115 \\ Miscentur pariter sociis incendia lymphis \\ Et gelidos imbres proximus ignis habet. \\ Vtilis hic flamma est et nullos pascitur artus \\ Optaturque magis per nova vota calor. \\ Longior hic aegros morborum cura relinquit \\ Nec lavat in vitreis hic moriturus aquis. \\ 0 1 0 \\ 
      \end{verse}
  
            \subsection*{}
      \begin{verse}
      \poemtitle{EIVSDEM}II 36. \\ 1. 23. \\ Aliter \\ B. IV 335. \\ Regia praeclaras erexit iussio moles, \\ Sensit et imperium calx lapis unda focus. \\ 
        \pagebreak 
    \begin{center} \textbf{CODICIS SALMASIANI.} \end{center} \marginpar{[181]} Inclusus Vulcanus aquis algentibus hic est \\ Et pacem liquidis fontibus ignis habet. \\ Cum lymphis gelidis extat concordia flammae \\ Ac stupet ardentes frigida nympha lacus. \\ Vritur hic semper gaudens neque laeditur hospes \\ Et vegetat medicus pectora fota vapor. \\ Mxima sed quisquis patitur fastidia solis \\ Aut gravibus madido corpore torpet aquis, \\ Hi Thrasamundiacis properet se tinguere thermis: \\ Protinus effugiet tristis uterque labor. \\ 1q \\ 
      \end{verse}
  
            \subsection*{}
      \begin{verse}
      \poemtitle{EIVSDEM}B. III 37. \\ M. 294. \\ Aliter \\ B. IV 36. \\ Publica qui celsis educit moenia tectis, \\ Hic pia rex populis Thbrasamundus vota dicavit, \\ Per quem cuncta suis consurgunt pulchra ruinis \\ Et nova transcendunt priscas fastigia sedes. \\ lic quoque post sacram meritis altaribus aedem \\ Egregiasque aulas, quas grato erexit amore, \\ Condidit ingentes proprio sub nomine thermas. \\ Hic bonus inriguis decertat fontibus ignis, \\ Hic etiam ardentis  \lbrack nemo timet ora camii, \\ p. 116 \\ Plurimus hic imber gelidas adcommodat undas, \\ Hic aestus levis est, hic nullum frigora torrent, \\ Hic geminata dies per candida marmora fulget. \\ 
        \pagebreak 
     \marginpar{[182]} \begin{center} \textbf{CARMNA} \end{center}
      \end{verse}
  
            \subsection*{214}
      \begin{verse}
      \poemtitle{EIVSDEM}B. III 33. \\ M. 892. \\ Aliter \\ B. IV 336. \\ Tranquillo, nymfae, deCurrite fluminis ortV; \\ Huc, proba, flagranti sVccedite, numina, Foeb0, \\ Rupibus ex celsis ubi Nunc fastigia surgun? \\ Aequanturque polo toTis praecelsa lavacr \\ Sedibus. hic magnis exArdent marmora signiS, \\ Ardua sublimes praevlncunt culmina termaE, \\ Muneraque eximius taNti dat liminis auctoR \\ Vnica continuae praeNoscens praemia fama. \\ Non hic flamma nocet. vOtum dinoscite carmeN, \\ Discite vel quanto viVat sub gurgite lymph. \\ Vandalicum hic renovAt caro de semine nomeN, \\ Sub cuius titulo meriTis stat gratia factiS. \\ IB. V 182. \\ 
      \end{verse}
  
            \subsection*{215}
      \begin{verse}
      M. I112. \\ In Anels \\ B. IV 337. \\ Vandalirice potens, gemini diadematis bheres, \\ Ornasti proprinm per facta ingentia nomen. \\ 
        \pagebreak 
    \poemtitle{CODICIS SALMASIANI.}Belligeras acies domuit Theodosius ultor, \\ Captivas facili reddens certamine gentes. \\ Adversos placidis subiecit Honorius armis, \\ Cuius prosperitas melior fortissima fecit. \\ Ampla Valentiniani virtus cognita mundo \\ Hostibus addictis ostenditur arce nepotis. \\ 
      \end{verse}
  
            \subsection*{216}
      \begin{verse}
      B. III 156 \\ M. 940. \\ Postulatio muneris \\ p.211 \\ B. IV 37. \\ Sie tibi florentes aequaevo germine nati \\ Indolis aetheriae sidera celsa petant, \\ p.212 \\ Sic priscos vincant atavos clarosque parentes \\ Exuperent meritis saeclaque longa gerant, \\ Sic subolis numerum transcendat turba nepotum \\ Nobilibusque iuges gaudia tanta toris: \\ Ne sterilem praestes indigno munere Musam, \\ Vtque soles, largus carmina nostra fove, \\ Imperiis ut nostra tuis servire Thalia \\ Possit et in melius personet icta chelys! \\ 
      \end{verse}
  
            \subsection*{0}
      \begin{verse}
      7 1 4 \\  \lbrack  p \\ B. III 253. \\ M. 183. \\ Epistula. Amans amanti \\ p. 116 \\ B. IV 338. \\ Candida sidereis ardescunt lumina flammis, \\ Fundunt colla rosas et cedit crinibus aurum, \\ 
        \pagebreak 
    \poemtitle{CARMINA}Mollia purpureum †depromunt ora ruborem \\ p. 777 \\ Lacteaque admixtus sublimat pectora sanguis, \\ Ac totus tibi servit honos formaque dearum \\ Funlges et Venerem caelesti corpore vincis. \\ Argento stat facta manus digitisque tenellis \\ Serica flla trahens pretioso in stamine ludis. \\ Planta decens nescit modicos calcare lapillos \\ Et dura laedi scelus est vestigia terra; \\ Ipsa tuos cum ferre velis per lilia gressus \\ Nullos interimes leviori pondere flores. \\ Guttura nunc aliae magnis \lbrack pve \rbrack  monilibus ornent \\ Aut gemmas aptent capiti: tu sola placere \\ Vel spoliata potes. nulli laudabile totum est: \\ In te cuncta probat, si quisquam cernere possit. \\ Sirenum cantus et dulcia plectra Thaliae \\ Ad vocem tacuisse rear, qua mella propagas \\ Dulcia et in miseros telum iacularis amoris. \\ Langueo deficio marcesco punior uror \\ Aestuo suspiro pereo debellor anhelo, \\ Et grave vulnus alo nullo sanabile ferro. \\ 
        \pagebreak 
    \begin{center} \textbf{CODICIS SALMASIANI.} \end{center} \marginpar{[185]} Sed tua labra meo saevum de corde dolorem \\ Depellant morbumque animae medicaminis huius \\ Cura fuget, ne tanta putres violentia nervos \\ Dissecet atque tuae moriar pro crimine causae. \\ Sed si hoc grande putas, saltem concede precanti, \\ Vt iam defunctum niveis ambire lacertis \\ Digneris vitamque mibi post fata reducas. V \\ B.III27. M.379. \\ 
      \end{verse}
  
            \subsection*{218}
      \begin{verse}
      B. IV 96. \\ Bucoh. Petron. \\ \poemtitle{PETRONII}. 48. \\ De malis aureis amator ab amata missis \\ Aurea mala mihi, dulcis mea Martia, mittis, \\ Mittis et hirsutae munera castaneae. \\ p.778 \\ Omnia grata putem; sed si magis ipsa venires, \\ Ornares donum, pulcra puella, tuum. \\ Tu licet adportes stringentia mala palatum, \\ Tristia mandenti est melleus ore sapor. \\ At si dissimulas, multum mihi cara, venire, \\ Oscula cum pomis mitte: vorabo libens. \\ 
        \pagebreak 
     \marginpar{[186]} \begin{center} \textbf{CARMINA} \end{center}
      \end{verse}
  
            \subsection*{219}
      \begin{verse}
      B. I 12. \\ M. 246. \\ \poemtitle{De Narcisso}B. IV 340. \\ Se Narcissus amat captus lenonibus undis. \\ Cui si tollis aquas, non est ubi saeviat ignis. \\ 
      \end{verse}
  
            \subsection*{220}
      \begin{verse}
      B. II 1I et V 93. \\ M. 706. \\ \poemtitle{De Perdicca}B. IV 310. \\ Eximius Perdicca fuit, qui corpore eburno \\ Fulgebat roseisque genis, cui lumina blanda \\ Fundebant flammas, crocei per colla capilli \\ Pendebant variosque dabant sibi saepe colores. \\ Fulvus poples erat, nitidus pes. omnia rident, \\ Quidquid habet iuvenis. solus vincebat Adonem. \\ 001 \\ IB. I 28. \\ M. 586. \\ 
      \end{verse}
  
            \subsection*{}
      \begin{verse}
      \poemtitle{De Cupidine}B. IV 340. \\ Sol calet igne meo. flagrat Neptunus in undis. \\ Pensa dedi Alcidae. Bacchum servire coegi, \\ Quamvis ‘Liber’ erat.  \lbrack feci mugire Tonantem \rbrack  \\ 
        \pagebreak 
    \begin{center} \textbf{CODICIS SALMASIANI.} \end{center} \marginpar{[187]} 
      \end{verse}
  
            \subsection*{000}
      \begin{verse}
      B. II 183. \\ M. 835. \\ De libris Vergili ab asino oomestis . V \\ Carminis liaci libros consumpsit asellus. \\ O fatum Troiael aut ecus aut asinus! \\ 
      \end{verse}
  
            \subsection*{223}
      \begin{verse}
      B. I 176. \\ \poemtitle{C OR 0 N AT I}M. 549. \\ B. IV 186. \\ vrl clrisslml \\ Locus Vergilianus: \\ Vivo equidem vitamque extrema per oni dueo’ \\  \lbrack sunt vero \rbrack  versus XXIII \\ X Aspera diverso lassatur vita dolore \\ Et morti vicinus eo vivitque dolori \\ Aegra salus. nostras cruciant dum flamina mentes, \\ Vitae flatus abit. quantos post Pergama casus \\ (Vidit enim praeclara manus), quos saepe dolores \\ Pertulit atque iterum quas sensit Troia ruinas! \\ Plus cecidit post fata Phrygum: nunc ipsa cremata est \\ In natis incensa suis recidivaque morti \\ 
        \pagebreak 
     \marginpar{[188]} \begin{center} \textbf{CARMINA} \end{center}Iuncta est et casum †meruit miseranda †maritum. \\ Traditur infestis semper nova nupta procellis \\ Et flammas habet ipsa suas taedasque ministrat \\ Sola sibi. non †lucis eget, non ignibus umquam: \\ De facibus micat ipsa suis lumenque nigellum \\ Possidet infelix semper. quas aequoris iras, \\ Quas caeli terraeque lues miserabile vulgus \\ Pertulit, errantes saevis Tritonis in undis! \\ Atque domus mihi pontus erat Phrygiique penates \\ Et quasi iam pirata fui, lacrimisque profusis \\ Inter aquas †siccabat homo nostrasque proferto \\ Fluctibus addebant; fluctus crescebat in astra, \\ Et mihi naufragium nostri fecere liquores. \\ Iam mulier sibi nauta fuit, iam Virginis astrum \\ Vidit virgo, potens fortes torquere rudentes \\ Et remis aptare manus; curaque vigente \\ Nocte vigil fuerat. non norat femina somnum; \\ Sideribus iam docta poli (labor ipse magistram \\ Fecerat astrornm) norat venientia fata. \\ Non ver tranquillum fuerat, non mollior aestas; \\ Sideribus tempus, requiem non denique uorat. \\ 
        \pagebreak 
    \begin{center} \textbf{CODICIS SALMASIANI.} \end{center} \marginpar{[189]} 
      \end{verse}
  
            \subsection*{224}
      \begin{verse}
      M. 925. \\ \poemtitle{De eleetione conlugii}B. IV 341. \\ Moribus et vultu coniunx quaeratur habenda. \\ Horrida nam facies nullo celatur ab auro. \\ p.120 \\ Si quis erit sponsus, talem qui ducat, avarus, \\ Horret et ipse suam: feditas dissolvet amorem. \\ Sequitur illa calens, dorsum dabit ille calenti; \\ Cogetur fervore suo clunem submittere asello, \\ Concubitu turpi monstrum paritura biforme, \\ Quem mater ipsa suum timeat contingere natum. \\ Discite, formosas aurum superare puellas. \\ 
      \end{verse}
  
            \subsection*{225}
      \begin{verse}
      B V 154. \\ . 1088. \\ \poemtitle{De esiciata}B. IV 341. \\ Amisit proprias vacuato corpore carnes \\ Accepitque novas. dedit ac tulit et grave lucrum \\ Perdendo adquirit. damno bene crevit in illo. \\ 
        \pagebreak 
     \marginpar{[190]} \begin{center} \textbf{CARMINA} \end{center}
      \end{verse}
  
            \subsection*{226}
      \begin{verse}
      \poemtitle{C0 R 0 N A TI}, y155. \\ M. 550. \\ vlr elrlssflmi \\ B. IV 342. \\ Aliter unde upra \\ Mortua fit praedo; pullorum turgida membris \\ Ex aliis crescit nec sese repperit in se. \\ 
      \end{verse}
  
            \subsection*{227}
      \begin{verse}
      \poemtitle{DONATI}. V 156. \\ M. 552 \\ \poemtitle{De ovat}B. IV 342. \\ Ovorum copiosa phalanx in ventre tumentis \\ Conditur et membris crescit gallina repletis. \\ Mortua concepit, quantum nec viva creavit. \\ 
      \end{verse}
  
            \subsection*{228}
      \begin{verse}
      \poemtitle{CORONATI}B. I 177. V 157. \\ M. 551. \\ trl earlsslml \\ B. IV 342. \\ Vnde supa \\ Medeam fertur natos Prognenque necasse: \\ Iaec natis atavisque simul vel caede sororum \\ Crescit; plus moriens sumpsit de prole tumorem. \\ 
      \end{verse}
  
            \subsection*{229}
      \begin{verse}
      B. V 1583. \\ M. 1089. \\ \poemtitle{De pressa}B. IV 342. \\ Turgida sum moriens sollertique arte defixa. \\ Crescens decresco, nomine pressa vocor. \\ 
        \pagebreak 
    \begin{center} \textbf{CODICIS SALMASIANI.} \end{center} \marginpar{[191]} 
      \end{verse}
  
            \subsection*{230}
      \begin{verse}
      B. V 159. \\ M. 100. \\ \poemtitle{De mistu asso}B. V 342. \\ Perna, lepus, turtur, perdix, Iunonius ales, \\ Agnus, porcellus iunguntur, candidus anser. \\ Aethera quod pontusque, altrix quod terra creavit, \\ Cernimus esciferum paulatim sumere ventrem. \\ I 4 \\ B. V 160. \\ M. 1091. \\ 
      \end{verse}
  
            \subsection*{}
      \begin{verse}
      \poemtitle{De pastllo coeeti}B. IV 34. \\ Blandum mellis opus sollerti fingitur arte. \\ Faucibus hoc dulce est, dentibus interitus. \\ 040 \\ 
      \end{verse}
  
            \subsection*{}
      \begin{verse}
      \poemtitle{SENECAE}B. III 95. \\ M. 131. \\ 
      \end{verse}
  
            \subsection*{}
      \begin{verse}
      \poemtitle{De qualitte temporis}. IV 55. \\ XI Omnia tempus edax depascitur, omnia carpit, \\ Omnia sede movet, nil sinit esse diu. \\ Flumina deficiunt, profugum mare litora siccant, \\ Subsidunt montes et iuga celsa ruunt. \\ Quid tam parva loquor? moles pulcherrima caeli \\ Ardebit flammis tota repente suis. \\ Omnia mors poscit. lex est, non poena, perire: \\ Hic aliquo mundus tempore nullus erit. \\ 
        \pagebreak 
     \marginpar{[192]} \begin{center} \textbf{CARMINA} \end{center}02 \\ 
      \end{verse}
  
            \subsection*{}
      \begin{verse}
      \poemtitle{CAESARIS}B. I1 230. \\ M. 554. \\ 
      \end{verse}
  
            \subsection*{}
      \begin{verse}
      \poemtitle{De libris Lucni}B. IV 102. \\ Mantua, da veniam, fama sacrata perenni: \\ Sit fas Thessaliam post Simoenta legi. \\ 0 \\ 
      \end{verse}
  
            \subsection*{234}
      \begin{verse}
      \poemtitle{PETADI}B. III 105. \\ M. 251. \\ D e fortuna \\ B. IV 343. \\ Res eadem adsidue momento volvitur uno \\ Atque redit dispar res eadem adsidue. \\ Vindice functa manu Progne pia dicta sorori, \\ Impia sed nato vindice functa manu. \\ Carmine nisa suo Colchis fuit ulta maritum, \\ Sed scelerata fuit carmine nisa suo. \\ Coniugis Eurydice precibus remeabat ad auras: \\ Rursus abit vitio coniugis Eurydice. \\ Sanguine poma rubent Thisbae nece tincta repente: \\ Candida quae fuerant, sanguine poma rubent. \\ Daedalus arte sua fugit Minoia regna: \\ Amisit natum Daedalus arte sua. \\ p. 122 \\ 
        \pagebreak 
    \begin{center} \textbf{CODCIS SALMASIAN.} \end{center} \marginpar{[193]} Munere Palladio laeti qua nocte fuere, \\ Hac periere Phryges munere Palladio. \\ Nate, quod alter ades caelo, sunt gaudia Ledae; \\ Sed maeret mater, nate, quod alter ades. \\ Hostia †saepe fuit diri Busiridis hospes, \\ Busirisque aris hostia †saepe fuit. \\ Theseus Hippolyto vitam per vota rogavit,, \\ Optavit mortem Theseus Hippolyto. \\ Stipite fatifero iuste quae fvatribus usa est, \\ Mater saeva fuit stipite fatifero. \\ Solarelictatorisflevisti \lbrack in \rbrack litore,Gnosis; \\ Laetatur caelo sola relicta toris. \\ Aurea lana fuit, Phrixum quae per mare vexit; \\ Helle qua lapsa est, aurea lana fuit. \\ Tantalis est numero natorum facta superba, \\ Natorum adflicta Tantalis est numero. \\ Pelias hasta fuit, vulnus grave quae dedit hosti; \\ loc quae sanavit, Pelias hasta fuit. \\ Per mare iacta ratis pleno subit ostia velo, \\ In portu mersa est per mare iacta ratis. \\ Lux cito summa datur natusque extinguitur infans \\ Atque animae prima lux cito summa datur. \\ 
        \pagebreak 
    \begin{center} \textbf{CARMINA} \end{center} \marginpar{[194]} Sunt mala laetitiae diversa lege creata, \\ Iuncta autem adsidue sunt mala laetitiae. \\ 0 \\ 5a0a \\ 
      \end{verse}
  
            \subsection*{}
      \begin{verse}
      \poemtitle{EIVSDEM}B. V 69. \\ M. 252. \\ 
      \end{verse}
  
            \subsection*{}
      \begin{verse}
      \poemtitle{De adventu veris}B. IV 344. \\ hiems;EephyrisqueanimantibusorbemSentio,fugit \\ Iam tepet Eurus aquis. sentio, fugit hiems. \\ Parturit omnis ager, persentit terra calores \\ Germinibusque novis parturit omnis ager. \\ . \\ ,Laeta virecta tument, foliis sese induit arbor, \\ Vallibus apricis laeta virerta tument. \\ Iam Philomela gemit modulis, ltyn impia mater \\ Oblatum mensis iam Philomela gemit. \\ lonte tumultus aquae properat per laevia saxa \\ Et late resonat monte tumultus aquae. \\ Floribus innumeris pingit sola flatus Eoi \\ Tempeaque exhalant floribus innumeris. \\ Per cava saxa sonat pecudum mugitibus Ecbo \\ Bisque repulsa iugis per cava saxa sonat. \\ Vitea musta tument vicinas iuncta per ulmos; \\ Fronde maritata vitea musta tument. \\ Dum recolit nidos, nota tigilla linit. \\ 
        \pagebreak 
    \begin{center} \textbf{CODICIS SALMASIANI.} \end{center} \marginpar{[195]} Sub platano viridi iucundat somnus in umbra, \\ Sertaque texuntur sub platano viridi. \\ Tunc quoque dulce mori, tunc fila recurrite fusis, \\ Inter et amplexus tunc quoque dulce mori. \\ 
      \end{verse}
  
            \subsection*{236}
      \begin{verse}
      \poemtitle{SENECAE}B. III 12. \\ M. 129. \\ D e Corsiea \\ B. IV 55. \\ Corsica Phocaico tellus habitata colono, \\ Corsica quae Graio nomine Cyrnos eras, \\ Corsica Sardinia brevior, porrectior lva, \\ Corsica piscosis pervia fluminibus, \\ Corsica terribilis, cum primum incanduit aestas, \\ Saevior, ostendit cum ferus ora Canis: \\ Parce relegatis; hoc est: iam parce solutis! \\ Vivorum cineri sit tua terra levis! \\ p. 12 \\ \poemtitle{ \lbrack EIVSDEM)}B. III 13. \\ M. 130. \\  \lbrack tem \rbrack  \\ B. IV 56. \\ Barbara praeruptis inclusa est Corsica saxis, \\ Horrida, desertis undique vasta locis. \\ 
        \pagebreak 
     \marginpar{[196]} \begin{center} \textbf{CARMNA} \end{center}Non poma autumnus, segetes non educat aestas \\ Canaque Palladio munere bruma caret. \\ Imbriferum nullo ver est laetabile fetu \\ Nullaque in infausto nascitur herba solo. \\ Non panis, non haustus aquae, non ultimus ignis; \\ Hic sola haec duo sunt: exul et exilium. \\ 
      \end{verse}
  
            \subsection*{238}
      \begin{verse}
      B. V 14. \\ M. I026. \\ De  \lbrack die oeciduo \\ B. IV 56. \\ Iam nitidum tumidis Phoebus iubar intulit undis, \\ Emeritam renovans Tethyos amne facem. \\ Astra subit niveis Phoebe subvecta iuvencis, \\ Mitis et aetherio labitur axe sopor. \\ 
      \end{verse}
  
            \subsection*{0}
      \begin{verse}
      Adludunt pavidi tremulis conatibus agni \\ Lacteolasque animas lacteus umor alit. \\ 
      \end{verse}
  
            \subsection*{239}
      \begin{verse}
      B. II 12. \\ M. 186. \\ aus ersis \\ B. IV 77. \\ Xerses magnus adest. totus comitatur euntem \\ Orbis. quid dubitas, Graecia, ferre iugum? \\ 
        \pagebreak 
    \begin{center} \textbf{CODICIS SALMASAN1.} \end{center} \marginpar{[197]} Tellus iussa facit, caelum texere sagittae, \\ Abscondunt clarum Persica tela diem. \\ Classes fossus Athos intra sua viscera vidit, \\ Phryxeae peditem ferre iubentur aquae. \\ Quis novus hic dominus terramque diemque fretumque \\ Permutat? certe sub Iove mundus erat. \\ 
      \end{verse}
  
            \subsection*{240}
      \begin{verse}
      B. I 30. \\ M. 588. \\ Cupido amns \\ B. IV 345. \\ Quis me fervor agit? nova sunt suspiria menti. \\ Anne aliquis deus est nostro vehementior arcu? \\ Quem mihi germanum fato fraudante creavit \\ p. 125 \\ Diva parens?  \lbrack satis \rbrack  an  \lbrack mea \rbrack  spicula fusa per orbem \\ Vexavere polum laesusque in tempore mundus \\ Invenit poenam? sed si mea vulnera novi, \\ Hic meus est ignis; meus est, qui parcere nescit. \\ In furias ignesque trahor! licet orbe superno, \\ Iuppiter, exultes; undis, Neptune, tegaris; \\ Abdita poenarum te cingant Tartara, Pluton: \\ lnpositum rumpemus onus! volitabo per axem \\ Mundigerum caeliquc plagas pontique procellas \\ Vmbriferumque Chaos; pateant adamantina regna, \\ Torva venenatis cedat Bellona flagellis! \\ Poenam mundus amet, stupeat, vincatur, anhelet! \\ Instat saevus Amor fraudemque in vulnere quaerit! \\ 
        \pagebreak 
     \marginpar{[198]} \begin{center} \textbf{CARMINA} \end{center}
      \end{verse}
  
            \subsection*{241}
      \begin{verse}
      B. . \\ \poemtitle{De rore}B. IV 346. \\ Cumque serenifluo sudat nox humida caelo, \\ Mane rigent herbae vitreaque aspergine lucent \\ CGramina et in emmis stabiles tenet aura liquores \\ 
      \end{verse}
  
            \subsection*{22}
      \begin{verse}
      B. II 176. \\ M. 869 \\ Vnde supra \\ B. IV 183. \\ Temporibus laetis tristamur, maxime Caesar, \\ loc uno: amissum (quod gemo) Vergilium. \\ Et vetuit relegi, si tu patiere, libellos, \\ In quibus Aenean condidit ore sacro. \\ Roma rogat, precibusque isdem tibi supplicat orbis, \\ Ne pereant flammis tot monumenta ducum. \\ Anne iterum Troiam, sed maior, flamma cremabit? \\ Fac laudes Htalum, fac tua gesta legi, \\ Aeneidemque suam fac maior †Mincius ornet: \\ Plus fatis possunt Caesaris ora dei. \\ 
        \pagebreak 
    \begin{center} \textbf{CODICIS SALMASIANI.} \end{center} \marginpar{[199]} 
      \end{verse}
  
            \subsection*{243}
      \begin{verse}
      B. V 175. \\ M. 1105. \\ \poemtitle{De equis aeneis}B. IV 183. \\ Quae manus hos animavit equos? quis  \lbrack in \rbrack  aere rigenti \\ Currere velle dedit et in aethere quaerere cursus? \\ Spirant aerias involvere cursibus auras, \\ Arte citi sed mole graves, properante metallo. \\ 
      \end{verse}
  
            \subsection*{244}
      \begin{verse}
      B. I 175. \\ M. 1612. \\ Themas: Purne, in te suprema salus †’ \\ Turne, spes ltalum, custos fortissime regni, \\ Si properes! muros arces civesque Latinos \\ Non solitis urguet Troianus viribus hostis. \\ Sed potiora premunt. Quem nunc (ignosce) fugavit, \\ Nec pugnae inferior nec belli (crede) relictor \\ Adserar. armatus timendo fulmine miles \\ Nec culpam nec crimen habet; nam bella minatus \\ Nunc gerit Aeneas: reverendi fragor Olympi \\ Ex umbone tonat, caelestes concita flammas \\ Hasta iacit, oculis elatus ingerit ensis \\ Fulgura, progeniesque dei testata vigorem \\ Numinis antiqui spoliavit fulmine caelum \\ Et tulit arma Iovi. quod felix turba deorum \\ In saevos quondam potuit conferre Gigantes, \\ Aeneas nunc solus habet urbisque ruinam \\ 
        \pagebreak 
    \begin{center} \textbf{CARMINA} \end{center} \marginpar{[200]} Saevo Marte parat, ltalasque evertere turres \\ Non ariete gravi, non torti turbine saxi \\ Disponit, spernit belli tormenta magister; \\ Et memor, exitio Troiam sic esse sepultam, \\ Iam faces, non tela iaci. scit quippe tyrannus, \\ liacas sine face domus stetisse per aevum. \\ p. 12 \\ Heia age, curre, precor; te mater rexque Latinus, \\ Turba senum, lacrimans Lavinia virgo requirit. \\ 245 252 \\ 
      \end{verse}
  
            \subsection*{}
      \begin{verse}
      \poemtitle{TLORI}
      \end{verse}
  
            \subsection*{}
      \begin{verse}
      \poemtitle{De qulitate vitae}B. I 20. \\ M. 213. \\ 
      \end{verse}
  
            \subsection*{245}
      \begin{verse}
      B. IV 346 348. \\ Bacche, vitium repertor, plenus adsis vitibus, \\ Effluas dulcem liquorem conparandum nectari, \\ Conditumque fac vetustum, nec malignis venulis \\ Asperum ducat saporem versus usum in alterum. \\ B. III 114. \\ 
      \end{verse}
  
            \subsection*{246}
      \begin{verse}
      M. 214. \\ Omnis mulier intra pectus celat virus pestilens: \\ Dulce de labris locuntur, corde vivunt noxio. \\ 
        \pagebreak 
    \begin{center} \textbf{CODICIS SALMASIANI.} \end{center} \marginpar{[201]} B. I 17. \\ 
      \end{verse}
  
            \subsection*{24}
      \begin{verse}
      M. 215. \\ Sic Apollo, deinde Liber sic videtur ignifer: \\ Ambo sunt flammis creati prosatique ex ignibus; \\ Ambo de donis calorem, vite et radio, conferunt; \\ Noctis hic rumpit tenebras, hic tenebras pectoris. \\ B. II 265. \\ 
      \end{verse}
  
            \subsection*{248}
      \begin{verse}
      M. 216. \\ Quando ponebam novellas arbores mali et piri, \\ Cortici summae notavi nomen ardoris mei. \\ Nlla fuit exinde finis vel quies cupidinis: \\ Crescit arbor, gliscit ardor: animus implet litteras. \\ B. III 113. \\ 
      \end{verse}
  
            \subsection*{249}
      \begin{verse}
      M. 217. \\ Qui mali sunt, non fuere matris ab aluo mali, \\ Sed malos faciunt malorum falsa contubernia. \\ B. II 111. \\ 
      \end{verse}
  
            \subsection*{250}
      \begin{verse}
      M. 218. \\ ‘Sperne mores transmarinos, mille habent ofucias. \\ Cive lomano per orbem nemo vivit rectius. \\ Quippe malim unum Catonem quam trecentos Socratas.’ \\ Nemo non haec vera dicit: nemo non contra facit. \\ 
        \pagebreak 
    \begin{center} \textbf{CARMINA} \end{center} \marginpar{[202]} B. III I112. \\ 01 \\ M. 219. \\ Tam malum est habere nummos, non habere quam \\ malum est. \\ Tam malum est audere semper, quam malum est semper \\ pudor. \\ Tam malum est tacere multum, quam malum est multum \\ loqui. \\ Tammalumest forisamica, quam malum estuxordomi. \\ B. III 115. \\ 070 \\ M. 220. \\ Consules fiunt quotannis et novi proconsules: \\ Solus aut rex aut poeta non quotannis nascitur. \\ 
      \end{verse}
  
            \subsection*{}
      \begin{verse}
      \poemtitle{REPOSIANI}B. I 72. \\ M. 559. \\ Vs. De eoncubitu Martis et Veneris \\ Discite securos non umquam credere amores. \\ Ipsa Venus, cui flamma potens, cui militat ardor, \\ Quae tuto posset custode Cupidine amare, \\ Quae docet et fraudes et amorum furta tuetur, \\ Nec sibi securas valuit praebere latebras. \\ Ilnprobe dure puer, crudelis crimine matris, \\ Pompam ducis, Amor, nullo satiate triumpbo! \\ Quid conversa lovis laetaris fulmina semper? \\ Vt mage flammantes possis laudare sagittas, \\ Iunge puer teretes Veneris Martisque catenas, \\ Gestet amans Mavors titulos et vincula portet \\ Captivus, quem bella timent! utque ipse veharis, \\ Iam roseis fera colla iugis submittit amator; \\ 
        \pagebreak 
    \begin{center} \textbf{CODICIS SAMASIANI.} \end{center} \marginpar{[203]} Post vulnus, post bella potens Gradivus anhelat \\ ln castris modo tiro tuis, semperque timendus \\ Te timet et sequitur qua ducunt vincla marita. \\ Ite, precor, Musae; dum Mars, dum blanda Cythere \\ Imis ducta trahunt suspiria crebra medullis \\ Dumque intermixti captatur spiritus oris, \\ Carmine doctiloquo Vulcani vincla parate, \\ Quae Martem nectant Veneris nec brachia laedant \\ Inter delicias roseo prope livida serto. \\ Namque ferunt Paphien, Vulcani et Martis amorem, \\ Inter adulterium vel iusti iura mariti \\ Indice sub Phoebo captam gcssisse catenas. \\ p.29 \\ Illa manu duros nexus tulit, illa mariti \\ Ferrea vincla sui. quae vis fuit ista doloris? \\ An fortem faciebat amor? quid, saeve, laboras? \\ Cur nodos Veneri Cyclopia flamma paravit? \\ 
      \end{verse}
  
            \subsection*{30}
      \begin{verse}
      \poemtitle{De roseis conecte manus, Vulcane, catenis!}Nec tu deinde liges, sed blandus vincla Cupido, \\ Ne palmas duro †comodus vulnere laedat. \\ Lucus erat Marti gratus post vulnera Adonis \\ †Pictus amore deae, si Phoebi lumina desint \\ Tutus adulterio, dignus quem Cypris amaret, \\ Quem Byblis coleret, dignus quem Gratia †servet. \\ Vilia non illo surgebant gramina luco, \\ 
        \pagebreak 
     \marginpar{[204]} \begin{center} \textbf{CARMINA} \end{center}Pingunt purpureos candentia lilia flores, \\ Ornat terra nemus. nunc lucos vitis inumbrat, \\ Nunc laurus nunc nyrtus. labent sua munera rami: \\ Namque hic per frondes redolentia lilia pendent, \\ Hic rosa cum violis, hic omnis gratia florum, \\ Hic inter violas coma mollis laeta hyacintbi: \\ Dignus amore locus, cui tot sint munera rerum. \\ Non tamen in lucis aurum, non purpura fulget: \\ Flos lectus, flos vincla toris, substramina flores; \\ Deliciis Veneris dives natura laborat. \\ Texerat hic liquidos fontes non vilis arundo, \\ Sed qua saeva puer conponat tela Cupido. \\ Hunc solum Paphie puto lucum fecit amori, \\ Hic Martem expectare solet. quid Gratia cessat, \\ Quid Charites? cur, saeve puer, non lilia nectis? \\ Tu lectum consterne rosis! u serta parato \\ Et roseis crinem nodis subnecte decenter! \\ Haec modo purpureum decerpens pollice florem \\ Cum delibato suspiria ducat odore! \\ Ast tibi blanda manus  \lbrack flores \rbrack  sub pectore condat! \\ Tu, ne purpurei laedat te spina roseti, \\ Destrictis teneras foliis constringe papillas! \\ Sic decet in Veneris luco gaudere puellas, \\ Vt tamen inlaesos Paphiae servetis amores. \\ Vincula sic mixtis caute constringite ramis, \\ 
        \pagebreak 
    \begin{center} \textbf{CODICS SALMASANI. .} \end{center} \marginpar{[205]} Ne diffusa ferat per frondes lumina Titan! \\ His igitur lucis Papbie, dum praelia Mavors \\ †Horrida, dum populos diro terrore fatigat, \\ Ludebat teneris Bybli permixta puellis. \\ Nunc varios cantu divum referebat amores \\ Inque modnum vocis nunc motus forte decentes \\ Corpore laeta dabat, nunc miscens †denique plantas, \\ o Nunc alterna movens suspenso pollice crura, \\ Molliter inflexo subnitens poplite sidit. \\ Saepe comam pulcro collectam flore ligabat, \\ Ornans ambrosios divino pectine crines. \\ Dum ludos sic blanda Venus, dum gaudia miscet, \\ 6 Dum flet, quod sero veniat sibi grata voluptas, \\ Et dum suspenso solatia quaerit amori: \\ Ecce furens post bella deus, post praelia victor \\ Victus amore venit. cur gestas ferrea tela? \\ Ne metuat Cypris, comptum decet ire rosetis. \\ A, quotiens Paphie vultum mentita furentis \\ Lumine converso serum incusavit amantem! \\ p. 431 \\ Verbera saepe dolens minitata est dulcia serto, \\ Aut, ut forte magis succenso Marte placeret, \\ Admovit teneris suspendens oscula labris, \\ Nec totum effundens medio blanditur amore. \\ Decidit aut posita est devictis lancea palmis \\ Et dum forte cadit, myrto retinente pependit. \\ Ensem tolle, puer! galeam tu, Gratia, solve! \\ Solvite, Bybliades, praeduri pectora Maris! \\ 
        \pagebreak 
    \begin{center} \textbf{CARMNA} \end{center} \marginpar{[206]} Haec laxet nodos, haec ferrea vincula temptet \\ Loricaeque moras! vos scuta et tela tenete! \\ Nunc violas tractare decet. laetare, Cupido, \\ Terribilem divum tuo solo numine victum: \\ Pro telis lores, pro scuto myrtea serta, \\ Et rosa forte loco est gladii, quem iure tremescunt. \\ Iverat ad lectum Mavors et pondere duro \\ Floribus incumbens totum turbarat honorem. \\ bat pulcra Venus vix presso pollice cauta, \\ Florea ne teneras violarent spicula plantas, \\ Et nunc innectens, ne rumpant oscula, crinem, \\ Nunc vestes fluitare sinens vix laxa retentat, \\ Cum nec tota latet nec totum nudat amorem. \\ Ille inter flores furtivo lumine tectus \\ Spectat hians Venerem totoque ardore tremescit. \\ Incubuit lecis Paphie. pro sancte Cupido, \\ Quam blandas voces, quae tunc ibi murmura fundunt! \\ Oscula permixtis quae tunc fixere labellis! \\ Quam bene consertis haeserunt artubus artus! \\ Stringebat Paphiae Mavors tunc pectora dextra \\ Et collo innexam ne laedant pondera laevam, \\ Lilia cum roseis supponit candida sertis. \\ Saepe levi cruris tactu conmovit amantem \\ In flammas, quas diva movet. iam languida fessos \\ Forte quies Martis tandem conpresserat artus; \\ Non tamen omnis amor, non omnis pectore cessit \\ 
        \pagebreak 
    \begin{center} \textbf{CODICIS SALMASIANI.} \end{center} \marginpar{[207]} Flamma dei, trahit in medio suspiria somno \\ Et Venerem totis pulmonibus ardor anhelat. \\ Ipsa Venus tunc tunc calidis succensa venenis \\ Vritur ardescens, nec somnia parta quieta. \\ O quam blanda quies! o quam bene presserat artus \\ Nudos forte sopor! niveis suffulta lacertis \\ Colla nitent, pectus gemino quasi sidere turget. \\ Non omnis resupina iacet, sed corpore flexo \\ Molliter et laterum qua se confinia iungunt. \\ Martem respiciens deponit lumina somno, \\ Sed gratiosa, decens. Pro lucis forte Cupido \\ Martis tela gerit; quae postquam singula  \lbrack sumpsit, \\ Loricam clypeum gladium galeaeque minacis \\ Cristas, flore ligat. tunc hastae pondera temptat \\ Miraturque suis tantum licuisse sagittis. \\ Ilam medium Phoebus radiis possederat orbem, \\ Iam tumidis calidum spatium libraverat horis. \\ Flammantes retinebat equos. pro conscia facti \\ Invida luxl Veneris qui nunc produntur amores \\ Lumine, Phoebe, tuo! stant capti iudice tanto \\ Mars Amor et Paphie, ramisque inserta tremescunt \\ Lumina, nec crimen possunt te teste negare. \\ Viderat effusis Gradivum Pboebus habenis \\ 
      \end{verse}
  
            \subsection*{}
      \begin{verse}
      \poemtitle{. 133}In gremio Paphiae spirantem incendia amoris. \\ O rerum male tuta fides! o gaudia et ipsis \\ Vix secura deis! quis non, cum Cypris amaret, \\ 
        \pagebreak 
    \begin{center} \textbf{CARMNA} \end{center} \marginpar{[208]} Praeside sub tanto tutum speraret amorem? \\ Criminis exemplum si iam de numine habemus, \\ Quid speret mortalis amor? qua vota ferenda? \\ Quod numen poscat, quo sit securus, adulter? \\ Cypris amat, nec tuta tamen. coupressit habenas \\ Phoebus et ad lucos tantummodo lumina vertit \\ Et sic pauca refert: ‘nunc spargis tela, Cupido, \\ Nunc nunc, diva Venus, nati devicta sagittis \\ Das mihi solamen; sub te securus amavi. \\ Fabula, non crimen, nostri dicentur amores’ \\ Haec ait et dictis Vulcanum instigat amaris: \\ ‘Dic ubi sit Cytherea decens, secure marite! \\ Te expectat lacrimans, tibi castum servat amorem? \\ Vel si forte tuae Veneris fera crimina nescis, \\ Quaere simul Martem, cui tu modo tela parasti.’ \\ Dixit et infuso radiabat lumine lucos \\ lnque fidem sceleris totos demiserat ignes. \\ Haeserat lgnipotens stupefactus crimine tanto. \\ Iam quasi torpescens (vix sufficit ira dolori) \\ Ore fremit maestus \lbrack que \rbrack  dolum gemit, ultima pulsans \\ Ilia et indignans suspiria pressa fatigat. \\ Antra furens Aetnaea petit. vix iusserat: omnes \\ Incubuere mnus, multum dolor addidit arti. \\ Quam cito cuncta gerunt ars numen flamma maritus \\ Ira dolor! nam vix causam tunc forte iubendo \\ Dixerat, et vindex coniunx iam vincla ferebat. \\ Pervenit ad lucos, non ipsi visus Amori, \\ 
        \pagebreak 
    \begin{center} \textbf{CODICIS SALMASIANI.} \end{center} \marginpar{[209]} Non Chariti: totas arti mandaverat iras. \\ Vincula tunc manibus suspenso molliter ictu \\ Inligat et teneris conectit brachia palmis. \\ Excutitur somno Mavors et pulcra Cythere. \\ Posset Gradivus validos disrumpere nexus, \\ Sed retinebat amor, Veneris ne brachia laedat. \\ Tunc tu sub galea, tunc inter tela latebas, \\ Saeve Cupido, timens. stat Mavors lumine torvo \\ Atque indignatur, quod sit deprensus adulter. \\ At Paphie conversa dolet non crimina facti, \\ Sed quae sit vindicta sibi. dum singula volvens \\ Cogitat, hanc poenam sentit, si Phboebus amaret. \\ Iamque dolos properans decorabat cornua tauri, \\ Passiphaae crimen mixtique cupidinis iram. \\ 
      \end{verse}
  
            \subsection*{254}
      \begin{verse}
      \poemtitle{PLAVII ELICIS}B. d Lux. 6. \\ M. 95. \\ viri clarissimi \\ B. IV 356. \\ Postulatio honoris aputc Victorinianum virum \\ nlustrem et primiscrinirium \\ Snt versus XXXII \\  \lbrack XXI \rbrack  Aspera dum quaterent humanas praelia mentes \\ Aut raperet vastum nigra procella fretum, \\ 
        \pagebreak 
    \begin{center} \textbf{CARMINA} \end{center} \marginpar{[210]} Cum dubiis fortuna suis penderet habenis \\ Torqueretque vagus stolida corda metus, \\ Anxia Pboebeis sanabant pectora templis \\ Et sacros tripodas nosse salubre fuit. \\ p. 135 \\ Nunc etiam saevis agitat quos cura periclis \\ Atque inopi vexat dura labore fames, \\ Altera Parnasi currunt ad numina montis \\ Castalioque lacu viscera maesta fovent. \\ Te nudi trlstesque rogant, tibi flebile plangit \\ Auxilium poscens paupera turba tuum. \\ Tu mihi numen eis, Phoebeo munere plenus \\ Qui potes infirmos morte levare manu. \\ Erige languetem, morbos depelle meroris \\ Et miserum melior factus Apollo iuva, \\ Quaeque meos domus est proavos miserata patremque, \\ laec eadem natis praemia nota ferat. \\ Non ego litigeros cupio cognoscere fasces, \\ Nec mihi reddantur iura superba precor; \\ Triste forum nolo, vexant quod praelia pacis \\ Fraternisque odiis alea caeca subit;r \\ Confictus audire piget rixasque togarum, \\ Quas inter fictus concrepat ore furor, \\ Iustitiam calcat seductus faenore praedae \\ Et, quas defendit, partibus arma parat; \\ Nec mihi qui pereant ulli poscuntur honores \\ Aut avidus pompae maxima lucra peto: \\ Sit mihi fas audire sacros et cernere cultus, \\ Ecclesiae spectans dona venire mea. \\ 
        \pagebreak 
    \begin{center} \textbf{CODICIS SALMASANM.} \end{center}\begin{center} \textbf{0 1 1} \end{center}Sic tibi Phoenicis transcendere congruat annos \\ Incolumique aevum coniuge laetus agas, \\ Sic thalamis prolem socies videasque nepotes \\ Prudentis gremio ludere semper avi, \\ Inclita sic celsi praevectus fata parentis \\ p. 136 \\ Exsuperes meritis saeclaque longa geras; \\ Sic priscos vincas atavos clarosque parentes \\ Et placido regi nuntius orsa feras: \\ Adnue poscenti, miserum sustolle ruinae; \\ Clericus ut fiam, dum velis ipse, potes. \\ 
      \end{verse}
  
            \subsection*{255}
      \begin{verse}
      Thema Vergilianum \\ B. I 171. \\ M. 1611. \\ ‘Nee tibi, diva parens’ \\ B. IV 1e5. \\ Dedecus o iuvenum turpisque infamia Teucrum, \\ Qui segnis per bella lates, †gens perfida et amens, \\ Non virtute potens, non belli maximus auctor, \\ gnavus tu semper eris semperque fuisti, \\ Naufragus atque miser segnisque in praelia ductor! \\ Nec non est aliud, quod maius crimen obibis. \\ lamque tuo generi quia sempcr perfidus extas, \\ Non equidem miror. non est ex tempore natum; \\ Antiquos imitaris avos, periuria patrum. \\ Nec non haut Veneris pulcra de stirpe crearis \\ Nec pater Anchises vestrae  \lbrack est \rbrack  aut Dardanus auctor \\ Gentis, sed durae tigres lapidesque sinistri \\ 
        \pagebreak 
    \begin{center} \textbf{CARMINA} \end{center}\begin{center} \textbf{0 M 0} \end{center}Te genuere virum, silvae montesque profani. \\ Vberaque tibi et potum admovere malignum, \\ Quae tibi perfidiam mixto cum lacte dederunt. \\ 
      \end{verse}
  
            \subsection*{256}
      \begin{verse}
      B. II 68. \\ M. 88. \\ r r \\ B. IV 156. \\ Nocte pluit tota, redeunt at mane serena. \\ Commune imperium cum love, Caesar, agis. \\ 
      \end{verse}
  
            \subsection*{257}
      \begin{verse}
      0. \\ B. II 69. \\ M. 88. \\ \poemtitle{EIVSDEM}B. IV 156. \\ Hos ego versiculos feci. tulit alter honorem. \\ Sic vos non vobis mellificatis apes. \\ 
        \pagebreak 
    \begin{center} \textbf{CODICIS SALMASMANI.} \end{center} \marginpar{[01]} 
      \end{verse}
  
            \subsection*{258}
      \begin{verse}
      B. III 20. \\ M. 96. \\ \poemtitle{EIVSDEM}B. IV 157. \\ Pars uibit, Nise, datur Baccbi, pars deinde negatur: \\ p.13 \\ Esse potes liber, non potes esse pater. \\ 
      \end{verse}
  
            \subsection*{259}
      \begin{verse}
      B. III 142. \\ M. 91. \\ \poemtitle{EIVSDEM}B. IV 157. \\ Arretine calix, mensis decor ante paternis, \\ Ante manus medici quam bene sanus eras! \\ 
      \end{verse}
  
            \subsection*{260}
      \begin{verse}
      B. III 148. \\ M. 99. \\ \poemtitle{EIVSDEM}B. ib. \\ Humor alit segetem; segeti contrarius humor. \\ Quod iuvat, \lbrack hoc \rbrack  dulce est; quod cogitur, alteramors est. \\ 
      \end{verse}
  
            \subsection*{261}
      \begin{verse}
      . \\ M. 97. \\ \poemtitle{EIVSDEM} \lbrack B. b. \\ Monte sub hoc lapidum premitur Ballista sepultus: \\ Nocte die tutum carpe viator iter. \\ 
        \pagebreak 
    \begin{center} \textbf{CARMINA} \end{center} \marginpar{[214]} 
      \end{verse}
  
            \subsection*{262}
      \begin{verse}
      B . \\ . \\ \poemtitle{EIVSDEM}B. IV 157. \\ Si quotiens peccant homines, sua fulmina mittat yi. \\ Trist. \\ Iuppiter, exiguo tempore inermis erit. \\ II 33sq. \\ 
      \end{verse}
  
            \subsection*{263}
      \begin{verse}
      B. III 268. \\ M. 95. \\ \poemtitle{EIVSDEM}B. IV 158. \\ Dum dubitat natura, marem faceretne puellam, \\ Natus es, o pulcher, paene puella, puer. \\ 
      \end{verse}
  
            \subsection*{264}
      \begin{verse}
      \poemtitle{SEXTI PROPERTI}. \\ \poemtitle{De Vergilio}B. IV 158. \\ Propert.CediteRomaniscriptores,cediteGrai: \\ III 32, 65sq. \\ Nescio quid maius nascitur Iliade. \\ 
      \end{verse}
  
            \subsection*{265}
      \begin{verse}
      \poemtitle{PENTADI}B. I 139. \\ M. 242. \\ \poemtitle{ \lbrack De Narcisso \rbrack }B. IV 358. \\ Cui pater amnis erat, fontes puer ille colebat \\ Laudabatque undas, cui pater amnis erat. \\ Se puer ipse videt, patrem dum quaerit in amne, \\ Perspicnoque lacu se puer ipse videt. \\ Quod Dryas igne calet, puer hunc inridet amorem \\ Nec putat esse decus, quod Dryas igne calet. \\ 
        \pagebreak 
    \begin{center} \textbf{CODICIS SALMASIANI.} \end{center} \marginpar{[01]} Stat stupet haeret amat rogat innuit aspicit ardet \\ Blanditur queritur stat stupet haeret amat. \\ Quodque amat, ipse facit vultu prece lumine fletu; \\ Oscula dat fonti, quodque amat ipse facit. \\ p.138 \\ 
      \end{verse}
  
            \subsection*{266}
      \begin{verse}
      \poemtitle{EIVSDEM}B. I 141. \\ M. 244. \\  \lbrack Aliter \rbrack  \\ B. IV 358. \\ Hic est ille, suis nimium qui credidit undis, \\ Narcissus vero dignus amore puer. \\ Cernis ab inriguo repetentem gramine ripas, \\ Vt per quas periit crescere possit aquas. \\ 
      \end{verse}
  
            \subsection*{267}
      \begin{verse}
      B. I 165. \\ M. 248. \\ \poemtitle{EIVSDEM}B. IV 358. \\ Chrysocome gladium fugiens stringente marito \\ Texit adulterium iudice casta reo. \\ 
      \end{verse}
  
            \subsection*{268}
      \begin{verse}
      B. III 83. \\ M. 245. \\ \poemtitle{EIVSDEM}B. IV 359. \\ Crede ratem ventis, anuimum ne crede puellis; \\ Namque est feminea tutior unda lide. \\ Femina nulla bona  \lbrack est \rbrack , vel, si bona contigit una, \\ Nescio quo fato res mala facta bona est. \\ 
        \pagebreak 
    \begin{center} \textbf{CARMINA} \end{center}\begin{center} \textbf{0 2} \end{center}
      \end{verse}
  
            \subsection*{269}
      \begin{verse}
      \poemtitle{OVIDI}\poemtitle{De aetate}B. M. B \\ Ouid.Vtendumestaetate;citopedelabituraetas, \\ Necbonatamsequior quam boa prima fuit.... †,;; \\ III 65sq. \\ le me nunc miserum! laxatur corpora rugis \\ Et perit, in nitido qui fuit ore, color. \\ 
      \end{verse}
  
            \subsection*{270}
      \begin{verse}
      B. III 38. \\ M. 537. \\ \poemtitle{REGIANI}B. IV 359. \\ Quis deus has incendit aquas? quis fontibus ignes \\ Miscuit et madidas fecit decurrere flammas? \\ In regnis, Neptune, tuis Vulcanus anhelat! \\ r \\ B. III 28. \\ M 536. \\ 
      \end{verse}
  
            \subsection*{}
      \begin{verse}
      \poemtitle{EIVSDEM}B. IV 359. \\ Ante bonam Venerem gelidae per litora Baiae \\ Illa natare lacus cum lampade iussit Amorem. \\ Dum natat, algentes cecidit scintilla per undas; \\ Iinc vapor ussit aquas: quicumque natavit, amavit. \\ 
      \end{verse}
  
            \subsection*{272}
      \begin{verse}
      B. III 53. \\ M. 538. \\ \poemtitle{EIVSDEM}B. IV 359. \\ Bellipotens Mavors, Veneris gratissime furto, \\ p.139 \\ lic securus ama. locus hic amplexibus aptus. \\ Vulcanus prohibetur aquis, Sol pellitur umbra. \\ 
        \pagebreak 
    \begin{center} \textbf{q} \end{center}\begin{center} \textbf{CODICIS SALMASIANI.} \end{center}
      \end{verse}
  
            \subsection*{273}
      \begin{verse}
      B. I 31. \\ M. 231. \\ \poemtitle{MODESTINI}B. IV 360. \\ Forte iacebat Amor victus puer alite somuo \\ Myrti inter frutices pallentis roris in herba. \\ Hunc procul emissae tenebrosa Ditis ab aula \\ Circueunt animae, saeva face quas cruciarat. \\ ‘Ecce meus venator’, ait ‘bunc’ Phaedra ‘ligemus!’ \\ Crudelis ‘crinem’ clamabat Scylla ‘metamus!’ \\ Colchis et orba Progne numerosa caede: ‘necemus!’ \\ Dido et Canace: ‘saevo gladio perimamus!’ \\ Myrrha: ‘meis ramis’, Euhadne: ‘igne crememus!’ \\ ‘Hlunc’ Arethusa ‘in aquis’, Byblis ‘in fonte necemus!’ \\ Ast Amor evigilans dixit: ‘mea pinna, volemus.’ \\ B. II 71. \\ 4 \\ M. 539. \\ 
      \end{verse}
  
            \subsection*{}
      \begin{verse}
      \poemtitle{PONNANI}B. IV 360. \\ Picta fuit quondam Pharii regina Canopi \\ Artificis formata manu. nam vivere serpens \\ Creditur et morsu gaudens dare fata papillae. \\ O quam vivit opus, quam paene figura dolorem \\ Sentit et ex ipso moritur pictura veneno! \\ B III 261. \\ 
      \end{verse}
  
            \subsection*{}
      \begin{verse}
      \poemtitle{MATIALIST}M. B. \\ Qualem, lacce, velim quaeris nolimve puellam? Maril. \\ 
      \end{verse}
  
            \subsection*{}
      \begin{verse}
      \poemtitle{I 57}Nolo nimis facilem difficilemque nimis. \\ 
        \pagebreak 
    \begin{center} \textbf{CARMNA} \end{center} \marginpar{[218]} llud quod medium est atque inter utrumque, probatur: \\ Nec volo quod satiet nec volo quod cruciet. \\ 
      \end{verse}
  
            \subsection*{276}
      \begin{verse}
      B \\ . 200. \\ \poemtitle{EIVSDEM}B. IV 117. \\ Apud \\ Nec volo me summis ortuna neque adplicet imis, nial \\ Sed medium vitae temperet illa gradum. non ertat \\ Invidia excelsos, inopes iniuria vexat: \\ p. 140 \\ Quam felix vivit, quisquis utraque caret! \\ 7 4 4 \\ B. III 281. \\ M. 545. \\ ru \\ 
      \end{verse}
  
            \subsection*{}
      \begin{verse}
      \poemtitle{TVCCIANI}B. IV 360. \\ Cantica gignit amor et amorem cantica gignunt. \\ Cantandum est, ut ametur, et ut cantetur, amandum. \\ 
      \end{verse}
  
            \subsection*{278}
      \begin{verse}
      B. III 22. \\ M. 546. \\ \poemtitle{EIVSDEM}B. IV 361. \\ Pallas tota Iovis patrio de vertice nata \\ In caelum cecidit . \\ 
      \end{verse}
  
            \subsection*{279}
      \begin{verse}
      \poemtitle{VINCENTI}B. I 154. \\ M. 548. \\ Phaedra \\ B. IV 361. \\ Victe pudor, conpone preces! secreta duorum \\ Tertius ignoret; sit tantum littera testis \\ 
        \pagebreak 
    \begin{center} \textbf{CODICIS SALMASIAN1.} \end{center} \marginpar{[219]} Inmensum locutura nefas. metus unicus urget, \\ Ne speret iuvenis lacrimas finxisse novercam. \\ Hinc tamen incipiam. caeli regnator amavit \\ Et tbalamos tenet una soror; nec crimine dignum est, \\ Quod mundus cum lege facit. Telluris alumnus \\ Vnus in orbe fuit: nulla est cognatio plebis? \\ Venimus ex uno mixti patre, matre, sorore. \\ Ne metuas facinus! ‘on vult hoc scire Cupido.’ \\ Pignus habes tabulas: rea sum semper, rea sola. \\ Sed si virgineum suffundunt ora ruborem, \\ Vel negature veni. liceat sperare rogantem \\ Et fastus tolerare tuos. sequar ora superbi, \\ Oscula suspensis defigam singula plantis. \\ Me cruciet captivus amor, nemus omne resultet \\ HIippolytum, secumque ferat vox ultima nomen. \\ Testor amoris onus Venerisque insignia nostrae: \\ Dum volo, dum nolo, dum vincor scribo repugno, \\ Suspensos collo laqueos rogatura resolvi. \\ p. 141 \\ 
      \end{verse}
  
            \subsection*{280}
      \begin{verse}
      \poemtitle{BONOSI}B . . \\ Vnde supra \\ B. IV 362. \\ Incestum si promiseris, petii; \\ Si negaveris, tu petisti. \\ 
        \pagebreak 
    \begin{center} \textbf{CARMNA} \end{center} \marginpar{[220]} 
      \end{verse}
  
            \subsection*{281}
      \begin{verse}
      B. III 156. \\ M. 921. \\ \poemtitle{De unambulo}B. IV 362. \\ Vidi hominem pendere cum via, \\ Cui latior era planta quam semita. \\ 
      \end{verse}
  
            \subsection*{282}
      \begin{verse}
      , v 7s. De ursa aenea, ln qua serpens uit, \\ M. 1196. \\ FS. ubi inscius puer manum misit \\ Aere cavo falsam serpens impleverat ursam, \\ Addidit et morsum et iubet esse feram \\ lnplevit serpens quod minus artis erat. \\ 
      \end{verse}
  
            \subsection*{283}
      \begin{verse}
      B. V 183. \\ \poemtitle{De pectine}B. IV 362. \\ Crinibus ambrosiis Veneris decus addidit, ut se \\ Pulchras iactarent Pallas vel pronuba Iuno. . . \\ 
      \end{verse}
  
            \subsection*{284}
      \begin{verse}
      B V 181. \\ M. 1111. \\ \poemtitle{De antlia}B. IV 363. \\ Fundit et haurit aquas, pendentes evomit undas, \\ Et fluvium vomiura bibit. mirabile factum! \\ 
        \pagebreak 
    \begin{center} \textbf{CODCIS SALMASIANI.} \end{center} \marginpar{[001]} Portat aquas, portatur aquis. sic unda per undas \\ Volvitur et veteres haurit nova machina lymphas. \\ 
      \end{verse}
  
            \subsection*{285}
      \begin{verse}
      B. V 161. \\ M. 1092. \\ \poemtitle{De convlviis barbaris}B. IV 364. \\ Inter ‘eils’ goticum ‘scapia matia ia drincan’ \\ Non audet quisquam dignos edicere versus. \\ 
      \end{verse}
  
            \subsection*{285a}
      \begin{verse}
       \lbrack Item \rbrack  \\ Calliope madido trepidat se iungere Baccho, \\ Ne pedibus non stet ebria Musa suis. \\ 
      \end{verse}
  
            \subsection*{286}
      \begin{verse}
      \poemtitle{SYMPIOSII}scholastici \\ . M \\ A e n i g m a t a \\ B. IV 365 385. \\ Praefatio \\  \lbrack XXII) llaec quoque Symphosius de carmine lusit inepto \\ (Sic tu, Sexte, doces; sic te deliro magistro), \\ 
        \pagebreak 
    \begin{center} \textbf{CARMINA} \end{center} \marginpar{[222]} Annua Saturni dum tempora festa redirent \\ Perpetuo semper nobis sollemnia ludo. \\ Post epulas laetas, post dulcia pocula mensae \\ Deliras inter vetulas puerosque loquaces, \\ Cum streperet late madidae facundiai linguae, \\ 
      \end{verse}
  
            \subsection*{}
      \begin{verse}
      \poemtitle{. 142}Tum verbosa cohors studio sermonis inepti \\ 
        \pagebreak 
    \begin{center} \textbf{CODICIS SALMASIANI.} \end{center} \marginpar{[002]} Nescio quas passim magno de nomine nugas \\ Est meditata diu; sed frivola multa locuta est. \\ Nec mediocre fuit, magni certaminis instar, \\ Ponere diverse vel solvere quaeque vicissim. \\ Ast ego, ne solus foede tacuisse viderer, \\ Qui nihil adtulerim mecum, quod dicere possem, \\ Hos versus feci subito †de carmine vocis. \\ Insanos inter sanum non esse necesse est. \\ Da veniam, lector, quod non sapit ebria Musa. \\ I Graphium \\ De summo planus, sed non ego planus in imo \\ Versor utrimque manu. diverso munere fungor. \\ Altera pars revocat, quidquid pars altera fecit. \\ II Harundo \\ Dulcis amica dei, ripae vicina profundae, \\ Suave canens Musis, nigro perfusa colore, \\ Nuntia sum linguae digitis signata magistris. \\ 
        \pagebreak 
    \begin{center} \textbf{CARMINA} \end{center} \marginpar{[004]} III Anulus cum gemma \\ Corporis extremi non magnum pondus adhaesi. \\ Igenitum dicas; ita pondere nemo gravatur. \\ Vna tamen facies plures habet ore figuras. \\ IIII Clavis \\ Virtutes magnas de viribus affero parvis. \\ Pando domos clausas, iterum sed claudo patentes. \\ Servo domum domino, sed rursus servor ab ipso. \\ V Catena \\ Nexa ligor ferro, multos habitura ligatos. \\ Vincior ipsa prius, sed vincio vincta vicissim. \\ Et solvi multos, nec sum tamen ipsa soluta. \\ VI Tegula \\ Terra mihi corpus, vires mihi praestitit ignis; \\ De terra nascor; sedes est semper in alto; \\ Et me perfundit, qui me cito deserit, umor. \\ 
        \pagebreak 
    \begin{center} \textbf{CODICIS SALMASIANI.} \end{center} \marginpar{[225]} VII Fumus \\ Sunt mihi, sunt lacrimae, sed non est causa doloris. \\ Est iter ad caelum, sed me gravis inpedit aer, \\ Et qui me genuit, sine me non nascitur ipse. \\ VIII Neula \\ Nox ego sum facie, sed non snm nigra colore \\ Inque die media tenebras tamen affero mecum. \\ Nec mihi dant stellae lucem nec Cynthia lumen. \\ VIIII Pluvia \\ Ex alto venio, longa delapsa ruina. \\ De caelo cecidi medias transmissa per auras: \\ Sed sinus excepit, qui me simul ipse recepit. \\ X Glacies \\ Vnda fui quondam, quod me cito credo futuram. \\ Nunc rigidi caeli duris conexa catenis \\ Nec calcata pati possum nec nuda teneri. \\ XI Nix \\ Pulvis aquae tenuis modico cum pondere lapsus, \\ Sole madens, aestate fluens, in frigore siccus, \\ Flumina facturus totas prius occupo terras. \\ 
        \pagebreak 
     \marginpar{[226]} \begin{center} \textbf{CARMINA} \end{center}XII Tlumen et piscis \\ p. 144 \\ Est domus in terris, clara quae voce resultat. \\ lpsa domus resonat, tacitus sed non sonat hospes. \\ Ambo tamen currunt, hospes simul et domus una. \\ XII Navis \\ Vrr \\ Longa feror velox formosae filia silvae, \\ Innumeris pariter comitum stipata catervis. \\ Curro vias multas, vestigia nulla relinquens. \\ XIIII Pullus in ovo \\ Mira tibi referam nostrae primordia vitae. \\ Nondum natus eram nec eram iam matris in alvo. \\ Iam posito partu natum me nemo videbat. \\ XV Vipera \\ Non possum nasci, si non occidero matrem. \\ Occidi matrem, sed me manet exitus idem. \\ Id mea mors patitur, quod iam mea fecit origo. \\ XVI Tine \\ Littera me pavit, nec quid sit littera novi. \\ In libris vixi nec sum studiosior inde. \\ Exedi Musas nec adhuc tamen ipsa profeci. \\ 
        \pagebreak 
    \begin{center} \textbf{CODICIS SALMASIANI.} \end{center} \marginpar{[227]} XVII Arane \\ Pallas me docuit texendi nosse laborem. \\ Nec pepli radios poscunt nec licia telae. \\ Nulla mihi manus est, pedibus tamen omnia fiunt. \\ XVIII Coclea \\ Porto domum mecum, semper migrare parata, \\ Mutatoque solo non sum miserabilis exul, \\ Sed mihi concilium de caelo nascitur ipso. \\ \poemtitle{XVIII Rana}aucisonans ego sum media vocalis in unda, \\ Sed vox laude sonat, †quasi se quoque laudet et ipsa. \\ Cumque canam semper, nullus mea carmina laudat. \\ XX Testudo \\ Tarda, gradu lento, specioso praedita dorso; \\ Docta quidem studio, sed saevo prodita fato \\ Viva nihil dixi; quae sic modo mortua canto. \\ \poemtitle{XXI Tpa}Caeca mibi facies atris obscura tenebris. \\ Nox est ipse dies nec sol mihi cernitur ullus. \\ Malo tegi terra: sic me quoque nemo videbit. \\ 
        \pagebreak 
     \marginpar{[228]} \begin{center} \textbf{CARMINA} \end{center}XXII Formica \\ Providasumvitae,durononpigralabori, \\ psa ferens umeris securae praemia brumae. \\ Nec gero magna simul, sed congero multa vicissim. \\ \poemtitle{XXIII Musca}V \\ lnproba sum, fateor. quid enim gula turpe veretur? \\ Frigora vitabam, quae nunc aestate revertor; \\ Sed cito submoveor falso conterrita vento. \\ \poemtitle{XXIIII Crclio}Nou bonus agricolis, non frugibus utilis hospes, \\ Non magnus forma, non recto nomine dictus, \\ Non gratus Cereri non parvam sumo saginam. \\ r \\ \poemtitle{XXV Mus}Parva mihi domus est, sed ianua semper aperta. \\ Exiguo sumptu furtiva vivo sagina. \\ Quod mihi nomen inest, omae quoque consul habebat. \\ XXVI 6rus \\ Littera sum caeli pinna perscripta volanti, \\ Bella cruenta gerens volucri discrimine Martis. \\ Nec vereor pugnas, dum non sit longior hostis. \\ 
        \pagebreak 
     \marginpar{[229]} \begin{center} \textbf{CODICIS SALMASIANI.} \end{center}\poemtitle{XXVII Corix}V c r \\ Vivo novem vitas, si me non Graecia fallit, \\ Atraque sum semper nullo conpulsa dolore, \\ Et non irascens ultro convitia dico. \\ XXVIII Vespertilio \\ V \\ Nox mihi dat nomen primo de tempore noctis. \\ Pluma mihi non est, cum sit mibi pinna volantis; \\ Sed redeo in tenebris nec me committo diebus. \\ \poemtitle{XXVIIII Eicius}rr \\ Plena domus spinis; parvi sed corporis hospes \\ Incolumi dorso telis conpletus acutis \\ Sustinet armatas sedes, habitator inermis. \\ XXX Peduculus \\ Est nova nostrarum cunctis captura ferarum, \\ Vt si quid capias, id tu tibi ferre recuses, \\ Et quod non capias, tecum tamen ipse reportes. \\ 
        \pagebreak 
     \marginpar{[230]} \begin{center} \textbf{CARMNA} \end{center}XXXI Pboenix \\ Vita mihi mors est; morior si coepero nasci. \\ Sed prius est fatum leti, quam lucis origo. \\ Sic solus Manes ipsos mihi dico parentes. \\ \poemtitle{XXXII Taus}Moechus eram regis, sed lignea membra sequebar. \\ Et Cilicum mons sum, sed mons sum nomine solo. \\ vehor in caelis et in ipsis ambulo terris. \\ \poemtitle{XXXIII Lpus}Dentibus insanis ego sum, qui vinco bidentes, \\ Sanguineas praedas quaerens victusque cruentos. \\ Multa cum rabie vocem quoque tollere possum. \\ \poemtitle{XXIIII Vulpes}Exiguum corpus, sed cor mihi corpore maius. \\ Sum versuta dolis, arguto callida sensu, \\ Et fera sum sapiens, sapiens fera si qua vocatur. \\ \poemtitle{XXXV Cpra}Alma Iovis nutrix, longo vestita capillo, \\ Culmina difficili peragrans super ardua gressu, \\ Custodi pecoris tremula respondeo lingua. \\ 
        \pagebreak 
    \begin{center} \textbf{CODICIS SALMASIANI.} \end{center} \marginpar{[0221]} \poemtitle{XXXVI Porcs}Setigerae matris fecunda natus in alvo \\ Desuper ex alto virides expecto saginas, \\ Nomine numen habens, si littera prima periret. \\ 
      \end{verse}
  
            \subsection*{}
      \begin{verse}
      \poemtitle{XXXVII Ml}Dissimilis patri, matri diversa figura \\ Confusi generis; generi non apta propago. \\ Ex aliis nascor nec quisquam nascitur ex me. \\ \poemtitle{XXXVIII ris}A fluvio dicor, fluvius vel dicitur ex me. \\ Iunctaque sum vento, quae sum velocior ipso. \\ Et mihi dat ventus natos nec quaero maritum. \\ XXXVIIII Centaurus \\ Quattuor insignis pedibus manibusque duabus \\ Dissimilis mihi sum, quia sum non unus et unus. \\ Et vehor et gradior, qua me mea corpora portant. \\ XL Papaver \\ Grande mihi caput est, intus sunt membra minuta. \\ Pes unus solus, sed pes longissimus unus. \\ Et me somnus amat, proprio nec dormio somno. \\ 
        \pagebreak 
     \marginpar{[232]} \begin{center} \textbf{CARMINA} \end{center}XLI Matva \\ Anseris esse pedes similes mihi, nolo negare. \\ Nec duo sunt tantum, sed plures ordine cernis; \\ Et tamen hos ipsos omnes ego porto supinos. \\ XLII 3et \\ Tota vocor graece, sed non sum tota latine. \\ A nte tamen mediam cauponis scripta tabernam \\ In terra nascor, lympha lavor, unguor olivo. \\ XLHII Cucurbita \\ Pendeo, dum nascor; rursus, dum pendeo, †nascor. \\ Pendens commoveor ventis et nutrior undis. \\ Pendula si non sim, non sum iam iamque futura. \\ \poemtitle{XLIIII Cepa}Mordeo mordentes, ultro non mordeo quemquam; \\ Sed sunt mordentem multi mordere parati. \\ Nemo timet morsum, dentes quia non habet ullos. \\ XLV Rosa \\ Purpura sum terrae, pulcro perfusa rubore, \\ Saeptaque, ne violer, telis defendor acutis. \\ 0 felix,longo si possim vivere fato! \\ 
        \pagebreak 
    \begin{center} \textbf{CODICIS SALMASIANI.} \end{center} \marginpar{[233]} \poemtitle{XLVI Vo}Magna quidem non sum, sed inest mihi maxima virtus. \\ Spiritus est magnus, quamvis sim corpore parvo. \\ Nec mihi germen habet noxam nec culpa ruborem. \\ cr r \\ \poemtitle{XVII Tu}Dulcis odor nemoris flamma fumoque fatiger, \\ Et placet hoc superis, medios quod mittor in ignes: \\ Nec mihi poena datur, sed habetur gratia \\ dandi. \\ p. 19 \\ XLVIII Suecinum \\ De lacrimis et pro lacrimis mea coepit origo. \\ Ex oculis fluxi, sed nunc ex arbore nascor. \\ Laetus honor frondis, tristis sed imago doloris. \\ r r rrr \\ \poemtitle{XLVIIII Ebur}Dens ego sum magnus populis cognatus Eois. \\ Nunc ego per partes in corpora multa recessi. \\ Nec remanent vires, sed formae gratia mansit. \\ 
        \pagebreak 
    \begin{center} \textbf{CARMINA} \end{center} \marginpar{[234]}  \marginpar{[02]} Fenum \\ Herba fui quondam viridi de gramine terrae; \\ Sed chalybis duro mollis praecisa metallo \\ Mole premor propria, tecto conclusa sub alto. \\ LI Mol \\ Ambo sumus lapides, una sumus, ambo iacemus. \\ Quam piger est unus, tantum non est piger alter: \\ Hic manet inmotus, non desinit ille moveri. \\ LII Farina \\ Inter saxa fui, quae me contrita premebant; \\ Vix tamen effugi totis conlisa medullis. \\ Et iam forma mihi minor est, sed copia maior. \\ III Viis \\ Nolo toro iungi, quamvis placet esse maritam. \\ Nolo virum thalamo: per me mea nata propago est. \\ Nolo sepulcra pati; scio me submergere terrae. \\ \poemtitle{LIIII Amus}Exiguum munus flexu mucronis adunci \\ Fallaces escas medio circumfero fluctu. \\ Blandior, ut noceam; morti praemitto saginam. \\ 
        \pagebreak 
    \begin{center} \textbf{CODICS SALMASIANI.} \end{center}\begin{center} \textbf{0.} \end{center}LV Acula \\ p. 750 \\ Longa, sed exilis, tenui producta metallo, \\ Mollia duco levi comitantia vincula ferro, \\ Et faciem laesis et nexum reddo solutis. \\ \poemtitle{LVI Cli}Maior eram longe quondam, dum vita manebat; \\ Sed nunc exanimis lacerata ligata revulsa \\ Dedita sum terrae, tumulo sed condita non sum. \\ LVII Clavus caligaris \\ In caput ingredior, quia de pede pendeo solo. \\ Vertice tango solum, capitis vestigia signo; \\ Sed multi comites casum patiuntur eundem. \\ LVIII Capillus \\ Findere me nulli possunt, praecidere multi. \\ Sed sum versicolor, albus quandoque futurus. \\ Malo manere niger: minus ultima fata verebor. \\ 
        \pagebreak 
     \marginpar{[236]} \begin{center} \textbf{CARMNA} \end{center}
      \end{verse}
  
            \subsection*{}
      \begin{verse}
      \poemtitle{LVIIII P}rr \\ Non sum cincta comis et non sum compta capillis. \\ Intus enim crines mihi sunt, quos non videt ullus. \\ Meque manus mittunt manibusque remittor in auras. \\ LX Serra \\ Dentibus innumeris sum toto corpore plena. \\ Frondicomam subolem morsu depascor acuto. \\ Mando tamen frustra, quia respuo praemia dentis. \\ LXI Ancora \\ Mucro mihi geminus ferro coniungitur uno. \\ Cum vento luctor, cum gurgite pugno profundo. \\ Scrutor aquas medias, ipsas quoque mordeo terras. \\ XII Pons \\ 
      \end{verse}
  
            \subsection*{}
      \begin{verse}
      \poemtitle{. 7151}Stat nemus in lympbis, stat in alto gurgite silva, \\ Et manet in mediis undis inmobile robur. \\ Terra tamen mittit, quod terrae munera praestat. \\ 
        \pagebreak 
    \begin{center} \textbf{CODICIS SALMASIAN1.} \end{center} \marginpar{[237]} LXII Spoogi \\ Ipsa gravis non sum, sed aquae mihi pondus inhaeret. \\ Viscera tota tument patulis diffusa cavernis. \\ Intus lymphba latet, sed non se sponte profundit. \\ \poemtitle{LXIIII Trides}Tres mihi sunt dentes, unus quos continet ordo. \\ Vnus praeterea dens est et solus in imo. \\ Meque tenet numen, ventus timet, aequora curant. \\ LXV Sitt \\ Saepta gravi ferro, levibus circumdata pinnis \\ Aera per medium volucri contendo meatu, \\ Missaque discedens nullo mittente revertor. \\ LXVI Tlagellum \\ De pecudis dorso pecndes ego terreo cunctas, \\ Obsequio cogens moderati lege doloris. \\ Nec volo contemni, sed contra nolo nocere. \\ LXVII Lantera \\ Cornibus apta cavis, tereti perlucida gyro, \\ Lumen habens intus, divini sideris instar, \\ Noctibus in mediis faciem non perdo dierum. \\ 
        \pagebreak 
    \begin{center} \textbf{CARMINA} \end{center} \marginpar{[238]} LXVIII Speculr \\ Perspicior penitus nec luminis arceo visus, \\ Transmittens oculos intra mea membra meantes. \\ Nec me transit hiems, sed sol tamen emicat in me. \\ \poemtitle{LXVIIII Speelm}Nulla mihi certa est, nulla est peregrina figura. \\ Fulgor inest intus radianti luce coruscus, \\ Qui nihil ostendit, nisi  \lbrack si \rbrack  quid viderit ante. \\ LXX Clepsydr \\ Lex bona dicendi, lex sum quoque dura tacendi, \\ Ius avidae linguae, finis sine fine loquendi, \\ Ipsa fluens, dum verba fluunt, ut lingua quiescat. \\ LXXI Puteus \\ Mersa procul terris in cespite lympha profundo \\ Non nisi perfossis possum procedere rimis, \\ Et trahor ad superos alieno ducta labore. \\ 
        \pagebreak 
    \begin{center} \textbf{CODCIS SALMASIANI.} \end{center} \marginpar{[239]} r r \\ \poemtitle{LXXII Tbus}Truncum terra tegit, latitant in caespite lymphae. \\ Alveus est modicus, qui ripas non habet ullas. \\ In ligno vehitur medio, quod ligna vebebat. \\ \poemtitle{LXXIII V6e:}Non ego continuo morior, dum spiritus exit: \\ Nam redit adsidue, quamvis et saepe recedit. \\ Nunc \lbrack que \rbrack  mihi magna est animae, nunc nulla facultas. \\ \poemtitle{LXXIIII Lopie}Deucalionea proveni sospes ab unda, \\ Afinis terrae, sed longe durior illa. \\ Littera decedat: volucrum quoque nomen habebo. \\ r r \\ \poemtitle{LXXV Cx}Evasi flammas, ignis tormenta profugi. \\ Ipsa medella meo pugnat contraria fato: \\ Ardeo de lymphis; mediis incendor ab undis. \\ 
        \pagebreak 
    \begin{center} \textbf{CARMINA} \end{center} \marginpar{[20]} \poemtitle{LXXVI Siex}Semper inest intus, sed raro cernitur ignis. \\ Intus enim latitat, sed solos prodit ad ictus. \\ Nec lignis ut vivat eget, nec ut occidat undis. \\ \poemtitle{XXVII Rotae}p. 153 \\ Quattuor aequalcs currunt ex arte sorores \\ Sic quasi certantes, cum sit labor omnibus unus. \\ Et prope sunt pariter nec se contingere possunt. \\ r1 \\ \poemtitle{XXVIII Scale}Nos sumus, ad caelum quae tendimus, alta petentes, \\ Concordi fabrica quas unus continet ordo, \\ Vt simul haerentes per nos nitantur ad auras. \\ \poemtitle{LXXVIIII Seop}Mundi magna parens, laqueo conexa tenaci, \\ Vincta solo plano, manibus conpressa duabus \\ Ducor ubique sequens et me quoque cuncta sequuntur. \\ 
        \pagebreak 
    \begin{center} \textbf{CODICIS SALMASIAN.} \end{center} \marginpar{[241]} rr r \\ LXXX Tntinnbulum \\ Aere rigens curvo patulum conponor in orbem. \\ Mobilis est intus lingnae crepitantis imago. \\ Non resono positus, sed motus saepe resulto. \\ x \\ \poemtitle{LXXXI Laee}Mater erat Tellus, genitor est ipse Prometheus, \\ Auriculae \lbrack que \rbrack  regunt redimito ventre cavato. \\ Dum cecidi, subito mater mea me laniavit. \\ LXXXII Coditum \\ Tres olim fuimus, qui nomine iungimur uno. \\ Ex tribus est unus, et tres miscentur in uno. \\ Quisque bonus per se; melior, qui continet omnes. \\ LXXXIII Vinum in acetum covesum \\ Sublatum nihil est, nibil est extrinsecus auctum; \\ Nec tamen inveni, quicquid prius ipse reliqui. \\ Quod fuerat, non est; coepit, quod non erat, \\ eSs5e. \\ 
        \pagebreak 
    \begin{center} \textbf{CARMINA} \end{center}\begin{center} \textbf{0 0} \end{center}cc \\ \poemtitle{LXXIIII Mlum}Nomen ovis graece, contentio magna dearum, \\ Fraus iuvenis Phrygii, multarum cura sororum. \\ Hoc volo, ne breviter mihi syllaba prima \\ legatur. \\ \poemtitle{LXXXV Pera}Nobile duco genus magni de gente Catonis. \\ Vna mihi soror est, plures licet esse putentur. \\ De fumo facies, sapientia de sale nata est. \\ \poemtitle{LXXXVI Mates}Non ego de toto mihi corpore vindico vires, \\ Sed capitis pugna nulli certare recuso. \\ CGrande tamen caput est, totum quoque pondus in illo. \\ \poemtitle{LXXXVII Psiiu}Contero cuncta simul virtutis robore magno. \\ Vna mihi cervix, capitum sed forma duorum. \\ Pro pedibus caput est: nam cetera corpore non sunt. \\ 
        \pagebreak 
    \begin{center} \textbf{CODICIS SALMASIANI.} \end{center} \marginpar{[01]} LXXXVIII Strigilis aenea. \\ †Rubida curva capa, alienis humida guttis, \\ Luminibus falsis auri mentita colorem, \\ Dedita sudori, modico subcumbo labori. \\ \poemtitle{LXXXVIIII 3alneum}Per totas aedes innoxius introit ignis. \\ Est calor in medio magnus, quem nemo veretur. \\ Non est nuda domus, sed nudus convenit hospes. \\ XXXX Tessemr \\ p. 157 \\ E85 Dedita sum semper voto, non certa futuri. \\ Iactor in ancipites varia vertigine casus \\ N on ego maesta malis, non rebus laeta secundis. \\ \poemtitle{LXXXXI Peeuia}Terra fui primo, latebris abscondita terrae; \\ Nunc aliud pretium flammae nomenque dederunt, \\ Nec iam terra vocor, licet ex me terra paretur. \\ 
        \pagebreak 
    \begin{center} \textbf{CARMNA} \end{center} \marginpar{[244]} V V \\ XXXXlI Mulier quae geminos pariebat \\ Plus ego sustinui, quam corpus debuit unum. \\ Tres animas habui, quas omnes intus habebam: \\ Discessere duae, sed tertia paene secuta est. \\ r V 1 \\ LXXXXI1I Miles podaricus \\ Bellipotens olim, saevis metuendus in armis, \\ Quinque pedes habui, quod numquam nemo negavit. 295s \\ Nunc mihi vix duo sunt; inopem me copia fecit. \\ V V c 1 \\ LXXXXIII Luscus alium tenens \\ Cernere iam fas est, quod vix tibi credere fas est. \\ Vnus inest oculus, capitum sed milia multa. \\ Qui quod habet vendit, quod non habet unde parabit? \\ r \\ LXXXXV Funambulus \\ Inter luciferum caelum terrasque iacentes \\ Ara per medium docta meat arte viator. \\ Semita sed brevis est, pedibus nec sufficit ipsis. \\ 
        \pagebreak 
    \begin{center} \textbf{CODICIS SALMASIANI.} \end{center} \marginpar{[245]} 
      \end{verse}
  
            \subsection*{}
      \begin{verse}
      \poemtitle{LXXXXVI} \lbrack Nunc mihi iam credas, fieri quod posse negatur. \\ Octo tenes manibus, sed me monstrante magistro \\ Sublatis septem reliqui tibi quinque manebunt \rbrack  \\ \poemtitle{LXXXVII Vmb}Insidias nullas vereor de fraude latenti; \\ Nam deus attribuit nobis haec munera formae, \\ Quod me nemo movet, nisi qui prius ipse movetur. \\ \poemtitle{LXXXXVIII Eio}Virgo modesta nimis legem bene servo pudoris. \\ Ore procax non sum nec sum temeraria linguae. \\ ltro nolo loqui, sed do responsa loquenti. \\ \poemtitle{XXXXVIIII Somnus}Sponte mea veniens varias ostendo figuras. \\ Fingo metus vanos nullo discrimine veri. \\ Sed me nemo videt, nisi qui sua lumina claudit. \\ 
        \pagebreak 
    \begin{center} \textbf{CARMINA} \end{center} \marginpar{[246]} C Monumentum \\ Nomen habens hominis post ultima fata relinquor. \\ Nomen inane manet, sed dulcis vita profugit. \\ Vita tamen superest morti post tempora vitae. \\ Mater me genuit, eadem mox gignitur ex me. \\ 
        \pagebreak 
    \begin{center} \textbf{B. V I 1 83.} \end{center}\begin{center} \textbf{LVXORII} \end{center}\begin{center} \textbf{M. 299 381.} \end{center}B. IV 356 425. \\ irt c ari ssi m i e t sp e et abo i i s \\ 
      \end{verse}
  
            \subsection*{}
      \begin{verse}
      \poemtitle{LIBE EPIGRAMMAT0N}Sunt  \lbrack verol versusLXXXXVII \\ 
      \end{verse}
  
            \subsection*{287}
      \begin{verse}
      Metro phalaecelo ad Paustum \\ \poemtitle{ \lbrack XXIV)}Ausus post veteres tuis, amice, \\ Etsi iam temere est, placere iussis, \\ Nostro Fauste animo probate conpar, \\ Tantus grammaticae magister artis, \\ Quos olim puer in foro paravi \\ Versus ex variis locis deductos \\ (llos scilicet, unde me poetam \\ Insulsum puto quam magis legendum), \\ Nostri temporis ut amavit aetas, \\ In parvum tibi conditos libellum \\ Transmisi memori tuo probandos \\ Primum pectore; deinde, si libebit, \\ p.157 \\ Discretos titulis, quibus tenentur, \\ 
        \pagebreak 
    \begin{center} \textbf{CARMINA} \end{center} \marginpar{[248]} Per nostri similes dato sodales. \\ Nam si doctiloquis nimisque magnis \\ Haec tu credideris viris legenda, \\ Culpae nos socios notabit index: \\ Tam te, talia qui bonis recenses, \\ Quam me, qui tua duriora iussa \\ Feci nescius, inmemor futuri. \\ Nec me paeniteat iocos secutum, \\ Quos verbis epigrammaton facetis \\ Diversos facili pudore lusit \\ Frigens ingenium, laboris expers. \\ Causam, carminis unde sit voluptas, \\ Edet ridiculum sequens poema. \\ 
      \end{verse}
  
            \subsection*{288}
      \begin{verse}
      Iambiei ad eetorem operis sui \\ Priscos cum haberes, quos probares, indices, \\ Lector, placere qui bonis possent modis, \\ Nostri libelli cur retexis paginam \\ Nugis refertam frivolisque sensibus, \\ Et quam tenello tiro lusi viscere? \\ An forte doctis illa cara est versibus, \\ Sonat pusillo quae laboris †schemate, \\ Nullo decoris, ambitus, seutentiae? \\ Hanc tu requiris et libenter inchoas, \\ Velut iocosa si theatra pervoles? \\ 
        \pagebreak 
    \begin{center} \textbf{CODICIS SALMASIANI.} \end{center} \marginpar{[249]} 
      \end{verse}
  
            \subsection*{289}
      \begin{verse}
      Asclepiadei ad librum suum \\ Parvus nobilium cum liber ad domos \\ Pomposique fori scrinia publica \\ Cinctus multifido veneris agmine, \\ p. 15 \\ Nostri defugiens pauperiem laris, \\ Quo dudum modico sordidus angulo \\ Squalebas, tineis iam prope deditus: \\ Si te despiciet turba legentium \\ Inter Romulidas et Tyrias manus, \\ Isto pro exequiis claudere disticho: \\ Contentos propriis esse decet focis, \\ Quos laudis facile est invidiam pati. \\ 
      \end{verse}
  
            \subsection*{290}
      \begin{verse}
      Epigrmmatae parva uod in hoc \\ Iibro sceripserit \\ Si quis boc nostro detrahit ingenio, \\ Adtendat modicis condi  \lbrack de \rbrack  mensibus annum, \\ Et graciles hiemis, veris et esse dies; \\ Noverit  \lbrack in \rbrack  brevibus magnum deprendier usum. \\ Vltra mensuram gratia nulla datur. \\ Sic mea concinno si pagina displicet actu, \\ Finito citius carmine clausa silet. \\ 
        \pagebreak 
     \marginpar{[250]} \begin{center} \textbf{CARMINA} \end{center}Nam si constaret libris longissima multis, \\ Fstidita forent plurima †vel vitio. \\ 
      \end{verse}
  
            \subsection*{291}
      \begin{verse}
      Troehaicum de piseibus, qui ab hominibus clbos \\ epiebant \\ Verna clausas inter undas et lacunas regias \\ Postulat cibos diurnos ore piscis parvolo \\ Nec manum fugit vocatus nec pavescit retia. \\ Roscidi sed amnis errans hinc et inde margine \\ Odit ardui procellas et dolosi gurgitis, \\ Ac suum, quo libet esse, transnatans colit mare. \\ Sic famem gestu loquaci et mitiori †vertice \\ Discit ille quam sit aptum ventris arte vincere. \\ 00 \\ Archiloehium de apro mitissimo in trielinio nutrito \\ Martis aper genitus iugis inesse montium \\ p. 159 \\ Frangere  \lbrack et \rbrack  horrisonum nemus ferocius solens, \\ Pabula porticibus capit libenter aureis \\ Et posito famulans furore temperat minas; \\ Nec Parios lapides revellit ore spumeo \\ Atria nec rabidis decora foedat ungulis, \\ Sed domini placidam manum quietus appetens \\ Fit magis ut Veneris dicatus ille sic sacris. \\ 
        \pagebreak 
    \begin{center} \textbf{CODICIS SALMASIANI.} \end{center} \marginpar{[01]} 
      \end{verse}
  
            \subsection*{293}
      \begin{verse}
      \poemtitle{De auriga Aegptio ui semper vineebat}Quamvis ab Aurora fuerit genetrice creatus \\ Memnon, Pelidae conruit ille manu. \\ At te Nocte satum, ni fallor, matre paravit \\ Aeolus et Eephyri es natus in antra puer. \\ Nec quisquam qui te superet nascetur Achilles: \\ Dum Memnon facie es, non tamen es genio. \\ 
      \end{verse}
  
            \subsection*{294}
      \begin{verse}
      Sapphicum in grammaticum furiosum \\ Carminum interpres meritique vatum, \\ Cum leves artem pueros docere \\ Diceris vel te iuvenes magistrum \\ Audiunt verbis veluti disertum, \\ Cur in horrendam furiam recedis \\ Et manu et telo raperis cruentus? \\ Non es, in quantum furor hic probatur, \\ Dignus inter grammaticos vocari, \\ Sed malos inter sociari Orestas. \\ 
      \end{verse}
  
            \subsection*{295}
      \begin{verse}
      lyconeum Mn advoeatum efeminatum \\ Execti species viri, \\ Naturae grave dedecus, \\ Vsu femineo Paris, \\ 
        \pagebreak 
    \begin{center} \textbf{CARMINA} \end{center} \marginpar{[252]} Foedae cura libidinis: \\ p. 160 \\ Cum sis ore facundior, \\ Cur causas steriles agis \\ Aut corrupta negotia \\ Et perdenda magis locas? \\ Agnovi: ut video, tuo \\ Ori quid bene credier \\ Non vis, sed puto podici. \\ 
      \end{verse}
  
            \subsection*{296}
      \begin{verse}
      B. III 161. \\ In clmosum Pygmaeum eorpore \\ M. 911. \\ et furiosum \\ Corpore par querulis es vel clamore cicadis: \\ linc potior, quod te tempora nulla vetant. \\ Dum loqueris, quaerunt cuncti, vox cuius oberret, \\ Atque sonum alterius corporis esse putant. \\ Miramur, tantum capiant qui membra furorem, \\ Cum sit forma levis, clamor et ira gravis. \\ 
      \end{verse}
  
            \subsection*{297}
      \begin{verse}
      Phalaeclum in moeehum. quod debriatus plorabat, \\ eum coitum inplere non posset \\ Saepius futuis nimisque semper, \\ Nec parcis, nisi forte debriatus \\ Effundis lacrimas, quod esse moechus \\ 
        \pagebreak 
    \begin{center} \textbf{CODCIS SALMASIANI.} \end{center}Multo non valeas mero subactus. \\ Plura ne futuas, peto, Lucine, \\ Aut semper bibe taediumque plange, \\ Aut, numquam ut futuas, venena sume. \\ 
      \end{verse}
  
            \subsection*{298}
      \begin{verse}
      In spadonem regium ui mitellam sumebat \\ Rutilo decens capillo \\ Roseoque crine ephebus \\ Spado regius mitellam \\ Capiti suo locavit; \\ Proprii memor pudoris, \\ p. 767 \\ Bene conscius quid esset, \\ Posuit cogente nullo, \\ Fuerat minus quod illi. \\ 
      \end{verse}
  
            \subsection*{299}
      \begin{verse}
      Anapaesticeum in maum mendieum \\ Tibi cum non sit diei panis, \\ Magicas artes inscius inples. \\ Ire per umbras atque sepulcra \\ Pectore egeno titubans gestis. \\ Nec tua Manes carmina sumunt, \\ Fame dum pulsus Tartara cantu \\ Omnia turbas, aliquid credens \\ Dare quod possit superis Pluton \\ 
        \pagebreak 
     \marginpar{[254]} \begin{center} \textbf{CARMNA} \end{center}Pauperibus. qui puto quod peius \\ Egeas totum semper in orbem, \\ Mage, si posces membra perempta. \\ 
      \end{verse}
  
            \subsection*{300}
      \begin{verse}
      In acceptorarium obesum et infelieem \\ Pondere detracto miseras, Martine, fatigas \\ Pressura crudelis aves. pinguedine tanta \\ Vt tu sis, frustra maciem patiuntur iniquam. \\ Debuerant, fateor, magis has tua pascere membra, \\ Vt numquam possent ieiuna morte perire. \\ 
      \end{verse}
  
            \subsection*{301}
      \begin{verse}
      In vetulam virginem nubentem \\ Virgo, quam Phlcgethon vocat sororem, \\ Saturni potior parens senecta, \\ Quam Nox atque Erebus tulit Chaosque, \\ Cui rugae totidem graves, quot anni, \\ Cui vultus elefans dedit cutemque, \\ Mater simia quam creavit arvis \\ Grandaeva in Libycis novo sub orbe, \\ p. 162 \\ Olim quae decuit marita Diti \\ Pro nata Cereris dari per umbras: \\ Quis te tam petulans suburit ardor, \\ Nunc cum iam exitium tibi supersit? \\ 
        \pagebreak 
    \begin{center} \textbf{CODICIS SALMASIANI.} \end{center} \marginpar{[255]} \begin{center} \textbf{0.} \end{center}An hoc pro titulo cupis sepulcri, \\ Vt te cognita fama sic loquatur, \\ Quod stuprata viro est anus nocenti? \\ In medieolenonem \\ Quod te pallidulum, Marine noster, \\ Cuncti post totidem dies salutant, \\ Credebam medicum velut peritum \\ Curam febribus et manum pudicam \\ 
      \end{verse}
  
            \subsection*{}
      \begin{verse}
      \poemtitle{De pactis logicae parare sectae}Aut de methodicis probare libris. \\ At tu fornice turpius vacabas, \\ Exercens aliis, quod ipse possis \\ Lenatis melius tibi puellis \\ Scortandi solito labore ferre. \\ Novi, quid libeat tuum, chirurge, \\ Conspectos animum videre cunnos: \\ Vis ostendere te minus virum esse, \\ Arrectos satis est mares videre. \\ 
      \end{verse}
  
            \subsection*{303}
      \begin{verse}
      In diaeonum festinantem ad prandium cauponis \\ Quo festinus abis gnla inpellente, sacerdos? \\ An tibi pro psalmis pocula corde sedent? \\ 
        \pagebreak 
    \begin{center} \textbf{CARMNA} \end{center} \marginpar{[256]} Pulpita templorum, ne pulpita quaere tabernae, \\ Numiua quo caeli, non fialas referas. \\ 
      \end{verse}
  
            \subsection*{304}
      \begin{verse}
      De turre in viridiario posita, ubi se Pridama \\ prum pinxit oeeidere \\ \poemtitle{1. 73}Extollit celsas uemoralis Aricia sedes, \\ Sternit ubi famulas casta Diana feras; \\ Froudosis Tempe cinguntur Thessala silvis \\ Pinguiaque Nemeae lustra Molorchus habet; \\ Haec vero aetherias exit quae turris in auras, \\ Consessum domino deliciosa parans, \\ Omnibus in medium lucris ornata refulget \\ Obtinuitque uno praemia cuncta loco. \\ Hinc uemus, hinc fontes exstructa cubilia cingunt \\ S1atque velut propriis ipsa Diana iugis. \\ Clausa sed in tanto cum sit splendore voluptas \\ Artibus ac variis atria pulcra micent, \\ Admiranda tuae tamen est virtutis imago, \\ Fridamal, et stratae gloria magna ferae. \\ Qui solitae accendens mentem virtutis amore \\ Aptasti digno pingere facta loco: \\ Hic spumantis apri iaculo post terga retorto \\ Fronte et cum geminis uaribus ora feris. \\ 
        \pagebreak 
    \begin{center} \textbf{CODICIS SALMASMANI.} \end{center} \marginpar{[257]} Ante ictum subita prostrata est bellua morte, \\ Cui prius extingui quam cecidisse fuit. \\ Iussit fata manus telo, nec vulnera sensit \\ Exerrans anima iam pereunte cruor. \\ 
      \end{verse}
  
            \subsection*{305}
      \begin{verse}
      De avibus marinis, quae post volatum a \\ domum remeabant \\ Felix marinis alitibus Fridamal, \\ Felix iuventa, prosperior genio, \\ Quem sponte poscunt aequoreae volucres \\ Nec stagna grato frigida concilio \\ p. 164 \\ Pigris strepentes gurgitibus retinent; \\ Sed quo tuorum temperiem nemorum \\ Monstrent, volatu praememores famulo \\ Pro te relictam non repetunt patriam. \\ 
      \end{verse}
  
            \subsection*{306}
      \begin{verse}
      In aurigam senem victum crimina in populos \\ Iactantem \\ Te quotiens victum circus, Cyriace, resultat, \\ Crimine victores polluis et populos. \\ Non visum quereris senio languente perisse \\ Castigasque tuae tarda flagella manus. \\ Sed quod in alterius divulgas crimina nomen, \\ Cur non illa magis credis inesse tibi? \\ 
        \pagebreak 
    \begin{center} \textbf{CARMNA} \end{center} \marginpar{[258]} Es meritis inpar, virtute, aetate relictus: \\ laec cum habeant alii, crimina vera putas. \\ Sola tamen falsis surgat tibi poena loquellis, \\ Vt victus semper nil nisi crimen agas. \\ In podagrum venationi studentem \\ Apros et capreas levesque cervos \\ Incurvus rapidis equis fatigat. \\ Tantum nec sequitur capitque quicquam. \\ Esse inter iuvenes cupit, vocari \\ Baudus, dum misero gemat dolore \\ Et nil praevaleat. quid ergo gestit? \\ Mori praecipiti furit caballo, \\ Cum lecto melius perire possit. \\ 
      \end{verse}
  
            \subsection*{308}
      \begin{verse}
      In supra seriptum quod multa seorta habuit \\ et eas custodiebat \\ Eelo  \lbrack agitas \rbrack  plures, Incurvus, clune puellas, \\ Sed nulla est, quae te sentiat esse virum. \\ Custodis clausas, tamquam sis omnibus aptus; \\ Est tamen internus luppiter ex famulis. \\ Si nihil ergo vales, †vacuo cur arrigis †oge \\ Et facis ignavus mentis adulterium? \\ 
        \pagebreak 
    \begin{center} \textbf{CODICIS SALMASIANI.} \end{center} \marginpar{[259]} 
      \end{verse}
  
            \subsection*{309}
      \begin{verse}
      Anacreontium in medicum inpotentem, qui ter \\ vViduam duxit uxorem \\ Post tot repleta busta \\ Et funerum catervas \\ Ac dispares maritos, \\ Rugosa quos peremit \\ FTatis anus sinistris, \\ Tu nunc, chirurge, quartus \\ Coniunx vocate plaudis. \\ Sed vivus es sepultus, \\ Dum parte qua decebat \\ Nil contines mariti. \\ Iam nosco, cui videtur \\ Nupsisse Paula rursus. \\ Nulli! quid ergo fecit? \\ Mutare mox lugubrem \\ Quam sumpserat cupivit \\ Vxor nefanda vestem, \\ Vt quartus atque quintus \\ Possit venire coniunx! \\ 
      \end{verse}
  
            \subsection*{310}
      \begin{verse}
      In pantomimam maeam, quae Andromaehae \\ fabulam freuenter saltabat et rptum elenae \\ Andromacham atque Helenam saltat Macedonia semper, \\ Et quibus excelso corpore forma fuit. \\ Haec tamen aut brevior Pygmaea virgine surgit \\ Ipsius aut quantum pes erat Andromachae. \\ 
        \pagebreak 
    \begin{center} \textbf{CARMINA} \end{center} \marginpar{[260]} Sed putat illarum fieri se nomine talem, \\ Motibus et falsis crescere membra cupit. \\ Hlac spe, crede, tuos incassum decipis artus: \\ Thersiten potius finge, quod esse soles! \\ In ebriosum nihil comedentem, sed solum \\ bibentem \\ Dum bibis solus pateras, quot omnes, \\ Saepe nec totis satiaris horis \\ Et tibi munus Cereris resordet \\ Ac nihil curas nisi ferre Bacchum, \\ Nerfa, iam te non hominem vocabo, \\ Sed nimis plenam et patulam lagunam. \\ 
      \end{verse}
  
            \subsection*{312}
      \begin{verse}
      \poemtitle{De Tama picta in stabulo eirci}Qualem te pictor stabulis formavit equorum, \\ Talem te nostris blanda referto iugis. \\ Semper et adsiduo vincendi munera porta \\ His, quorum limen fortis amica sedes. \\ Aliter \\ Verum, Fama, tibi vultum pictura notavit, \\ Dum vivos oculos iuncea forma gerit. \\ 
        \pagebreak 
    \begin{center} \textbf{CODICIS SALMASIANI.} \end{center} \marginpar{[261]}  \marginpar{[0]} Tu quamvis totum velox rapiaris in orbem, \\ Pulcrior hoc uno limine clausa sedes. \\ 
      \end{verse}
  
            \subsection*{314}
      \begin{verse}
      In vieinum invidum \\ Eeleris nimium cur mea, Marcie, \\ Tamquam si pereas, limina, nescio, \\ Cum sis proximior, una velut domus, \\ Et nostros paries dimidiet lares. \\ Sed gratum ferimus: talis es omnibus, \\ Nec quemquam nisi te  \lbrack vis \rbrack  miser aspici. \\ Contingat (quesumus, numina) quod cupis, \\ Te solum ut videas, Marcie, dum vivis! \\ p.16 \\ 
      \end{verse}
  
            \subsection*{315}
      \begin{verse}
      In gibberosum. qui se genrosum iactabat \\ Fingis superbum quod tibi patrum genus, \\ Nunc Iuliorum prole te satum tumens, \\ Nunc Memmiorum Martiique Romuli, \\ Prodesse gibbo forte quid putas tuo? \\ Nil ista falso verba prosunt ambitu. \\ Tace parentes, ne quietos moveas: \\ Natura nobis unde sis natus docet. \\ 
        \pagebreak 
    \begin{center} \textbf{CARMINA} \end{center} \marginpar{[262]} 
      \end{verse}
  
            \subsection*{316}
      \begin{verse}
      De eo qui se poetam dicebat quod in triviis \\ Cantaret et a pueris laudaretur \\ Conponis fatuis dum pueris melos, \\ Eenobi, et trivio carmine perstrepis \\ Indoctaque malis verba facis locis, \\ Credis tete aliquid laudibus indere \\ Tamamque ad teneros ducere posteros? \\ Hoc nostrae faciunt semper et alites: \\ Ni rite institues, sibila tum canunt. \\ 
      \end{verse}
  
            \subsection*{317}
      \begin{verse}
      In puellam hermaphroditam \\ Monstrum feminei bimembre sexus, \\ Quam coacta virum facit libido, \\ Quin gaudes futui furente cunno? \\ Cur te decipit inpotens voluptas? \\ Non das, quo pateris facisque, cunnum. \\ Illam, qua mulier probaris esse, \\ Partem cum dederis, puella tunc sis. \\ 
      \end{verse}
  
            \subsection*{318}
      \begin{verse}
      Ad eum qui per diem dormiens noetu vigilabat \\ Stertis anhelanti fessus quod corde, Lycaon, \\ Exhorrens lucis munera parta die, \\ 
        \pagebreak 
    \begin{center} \textbf{CODICIS SALMASIAN1.} \end{center} \marginpar{[263]} Et tibi vigilias semper nox tetra ministrat, \\ Iam scio te nostro vivere nolle die. \\ At si tale tibi studium natura paravit, \\ Vivas ad antipodas; sis vel ut inde, redi! \\ 
      \end{verse}
  
            \subsection*{319}
      \begin{verse}
      \poemtitle{De sareopho. ubi turpia sculpta fuerant}Turpia tot tumulo defixit crimina Balbus, \\ Post superos spurco Tartara more premens. \\ Pro facinus! finita nihil modo vita retraxit, \\ Luxuriam ad Manes moecha sepulcra gerunt. \\ 
      \end{verse}
  
            \subsection*{320}
      \begin{verse}
      tem unde supr seriptum: ubi eui ceirei \\ bebant \\ Crevit ad ornatum stabuli circique decorem \\ Purior egregio reddita nympha loco, \\ Quam cingunt variis insignia clara metallis \\ Crispatumque super scinditur unda gradum. \\ Excipit hanc patuli moles miranda sepulcri, \\ Corporibus vivis pocula blanda parans. \\ Nec iam sarcofagus tristis sua funera claudit, \\ Sed laetos dulci flumine conplet equos. \\ Fundit aquas duro signatum marmore flumen, \\ Falsa tamen species vera fluenta vomit. \\ 
        \pagebreak 
     \marginpar{[264]} \begin{center} \textbf{CAiRMNA} \end{center}Plaudite vos, Musae, diversaque, plaudite, sigua, \\ Quae circum docili continet arte decor; \\ Et dum palmiferis post praelia tanta quadrigis \\ Garrula victores turba resolvit equos, \\ Praebete innocuos potus potusque salubres, \\ Vt domino proprius gaudia circus agat. \\ 
      \end{verse}
  
            \subsection*{101}
      \begin{verse}
      In cinedum bon sua corruptoribus dantem \\ Divitias grandesque epulas et munera multa, \\ Quod proavi atque atavi quodque reliquit avus, \\ Des licet in cunctos et spargas, ecca, maritos: \\ Plus tamen ille capit, cui dare saepe cupis. \\ Nescio quid miserum est, quod celas, Becca: talento \\ Vendere debueras, si bona membra dares. \\ 00 \\ De eo qui uxorem suam prostare faeiebat \\ pro fliis habendis \\ Stirpe negata patrium nomen \\ Non pater audis; carus adulter \\ Coiugis castae viscera damas, \\ Pariat spurios ut tibi natos, \\ Inscia, quo sint semine creti. \\ Fuerant forsan ista ferenda \\ Foeda, Proconi, vota parumper, \\ 
        \pagebreak 
    \begin{center} \textbf{CODICIS SALMASIAN1.} \end{center} \marginpar{[265]} †Scire vel ipsa si tuus umquam \\ †Posset adultus dicere matrem. \\ 
      \end{verse}
  
            \subsection*{323}
      \begin{verse}
      \poemtitle{De aleatore in pretio lenoeinii udente}Ludis, nec superas, Eltor, ad aleam, \\ Nec quicquam in tabula das, nisi virginem, \\ Spondens blanditias et coitus simul. \\ loc cur das aliis, quod poteras tibi? \\ An tali melius praemia grata sunt? \\ Aut prodest vitium tale quod impetras? \\ Si vincas, ego te non puto virginem \\ In luxum cupere, sed mage vendere. \\ 
      \end{verse}
  
            \subsection*{324}
      \begin{verse}
      In nomen Aeptii, quo eui cirei infortunium \\ piebant \\ ‘Icarus’ et ‘Phaethon’ Veneto nolente vocaris \\ Atque ‘agilis’, pigro cum pede cuncta premas. \\ Sed tamen et Phaethon cecidit super aethera flammis, \\ Dum cupit insolitis nescius ire plagis; \\ Tu quoque confractis defectus in aequore pinnis, \\ Icare, Phoebeo victus ab igne cadis. \\ Digna his ergo tibi praebentur nomina fatis, \\ Per te iterum ut pereant, qui periere prius. \\ 
        \pagebreak 
     \marginpar{[266]} \begin{center} \textbf{CARMINA} \end{center}
      \end{verse}
  
            \subsection*{325}
      \begin{verse}
      De Romulo pieto, ubi in muris fratrem oeeldit \\ Disce pium facinus: percusso, omule, fratre \\ Sic tibi Roma datur. huius iam nomine culpet \\ Nemo te c(aedis, murorum si decet omen. \\ 
      \end{verse}
  
            \subsection*{326}
      \begin{verse}
      De eo qui amteos ad prandium clamabat, \\ ut plura exposceret xenia \\ Gaudeo quod me nimis ac frequenter \\ Ambitu pascis, Blumarit, superbo. \\ Vnde sed pascor? ea sunt per omnes \\ Sparsa convivas bona. nec volebam, \\ Pasceres quemquam peteresque mecum, \\ Ne tibi quicquam detur, unde pascas. \\ Hoc tamen sed si vitio teneris, \\ Me precor numquam iubeas vocari. \\ 
      \end{verse}
  
            \subsection*{327}
      \begin{verse}
      \poemtitle{De auriga elato freuenter cadente}Pascasium aurigam populi fortem esse fatentur, \\ Ast ego non aliud quam turgida membra notabo \\ Inflatumque caput papulis et amica ruinis \\ Brachia, quae numquam recto moderamine frenant. \\ Mox cadit et surgit, rursum cadit, inde resurgit \\ Et cadit, ut miseris frangantur crura caballis. \\ 
        \pagebreak 
    \begin{center} \textbf{CODICIS SALMASIANI.} \end{center} \marginpar{[267]} Non iste humano dicatur nomine natus: \\ Hunc potius gryphum proprium vocet Africa circo. \\ 
      \end{verse}
  
            \subsection*{328}
      \begin{verse}
      \poemtitle{De laude aurige prasini}Iectofian prasino felix auriga colore, \\ Priscorum conpar, ars quibus ipsa fuit. \\ p.171 \\ Suetus equos regere  \lbrack et \rbrack  metas lustrare quadrigis \\ Et quocumque velis ducere frena manu. \\ Non sic Tantalides humero stat victor eburno; \\ Vna illi palma est, at tibi multa manet. \\ 
      \end{verse}
  
            \subsection*{329}
      \begin{verse}
      In eum qui foedas amabat \\ Diligit informes et foedas Myrro puellas; \\ Quas aliter pulcro viderit ore, timet. \\ Iudicium hoc quale est oculorum, Myrro, fateri, \\ Vt tibi non placeat Pontica, sed Garamas! \\ Ilam tamen agnosco, cur tales quaeris amicas: \\ Pulcra tibi numquam, sed dare foeda solet. \\ 
      \end{verse}
  
            \subsection*{330}
      \begin{verse}
      \poemtitle{De simii canum dorso fnpositis}Reddita post longum Tyriis est mira voluptas, \\ Quem pavet ut sedeat simia blanda canem. \\ 
        \pagebreak 
     \marginpar{[268]} \begin{center} \textbf{CARMNA} \end{center}Quantum magna parant felici tempora regno, \\ Discant ut legem pacis habere ferae! \\ 0 1 \\ \poemtitle{De partu ursae}Lambere nascentis fertur primordia prolis \\ Vrsa ferox, placido cum facit ore genus. \\ Expolit informes labris parientibus artus \\ Et pietas subolem rursus amore creat. \\ Attrito truncum formatur corpore pignus, \\ ‘t sculpendo facit crescere membra faber. \\ Officium natura suum permisit amanti: \\ Formam post uterum lingua magistra parit. \\ 1210 \\ 
      \end{verse}
  
            \subsection*{}
      \begin{verse}
      \poemtitle{De laude horti Eugeti}Hortus, quo faciles fluunt Napaeae, \\ Quo ludunt Dryades virente choro, \\ Quo fovet teneras Diana Nympbas, \\ p. 72 \\ Quo Venus roseos recondit artus, \\ Quo fessns teretes Cupido flammas \\ Suspensis reficit liber pharetris, \\ Quo se Laconides ferunt puellae, \\ 
        \pagebreak 
    \begin{center} \textbf{CODICIS SALMASIANI.} \end{center} \marginpar{[269]} Cui numquam minus est amoena frondis, \\ Cui semper redolent amoma verni, \\ Cui fons perspicuis tener fluentis \\ Muscoso riguum parit meatu, \\ Quo dulcis avium canor resultat: \\ Quidquid per varias refertur urbes, \\ Hoc uno famulans loco subaptat. \\ 
      \end{verse}
  
            \subsection*{333}
      \begin{verse}
      \poemtitle{De tablista furloso quasi tesseris imperante}Ludit cum multis Vatanans, sed ludere nescit, \\ Et putat imperio currere puncta suo. \\ Sed male dum numeros contraria tessera mittit, \\ Clamat et irato pallidus ore fremit. \\ Tum verbis manibusque furens miserandus anhelat, \\ IDe solitis faciens proelia vera iocis. \\ Efundit tabulam meusam subsellia pyrgum \\ Perditaque larpyacis aera rapit manibus. \\ lic si forte unam tabulam non arte sed errans \\ Vicerit aut aliam, nil bene dante manu, \\ Mox inflat venas ct †pallida guttura tendit \\ Plusque furit vincens, quam superatus erat. \\ Non iam huic ludum sapientum calculus aptet \\ 
        \pagebreak 
     \marginpar{[270]} \begin{center} \textbf{CARMINA} \end{center}
      \end{verse}
  
            \subsection*{334}
      \begin{verse}
      De venatore pieto in manibus oeulos habente \\ Docta manus saevis quotiens se praebuit ursis, \\ Numquam fallentem tela dedere necem. \\ Hinc etiam digitis oculos pictura locavit, \\ Quod visum frontis provida dextra tulit. \\ 
      \end{verse}
  
            \subsection*{335}
      \begin{verse}
      Aliter unde supr \\ Venatori oculos manibus pictura locavit \\ Et geminum egregia lumen ab arte manet. \\ Hic quocumque modo venabula fulgida pressit, \\ Signatum veluti contulit exitium. \\ Naturae lucem vicerunt fortia facta: \\ Iam visus proprios coepit habere manus. \\ 
      \end{verse}
  
            \subsection*{336}
      \begin{verse}
      In aurigam efeminatum numuam vincentem \\ Praecedis, Vico, nec tamen praecedis, \\ Et quam debueras tenere partem, \\ Iac mollis misero teneris usu. \\ Vmquam vinccre possis ut quadrigis, \\ Corruptor tibi sit retro ponendus. \\ 0 4 \\ De paranmpho delatore,. ui se ad hoc offcium \\ omnibus ingerebat \\ Hermes cunctorum thalamos et vota pererrat, \\ Omnibus ac sponsis pronubus esse cupit. \\ 
        \pagebreak 
    \begin{center} \textbf{CODICIS SALMASIANI.} \end{center}\begin{center} \textbf{r} \end{center}Hunc quisquam  \lbrack si \rbrack  forte velit contemnere dives, \\ Mox eius famam rodit iniqua ferens. \\ Nec tutum obsequium nuptis: famulatur, amicis \\ Indicet ut potius, quae videt, ille nocens. \\ Non sua sortitur, te qui facit auspice vota, \\ Sed tua; cui multum conferet, ut taceas. \\ 
      \end{verse}
  
            \subsection*{338}
      \begin{verse}
      De funere mulieris formosae, uae litigiosa fuit \\ Gorgoneos vultus habuit Catucia coniunx: \\ Haec dum pulcra foret, iurgia saepe dabat. \\ p. 174 \\ Fecerat atque  \lbrack suum \rbrack  semper rixando maritum, \\ Esset ut insano stultius ore tacens. \\ Et quotiens illam trepido cernebat amore, \\ Horrebat, tamquam vera Medusa foret. \\ Defuncta est tandem, haec iurgia ferre per umbras \\ Cumque ipsa litem reddere Persephone. \\ 
      \end{verse}
  
            \subsection*{339}
      \begin{verse}
      De duobus ui se onpedibus, uibus vincti \\ erant, eeeiderunt \\ Conpedibus nexi quidam duo forte sedebant \\ Criminis ob causam carceris ante fores. \\ Hi secum subitae moverunt iurgia rixae; \\ Ebrietatis opns gessit iniqua fames. \\ Nec caedem pugnis aut calcibus egit uterque: \\ Vincla illis telum, vincla fuere manus. \\ 
        \pagebreak 
    \begin{center} \textbf{CARMINA} \end{center}Nemo truces posthac debet pavitare catenas, \\ Si reus e poenis ingerit arma suis. \\ 
      \end{verse}
  
            \subsection*{340}
      \begin{verse}
      De causidico turpi, qui eoneubam suam Charitem \\ Vocabat \\ Esset causidici si par facundia nervo, \\ mpleret cuncti viscera negotii. \\ At tamen invigilat causis, quae crimina pandunt: \\ Cum Veneris famula iure Priapus agit. \\ 
      \end{verse}
  
            \subsection*{341}
      \begin{verse}
      In ministrum regis. qui alienas faeultates vi \\ extorquebat \\ Bella die noctuque suis facit Eutychus armis, \\ Divitias cunctis e domibus rapiens. \\ Huic si forte aliquis nolit dare sive repugnet, \\ Vim facit et clamat, regis habenda, nimis. \\ Quid gravius hostis, fur aut latrunculus implet, \\ Talia si dominus atque minister agit? \\ 
      \end{verse}
  
            \subsection*{342}
      \begin{verse}
      \poemtitle{De eodem aliter}p.775 \\ Cum famulis telisque furens penetralia cuncta \\ Eutychus inrumpit divitiasque rapit. \\ 
        \pagebreak 
    \begin{center} \textbf{CODCIS SALMASIANI.} \end{center} \marginpar{[0]}  \marginpar{[40]} Iunc nullus vetat ire parens, non forsan amicus; \\ Deterior precibus redditus ille manet. \\ Quae sunt ergo manus aut ferrea tela ferenda, \\ Quisve aries talem quodve repellat opus? \\ Huic si nemo potest ullas opponere vires, \\ Obvia sint illi fulmina sola dei! \\ 
      \end{verse}
  
            \subsection*{343}
      \begin{verse}
      In eum ui, cum senior dici nollet, multas sibi \\ eOncubas faeiebat et . . \\ Accusas proprios cur longo ex tempore canos, \\ Cum sis Phoenicis grandior a senio, \\ Et quotiens tardam quaeris celare senectam, \\ Paelicibus multis te facis esse virum? \\ Incassum reparare putas hac fraude iuventam; \\ Harum luxus agit, sis gravis ut senior. \\ 
      \end{verse}
  
            \subsection*{344}
      \begin{verse}
      ItCem in supra scriptum, uod se mori numquam \\ dieeret \\ Quantum tres Priami potuissent vivere mundo \\ Aut quantum cornix atque elefans superest, \\ Tantam dum numeres longaeva aetate senectam, \\ Te numquam firmas Tartara posse pati \\ Et credis Lachesim numquam tua rumpere fata \\ Aeternoque putas stamine fila trahi. \\ Quamvis tarda, tibi veniet mors ultima tandem, \\ Cum magis oblitus coeperis esse tui. \\ 
        \pagebreak 
    \begin{center} \textbf{CARMMNA} \end{center} \marginpar{[4]} Nam poena est potius morbis producere vitam; \\ Quod non semper habes, tristius, esse diu, est. \\ 
      \end{verse}
  
            \subsection*{345}
      \begin{verse}
      Epitphion de flia Oageis infantula \\ Heu dolor! est magnis semper mors invida fatis, \\ Quae teneros artus inimico sidere mergit! \\ Damira hoc tumulo regalis clauditur infans, \\ Cui vita iunocua est quarto dirupta sub anno. \\ Quam facile ofuscant iucundum tristia lumen \\ Nemo rosam albentem, fuerit nisi quae bona, carpit. \\ Iaec parvam aetatem cuncta cum laude ferebat. \\ Grata nimis specie, verecundo garrula vultu \\ Naturae ingenio modicos superaverat annos. \\ Dulce loquebatur, quidquid praesumpserat ore, \\ Linguaque diversum fundebat mellea murmur, \\ Tamquam avitm verna resonat per tempora cantus. \\ Huius puram animam stellantis regia caeli \\ Possidet et iustis inter videt esse catervis. \\ At pater Oageis, L.ibyam dum protegit armis, \\ Audivit subito defunctam funere natam. \\ Nuntius hic gravior cunctis fuit hostibus illi, \\ Ipsaque sub tali flevit Victoria casu. \\ 
      \end{verse}
  
            \subsection*{346}
      \begin{verse}
      B.III1. De amphitheatro in villa vioina mari \\ fabricato \\ M. 916. \\ Amphitheatrales mirantur rura triumphos \\ Et nemus ignotas cernit adesse feras. \\ 
        \pagebreak 
    \begin{center} \textbf{CODICIS SALMASANI.} \end{center} \marginpar{[275]} Spctat arando novos agrestis turba labores \\ Nautaque de pelago gaudia mixta videt. \\ ecundus nil perdit ager, plus germina crescunt, \\ Dum metuunt omnes hic sua fata ferae. \\ 
      \end{verse}
  
            \subsection*{347}
      \begin{verse}
      B. I 29. \\ \poemtitle{De sieillo Gupidinis uas fundentis !}gne salutifero Veneris puer omnia flammans \\ Pro facibus proprias arte ministrat aquas. \\ 
      \end{verse}
  
            \subsection*{348}
      \begin{verse}
      \poemtitle{De Neptuno in marmoreo alveo au}B. I 16. \\ fundente \\ M. 571. \\ Quam melior, Neptune, tuo sors ista tridente est: \\ Post pelagus dulces hic tibi dantur aquae! \\ 
      \end{verse}
  
            \subsection*{349}
      \begin{verse}
      B. I1I 25. \\ De puteo cavato in monte arido ’ \\ IHunc quis non credat ipsis dare Syrtibus amnes, \\ Qui dedit ignotas viscere montis aquas? \\ 
      \end{verse}
  
            \subsection*{350}
      \begin{verse}
       \lbrack B. etiam \\ \poemtitle{De aquis ealidis Cinensibus}III 32. \\ Ardua montanos inter splendentia lucos \\ Culmina et indigenis nunc metuenda feris, \\ 
        \pagebreak 
    \begin{center} \textbf{CARMINA} \end{center} \marginpar{[276]} Cum deserta prius solum nemus atra tenebat \\ Tetraque inaccessam sederat umbra viam: \\ Qua vos laude canam quantoque in carmine tollam, \\ ln quibus extructa est atque locata salus? \\ lic etiam ignitus tepet ad praetoria fervor, \\ Plenior et calidas terra ministrat aquas. \\  \lbrack Quis sterilem non credat humum? fumantia vernant \\ Pascua, luxuriat gramine cocta silex \rbrack  \\ Innocuos fotus membris parit intima tellus \\ Naturamque pio temperat igne calor. \\  \lbrack Et cum sic rigidae cautes fervore liquescaut, \\ Contemtis audax ignibus herba viret. \rbrack  \\ 
      \end{verse}
  
            \subsection*{351}
      \begin{verse}
      \poemtitle{De sententiis septem philosophorum distieht}Solon praecipuus, fertur qui natus Athenis, \\ Finem prolixae dixit te cernere vitae. \\ Chilon, quem patria egregium Lacedaemona misit, \\ Hoc prudenter ait, te ipsum ut cognoscere possis. \\ Ex Mitylenaeis fuerat qui Pittacus oris, \\ Te, ne quid nimis ut cupias, exquirere dixit. \\ 
        \pagebreak 
    \begin{center} \textbf{CODICS SALMASIANI.} \end{center}Thales ingenio sapiens Milesius acri \\ Errorem in terris firmat non caelitus esse. \\ Inde Prienaea Bias tellure creatus \\ p. 73 \\ Plures esse malos divina voce probavit. \\ Vrbe Periander genitus, cui fama Corintho est, \\ Omnia constituit tecnm ut meditando revolvas. \\ Cleobolus, proprium clamat quem Lindia civem, \\ Omne, inquit, magnum est, quod mensura optima librat. \\ 
      \end{verse}
  
            \subsection*{}
      \begin{verse}
      \poemtitle{De Ianuario mense}Lucifer annorum et saeclis, Sol, lane, secundus \\ Est rota certa tui tecum sine fine laboris; \\ ltque reditque tibi, quidquid in orbe venit. \\ Omnia perpetuis praecedis frontibus ora; \\ Quae necdum venient quaeve fuere vides. \\ 59a5 \\ 
      \end{verse}
  
            \subsection*{}
      \begin{verse}
      \poemtitle{De Oympio venatore Aegptio}Grata voluptatis species et causa favoris, \\ Fortior innumeris, venator Olympie, palmis, \\ 
        \pagebreak 
    \begin{center} \textbf{CARMNA} \end{center} \marginpar{[278]} Tu verum nomen membrorum robore signas, \\ Alcides collo scapulis cervice lacertis, \\ Admirande audax velox animose parate. \\ Nil tibi forma nocet igro fuscata colore. \\ Sic ebenum pretiosum atrum natura creavit, \\ Purpura sic †magno depressa in murice fulget, \\ Sic nigrae violae per mollia gramina vernant, \\ Sic tetras quaedam conmendat gratia gemmas, \\ Sic placet obscuros elefans inmanis ad artus, \\ Sic turis piperisque Indi nigredo placessit. \\ Postremum tauto populi pulcrescis amore, \\ Foedior est quantum pulcler sine viribus alter. \\ a \\ n epitphion supra sceripti Olympii \\ r \\ Venator iucunde nimis atque arte ferarum \\ p. 179 \\ Saepe placens, agilis gratus fortissimus audax, \\ Qui puer ad iuvenes dum non advixeris annos, \\ Omnia maturo conplebas facta labore. \\ Qui licet ex propria populis bene laude placeres, \\ Praestabas aliis, ut tecu viucere possent. \\ Tantaque miraudae fuerant tibi praemia formae, \\ Vt te post fatum timeant laudentque sodales. \\ 
        \pagebreak 
    \begin{center} \textbf{CODICIS SALMASHANI.} \end{center} \marginpar{[279]} Heu nunc tam subito mortis livore peremtum \\ Iste capit tumulus, quem non Carthaginis arces \\ Amphitheatrali potuerunt ferre triumpho! \\ Sed nihil ad Manes hoc funere perdis acerbo: \\ Vivet fama tui post te longaeva decoris \\ Atque tuum nomen semper Carthago loquetur. \\ 
      \end{verse}
  
            \subsection*{355}
      \begin{verse}
      \poemtitle{De Chfimaera aenea}Aeris fulgiduli nitens metallo \\ Ines pertulit, ante quos vomebat, \\ Et facta est melior Chimaera flammis. \\ 
      \end{verse}
  
            \subsection*{356}
      \begin{verse}
      De statua Veneris, in cuius capite violae sunt \\ natae \\ Cypris †candenti reddita marmore \\ Veram se exanimis corpore praebuit: \\ Infudit propriis membra caloribus, \\ Per florem in statuam viveret ut suam. \\ Nec mendax locus est: qui violas feret, \\ Servabit famulas inguinibus rosas. \\ 0 \\ In eaeeum, qui pulcras mulieres taetu noseebat \\ Lucis egenus, viduae frontis, \\ Iter amittens, caecus amator \\ 
        \pagebreak 
    \begin{center} \textbf{CARMNA} \end{center} \marginpar{[280]} Corpora tactu mollia palpat \\ Et muliebres iudicat artus, \\ p. 180 \\ Nivei cui sit forma decoris. \\ Credo quod ille nolit habere \\ Oculos, per quos cernere possit, \\ Cui dedit plures docta libido. \\ 
      \end{verse}
  
            \subsection*{358}
      \begin{verse}
      In philosophum hirsutum. nocte tantum cum \\ puelis eoncumbentem \\ Hispidus tota facie atque membris, \\ Crine non tonso capitis verendi, \\ Omnibus clares Stoicus magister. \\ Te viris tantum simulas modestum \\ Nec die quaeris coitum patrare, \\ Ne capi possis lateasque semper; \\ Fervidus sed cum petulante lumbo \\ Nocte formosas subigis puellas. \\ Incubus fies subito per actus, \\ Qui Cato dudum fueras per artes. \\ 
      \end{verse}
  
            \subsection*{359}
      \begin{verse}
      De catula sua brevissima, ad domini sui nutum \\ currente \\ Forma meae catulae brevis  \lbrack est \rbrack , sed amabilis inde, \\ lanc totam ut possit concava ferre manus. \\ Ad domini vocem famulans et garrula currit, \\ Iumanis tamquam motibus exiliens. \\ 
        \pagebreak 
    \begin{center} \textbf{CODICIS SALMASIANI.} \end{center} \marginpar{[281]} Nec monstrosum aliquid membris gerit illa decoris; \\ Omnibus exiguo corpore visa placet. \\ Mollior huic cibus est somnusque in stramine molli, \\ Muribus infensa est, saevior atque catis. \\ Vincit membra nimis latratu fortia torvo; \\ Si natura daret, posset ab arte loqui. \\ 
      \end{verse}
  
            \subsection*{360}
      \begin{verse}
      De pardis mansuetis, ui cum cenbus venationem \\ faclebant \\ Cessit Lyaei sacra fama numinis \\ p. 181 \\ Lynces ab oris qui subegit Hndicis. \\ Curru paventes duxit ille bestias \\ Mero gravatas ac minari nescias \\ Et quas domarent vincla coetu garrulo. \\ Sed mira nostri forma constat saeculi, \\ Prdos feroces saeviores tigribus \\ Praedam sagaci nare mites quaerere \\ Canum inter agmcn et, famem doctos pati, \\ Quidquid capessunt, ore ferre baiulo. \\ O qui magister terror est mortalium, \\ Diros ferarum qui retundit impelus, \\ Morsum repertis ut cibis non audeant! \\ 
        \pagebreak 
     \marginpar{[020]} \begin{center} \textbf{CARMINA} \end{center}01. \\ In psaltriam foedam \\ Cum saltas misero, Gattula, corpore \\ Nec cuiquam libitum est, horrida, quod facis, \\ Insanam potius te probo psaltriam, \\ Quae foedam faciem motibus ingraves \\ Et, dum displiceas, quosque feras iocos. \\ Credis, quod populos cymbala mulceant? \\ Nemo iudicium tale animi gerit, \\ Pro te ut non etiam gaudia deserat. \\ 70 \\ 
      \end{verse}
  
            \subsection*{50}
      \begin{verse}
      Item de ea quod ut amaretur praemia \\ promittebat \\ Qui facis, ut pretium promittens, Gattula, ameris? \\ Da pretium, ne te oderis ipsa simul! \\ Praemia cur perdis? cur spondes munera tantis? \\ Accipe tu pretium, ne mihi dona feras! \\ Non est tam petulans pariterque insanus amator, \\ Qui te non credat prodigiale malum. \\ Sed si forte aliquis moechus surrexit ab umbris, \\ Cui talis placeas, huic tua dona dato. \\ 
        \pagebreak 
    \begin{center} \textbf{CODICIS SALMASIANI.} \end{center} \marginpar{[283]} 
      \end{verse}
  
            \subsection*{363}
      \begin{verse}
      In ebriosam et satis meientem \\ Quod bibis  \lbrack et \rbrack  totum dimitis ab inguine Bacchum, \\ Pars tibi superior debuit esse femur. \\ Potabis recto (poteris, Follonia!) Baccbo, \\ Si parte horridius inferiore bibas. \\ 
      \end{verse}
  
            \subsection*{364}
      \begin{verse}
      (B. em III 222.3 \\ n mulierem pulcram eastitati studentem \\ Pulcrior et nivei cum sit tibi forma coloris, \\ Cuncta pudicitiae iura tenere cupis. \\ Mirandum est, quali naturam laude gubernes, \\ Moribus ut Pallas, corpore Cypris eas. \\ Te neque coniugii libet excepisse levamcn, \\ Saepius exoptas nolle videre mares. \\ Haec tamen est auimo quamvis exosa voluptas: \\ Numquid non mulier conparis esse potes? \\ 
      \end{verse}
  
            \subsection*{365}
      \begin{verse}
      De eo qui ceum Burdo diceretur fliae suae \\ Passiphaae nomen inposuit \\ Disciplinarum esse hominem risnsque capacem, \\ Quod nulli est pecudi, dixit Aristoteles. \\ 
        \pagebreak 
    \begin{center} \textbf{CARMNA} \end{center} \marginpar{[284]} Sed  \lbrack cum)Burdo homo sit, versum est sophismate verum: \\ Nam et ridere solet vel ratione viget. \\ Surrexit duplex nostro sub tempore monstrum, \\ Quod pater est burdo Passiphaeque redit. \\ 
      \end{verse}
  
            \subsection*{366}
      \begin{verse}
      \poemtitle{De laude rosae centumfoliae}Hanc puto de proprio tinxit Sol aureus ortu \\ Aut unum ex radiis maluit esse suis. \\ Sed si etiam centum foliis rosa Cypridis extat, \\ Fluxit in hanc omni sanguine tota Venus. \\ Haec florum sidus, haec Lucifer almus in agris, \\ Iuic odor et color est dignus honore poli. \\ 
      \end{verse}
  
            \subsection*{367}
      \begin{verse}
      De statua Heetoris in lio, uae videt Achillem \\ et sudat \\ lion in medium Pario de marmore facti \\ Stant contra Phrygius lector vel Graius Achilles. \\ Priamidae statuam sed verus sudor inundat \\ Et falsum fictus lector formidat Achillem. \\ Nescio quid mirum gesserunt Tartara saeclo: \\ Credo quod aut superi animas post funera reddunt \\ Aut ars mira potest legem mutare barathri. \\ 
        \pagebreak 
    \begin{center} \textbf{CODCIS SALMASMANI.} \end{center} \marginpar{[285]} Sed si horum nihil est, certe exstat marmore Hector \\ Testaturque suam viva formidine mortem. \\ 
      \end{verse}
  
            \subsection*{368}
      \begin{verse}
      \poemtitle{De muliere Marina vocabulo}Quidam concubitu futuit fervente Marinam. \\ Fluctibus in salsis fecit adulterium. \\ Non hic culpandus, potius sed laude ferendus, \\ Qui memor est Veneris, quod mare nata foret. \\ 
      \end{verse}
  
            \subsection*{369}
      \begin{verse}
      De horto domni Oageis, ubi omnes herbae \\ mediinales plantatae sunt \\ Constructas inter moles parietibus altis \\ lortus amoenus inest aptior et domino. \\ Hic vario frondes vitales semine crescunt, \\ ln quibus est Genio praemedicante salus. \\ Nl Phoebi Asclepique tenet doctrina parandum: \\ Omnibus hinc morbis cura sequenda placet. \\ Iam puto, quod caeli locus est, ubi numina rcegnant, \\ Cum datur his herbis vincere mortus onus. \\ 
      \end{verse}
  
            \subsection*{370}
      \begin{verse}
      De piea, quae humanas vocees imitabatur \\ Pica hominum voces cuncta ante animalia monstrat \\ Et docto exterum perstrepit ore melos. \\ p. 184 \\ 
        \pagebreak 
    \poemtitle{CARMNA}Nec nunc oblita est, quidnam prius esset in orbe: \\ Aut hae Picns erat aut homo rursus inest. \\ 4 \\ De rustica in diseo fact, quae spinam tollitc de \\ plnta Satri \\ Cauta nimis spinam Satyri pede rustica tollit, \\ Luminibus certis vulneris alta notans. \\ llam panduri solatur voce Cupido, \\ Inridens, parili teste carere virum. \\ Nl falsum credas artem lusisse figuris: \\ Viva minus speciem reddere membra solent. \\ 0 \\ 
      \end{verse}
  
            \subsection*{}
      \begin{verse}
      \poemtitle{De coloceasia herba in teeto populante}Nilus quam riguis parit fluentis, \\ Extendens colocasia †eorum \\ Ramos, per spatium virens amoenum, \\ laec nostris laribus creata frondet. \\ Naturam famulans opella vertit, \\ Plus tecto ut vigeat, solet quam horto. \\ 
      \end{verse}
  
            \subsection*{}
      \begin{verse}
      \poemtitle{De eo qui podium mphitheatri saliebat}Amphitheatralem podium transcendere saltu \\ Velocem audivi iuvenem, nec credere quivi \\ 
        \pagebreak 
    \begin{center} \textbf{COICIS SALMASIANI.} \end{center} \marginpar{[287]} Hunc hominem, potius sed avem, si talia gessit. \\ Et posui huic, fateor, me Dorica vina daturum, \\ Conspicere ut possem tanti nova facta laboris. \\ Aspexi victusque dedi promissa petenti \\ Atque meo gravior levis extitit ille periclo. \\ Non iam mirabar sumtis te, Daedale, pinnis \\ Isse per aetherios natura errante meatus: \\ Iunc magis obstipui, coram qui plebe videnti \\ Corpore, non pinnis, fastigia summa volavit. \\ 6 t \\ De Dicgene pieto, ubi laseivienti menetrix \\ barbam evellit et Cupido mingit in podiee eius \\ Diogenem meretrix derisum Laida monstrat \\ Barbatamque comam frangit amica Venus. \\ Nec virtus animi nec castae semita vitae \\ Philosophum revocat, turpiter esse virum. \\ Hoc agit infelix, alios quo saepe notavit. \\ Quodque nimis miserum est: mingitur archisophus. \\ 0 49 \\ De ceatto qui, eum sorieem maiorem devoraseet, \\ pOplexiam passus oceubouit \\ B. V 163. \\ M. 1091. \\ B. IV 25 \\ Deliciis periit crudior ille suis. \\ 
        \pagebreak 
     \marginpar{[288]} \begin{center} \textbf{CARMINA} \end{center}Pertulit adsuetae damnum per viscera praedae; \\ Per vitam moriens concipit ore necem. \\ 0 t0 \\ 
      \end{verse}
  
            \subsection*{}
      \begin{verse}
      \poemtitle{L.ORENTINI}V I \\ B. Lux. 85. \\ M. 290. \\ In laudem regis \\ B. IV 426. \\ Regia festa canam sollemnibus annua votis. \\ Imperiale decus Thrasamundi gloria mundi, \\ egnantis Libyae. toto sic clarior orbe \\ Sol radiante micans cunctis super enitet astris. \\ In quo concordant pietas prudentia mores \\ Virtus forma decus animus sensusque virilis, \\ Invigilans animo sollers super omnia †sensus. \\ Sed quid plura moror? vel quo me ad devia ducam? \\ Solus habet, toto quidquid praefertur in orbe. \\ Parthia quot radiat sublimibus ardua gemmis, \\ Lydia Pactoli rutilas quot sulcat harenas, \\ Vellera quot Seres tingunt variata colore, \\ Regnantum meritis pretioso praemia dantes \\ Tegmine, quo fulgent admisto murice vestes, \\ p. 166 \\ Africa quot fundit fructus splendentis olivae, \\ Et si quid tellus gignit laudata per orbem, \\ In regnis venere tuis, cui maximus auctor \\ Contulit et soli tribuit haec cuncta potiri. \\ 
        \pagebreak 
     \marginpar{[289]} \begin{center} \textbf{CODICIS SALMASIANI.} \end{center}Te regnante diu fulgent Carthaginis arces, \\ Filia quam sequitur Alianas inpare gressu, \\ Nec meritis nec honore minor, cui plurimus ardens \\ Regnantis increvit amor, quam surere fecit \\ Delectisque locis claram  \lbrack et vitalibus auris, \\ Quae meruit celsum meritis sufferre regentem. \\ linc freta marmoreo resonant sub gurgite ponti, \\ linc telluris opes viridanti vertice surgunt, \\ Vt maris et terrae dominus splendore fruatur. \\ Nam Carthago suam retinet per culmina laudem, \\ Carthago regimen; victrix Carthago triumphat, \\ Carthago Asdingis genetrix, Carthago coruscat, \\ Carthago excellens Libycas Carthago per oras, \\ Cartbago studiis, Carthago ornata magistris, \\ Carthago populis pollet, Carthago refulget, \\ Carthago in domibus, Carthago in moenibus ampla, \\ Carthago et dulcis, Carthago et nectare suavis, \\ Carthago lorens, Thrasamundi nomine regnans! \\ Cuius ut imperium maneat per saecula felix, \\ Optamus domino multos celebrare per annos \\ Annua, dum repetit fulgentia gaudia regui. \\ B. III 41. \\ 0 4 4 \\ M 384. \\ Versus balnearum \\ B. IV 427. \\ ln parvo magnas fecit manus ardua Baias, \\ Culmina distendit niveis snspensa columnis, \\ Curvavitcamerisvarioproschemateconchas.p. \\ 
        \pagebreak 
    \begin{center} \textbf{CARMNA} \end{center} \marginpar{[290]} Marmora resplendent solida conpage decora, \\ Vnitumque nitet decus  \lbrack et \rbrack  sartura magistri. \\ Murmure raucisono fornacibus aestuat ardor, \\ In flammis dominantur aquae, furit ignis anhelus. \\ Stat tutus lautor multo circumdatus igne \\ lnnocuas inter flammas (mirabile dictu), \\ Ignibus ut possit fervores vincere solis; \\ Inde petit latices glaciali fonte fuentis: \\ IHinc calor, inde nives reparant per membra salutem. \\ Discurrunt nymphae, piscinas flumina replent. \\ Vitrea crispatur labris refluentibus unda. \\ Haec laudis monumenta tibi natisque manebunt, \\ Et decorabit avus claros per saecla nepotes. \\ Tu tamen excelsus per tempora longa fruaris. \\ 0 40 \\ 
      \end{verse}
  
            \subsection*{}
      \begin{verse}
      \poemtitle{CALBVI GRAMMATICI}Versus fontis \\ B. . II p. 623. \\ M . \\ A parte episcopi \\ B. IV 28. \\ Crede prius veniens, Christi te fonte renasci: \\ Sic poteris mundus regna videre dei. \\ Tinctus in hoc sacro mortem non sentiet umquam; \\ Semper enim vivit, quem semel unda lavit. \\ 
        \pagebreak 
    \begin{center} \textbf{CODICIS SALMASIANI.} \end{center} \marginpar{[291]} Descensio fontis \\ Descende intrepidus: vitae fomenta perennis \\ Aeternos homines ista lavacra creant. \\ Ascensio fontis \\ Ascende in caelos, animam qui in fonte lavisti, \\ Idque semel factum sit tibi perpetuum. \\ Econtra epicopum \\ Peccato ardentes hoc fonte extinguite culpas. \\ Currite! quid statis? tempus et hora fugit. \\ p.I88 \\ Et in circuitu fontis \\ Marmoris oblati speciem, nova munera, supplex \\ Calbulus exhibuit. fontis memor, unde renatus, \\ Pr formam cervi gremium perduxit aquarum. \\ 0 49 \\ B. b. M. \\ Versus sancetae cruci \\ B. IV 29. \\ Hinc crux sancta potens caelo successit et astris: \\ Dum retinet corpus, misit in astra deum. \\ Qui fugis insidias mundi, crucis utere signis: \\ Hac armata fides protegit omne malum. \\ Crux domini mecum, crux est quam semper adoro, \\ Crux mihi refugium, crux mihi certa salus. \\ Virtutum genetrix, fons vitae, ianua caeli, \\ Crux Christi totum destruit hostis opus. \\ Pax Domini tecum, puro quam pectore quaeris. \\ 
        \pagebreak 
     \marginpar{[000]} \begin{center} \textbf{CARMNA} \end{center}
      \end{verse}
  
            \subsection*{380}
      \begin{verse}
      B. d Lux. 88. \\ M. 555. \\ \poemtitle{v e. DOMNI PETRI REERENDARII}IV I m \\ p. 92 \\ Versus in basiliea palatii sanctae Mariae \\ Qualiter intacta processit Virgine partus \\ Vtque pati voluit Natus, perquirere noli. \\ Haec nulli tractare licet, sed credere tantum. \\ 
      \end{verse}
  
            \subsection*{381}
      \begin{verse}
       \lbrack A ulcem \rbrack  \\ B M. B. 1. \\ Felices illos qui te genuere parentes, \\ Felicem solem qui te videt omnibus horis, \\ Felicem terram quam tu pede candida calcas, \\ Felices fascias cingentes corpus amatae, \\ Felices \lbrack que \rbrack  toros quibus, Dulcis, nuda recumbis! \\ Vt visco capiuntur aves, ut retibus apri, \\ Sic ego nunc Dulcis diro sum captus amore. \\ Vidi nec tetigi: video nec tangere possum. — \\ Totus in igne fui: non sum consumptus et arsi. \\ 
      \end{verse}
  
            \subsection*{382}
      \begin{verse}
       \lbrack Ad meretricem \rbrack  \\ B. M. B. i. \\ Post mille amplexus, post dulcia savia penem \\ p. 93 \\ Continiis laterum retortum suscipe, posce, \\ Viribus ut propriis mollem tu reddas, ab alvo \\ In alvum sumpltura, iterum qucm tempore certo \\ 
        \pagebreak 
    \begin{center} \textbf{CODICIS SALMASIANI.} \end{center} \marginpar{[293]} Et tunc, si suffert, tertio supplere conabor. \\ Nec volo plus cupias: nam me si cogis, iniquum est, \\ Vt tu victorem superes. noctisque futurae \\ Incipiant tenebrae, numerum qua spondeo ternum! \\ 
        \pagebreak 
    \begin{center} \textbf{DE SIN 6v.IS CAVSIS} \end{center}\begin{center} \textbf{r r} \end{center} \marginpar{[7]}  \marginpar{[I]} \begin{center} \textbf{p.2 3} \end{center}\poemtitle{ \lbrack VAE SVPERSsvNr)}Sunt uersus LXX \\ 
      \end{verse}
  
            \subsection*{383}
      \begin{verse}
      B. V 172. \\ M. 102. \\ \poemtitle{De alcyonibus}B. IV 432. \\ Stat domus exerrans constructa lymphis et ulva. \\ Fervet amor lymphis, liquescunt ova tepore. \\ Exposito partu natus non cernitur ullus. \\ p. 274 \\ Albus in albo manens rursus succendor ab albo, \\ Miraturque novos fetus consurgere ab undis \\ Alcyon, cui totus pelagus sunt parvoli nidus. \\ B. V 177. \\ M. 1107. \\ 
      \end{verse}
  
            \subsection*{}
      \begin{verse}
      \poemtitle{De Iuvenale venatore}. IV 32. \\ Excipit ingentem luvenalis fortior aprum. \\ Incumbens umero laevo pede pronior instat \\ Et spumantis adbuc morsum de vulnere fraudat. \\ 
        \pagebreak 
     \marginpar{[295]} \begin{center} \textbf{CODICIS SALMASIAN1;} \end{center}
      \end{verse}
  
            \subsection*{385}
      \begin{verse}
      B. V 1783. \\ M. 1108. \\ Item de apro \\ B. IV 33. \\ Invadunt post terga suem. stat torvus in ira: \\ Creticus excutitur lunato dente rebellis, \\ Quem socius morsu auxilians defendit in hostem. \\ 
      \end{verse}
  
            \subsection*{386}
      \begin{verse}
      . M. \\ \poemtitle{De Mandrite mimo}B. IV 433. \\ Mandris notus olim felix fur, cautus et audax, \\ Quattuor in medio dicit peccasse colonas. \\ ‘Sive ego sive lupus’, dixit, ‘tollatur et anser’. \\ 0 \\ B M. \\ \poemtitle{CAT0NIS}B. IV 33. \\ Rex Hunerix, manifesta fide quem fama perennis \\ lnclitat, hominibus spargit memorabile factum, \\ Quod verbo divisit aquas molemque profundi \\ Discidit iussis, semel  \lbrack ut \rbrack  nudata natantum \\ lugera calcet homo. pelagus fodisse ligones \\ Expavit natura maris: subducitur unda, \\ Tortilis anfractu liquidus converritur imber, \\ Oceanumque movent manibus; mare coclea sorbet. \\ 
        \pagebreak 
    \begin{center} \textbf{CARMINA} \end{center} \marginpar{[296]} 
      \end{verse}
  
            \subsection*{388}
      \begin{verse}
      B. I 124. \\ M. 692. \\ \poemtitle{De Pegaso ellerephonte et Chimaera}B. IV 433. \\ Vectum Pegaseo volucri pendente caballo \\ Conpita senserunt Bellerephonta denm. \\ Cuius iter vacuos et calcans ungula ventos \\ Rlia est. \\ 
        \pagebreak 
    \poemtitle{CARMNA}\poemtitle{CODICVM SAECVLI IX}
        \pagebreak 
    \begin{center} \textbf{M I} \end{center}\begin{center} \textbf{CAREN} \end{center}\poemtitle{CODICIS BEROLINENSIS DIE. 3 66}r \\ 
      \end{verse}
  
            \subsection*{388a}
      \begin{verse}
      B. M. \\  \lbrack Celeuma \rbrack  \\ B. III 167. \\ Heia, viri, nostrum reboans echo sonet heia! \\ Arbiter effusi late maris ore sereno \\ Placatum stravit pelagus posuitque procellam, \\ Edomitique vago sederunt pondere fluctus. \\ lleia, viri, nostrum reboans echo sonet heia! \\ Annisu parili tremat ictibus acta carina. \\ Nunc dabit arridens pelago concordia caeli \\ Ventorum motu praegnanti currere velo. \\ Heia, viri, nostrum reboans echo sonet heia! \\ Aequora prora secet delphinis aemula saltu \\ Atque gemat largum, promat seseque lacertis, \\ Pone trahens canum deducat  \lbrack et orbita sulcum. \\ Heia, viri, nostrum reboans echo sonet heia! \\ Aequoreos volvens fluctus  \lbrack ratis audiat \rbrack  heia! \\ Convulsum remis spumet mare; os tamen: heia. \\ Vocibus adsiduis litus re \lbrack duci \rbrack  sonet heia! \\ 
        \pagebreak 
    \begin{center} \textbf{CARMINA} \end{center}\begin{center} \textbf{VI} \end{center}
      \end{verse}
  
            \subsection*{}
      \begin{verse}
      \poemtitle{CODICIS PARISINI 8071}
      \end{verse}
  
            \subsection*{}
      \begin{verse}
      \poemtitle{OIM TIVANEI}
      \end{verse}
  
            \subsection*{389}
      \begin{verse}
      B. V 1. \\ M. 1021 q. \\ In laudem solis \\ B. IV 43. \\ Dum mundum Natura potens terramque dicaret, \\ Sol dedit ipse diem. horrentia nubila caelo \\ Dispulit et faciem roseo diffudit in orbe. \\ Pulcra serenigero fulserunt sidera motu. \\ Nam chaos est sine sole dies. tum discere lucem \\ Coepimus et croceum caeli sentire teporem. \\ Gurgite  \lbrack de \rbrack  roseo surgunt ex more iugales: \\ Naribus elatis eflant e pectore lucem. \\ Sol rumpit tenebras, rutiloque ut fulet ab ortu, \\ Spargit in aethereos flammantia lumina campos. \\ Hlaec armenta hominesque simul dant semina rerum, \\ Hinc alites,pecudumbinc \lbrack genusetlgenusomne natantum, \\ 
        \pagebreak 
    \begin{center} \textbf{CODICIS PARISNI 8071.} \end{center}\begin{center} \textbf{q} \end{center}Quod caelum, quod terra tenuet, quod sustinet aequor. \\ linc calor infusus, totum qui continet orbem, \\ Dulcia mellifluae dum pandit munera vitae. \\ Ast ubi iam Titau croceum conscendit in orbem \\ 2 Fluctibus ac nitidum tollit caput aethera in altum, \\ Cuncta patent, quaccunque tacens nox clauserat atra, \\ Et silvae campique virent et florea rura. \\ Tunc placidum iacet omne mare et vernantibus undis \\ Flumina: per tremulos currit lux aurea fluctus. \\ Sic regit imperium mundi, sic tempora sancit. \\ Mox tamen alipedum gemmautia lora rigescunt. . . \\ Aureus axis inest, currus ardescit ab auro, \\ Du pretio fulgens imitatur lumina Phoebi. \\ llc solus viget orbe dens, qtem cernere nobis \\ as nims es, cc iurat per florea rura. \\ O rm irtutis opus, quod flamma gubernat! \\ cc non igne suo praestat cum lumine sensus, \\ Iinc cops, hinc vita redit, hinc cuncta resurguut. \\ Namque docet Phoenix, ustis reparata favillis \\ Omnia Phoebeo vivesccre corpora tactu. \\ Iaec vitam de morte petit, post fata vigorem, \\ Vascitur ut pcreat, perit ut nascatur ab igni, \\ 
        \pagebreak 
     \marginpar{[302]} \begin{center} \textbf{CARMINA} \end{center}Vna cadit totiens surgitque ac deficit una: \\ Rupe sedet, capitur radiis, et lumine Phoebi \\ Suscipit inmissum recidiva morte calorem.. \\ Sol qui purpureo diffundit lumine terras, \\ Sol cui vernanti tellus respirat odorem, \\ Sol cui picta virent fecundo gramine prata, \\ Sol speculum caeli, divini numinis instar, \\ Sol semper iuvenis, rapidum qui dirigit axem, \\ Sol facies mundi caelique volubile templum, \\ Sol Liber, Sol alma Ceres, Sol Iuppiter ipse, \\ Sol frater Triviae, insunt cui numina mille, \\ Sol qui quadriiugo difundit lumina curru, \\ Sol et Hlyperboreo fulget matutinus in ortu, \\ Sol reddit cum luce diem, Sol pingit Olympum. \\ Sol aestas, autumnus, hiems, Sol ver quoque gratum, 4 \\ Sol saeclum mensisque, dies Sol, annus et hora, \\ Sol globus aethereus, Sol est lux aurea mundi. \\ Sol bonus agricolis, nautis quoque prosper in undis, \\ Sol repetit quaecumque potest transcendere semper. \\ Sol cui sereno pallescunt sidera motu, \\ Sol cui tranquillo resplendet lumine pontus, \\ Sol cui cuncta licet rpido lustrare calore, \\ 
        \pagebreak 
    \begin{center} \textbf{CODICIS PARISINI 8071.} \end{center} \marginpar{[303]} 9 Sol cui surgenti resonat levis unda canorem, \\ Sol cui mergenti servat maris unda teporem, \\ Sol mundi caelique decus, Sol omnibus idem, \\ C0 Sol noctis lucisque decus, Sol finis et ortus. \\ 
      \end{verse}
  
            \subsection*{390}
      \begin{verse}
      B. V 133. \\ M. 385. \\ \poemtitle{EVCIIERIAE}fl. 58 r. \\ B. V 361. \\ Aurea concordi quae fulgent fila metallo \\ Setarum cumulis consociare volo: \\ Sericeum tegmen, gemmantia texta Lacouum \\ Pellibus hircinis aequiperanda loquor: \\ Nobilis horribili iungatur purpura burrae, \\ Nectatur plumbofulgida gemma gravi: \\ 
        \pagebreak 
    \begin{center} \textbf{CARMNA} \end{center} \marginpar{[304]} Sit captiva sui nunc margarita nitoris \\ Et clausa obscuro fulgeat in chalybe: \\ Lingonico pariter claudatur in aere smaragdus, \\ Conpar silicibus nunc hyacinthus eat: \\ Rupibus atque molis similis dicatur iaspis, \\ Eligat infernum iam modo luna chaos: \\ Nunc etiam urticis mandemus lilia iungi, \\ Purpureamque rosam dira cicuta premat: \\ Nunc simul optemus despectis piscibus ergo \\ Delicias magni nullificare freti: \\ Auratam crassantus amet, saxatilis angucem, \\ Limacem pariter nunc sibi tructa petat: \\ Altaque iungatur vili cum vulpe leaena, \\ Perspicuam lyncem simius accipiat: \\ lungatur nunc cerva asino, nunc tigris onagro, \\ lungatur fesso concita damma bovi: \\ Nectareum vitient nunc lasera tetra rosatum \\ Mellaque cum fellis sint modo mixta malis: \\ Gemmantem sociemus aquam luteumque barathrum, \\ Stercoribus mixtis fons eat inriguus: \\ Praepes funereo cum vulture ludat hirundo, \\ Cum bubone gravi nunc philomela sonet: \\ 
        \pagebreak 
    \begin{center} \textbf{CODICIS PARISINI 8071.} \end{center} \marginpar{[305]} Tristis perspicua sit cum perdice cavannus \\ Iunctaque cum corvo pulcra columba cubet: \\ Iaec monstra incertis mutent sibi tempora fatis \\ Rusticus et servus sic petat Eucheriam! \\ B . \\ 
      \end{verse}
  
            \subsection*{391}
      \begin{verse}
      M. 874. \\ B. V 363. \\ Cervus aper coluber non cursu dente veneno \\ Vitarunt ictus, Maioriane, tuos. \\ 
        \pagebreak 
    \begin{center} \textbf{CARMINA} \end{center}\poemtitle{CODICIS VOSSIANI Q. 86}B. II 258. \\ Co. 3V \\ M. 210. \\ 
      \end{verse}
  
            \subsection*{392}
      \begin{verse}
      fol. 92 r. \\ B. IV 111. \\ Vt belli sonuere tubae, violenta peremit \\ lippolyte Teuthranta, Lyce Clonon, Oebalon Alce, \\ Oebalon ense, Clonon iaculo, Teuthranta sagitta. \\ Oebalus ibat equo, curru Clonus, at pede Teuthras, \\ Plus puero Teuthras, puer Oebalus, at Clouus heros. \\ Figitur ora Clonus, latus Oebalus, ilia Teuthras. \\ 
        \pagebreak 
    \begin{center} \textbf{CODICIS VOSSIANI Q. 86.} \end{center} \marginpar{[307]} Iphicli Teuthras, Dorycli Clonus, Oebalus ldae; \\ Argolicus Teuthras, Moesus Clonus, Oebalus Arcas. \\ B. II 257. \\ 
      \end{verse}
  
            \subsection*{393}
      \begin{verse}
      M. 253. \\ B. IV 112. \\ Almo Theon Thyrsis orti sub colle Pelori \\ Semine disparili: Laurente Lacone Sabina. \\ Vite Sabina, Lacon sulco, sue cognita Laurens. \\ Thyrsis oves, vitulos Theon egerat, Almo capellas, \\ Almo puer pubesque Theon et Thyrsis ephebus; \\ Canna Almo, Thyrsis stipula, Theon ore melodus. \\ Nais amat Thyrsin, Glauce Almona, Nisa Theonem; \\ Nisa rosas, Glauce violas dat, lilia Nais. \\ fol. 92 u. \\ 
      \end{verse}
  
            \subsection*{394}
      \begin{verse}
      Versu de †numero dierum ingulorum  \lbrack  \\ M. 1037. \\ menium \\ B. I 205. \\ Dira patet Iani Romanis ianua bellis. \\ Vota deo Diti Februa mensis habet. \\ 
        \pagebreak 
     \marginpar{[308]} \begin{center} \textbf{CARMNA} \end{center}Incipe, Mars, anni felicia fata reducti. \\ Tunc Aries Veneri lutea serta legit. \\ Dulcia, Maie, tuis ducis hexagona nonis. \\ Arce poli Geminos Iunius ecce locat. \\ Iulius ardenti divertit lumina soli. \\ Aera flammigero cuncta Leone calent. \\ Poma legit Virgo maturi mitia solis. \\ Fundit et October vina Falerna lacis. \\ Aret tota soli species vi dura Nepai. \\ Vde December, amat te genialis hiems. \\ 
        \pagebreak 
    \begin{center} \textbf{CODICIS VOSSIANI Q. 86.} \end{center}
      \end{verse}
  
            \subsection*{395}
      \begin{verse}
      M. 10 \\ singulis mensibus \\ Ianuarius \\ lic Hani mensis sacer est (en aspice ut aris \\ Tura micent, sumant ut pia iura Lares), \\ Annorum saeclique caput, natalis honorum, \\ Purpureos fastis qui numerat proceres. \\ Februariusfol.93r \\ At quem caeruleus nodo constringit amictus \\ Quique paludicolam prendere gaudet avem, \\ Daedala quem iactu pluvio circumvenit Iris, \\ Romuleo ritu Februa mensis habet. \\ Martius \\ Cinctum pelle lupae promptum est cognoscere mensem: \\ Mars olli nomen, Mars dedit exuvias. \\ Tempus ver \lbrack num \rbrack  aedus petulans et garrula hirundo \\ Indicat et sinus lactis et herba virens. \\ Aprilis \\ Contectam myrto Venerem veneratur Aprilis: \\ Lumen veris habet, quo nitet alma Thetis. \\ 
        \pagebreak 
    
      \end{verse}
  
            \subsection*{}
      \begin{verse}
      \poemtitle{CARMNA}Cereus et dexra flammas diffundit odoras, \\ Balsama nec desunt, quis redolet Paphie. \\ Maius \\ Cunctas veris opes et picta rosaria gemmis \\ Liniger in calathis, aspice, Maius habet, \\ Mensis Atlantigenae dictus cognomine Maiae, \\ Quem merito multum diligit Vranie. \\ Iunius \\ Nudus membra dehinc solares respicit horas \\ Iunius ac Phoebum flectere monstrat iter. \\ Lampas maturas Cereris designat aristas \\ Floralisque fugas lilia fusa docent. \\ Iulius \\ Ecce coloratos ostentat Iulius artus, \\ Crines cui rutilos spicea serta ligat. \\ Morus sanguineos praebet gravidata racemos, \\ Quae medio Cancri sidere laeta viret. \\ Augustus \\ Fontanos latices et lucida pocula vitro \\ Cerne ut demerso torridus ore bibat \\ Aeterno regni signatus nomine mensis, \\ Latona genitam quo perhibent Hecaten. \\ 
        \pagebreak 
    \begin{center} \textbf{CODICIS VOSSIANI Q. 86.} \end{center} \marginpar{[311]} September \\ Turgentes acinos varias et praesecat uvas \\ September, sub quo mitia poma iacent, \\ Captivam filo gandens religasse lacertam, \\ Quae suspensa manu mobile ludit opus. \\ Oetob s. \\ Dat prensum leporem cumque†ipso palite fetus \\ O0cober; piois dat ubi ru aes ?. 7 3 \\ Iam Bromios spumare lacus et musta sonare t.† \\ Apparet: vino vas calet ecce novo. \\ November \\ Carbaseos post hunc artus indutus amictus \\ Memphidos antiquae sacra deamque colit, \\ quo vix avidus sistro compescitur anser \\ Devotusque satis incola Memphideis. \\ December \\ Annua sulcatae †coniecti semina terrae \\ Pascit hiems; Pluvio de Iove cuncta madent. \\ Aurea nunc revocet Saturno festa December: \\ Nunc tibi cum domino ludere, verna, licet. \\ 
        \pagebreak 
    \begin{center} \textbf{CARMINA} \end{center} \marginpar{[312]} 
      \end{verse}
  
            \subsection*{396}
      \begin{verse}
      B. III 151. \\ M. 134. \\ Parcendum misero \\ B. IV 57. \\ Occisum iugulum quisquis scrutaris †amici, \\ Tu miserum necdum me satis esse putas? \\ Desere confossum victori vulnus iniquo \\ Mortiferum impressit mortua saepe manus. \\ B. II 48. \\ M. 744. \\ Mors Catonis \\ fol. 93 u. \\ B. IV 58. \\ Invictum victis in partibus, omnia Caesar \\ Vincere qui potuit, te, Cato, non potuit. \\ 
      \end{verse}
  
            \subsection*{398}
      \begin{verse}
      B. II 49. \\ M. 745. \\  \lbrack Iteml \\ B. IV 58. \\ Ictu non potuit primo Cato solvere vitam: \\ Defecit tanto vulnere victa manus. \\ Altius inseruit gladium: qua spiritus ingens \\ Exiret, magnum dextera fecit iter. \\ Opposuit Fortuna moram voluitque, Catonis \\ Sciremus ferro plus valuisse manum. \\ 
        \pagebreak 
    \begin{center} \textbf{CODICIS VOSSIANI Q. 86.} \end{center}
      \end{verse}
  
            \subsection*{399}
      \begin{verse}
      B. II 50. 51. \\ M. 746. \\  \lbrack Iteml \\ B. IV 58. \\ Iussa manus sacri pectus violare Catonis \\ Haesit et inceptum victa reliquit opus. \\ Ille ait infesto contra sua vulnera vultu: \\ ‘Estne aliquid magnum, quod Cato non potuit? \\ Dextera, nunc dubitas? durum est, iugulasse Catonem? \\ Sed quia liber erit, iam puto non dubitas! \\ Fas non est vivum cuiquam servire Catonem: \\ Rectius et vivit nunc Cato, si moritur’. \\ 
      \end{verse}
  
            \subsection*{400}
      \begin{verse}
      B. II 26. \\ M. 732. \\ Epitaphion Pompeiorum \\ B. IV 59. \\ Magne, premis Libyam; fortes tua pignera nati \\ Europam atque Asiam. nomina tanta iacent! \\ 
      \end{verse}
  
            \subsection*{401}
      \begin{verse}
      B. II 27. \\ M. 733. \\  \lbrack Item \rbrack  \\ B. b. \\ Quam late vestros duxit Fortuna triumphos, \\ Tam late sparsit funera, Magne, tua. \\ 
        \pagebreak 
    \begin{center} \textbf{CARMINA} \end{center} \marginpar{[314]} 
      \end{verse}
  
            \subsection*{402}
      \begin{verse}
      B. II 28 q. \\ M. 734 q. \\  \lbrack Item \rbrack  \\ B. IV 59. \\ Pompeius totum victor lustraverat orbem, \\ At rursus toto victus in orbe iacet: \\ Membra pater Libyco posuit male tecta sepulcro; \\ Filius Hispana est vix adopertus humo; \\ Sexte, Asiam sortite tenes. divisa ruina est: \\ Vno non potuit tanta iacere solo. \\ 
      \end{verse}
  
            \subsection*{403}
      \begin{verse}
      B. II 30. \\ M. 736. \\  \lbrack Item \rbrack  \\ B. ib. \\ Aut Asia aut Europa tegit aut Africa Magnum. \\ Quanta domus, toto quae iacet orbe, fuit! \\ 
      \end{verse}
  
            \subsection*{404}
      \begin{verse}
      B. II 31. \\ M. 737. \\  \lbrack Iteml \\ B. ib \\ Maxima civilis belli iactura †sub ipso est: \\ Quantus quam parvo vix tegeris tumulo! \\ 
      \end{verse}
  
            \subsection*{405}
      \begin{verse}
      B. III 152. \\ M. 135. \\ Ad amicum optimum \\ B. IV 60. \\ Crispe, meae vires laesarumque ancora rerum, \\ Crispe vel antiquo conspiciende foro, \\ Crispe, potens numquam, nisi cum prodesse volebas, \\ Naufragio litus tutaque terra meo, \\ 
        \pagebreak 
    \begin{center} \textbf{CODICIS VOSSIANI . 86.} \end{center} \marginpar{[315]} Solus honor nobis, arx et tutissima nobis \\ Et nunc afflicto sola quies animo, \\ Crispe, fides dulcis placideque acerrima virtus, \\ Cuius Cecropio pectora melle madent, \\ Maxima facundo vel avo vel gloria patri, \\ Quo solo careat si quis, in exilio est: \\ Intactae iaceo saxis telluris adhaerens, \\ Mens tecum est, nulla quae cohibetur humo. \\ 
      \end{verse}
  
            \subsection*{406}
      \begin{verse}
      B. II 35. \\ M. 741. \\ \poemtitle{De sacris evocantis animas Magnorum Vc}Fata per humanas solitus praenoscere fibras \\ Impius infandae religionis apex \\ Pectoris ingenui salientia viscera flammis fol. 94 r. \\ Imposuit; magico carmine rupit humum, \\ Ausus ab Elysiis Pompeium ducere campis! \\ Pro pudor, hoc sacrum Magnus ut aspiceret! \\ Stulte, quid infernis Pompeium quaeris in umbris? \\ Non potuit terris spiritus ille premi! \\ 
      \end{verse}
  
            \subsection*{407}
      \begin{verse}
      B. III 66 68. \\ M. 907 902. \\ \poemtitle{De vita humiliori}B. IV 61. \\ Vive et amicitias regum fuge’. pauca monebas: \\ Maximus hic scopulus, non tamen unus, erat. \\ 
        \pagebreak 
     \marginpar{[316]} \begin{center} \textbf{CARMINA} \end{center}Vive et amicitias nimio splendore nitentes \\ Et quicquid colitur perspicuum, fugito! \\ Ingentes dominos et famae nomina clarae \\ (90sI \\ Inlustrique graves nobilitate domos \\ Devita et longe tenuis cole; contrahe vela \\ Et te litoribus cymba propinqua vehat. \\ In plano semper tua sit fortuna paresque \\ Noveris: ex alto magna ruina venit. \\ Non bene cum parvis iunguntur grandia rebus: \\ Stantia namque premunt, praecipitata ruunt. \\ 
      \end{verse}
  
            \subsection*{408}
      \begin{verse}
      B. III 69. \\ M. 910. \\ Ixeml \\ B. IV 61. \\ Vice et amicitias omnes fge’: verius hoc est, \\ Quam ‘regum’ solas ‘efuge amicitias’. \\ Est mea sors testis: maior me afflixit amicus \\ Deseruitque minor. Turba cavenda simul. \\ Nam quicumque pares fuerant, fugere fragorem \\ Necdum conlapsam deseruere domum. \\  \lbrack I nunc et reges tantum fuge! Vivere doctus \\ Vni vive tibi; nam moriere tibi. \\ 
      \end{verse}
  
            \subsection*{409}
      \begin{verse}
      B. III 11. \\ M. 129. \\ \poemtitle{De se ad patriam}B. IV 62. \\ Corduba, solve comas et tristes indue vultus; \\ Inlacrimans cineri munera mitte meo. \\ 
        \pagebreak 
     \marginpar{[317]} \begin{center} \textbf{CODICIS VOSSIANM Q. 86.} \end{center}Nunc longinqua tuum deplora, Corduba, vatem, \\ Corduba non alio tempore maesta magis: \\ Tempore non illo, quo versis viribus orbis \\ Incubuit belli tota ruina tibi, \\ Cum geminis oppressa malis utrimque peribas \\ Et tibi Pompeius, Caesar et hostis erat; \\ Tempore non illo, quo ter tibi funera centum \\ Heu nox una dedit, quae tibi summa fuit; \\ Non, Lusitanus quateret cum moenia latro, \\ Figeret et portas lancea torta tuas. \\ llle tuus quondam magnus, tua gloria, civis \\ Infigor scopulo! Corduba, solve comas \\ Et gratare tibi, quod te natura supremo \\ Adluit Oceano: tardius ista doles! \\ 
      \end{verse}
  
            \subsection*{410}
      \begin{verse}
      B. III 153. \\ M. 137. \\ \poemtitle{De custodia sepuleri}B. IV 62. \\ Quisquis es (et nomen dicam? dolor omnia cogit!), \\ Qui nostrum cinerem nunc, inimice, premis \\ Et non conutentus tantis subitisque ruinis \\ Stringis in extinctum tela cruenta caput: \\ Crede mihi, vires aliquas natura sepulcris \\ Attribuit: tumulos vindicat umbra suos. \\ psos crede deos hoc nunc tibi dicere, livor, \\ Hoc tibi nunc Manes dicere crede meos: \\ Res est sacra miser; noli mea tangere fata: \\ Sacrilegae bustis abstinuere manus! \\ 
        \pagebreak 
    \begin{center} \textbf{CARMNA} \end{center} \marginpar{[318]} 
      \end{verse}
  
            \subsection*{411}
      \begin{verse}
      B. III 2. \\ M. 877. \\ \poemtitle{De Athenis}B. IV 63. \\ Quisquis Cecropias hospes cognoscis Athenas, \\ Quae veteris famae vix tibi signa dabunt, \\ ‘Hasne dei’ dices ‘caelo petiere relicto? \\ legnaque partitis haec sua cura deis?’ \\ Idem Agamemnonias dices cum videris arces: \\ ‘Heu victrix victa vastior urbe iacet!’ \\ Iae sunt, quas merito quondam est mirata vetustas: \\ Magnarum rerum magna sepulcra vides. \\ 10 \\ 
      \end{verse}
  
            \subsection*{412}
      \begin{verse}
      B. III 154. \\ M. 138. \\ In eum qui maligne ioetur \\ fol. 94 u. \\ B. IV 63. \\ Carmina mortifero tua sunt suffusa veneno, \\ At sunt carminibus pectora nigra magis. \\ Nemo tuos fugiet, non vir non femina, dentes; \\ Haut puer, haut aetas undique tuta senuis. \\ Vtque furens totas inmittit saxa per urbes \\ In populum, sic tu verba maligna iacis. \\ Sed solet insanos populus conpescere sanus \\ Et repetunt notum saxa remissa caput. \\ In te nunc stringit nullus non carmina vates \\ Inque tuam rabiem publica Musa furit. \\ Dum sua compositus nondum bene concutit arma \\ Miles, it e nostra lancea torta manu. — \\ 
        \pagebreak 
    \begin{center} \textbf{CODCIS VOSSIANI . 86.} \end{center} \marginpar{[319]} Bellus homo es? Valide capitalia crimina ludis \\ Deque tuis manant atra venena iocis! \\ Sed tu per \lbrack que iocum dicis vinumque? Quid ad rem, \\ Si plorem, risus si tuus ista facit? \\ Quare tolle iocos: non est iocus, esse malignum. \\ Numquam sunt grati, qui nocuere, sales. \\ 
      \end{verse}
  
            \subsection*{413}
      \begin{verse}
      B. II 3. \\ M. 742. \\ \poemtitle{De insepultis elaris}B. IV 64. \\ Litore diverso Libyae clarissima longe \\ Nomina vix ullo condita sunt tumulo, \\ Magnus et hoc Magno maior Cato. quam procul a te \\ Aspicis heu cineres, Roma, iacere tuos! \\ 
      \end{verse}
  
            \subsection*{414}
      \begin{verse}
      B. II 37. \\ M. 77. \\ \poemtitle{P. TERETII VARRONIS ATACINI .}u \\ Marmoreo Licinus tumulo iacet, at Cato nullo, \\ Pompeius parvo. credimus esse deos? \\ 
      \end{verse}
  
            \subsection*{414a}
      \begin{verse}
      B. M. ib. \\  \lbrack Responsum \rbrack  \\ B. IV 65. \\ Saxa premunt Licinum, levat altum fama Catonem, \\ Pompeium tituli: credimus esse deos. \\ 
        \pagebreak 
     \marginpar{[320]} \begin{center} \textbf{CARMINA} \end{center}
      \end{verse}
  
            \subsection*{415}
      \begin{verse}
      B. III e2. \\ M. 932. \\ \poemtitle{De spe}B. IV 65. \\ (Quertur per exempl \\ Spes fallax, spes dulce malum, spes summa malorum, \\ Solamen miseris, quos sua fata trahuut, \\ Credula res, quam nulla potest fortuna fugare, \\ Spes stat in extremis officiosa malis. \\ Spes vetat aeternis Mortis requiescere portis \\ Et curas ferro rumpere sollicitas. \\ Spes nescit vinci, spes pendet tota futuris; \\ Mentitur, credi vult tamen illa  \lbrack sibi \rbrack . \\ Improba, mentis inops, rebus gratissima laesis, \\ Quas fovet et verti fata subinde docet. \\ Sola tenet miseros in vita, sola moratur, \\ Sola perit numquam, nec venit atque redit. \\ Saepe bono rursusque malo blandissima saepe  \lbrack est \rbrack ; \\ Et quos decepit, decipit illa tamen. \\ Instabilis, dubio devexa ad tempora motu, \\ Audax et clausum quae putet esse nihil, \\ Omnia promittit nota levitate deorum; \\ Nil fixum et casus admonet esse leves. \\ 
        \pagebreak 
    \begin{center} \textbf{CODICIS VOSSIANI Q. 86.} \end{center} \marginpar{[201]} Naufragus hac cogente natat per feta procellis \\ Aequora, cum mersas viderit ante rates; \\ Captivus duras illa solante catenas \\ Perfert et victus vincere posse putat; \\ Noxius infami districtus stipite membra \\ Sperat et a fixa posse redire cruce. \\ Spem iussus praebere caput paloque ligatus, \\ Cum micat ante oculos stricta securis, habet. \\ Sperat et in saeva victus gladiator harena, \\ Sit licet infesto pollice turba minax: \\ Et cui descendit iugulato in pectora mucro, \\ Spem, quamuis lecto iam referatur, habet. \\ Spem recipit carcer foribus praeclusus aenis, \\ Spes et in horrendo robore parva manet. \\ Spes Marium movit, turpi se credere limo \\ Et tantum furto vivere velle virum; \\ Haec illum Libyae penetrare in litora victae \\ Iussit; et, o superi, quis fuit ille dies, \\ Quo Marium vidit suppar Carthago iacentem! \\ Tertia par illis nulla ruina fuit. \\ fol. 9 r. \\ Spes lagnum profugum toto discurrere in orbe \\ Iusserat et pueri regis adire pedes. \\ Spes uni numquam potuit dare verba Catoni, \\ Mendacisque deae non tulit ille dolos. \\ Quid non spes audet? Priamo post Iectora mansit; \\ Spes fuit uxori, Protesilae, tuae. \\ 
        \pagebreak 
    \begin{center} \textbf{CARMNA} \end{center} \marginpar{[322]} Orpheus infernas speravit tollere leges \\ Tartareum et cantu flectere posse canem. \\ Spe duce per medias enavit Daedalus auras \\ Et nova mirantes terruit ales aves. \\ Passiphae (quid non homini sperare licebit?) \\ Speravit torvo posse placere bovi. \\ Sperat qui curvo sulcos perrumpit aratro, \\ Sperat qui ventis vela ferenda dedit. \\ Spes hamis pisces, laqueo captare volucres \\ Erudit; haec orbem bella cruenta docet. \\ Spes sequitur gravibus rastris mala rura domantem, \\ In nova se nulla cum ratione parat. \\ Semper adulatur, semper male fida vagatur \\ Et populos urbes totaque regna capit. \\ Desertos medicis spes numquam deserit aegros, \\ Confessi numquam spem posuere rei. \\ Spes est, quae classis diverso ex hoste coactas \\ Ducit; spes cupidos tollit in arma viros. \\ Spes dicit ‘dura! nec te praesentia tangant: \\ Fors varias mutat mobilitate vices’. \\ Incerto ludit casu Fortuna per orbem: \\ Spes semper constat, nec fugit atque redit. \\ 
      \end{verse}
  
            \subsection*{416}
      \begin{verse}
      B. III 159. \\ M. 941. \\ Ad malivolum \\ B. IV 68. \\ Invisus tibi sum: peream si, Maxime, miror. \\ Odi te et, si vis, accipe cur faciam. \\ 
        \pagebreak 
    \begin{center} \textbf{CODICIS VOSSIANI Q. 86.} \end{center} \marginpar{[02]} Famam temptasti nostram sermone maligno \\ Laedere fellitis, invidiose, iocis. \\ Contra rem nuper pugnasti, livide, parvam: \\ Tu tamen in magna te nocuisse putas. \\ Hae peream nisi sunt animi mihi, Maxime, causae: \\ Odi, nec mentem res magis ulla iuvat, \\ Inque vicem ut facias oro pereoque timore, \\ Ne minus invisus sim tibi quam videor. \\ 
      \end{verse}
  
            \subsection*{417}
      \begin{verse}
      B. II 261. \\ M. 187. \\ Memoriam litteris permanere † \\ Haec urbem circa stulti monumenta laboris \\ Quasque vides moles, Appia, marmoreas, \\ Pyramidasque ausas vicinum attingere caelum, \\ Pyramidas, medio quas fugit umbra die, \\ Et Mausoleum, miserae solacia mortis, \\ Intulit externum quo Cleopatra virum, \\ Concutiet sternetque dies, quoque altius extat \\ Quodque opus, hoc illud carpet edetque magis. \\ Carmina sola carent fato mortemque repellunt; \\ Carminibus vives semper, lomere, tuis. \\ 
      \end{verse}
  
            \subsection*{418}
      \begin{verse}
      B. II 262. \\ M. 851. \\  \lbrack Item \rbrack  \\ B. IV 6. \\ Nullum opus exsurgit, quod non annosa vetustas \\ Expugnet, quod non vertat iniqua dies, \\ 
        \pagebreak 
    \begin{center} \textbf{CARMINA} \end{center} \marginpar{[324]} Tu licet extollas magnos ad sidera montes \\ Et calidas aeques marmore pyramidas. \\ Ingenio mors nulla nocet, vacat undique tutum; \\ Inlaesum semper carmina nomen habent. \\ 
      \end{verse}
  
            \subsection*{419}
      \begin{verse}
      B. II 84. \\ M. 762. \\ Laus Caesaris \\ B. IV 69. \\ Ausoniis numquam tellus violata triumphis \\ Icta tuo, Caesar, fulmine procubuit \\ Oceanusque tuas ultra se respicit aras: \\ Qui finis mundo est, non erat imperio. \\ 
      \end{verse}
  
            \subsection*{420}
      \begin{verse}
      B. II 85. \\ M. 763. \\  \lbrack Item \rbrack  \\ B. ib. \\ Victa prius  \lbrack nulli \rbrack , nullo spectata triumpho \\ Inlibata tuos gens patet in titulos. \\ Fabula visa diu medioque recondita ponto ol. 95 . \\ Libera victori quam cito colla dedit! \\ 
      \end{verse}
  
            \subsection*{421}
      \begin{verse}
      B. ib5. \\ M. 1b. \\  \lbrack Item \rbrack  \\ B. 1b. \\ Euphrates ortus, lhenus secluserat Arctos: \\ Oceanus medium venit in imperium. \\ 
        \pagebreak 
    \begin{center} \textbf{CODICIS VOSSIANI . 86.} \end{center} \marginpar{[325]} 
      \end{verse}
  
            \subsection*{422}
      \begin{verse}
      B. II 86. \\ M. 764. \\  \lbrack Iteml \\ B. IV 70. \\ Libera, non hostem, non passa Britannia regem \\ Externum, nostro quae procul orbe iaces, \\ Felix adversis et sorte oppressa secunda  \lbrack es \rbrack : \\ Communis nobis et tibi Caesar erit! \\ 
      \end{verse}
  
            \subsection*{423}
      \begin{verse}
      B. II 87. \\ M. 765. \\  \lbrack Item \rbrack  \\ B. IV 70. \\ Vltima cingebat Thybris tua, Romule, regna: \\ Hic tibi finis erat, religiose Numa. \\ Et tua, Dive, tuo sacrata potentia caelo \\ Extremum citra constitit Oceanum. \\ At nunc Oceanus geminos interluit orbes; \\ Pars est imperii, terminus ante fuit. \\ 
      \end{verse}
  
            \subsection*{424}
      \begin{verse}
      B. II 88. \\ M. 766. \\  \lbrack Item \rbrack  \\ B. IV 70. \\ Mars pater et nostrae gentis tutela Quirine \\ Et magno positus Caesar uterque polo, \\ Cernitis ignotos Latia sub lege Britannos: \\ Sol citra nostrum flectitur imperium. \\ Vltima cesserunt adaperto claustra profundo \\ Et iam Romano cingimur Oceano. \\ 
      \end{verse}
  
            \subsection*{425}
      \begin{verse}
      B. II 89q. \\ M. 767 0. \\  \lbrack Item \rbrack  \\ . IV 70. \\ Opponis frustra rapidum, Germania, Rhenum; \\ Euphrates prodest nil tibi, Parthe fugax; \\ 
        \pagebreak 
    \begin{center} \textbf{CARMINA} \end{center} \marginpar{[326]} Oceanus iam terga dedit, nec pervius ulli \\ Caesareos fasces imperiumque tulit: \\ Illa procul nostro semota exclusaque caelo \\ Alluitur nostra victa Britannis aqua. \\ 
      \end{verse}
  
            \subsection*{426}
      \begin{verse}
      B. II 91. \\ M. 769. \\  \lbrack Item \rbrack  \\ B. IV 71. \\ Semota et vasto disiuncta Britannia ponto \\ Cinctaque inaccessis horrida litoribus, \\ Quam pater invictis Nereus velaverat undis, \\ Quam fallax aestu circuit Oceanus, \\ Brumalem sortita polum, qua frigida semper \\ Praefulget stellis Arctos inocciduis, \\ Conspectu devicta tuo, Germanice Caesar, \\ Subdidit insueto colla premenda iugo. \\ Aspice, confundat populos ut pervia Tetbys: \\ Coniunctum est, quod adhuc orbis et orbis erat. \\ 
      \end{verse}
  
            \subsection*{4}
      \begin{verse}
      B . M. \\ \poemtitle{De voluptate adsidua per noetem}Sic et ames, mea lux, et rursus semper ameris, \\ Mutuus ut nullo tempore cesset amor. \\ Solis ad occasus, solis sic . ad orts \\ Hesperus hoc videat, Lucifer hoc videat. \\ 
        \pagebreak 
    \begin{center} \textbf{CODICIS VOSSIANI Q. 86.} \end{center} \marginpar{[327]} Si das saepe negas, si das saepe rc . . nis \\ Et pede preda tuo . atier re! . o . res \\ Nox mihi tota data est hite dampna peribunt \\ Hultac nimis debet timodius . \\ . S \\ Promittis, mea vita, semel non amplius una \\ .lisid..od. \\ Dum iaceam tecum permixtus corpore toto \\ Io.fito..re. \\ Nempeq.acd..m2ille \\ 
      \end{verse}
  
            \subsection*{}
      \begin{verse}
      \poemtitle{. 1 .}. 1i \\ m . \\ Lentus..rte \\ . p . \\ Femina cara semel, sed sine fine semel. \\ O certe numero vinces me . rera . t i \\ . . l . emper ego. \\ 
      \end{verse}
  
            \subsection*{428}
      \begin{verse}
      B. II 248. \\ M. 343. \\ \poemtitle{De tbns amicis bonis}B. IV 72. \\ Serranum Vegetumque simul iunctumque duobus \\ Herogenem, caros aspice Geryonas. \\ Esse putas fratres: tanta pietate fruuntur. \\ lmmo neges: sic est in tribus unus amor. \\ Triga mihi paucos inter dilecta sodales, \\ Triga sodalicii pars bene magna mei! \\ 
      \end{verse}
  
            \subsection*{429}
      \begin{verse}
      B. III 192. \\ M. 965. \\ ol. 96 . uod non severa carmina soribat † \\ Iam libet ad lusus lascivaque furta reverti. \\ Ludere, Musa, iuvat: Musa severa, vale! \\ 
        \pagebreak 
    \begin{center} \textbf{CARMINA} \end{center} \marginpar{[328]} Iam mihi narretur tumidis Arethusa papillis \\ Nunc astricta comas nunc resoluta comas; \\ Vt modo nocturno pulset mea limina signo \\ Intrepidos tenebris ponere docta pedes, \\ Nunc collo molles circum diffusa lacertos \\ Inflectat niveum semisupina latus, \\ Inque modos omnes, dulcis imitata tabellas, \\ Transeat et lecto pendeat illa meo, \\ Nec pudeat quicquam, sed me quoque nequior ipso \\ Exultet toto nou requieta toro. \\ Non deerit, Priamum qui defleat, Hectora narret: \\ Ludere, Musa, iuvat: Musa severa, vale! \\ 
      \end{verse}
  
            \subsection*{430}
      \begin{verse}
      iber IIII \\ B. III 22. \\ M. 996. \\ \poemtitle{ \lbrack De puero amto;}B. IV 73 \\ 0 sacros vultus Baccho vel Apolline dignos, \\ Quos vir, quos tuto femina nulla videt! \\ O digitos, quales pueri vel virginis esse \\ Vel potius credas virginis esse dei! \\ Felix, si qua tuum conrodit femina collum, \\ Felix, quae labris livida labra facit, \\ Quaeque puella tuo cum pectore pectora ponit \\ Et linguam tenero lassat in ore suam. \\ 
        \pagebreak 
    \begin{center} \textbf{CODICIS VOSSIANI Q. 86.} \end{center} \marginpar{[329]} 
      \end{verse}
  
            \subsection*{431}
      \begin{verse}
      B. III 153. \\ M. 966. \\ Excusatio insanioris mteriae \\ B. IV 73. \\ Esse tibi videor demens, quod carmina nolim \\ Scribere patricio digna supercilio? \\ Quod Telamoniaden non aequo iudice victum \\ Praeteream et pugnas, Penthesilea, tuas? \\ Quod non aut magni scribam primordia mundi \\ Aut Pelopis currus aut Diomedis equos? \\ Aut  \lbrack ut \rbrack  Acbilleis infelix Troia lacertis \\ Quassata lectoreo vulnere conciderit? \\ Vos mare temptetis, vos detis lintea ventis: \\ Me vehat in tuto parva carina lacu. \\ 
      \end{verse}
  
            \subsection*{432}
      \begin{verse}
      B. II 53. \\ M. 748. \\ \poemtitle{De sepulcro Catonis}B. IV 74. \\ Ne mirere sacri deformia busta Catonis: \\ Visuntur magni parva sepulcra lovis. \\ B. III 60. \\ M. 906. \\ \poemtitle{De bono vite humilioris}B. IV 74. \\ Est mihi rus parvum, fenus sine crimine parvum; \\ Sed facit haec nobis utraque magna quies. \\ Pacem animus nulla trepidus formidine servat \\ Nec timet ignavae crimina desidiae. \\ Castra alios operosa vocent sellaeque curules \\ Et quicquid vana gaudia mente movet. \\ 
        \pagebreak 
     \marginpar{[330]} \begin{center} \textbf{CARNA} \end{center}Pars ego sim plebis, nullo conspectus honore, \\ Dum vivam, dominus temporis ipse mei. \\ 
      \end{verse}
  
            \subsection*{434}
      \begin{verse}
      B. III 194. \\ M. 967. \\ . Excusat quod amoribus serviat \\ Insanus vobis videor. nec deprecor ipse, \\ Quominus hoc videar. cur tamen hoc videor? \\ Dicite nunc: ‘quod semper amas, quod semper amasti’ \\ Hic furor, hic, superi, sit mibi perpetuus! \\ 
      \end{verse}
  
            \subsection*{435}
      \begin{verse}
      B. III 195. \\ M. 968. \\ \poemtitle{De ea uae amat \rbrack }B. IV 75. \\ Quaedam me (si credis) amat. sed dissilit, ardet \\ Non sic, non leviter, sed perit et moritur. \\ Dum faciet gratis quaedam, simul atque rogaro, \\ Ostendam, quam non semper amatus amem. \\ 
      \end{verse}
  
            \subsection*{436}
      \begin{verse}
      B. III223. \\ M. 994. \\ \poemtitle{De eretatae facie}B. IV 75. \\ Cum cretam sumit, faciem Sertoria sumit. \\ Perdidit  \lbrack ut \rbrack  cretam, perdidit et faciem. \\ 
        \pagebreak 
    \begin{center} \textbf{CODICIS VOSSIANI Q. 86.} \end{center} \marginpar{[331]} 
      \end{verse}
  
            \subsection*{43}
      \begin{verse}
      B. II 16. \\ M. 703. \\ Mote omnes aeuari \\ B. IV 75. \\ Quisquis adhuc nondum fortunae mobile regnum \\ Nec sortem varias credis habere vices, \\ Aspice Alexandri positum venerabile corpus. \\ Abscondit tantum putris harena virum! ol. 96 u. \\ 
      \end{verse}
  
            \subsection*{438}
      \begin{verse}
      B. II 25. \\ M. 731. \\ Iteml \\ B. IV 76. \\ Iunxit magnorum casus fortuna virorum: \\ Hic parvo, nullo conditus ille loco est. \\ Ite, novas toto terras conquirite mundo: \\ Nempe manet ‘magnos’ parvula terra duces. \\ 
      \end{verse}
  
            \subsection*{439}
      \begin{verse}
      B. III 227. \\ M. 995. \\ \poemtitle{De puero amato}B. IV 76. \\ Quid saevis, Cypare? domiti modo terga iuvenci \\ Quid premis et tenerum currere cogis equum? \\ Dum stupet ac novus est et adhuc non novit amorem, \\ Parce: premendus erit, cum veteranus erit. \\ 
      \end{verse}
  
            \subsection*{440}
      \begin{verse}
      B. III65. \\ M. 913. \\ \poemtitle{De bono quietae vitae}B. IV 76. \\ Ante rates Siculo discurrent aequore siccae \\ Et deerit Libycis putris harena vadis, \\ 
        \pagebreak 
     \marginpar{[332]} \begin{center} \textbf{CARMNA} \end{center}Ante nives calidos demittent fontibus amnes \\ Et hbodanus nullas in mare ducet aquas, \\ Ante mari gemino semper pulsata Corinthos \\ Confundet fluctus pervia facta duos, \\ Ante feri cervis submittent colla leones \\ Saevaque dediscet proelia torvus aper, \\ Medus pila geret, pharetras Romana iuventus, \\ Fulgebit rutilis lndia nigra comis, \\ Quam mihi displiceat vitae fortuna quietae \\ Aut credat dubiis se mea puppis aquis. \\ 
      \end{verse}
  
            \subsection*{44}
      \begin{verse}
      B. II 155. \\ M. 139. \\ \poemtitle{De fratris lio parvulo}B. IV 77. \\ Sic mihi sit frater maiorque minorque superstes \\ Et de me doleant nil nisi morte mea; \\ Sic illos vincam, sic vincar rursus amando, \\ Mutuus inter nos sic bene certet amor; \\ Sic dulci Marcus qui nunc sermone fritinnit, \\ Facundo patruos provocet ore duos \\ 
      \end{verse}
  
            \subsection*{442}
      \begin{verse}
      e. B. II 12. \\ M. 186. \\ \poemtitle{De monte Atho}B. IV 77. \\ Xerses magnus adest. totus comitatur euntem \\ Orbis. quid dubitas, Graecia, ferre iugum? \\ 
        \pagebreak 
    \begin{center} \textbf{CODICIS VOSSIANI Q. 86.} \end{center}Mundus iussa facit: solem texere sagittae, \\ Calcatur pontus, fluctuat altus Athos. \\ 
      \end{verse}
  
            \subsection*{443}
      \begin{verse}
      B. III 70. \\ M. 911. \\ \poemtitle{De divitiis et inhonesto animo †}Quodtuamilledomussolidashabetaltacolumnas, \\ Quod tua marmoreo ianua poste nitet, \\ Aurea quod summo splendent laquearia tecto, \\ Imum crusta tegit quod pretiosa locum, \\ Atria quod circa dives †tegit omnia cultus: \\ Hoc animos tollit nempe, beate, tuos? \\ Aedibus in totis gemmae licet omnia claudant, \\ Turpe est, nil domino turpius esse suo. \\ 
      \end{verse}
  
            \subsection*{44}
      \begin{verse}
      B. III 93. \\ M. 250. \\ \poemtitle{De eodem}B. IV 78. \\ Non est (falleris) haec beata, non est, \\ Quam vos creditis esse, vita; non est, \\ Fulgentes manibus videre gemmas \\ Aut testudineo iacere lecto \\ Aut pluma latus abdidisse molli \\ Aut auro bibere et cubare cocco, \\ Regales dapibus gravare mensas \\ Et, quidquid Libyco secatur arvo, \\ Non una positum tenere cella: \\ 
        \pagebreak 
    \begin{center} \textbf{CARMINA} \end{center} \marginpar{[334]} Sed nullos trepidum timere casus \\ Nec vano populi favore tangi \\ Et stricto nihil aestuare ferro: \\ Hoc quisquis poteri, licebit ille \\ Fortunam moveat loco superbus. \\ 
      \end{verse}
  
            \subsection*{445}
      \begin{verse}
      B. II 249. \\ M. 136. \\ \poemtitle{De amiceo mortuo}B. IV 78. \\ Ablatus mihi Crispus est, amici, \\ Pro quo si pretium dari liceret, \\ Nostros dividerem libenter annos. \\ Nunc pars optima me mei reliquit, ol. 9 r. \\ Crispus, praesidium meum, voluptas, \\ Pectus, deliciae: nihit sine illo \\ Laetum mens mea iam putabit esse. \\ Consumptus male debilisque vivam: \\ Plus quam dimidium mei recessit. \\ 
      \end{verse}
  
            \subsection*{446}
      \begin{verse}
      B. III 204. \\ M. 976. \\ . De divite formose generosa inpudica \\ Formosa es, fateor, dives generosa venusta: \\ Confitear, si vis, omnia. redde vicem. \\ Nempe parum casta es, nempe es deprensa. negabis: \\ Res venit ad lites. ‘rursus et illa nego’ \\ Dic potius: ‘sed nempe semel, sed nempe puella; \\ Et cum deprensa quis nisi frater erat’ \\ 
        \pagebreak 
    \begin{center} \textbf{CODICIS VOSSIANI Q. 86.} \end{center} \marginpar{[335]} Frater erat? ‘nhil est, fecit quia luppiter illud.’ \\ Sed quod non fecit luppiter, hoc facitis! \\ 
      \end{verse}
  
            \subsection*{447}
      \begin{verse}
      B. III 1. \\ M. 876. \\ \poemtitle{De raeeiae ruina}B. IV 79. \\ Graecia bellorum longa succisa ruina \\ Concidit, inmodice viribus usa suis. \\ Fama manet, fortuna perit: cinis ipse iacentis \\ Visitur, et tumulo est nunc quoque sacra suo. \\ Exigua ingentis retinet vestigia famae \\ Et magnum infelix nil nisi nomen habet. \\ 
      \end{verse}
  
            \subsection*{448}
      \begin{verse}
      B. III 191. \\ M. 964. \\ B. IV 80. \\ Sic tua sit, quamcunque tuam vis esse puellam, \\ Sic quamcunque voles mutuus ignis edat, \\ Sic numquam dulci careant tua pectora flamma \\ Et sic laesuro semper amore vacent. \\ 
      \end{verse}
  
            \subsection*{449}
      \begin{verse}
      B. III 84. \\ M. 933. \\ \poemtitle{De vino et laetitia}B. IV 80. \\ Vince mero curas et, quidquid forte remordet, \\ Comprime deque animo nubila pelle tuo. \\ Nox curam, si prendit, alit: male creditur illi \\ Cura, nisi a multo marcida facta mero. \\ 
        \pagebreak 
    \begin{center} \textbf{CARMNA} \end{center} \marginpar{[336]} 
      \end{verse}
  
            \subsection*{450}
      \begin{verse}
      B. III 196. \\ M. 969. \\ \poemtitle{De silentio amoris}B. IV 80. \\ Iuratum tibi me cogis promittere, Galla, \\ Ne narrem. Iura rursus et ipsa mihi, \\ Ne cui tu dicas nimium est lex dura; remittam: \\ Praeterquam si vis dicere, Galla, viro! \\ 
      \end{verse}
  
            \subsection*{451}
      \begin{verse}
      B. III 197. \\ M. 17. \\ \poemtitle{De initio et flne amoris}B. IV 80. \\ Nescio quo stimulante malo pia foedera rupi. \\ Non capiunt vires crimina tanta meae. \\ Institit et stimulis ardentibus inpulit actum \\ Sive fuit fatum, seu fuit ille deus. \\ Arguimus quid vana deos? vis, Delia, verum? \\ Qui tibi me dederat, idem et ademit: amor. \\ 
      \end{verse}
  
            \subsection*{452}
      \begin{verse}
      B. III 202. \\ M. 974. \\ \poemtitle{De tinnitu auris}B. IV 81. \\ ‘Carrnla, quod totis resonant mihi noctibus aures, \\ Nescio quem dicis nunc meminisse mei’ \\ ‘lic quis sit, quaeris? resonant tibi noctibus aures, \\ Et resonant totis: Delia te loquitur’ \\ ‘Non dubie loquitur me Delia: mollior aura \\ Venit et exili murmure dulce fremit. \\ Delia nou aliter secreta silentia noctis \\ Summissa ac tenui rumpere voce solet, \\ Non aliter teneris collum complexa lacertis \\ Auribus admotis condita verba dare. \\ 
        \pagebreak 
    \begin{center} \textbf{CODICIS VOSSIANI 0. 86.} \end{center} \marginpar{[337]} Agnovi: verae venit mihi vocis image, \\ Blandior arguta tinnit in aure sonus. \\ Ne cessate, precor, longos gestare susurros! \\ Dum loquor haec, iam vos opticuisse queror. \\ 
      \end{verse}
  
            \subsection*{453}
      \begin{verse}
      B. III 193. \\ M. 970. \\ \poemtitle{De elotyp}B. IV 81. \\ Sic me custodi, Cosconia: neve ligata \\ Vincula sint nimium neve soluta nimis. \\ Effugiam laxata nimis, nimis aspera rumpam, \\ Sed neutrum faciam, commoda si fueris. (ol. u. \\ acu est) \\ 
      \end{verse}
  
            \subsection*{454}
      \begin{verse}
      B. II 32. \\ M. 738. \\ \poemtitle{De tumulis Manorum}fol. 98 r. \\ B. IV 82. \\ Alter Niliaco tumulo iacet, alter llibero, \\ Tertius Eois partibus occubuit. \\ Omnis habet Magnos mundi plaga. dis  \lbrack ita visum est: \\ Partem quisque suam nunc quoque victus habet. \\ 
      \end{verse}
  
            \subsection*{455}
      \begin{verse}
      B. II 33. \\ M. 739. \\  \lbrack Item \rbrack  \\ B. IV 2. \\ Patria, diverso terrarum litore Magnos \\ Spectas compositos heu sine nominibus, \\ Europa \lbrack que \rbrack  Asiaque simul Libyaque sepultos. \\ Victores victa sic potiuntur humo! \\ 
        \pagebreak 
    \begin{center} \textbf{CARMINA} \end{center} \marginpar{[338]} 
      \end{verse}
  
            \subsection*{456}
      \begin{verse}
      B. II 31. \\ M. 740. \\ Item \\ B. IV 82. \\ Diversis iuvenes Asia atque Europa sepulcris \\ Distinet; infida, Magne, iaces Libya \\ Distribuit Magnos mundo Fortuna sepultos, \\ Ne sine Pompeio terra sit ulla suo. \\ 
      \end{verse}
  
            \subsection*{457}
      \begin{verse}
      B. IV 1. \\ M. 1144. \\ \poemtitle{ \lbrack De Cascis fratribus \rbrack }B. IV 82. \\ Occidere simul Cascae, simul occubuere, \\ Dextra quisque sua, qua scelus ausus erat. \\ Castra eadem fovere, locus quoque vulneris idem; \\ Prtibus afflictis victus uterque iacet. \\ Quanta fuit mentis, tanta est coucordia fati, \\ Et tumulus cinerem parvus utrumque tegit. \\ Par fratrum multo celebrandum carmine vatum, \\ Vna si fierent parte minus gemini! \\ 
      \end{verse}
  
            \subsection*{458}
      \begin{verse}
      B. III 199. \\ M. 971. \\ V nterdum et neleetem formam placere \\ Semper munditias, semper, Basilissa, decores, \\ Semper dispositas arte decente comas \\ Et comptos scmper vultus unguentaque semper, \\ Omnia sollicita culta videre manu, \\ Non amo; neglectam, mihi se quae comit amica, \\ Se det: inoruata simplicitae valet. \\ 
        \pagebreak 
     \marginpar{[339]} \begin{center} \textbf{CODICIS VOSSIANI Q. 86.} \end{center}Vincula nec curet capitis discussa soluti \\ Nec decoret faciem: mel habet illa suum. \\ Fingere se semper non est confldere amori. \\ Quid, quod saepe decor, cum probibetur, adest? \\ 
      \end{verse}
  
            \subsection*{459}
      \begin{verse}
      B. III 200. \\ M. 972. \\  \lbrack Ad eandem \rbrack  \\ B. IV 83. \\ Ante dies multos nisi te, Basilissa, rogavi \\ Et nisi praemonui, te dare posse negas. \\ Vt subito crevere, solent ex tempore iunctae \\ Quam scriptae melius cedere deliciae. \\ 
      \end{verse}
  
            \subsection*{460}
      \begin{verse}
      B. III 201. \\ M. 973. \\  \lbrack Item \rbrack  \\ B. IV 83. \\ Cur difers, mea lux, rogata, semper? \\ Cur longam petis advocationem? \\ Primum hoc artificis scelus puellae est, \\ Deinde est difficile et laboriosum \\ In tentie tam diu morari. \\ Nil est praeterea, pell,niE est \\ Deprensa melis fitutione. \\ 
      \end{verse}
  
            \subsection*{461}
      \begin{verse}
      B. II 13. \\ M. 188. \\ \poemtitle{De Atho monte}B. IV 84. \\ Hic, quem cernis, Athos inmissis pervius undis \\ Flexibus obliquis circumeundus erat. \\ 
        \pagebreak 
    \begin{center} \textbf{CARMINA} \end{center} \marginpar{[340]} Accepit magno deductum Nerea fluctu \\ Perque latus misit maxima bella suum. \\ Sub tanto subitae sonuerunt pondere classes, \\ Caeruleus cana sub nive pontus erat. \\ Idem commisit longo duo litora ponte \\ Xerses, et fecit per mare miles iter. \\ Quale fuit regnum, mundo nova ponere iura! \\ ‘Hoc terrae fiat, hac mare’ dixit ‘ea’. \\ 
      \end{verse}
  
            \subsection*{462}
      \begin{verse}
      B. II 131. \\ M. 820. \\ \poemtitle{De malo belli civilis}B. IV 84. \\ Venerat Eoum quatiens Antonius orbem \\ Et coniuncta suis Partbica signa gerens, \\ Dotalemque petens Romam Cleopatra Canopo. \\ Hinc Capitolino sistra minata lovi, \\ Iinc invicta deo fidebat Caesare Roma, \\ Quae tunc paene suo pondere lapsa ruit. \\ Deserta est tellus, classis contexerat aequor, \\ Omnia permixti plena furoris erant. \\ fol. 98 u. \\ Fratribus heu fratres, patribus concurrere natos \\ Impia sors belli fataque saeva iubent. \\ lic generum, socerum ille petit, minimeque cruentus \\ Qui fuit,  \lbrack is \rbrack  sparsus sanguine civis erat. \\ Maevius, a castris miles melioribus, ausus \\ Hostilem saltu praecipitare ratem, \\ 
        \pagebreak 
    \begin{center} \textbf{CODICIS VOSSIANI 0. 86.} \end{center} \marginpar{[341]} In damnum felix et victor ut impius esset, \\ Nescius occiso fratre superbus erat. \\ Dum legit exuvias hostiliaque arma revellit, \\ Fraternos vultus oraque maesta videt. \\ Quod fuerat virtus, factum est scelus. haeret in hoste \\ Miles et e manibus mittere tela timet. \\ Inde ferox: ‘quid lenta manus? nunc denique cessem? \\ Iustius hoste tibi qui moriatur adest. \\ Fraternam res nulla potest defendere caedem; \\ Mors tua sola potest; morte luenda tua est. \\ Scilicet ad patrios referes spolia ampla penates? \\ Ad patrem victor non potes ire tuum, \\ Sed potes ad fratrem. nunc fortiter utere telo! \\ lmpius hoc telo es, hoc potes esse pius. \\ Vivere si poteris, potuisti occidere fratrem! \\ Nescisti: sed scis: haec mora culpa tua est. \\ Viximus adversis, iaceamus partibus isdem.’ \\ Dixit, et in dubio est, utrius ense cadat. \\ ‘Ense meo moriar, maculato morte nefanda? \\ Cui moreris, ferrum quo moriare dabit’ \\ Dixit et in fratrem fraterno concidit ense. \\ Victorem et victum condidit una manus. \\ 
      \end{verse}
  
            \subsection*{468}
      \begin{verse}
      B. II 132. \\ M. 821. \\  \lbrack Item \rbrack  \\ B. IV 86. \\ Sicine componis populos, Fortuna, furentis \\ Vt vinci levius, vincere sit gravius \\ 
        \pagebreak 
    \begin{center} \textbf{CARMINA} \end{center}\begin{center} \textbf{q0} \end{center}Occisum credens gaudebat Maevius hostem: \\ Infelix fratris vulnere laetus erat. \\ Nec licuit non nosse: ferox dum membra cruenti \\ Nudat, in exuvias incidit ipse suas. \\ Et scelus et fratrem pariter cognovit et amens \\ ‘loc age’, ait, ‘maius nunc tibi restat opus. \\ Vincere victorem debes, defendere fratrem. \\ Cessas? ad facinus quam modo fortis eras! \\ Terram, iura, deos, bellum iam polluis ipsum: \\ Quod civile fuit, sic quoque culpa gravis. \\ Hlis manibus patriae tu iam pia signa sequeris \\ Miles, in Antoni dignior ire rates? \\ Eripuit virtus pietatem, reddere virtus \\ Debet: qua rapuit, hac reparanda via est. \\ Quid moror absolvi’ Dixit, gladioque cruento \\ Incubuit, iungens fratris ad ora sua. \\ Sic, ortuna, regas semper civilia bella, \\ Vt victor victo non superesse velit! \\ B. III 122. M. 118. \\ 
      \end{verse}
  
            \subsection*{4654}
      \begin{verse}
      Petron. ed Buech. \\ fa 35. \\ Item \\ B. IV 88. \\ Inveniat, quod quisque velit. non omnibus unum est, \\ Quod placet. hic spinas colligit, ille rosas. \\ 
        \pagebreak 
    \begin{center} \textbf{CODICIS VOSSIANI Q. 86.} \end{center} \marginpar{[343]} 
      \end{verse}
  
            \subsection*{465}
      \begin{verse}
      B. M. \\ Buecb. 38. \\ Item \\ B. IV 88. \\ Iam nunc ardentes autumnus fregerat umbras \\ Atque hiemem tepidis spectabat Phoebus habenis, \\ Iam platanus iactare comas, iam coeperat uvas \\ Adnumerare suas defecto palmite vitis. \\ Ante oculos stabat, quidquid promiserat annus. \\ 
      \end{verse}
  
            \subsection*{466}
      \begin{verse}
      rr \\ \poemtitle{ETRONII}B. III 119. \\ M. 145. \\ uecb. 27. \\ Item \\ B. IV 88. \\ Primus in orbe deos fecit timor, ardua caelo \\ Fulmina cum caderent discussaque moenia flammis \\ Atque ictus flagraret Athos: mox Phoebus ad ortus \\ Lustrata devectus humo, Lunaeque resectus \\ Et reparatus honos; hinc signa effusa per orbem, \\ Et permutatis disiunctus mensibus annus. \\ Profecit vitium, iamque error iussit inanis \\ Agricolas primos Cereri dare messis honores, \\ Palmitibus plenis Bacchum vincire, Palemque ol. 99 r. \\ 
        \pagebreak 
     \marginpar{[344]} \begin{center} \textbf{CARMNA} \end{center}Pastorum gaudere manu: natat obrutus undis \\ Neptuni demersus aqua, Pallasque tabernas \\ Vindicat. et voti reus, et qui vendidit orbem, \\ Iam sibi quisque deos avido certamine fingit. \\ B. II 120. \\ 
      \end{verse}
  
            \subsection*{467}
      \begin{verse}
      M. 146. \\ Buech. 33. \\ Item \\ B. IV 89. \\ Nolo ego semper idem capiti sufundere costum \\ Nec noto stomachum conciliare mero. \\ Taurus amat gramen mutata carpere valle \\ Et fera mutatis sustinet ora cibis. \\ Ipsa dies ideo nos grato perluit haustu, \\ Quod permutatis hora recurrit equis. \\ B. III 121. \\ 
      \end{verse}
  
            \subsection*{468}
      \begin{verse}
      M. 147. \\ Buech. 34. \\  \lbrack Item \rbrack  \\ B. IV 90. \\ ‘Vxor legitimus debet quasi census amari’. \\ Nec censum vellem semper amare meum. \\ B. II 124. \\ 
      \end{verse}
  
            \subsection*{469}
      \begin{verse}
      M. 10. \\ uech. 37. \\ Item \\ B. IV 90. \\ Linque tuas sedes alienaque litora quaere, \\  \lbrack O \rbrack  iuvenis: maior rerum tibi nascitur ordo. \\ Ne succumbe malisx te noverit ultimus lister, \\ Te Boreas gelidus securaque regna Canopi, \\ 
        \pagebreak 
    \begin{center} \textbf{CODICIS VOSSIANI Q. 86.} \end{center} \marginpar{[345]} Quique renascentem Phoebum cernuntque cadentem: \\ Maior in externas lthacus descendat harenas. \\ B. III 123. \\ 
      \end{verse}
  
            \subsection*{470}
      \begin{verse}
      M. 149. \\ Buech. 36. \\ ItCem \\ B. IV 90. \\ Nam nihil est, quod non mortalibus afferat usum. \\ Rebus in adversis, quae iacuere, iuvant. \\ Sic rate demersa fulvum deponderat aurum: \\ Remorum levitas naufraga membra vehit. \\ Cum sonuere tubae, iugulo stat divite ferrum; \\ Barbara contemptu praelia pannus habet. \\ B. III 61. \\ M. 142. \\ Buech. 30. \\ Item \\ B. IV 91. \\ Parvula securo tegitur mihi culmine sedes \\ Vvaque plena mero fecunda pendet ab ulmo. \\ Dant rami cerasos, dant mala rubentia silvae \\ Palladiumque nemus pingui se vertice frangit. \\ Iam qua diductos potat levis area fontes, \\ Corycium mihi surgit olus malvaeque supinae \\ Et non sollicitos missura papavera somnos. \\ Praeterea sive alitibus contexere fraudem \\ Seu magis inbelles libuit circumdare cervos \\ Aut tereti lino pavidum subducere piscem, \\ Hos tantum novere dolos mea sordida rura. \\ I nunc et vitae fugientis tempora vende \\ 
        \pagebreak 
    \begin{center} \textbf{CARMINA} \end{center} \marginpar{[346]} Divitibus cenis! me si manet exitus idem \\ Hic precor inveniat consumptaque tempora poscat. \\ B. III 13. \\ A \\ M. 181. \\ Buech. 32. \\ Item \\ B. IV 91. \\ Non satis est, quod nos mergit furiosa iuventus, \\ Transversosque rapit fama sepulta probris: \\ En etiam famuli cognata faece soluti \\ Inter transgressas luxuriantur opes! \\ Vilis servus habet regni bona, cellaque capti \\ Deridet Vestam Romuleamqae casam. \\ Ilcirco virtus medio iacet obruta caeno, \\ Nequitiae classes candida vela ferunt. \\ 
      \end{verse}
  
            \subsection*{473}
      \begin{verse}
      B . M. \\ Buech. 39. \\ Item \\ B. IV 92. \\ Sic et membra solent auras includere ventris, \\ Quae penitus mersae cum rursus abire laborant, \\ Verberibus rimantur iter; nec desinit ante \\ Frigidus, adstrictis qui regnat in ossibus, horror, s \\ Quam tepidus laxo manavit corpore sudor. \\ 
        \pagebreak 
    \begin{center} \textbf{CODICIS VOSSIANI Q. 86.} \end{center}\begin{center} \textbf{0 4} \end{center}B. III 62. \\ 
      \end{verse}
  
            \subsection*{474}
      \begin{verse}
      M. 143. \\ uech. 51t. \\ Item \\ B. IV 92. \\ O0 litus vita mihi dulcius! o mare felix, \\ Cui licet ad terras ire subinde meas! \\ 0 formosa diesl hoc quondam rure solebam \\ †lliadas armatas sollicitare manus. \\ Hic fontis lacus est, illic sinus egerit algas, \\ Haec statio est tacitis fida Cupidinibus. \\ Pervixi: neque enim fortuna malignior umquam \\ Eripiet nobis, quod prior hora dedit. \\ 
      \end{verse}
  
            \subsection*{475}
      \begin{verse}
      B. M. \\ uech. 40. \\ Item \\ B. IV 93. \\ Haec ait et tremulo deduxit vertice canos \\ Consecuitque genas; oculis nec defuit imber, \\ Sed qualis rapitur per vallis improbus amuis, l. 99 u. \\ Cum gelidae periere nives et languidus Auster \\ Non patitur glaciem resoluta vivere terra, \\ Gurgite sic pleno facies manavit et alto \\ Insonuit gemitu turbato murmure pectus. \\ 
        \pagebreak 
    \begin{center} \textbf{CARMINA} \end{center} \marginpar{[348]} 
      \end{verse}
  
            \subsection*{476}
      \begin{verse}
      B. III 125. \\ \poemtitle{ \lbrack PETRoNI}. 178. \\ uech. 28. \\ Item \\ B. IV 83. \\ Nam citius flammas mortales ore tenebunt \\ Quam secreta tegant. quicquid dimittis in aula, \\ Efuit et subitis rumoribus oppida pulsat. \\ Nec satis est, vulgasse fidem. cumulatius exit \\ Proditionis opus famamque onerare laborat . . . \\ Sic commissa verens avidus reserare minister \\ Fodit humum regisque latentes prodidit aures. \\ Concepit nam terra sonos calamique loquentes \\ Invenere Midam, qualem narraverat index. \\ B. IM 62. \\ t \\ M d I43. \\ Buech. 1. \\ Item \\ B. IV 93. \\ llic alternis depugnat pontus et aer, \\ Hic rivo tenui pervia ridet humus. \\ Illic divisas conplorat navita puppis, \\ Hic pastor miti perluit amne pecus. \\ Illic inmanes mors obvia solvit hiatus, \\ Hic gaudet curva falce recisa Ceres. \\ 
        \pagebreak 
    \begin{center} \textbf{CODICIS VOSSIANI 0. 86.} \end{center} \marginpar{[349]} Illic inter aquas urit sitis arida fauces, \\  \lbrack IIic \rbrack  . \\  \lbrack I5c1 . \\ Hic dantur caro basia multo viro. \\ Naviget et fluctus lasset mendicus Vlixes: \\ In terris vivit candida Penelope! \\ B. III 63 \\ 
      \end{verse}
  
            \subsection*{478}
      \begin{verse}
      M. 144. \\ Buech. 52. \\ Item \\ B. IV 94. \\ Qui non vult properare mori nec cogere fata \\ Mollia praecipiti rumpere fila manu, \\ Hactenus iratum mare noverit. ecce refuso \\ Gurgite securos subluit unda pedes, \\ Ecce inter virides iactatur mytilus algas \\ Et rauco trahitur lubrica concha sono. \\ Ecce recurrentes qua versat fluctus harenas, \\ Discolor attrita calculus exit humo. \\ Haec quisquis calcare potest, in litore tuto \\ Ludat et hoc solum iudicet esse mare. \\ B. III 224. \\ 
      \end{verse}
  
            \subsection*{479}
      \begin{verse}
      M. 12. \\ uech. 31. \\ Item \\ B. IV 95. \\ Non est forma satis, nec, quae vult bella videri, \\ Debet vulgari more placere sibi. \\ Dicta, sales, lusus, sermonis gratia, risus \\ Vincunt, naturae candidioris opus. \\ 
        \pagebreak 
     \marginpar{[350]} \begin{center} \textbf{CARMINA} \end{center}Condit enim formam, quicquid consumitur artis; \\ Et nisi velle subest, gratia nuda perit. \\ s \\ 
      \end{verse}
  
            \subsection*{480}
      \begin{verse}
      B M. \\ \poemtitle{De peibus iber IIII}fol. 116 r. \\ B. IV 11. \\ Pes est spondius, longa qui constat utraque. \\ Hunc contra pariambus erit gemina breve innctus. \\ Longa brevisque pedem faciet postlata trochaeum. \\ At praelata brevis longae concludet iambum. \\ Dactylus ex longa veniet brevibusque duabus. \\ Idem anapaestus erit, brevibus si coeperit antc. \\ Cum brevis una duas distinguit syllaba longas, \\ Creticus est. . \\ 
        \pagebreak 
    \begin{center} \textbf{AENI6MATA} \end{center}\begin{center} \textbf{V} \end{center}
      \end{verse}
  
            \subsection*{}
      \begin{verse}
      \poemtitle{CODI CIS BERN EN SIS 611}I \\ 
      \end{verse}
  
            \subsection*{481}
      \begin{verse}
      B. M. I. \\ \poemtitle{De ola}Ego nata duos patres habere dinoscor. \\ Prior semper manet alterque morte finitur. \\ Tertia me mater dura mollescere cogit \\ Et tenera gyro formam assumo decoram. \\ Nullum dare victum frigenti corpore possum, \\ Calida sed cunctis salubres porrigo pastus. \\ \poemtitle{De lueerna}Me mater novellam vetus de germine finxit. \\ Et in nullo patris formata sumo figuram. \\ Oculi non mihi lumen ostendere possunt, \\ Patulo sed flammas ore produco coruscas. \\ 
        \pagebreak 
    \begin{center} \textbf{AENI6MTA} \end{center} \marginpar{[352]} Nolo me contingat imber nec flamina venti. \\ Sum amica lucis, domi delector in umbris. \\ B De sale \\ Me pater ignitus, ut nascar, creat urendo \\ Et pia defectu me mater donat ubique. \\ Is qui dura solvit, hic me constringere cogit. \\ Nullus me solutum, ligatum cuncti requirunt. \\ Opem fero vivis opemque reddo defunctis. \\ Patria me sine mundi nec ulla valebit. \\ 
      \end{verse}
  
            \subsection*{}
      \begin{verse}
      \poemtitle{De scamno}Mollior horresco semper consistere locis. \\ Vngula nam mihi firma, si caute ponatur. \\ Nullum iter agens sessorem dorso requiro. \\ Plures libens fero, meo dum stabulo versor. \\ Nolo frena mihi, mansueto iuveni, pendas, \\ Calcibus et senum nolo ne verberer ullis. \\ 
      \end{verse}
  
            \subsection*{5}
      \begin{verse}
      \poemtitle{De mensa}Pulcbhra mater ego natos dum collego multos, \\ Cunctis trado libens, quicquid in pectore gesto. \\ Nulli sicut mihi pro bonis mala redduntur. \\ Oscula nam mihi prius qui cara dederunt, \\ Vestibus exutam turpi me modo relinquunt. \\ Quos lactavi, nudam pede per angula versant. \\ 
        \pagebreak 
    \begin{center} \textbf{CODICIS BERNENSIS 611.} \end{center} \marginpar{[353]} 
      \end{verse}
  
            \subsection*{6}
      \begin{verse}
      \poemtitle{De caliee}Nullus ut mea lux sola penetrat umbram \\ Et natura vili miros postpono lapillos. \\ lgnem fero nascens: natus ab igne fatigor. \\ Nulla me putredo tangit nec funera turbant: \\ Pristina defunctus sospes in forma resurgo. \\ Et amica libens oscula porrigo cunctis. \\ 
      \end{verse}
  
            \subsection*{7}
      \begin{verse}
      \poemtitle{De vesica}Teneo liquentem, sequor membrana celatum. \\ Verbero nam cursu, visu quem cernere vetor. \\ Inpletur domus invisis sed vacua rebus. \\ Permanet dum cibum nullum de pondere gessi. \\ Quae dum clausa fertur, velox ad nubila surgit, \\ Patefacta nullum potest tenere manentem. \\ 
      \end{verse}
  
            \subsection*{8}
      \begin{verse}
      \poemtitle{De ovo}Nati mater ego natus ab utero mecum. \\ Prior illo non sum, semper qui mihi coaevus. \\ Virgo nisi manens numquam concipere possum. \\ Sed intacta meam infra concipio prolem. \\ Post si mihi venter disruptus ictu patescit, \\ Moriens viventem sic possum fundere fetum. \\ 
      \end{verse}
  
            \subsection*{9}
      \begin{verse}
      \poemtitle{De mol}Senior ab aevo, Eva sum senior ego. \\ Et senectam gravem nemo currendo revincit. \\ 
        \pagebreak 
    \begin{center} \textbf{AENIGMAA} \end{center} \marginpar{[354]} Milia prosterno, manu dum verbero nullum. \\ Vitam dabo cunctis, vitam si tulero multis. \\ Satura nam victum, ignem ieiuna produco \\ Et uno vagantes possum conprehendere loco. \\ 
      \end{verse}
  
            \subsection*{0}
      \begin{verse}
      \poemtitle{De scala}Singula si vivens firmis constitero plantis, \\ Viam me roanti directam ire negabo. \\ Gemina sed soror meo se latere iungat: \\ Coeptum valet iter velox percurrere quisquis. \\ Subito mihi pedem nisi calcaverit ille, \\ Manibus, quae cupit, numquam contingere valet. \\ \poemtitle{De nave}Mortua maiorem quam vivens porto laborem. \\ Dum iaceo, multos servo; si stetero, paucos. \\ Viscera si mihi foris detracta patescant, \\ Vitam fero cunctis victumque confero multis. \\ †Bestia defunctam avisque nulla me mordet, \\ Et onusta currens viam nec planta depingo. \\ I2 De greno \\ Mortem ego pater libens adsumo pro natis \\ Et tormenta simul, cara ne pignora tristem. \\ Mortuum me cuncti gaudent habere parentes \\ 
        \pagebreak 
    \begin{center} \textbf{CODICIS BERNENSIS 611.} \end{center} \marginpar{[355]} 0 Et sepultum nullus parvo vel funere plangit. \\ Vili subterrena pusillus tumulor urna, \\ Sed maiore possum post mortem surgere forma. \\ I3 De vite \\ Vno fixa loco longinquis porrigo victum. \\ Caput mihi ferrum secat et brachia truncat. \\ Lacrimis infecta plura per vincula nector, \\ Simili damnandos nece dum genero natos. \\ Sed defuncti solent ulcisci liberi matrem, \\ Sanguine dum fuso lapsis vestigia versant. \\ 
      \end{verse}
  
            \subsection*{}
      \begin{verse}
      \poemtitle{De oliva}Nullam ante tempus lustcri genero prolem \\ Annisque peractis superbos genero natos, \\ Quos domare quisquis valet industria parvus, \\ Cum eos marinus iunctus percusserit imber. \\ Asperi nam lenes sic creant filii nepotes, \\ Tenebris ut lucem reddant, dolori salutem. \\ 
      \end{verse}
  
            \subsection*{15}
      \begin{verse}
      \poemtitle{De palma}Pulchra semper comis locis consisto desertis. \\ Ceteris dum mihi cum lignis nulla figura, \\ Dulcia petenti de corde poma produco \\ 
        \pagebreak 
    \begin{center} \textbf{AENIGMAA} \end{center} \marginpar{[356]} Nulloque de ramis cultore confero fructum. \\ Nemo qui me serit de meis fructibus edit. \\ Et amata cunctis flore sum socia iustis. \\ 
      \end{verse}
  
            \subsection*{6}
      \begin{verse}
      \poemtitle{De cedria}Me mater ut spinis vivam enutrit iniquis, \\ Et dulcem faciat, inter acumina servat. \\ †Teretinam formam rubentem confringo ceratam \\ Et incisa nullam dono de corpore guttam. \\ Mellea cum mihi sit sine sanguine caro, \\ Acetum eructant . . exta clausa saporem. \\ 
      \end{verse}
  
            \subsection*{7}
      \begin{verse}
      \poemtitle{De cribro}Patulo sum semper ore nec labia iungo. \\ Incitor ad cursum freqnenti uerbere tactus. \\ Escae mihi ullae manu si forte ponantur, \\ Hlas amitto currens minuto vulnere ruptas. \\ Meliora cunctis, mihi nam vilia servant \\ Vacuumque bonis inane cuncti relinquuut. \\ I1 De scopa \\ Florigeras gero comas, dum maneo silvis, \\ Et honesto vivo modo, dum habito campis. \\ Sed acta vili solo depono capillos. \\ Turpius me nulla domi vernacula servit. \\ 
        \pagebreak 
    \begin{center} \textbf{CODICIS BERNENSIS 611.} \end{center}\begin{center} \textbf{D6 4} \end{center}Cuncti per horrenda me terrae pulvere iactant. \\ Sed amoena domus sine me nulla videtur. \\ I9 De pioce \\ Dissimilem sibi me mater concipit infra, \\ Et nullo virili creata de semine fundor. \\ Dum nascor sponte, gladio divellor a ventre. \\ Caesa vivit mater: ego nam flammis aduror. \\ Nullum clara manens possum concedere quaestum; \\ Plurimum fero lucrum, nigro si corpore mutor. \\ 
      \end{verse}
  
            \subsection*{20}
      \begin{verse}
      \poemtitle{De melle}Lncida de domo lapsus diffundor ubique; \\ Et quali dimissus modo, non invenit ullus. \\ Bisque idem natus, semel inde utero cretus, \\ Qualis in conceptu, talis in partu renascor. \\ Milia me quaerunt, ales sed invenit una \\ Cereamque mihi domum depingit ab ore. \\ \poemtitle{De apibus}Masculus qui non sum, sed neque femina, coniux \\ Filios ignoto patre parturio multos. \\ Vberibus prolem nullis enutrio tantam. \\ Quos ab ore cretos nullo de ventre sumsi. \\ 
        \pagebreak 
    \begin{center} \textbf{AENIGMAA} \end{center} \marginpar{[358]} Nomen quibus unum natisque conpar imago, \\ Meos inter cibos dulci conplector amore. \\ 
      \end{verse}
  
            \subsection*{}
      \begin{verse}
      \poemtitle{De ove}Exigua mihi virtus, sed magna facultas. \\ Opes ego nulli quaero, sed confero cunctis, \\ Paupera quae multum ipsos nam munero reges. \\ Modicos operans cibos egena requiro \\ Et ieiuna saepe cogor exsolvere censum. \\ Nullus sine meo mortalis corpore constat. \\ 
      \end{verse}
  
            \subsection*{23}
      \begin{verse}
      \poemtitle{De ine}Durus mihi pater, dura me generat mater: \\ Verbere nam multo huius de viscere funudor. \\ Modica prolatus feror a ventre figura, \\ Sed adulto mihi datur inmensa potestas. \\ Durum ego patrem duramque mollio matrem, \\ Et quae vitam cunctis, haec mihi funera praestat. \\ \poemtitle{De membran}Manibus me perquam reges et visu mirantur. \\ Lucrum viva manens toti nam confero mundo \\ Et defuncta mirum praesto de corpore quaestum, \\ Vestibus exuta multoque vinculo tensa. \\ Gladio sic mihi desecta viscera pendent, \\ Miliaque porto nullo sub pondere multa. \\ 
        \pagebreak 
    \begin{center} \textbf{CODICIS BERNENSIS 611.} \end{center} \marginpar{[359]} 
      \end{verse}
  
            \subsection*{5}
      \begin{verse}
      \poemtitle{De litteris}Nascimur albentibus locis sed nigrae sorores. \\ Tres unito simul nos creant ictu parentes. \\ Multimoda nobis facies et nomina multa, \\ Meritumque dispar, vox et diversa sonandi. \\ Numquam sine nostra nos domo detinet ullus \\ Nec una responsum dat sine pari roganti. \\ 
      \end{verse}
  
            \subsection*{8}
      \begin{verse}
      \poemtitle{De sinapi}Me si visu quaeras, multo sum parvolo †parvus, \\ Sed nemo maiorum mentis astutia vincit. \\ Cum feror sublimi parentis umero vectus, \\ Simplicem ignari me putant esse natura. \\ Verbere correptus saepe si giro fatigor, \\ Protinus occultum produco cordis saporem. \\ 
      \end{verse}
  
            \subsection*{27}
      \begin{verse}
      \poemtitle{De papiro}Amnibus delector, molli sub cespite cretus, \\ Et producta levi natus columna virdesco. \\ Vestibus sub meis nequeo cernere solem, \\ Alieno tectus possum producere lumen, \\ Filius profundi, dum fio lucis amicus. \\ Sic qui vitam dedit mater et lumina tollit. \\ 
      \end{verse}
  
            \subsection*{28}
      \begin{verse}
      \poemtitle{De serieo}Arbor una mihi, quae vilem conferat escam, \\ Qua repleta parvus produco vellera magna, \\ 
        \pagebreak 
    \begin{center} \textbf{AENIGMATA} \end{center} \marginpar{[360]} Exiguos conlapsa foetos pro munere fundo \\ Et ales effecta mortem adsumo libenter. \\ Nobili perfectus forma me Caesares ulnis \\ Eferunt et reges infra supraque mirantur. \\ \poemtitle{De speculo}Vterum si mihi praelucens texerit umbra, \\ Proprios volenti devota porrigo vultus. \\ Talis ego mater vivos non genero natos, \\ Sed petenti vanas diffundo visu figuras. \\ Exiguos licet mentita profero fetus, \\ Sed de vero suas videnti dirigo formas. \\ 
      \end{verse}
  
            \subsection*{30}
      \begin{verse}
      \poemtitle{Ee pisee}Nullo firma loco manens consistere possum \\ Et vagando vivens nullo conspicere quemquam. \\ Vita mihi mors est, mortem pro vita requiro \\ Et volanti domo semper amica delector. \\ Numquam ego lecto volo iacere tepenti, \\ Sed vitalem mibi torum sub frigore condo. \\ \poemtitle{De nmpha}Ore mihi nulla petenti pocula dantur, \\ Ebrius nec nullum reddo post inde fluorem. \\ 
        \pagebreak 
    \begin{center} \textbf{CODICIS BERNENSIS 611.} \end{center} \marginpar{[361]} Versa mihi datur vice bibendi facultas \\ Et vacuo ventri potus ab imo defertur. \\ Poplite depresso conceptas denego nymphas \\ Et sublato rursum diffuso confero nimbos. \\ 
      \end{verse}
  
            \subsection*{3}
      \begin{verse}
      \poemtitle{De ponia}Dissimilem sibi dat mihi mater figuram. \\ Caro nulla mihi, sed viscera vacua latebris. \\ Sumere nihil possum, si non absorbuero matrem. \\ Et quae me concepit, hanc ego genero, postquam \\ Manu capta levis gravis sum manu demissa. \\ Et quae sumpsi libens, mox cogor reddere sumptum. \\ 
      \end{verse}
  
            \subsection*{3}
      \begin{verse}
      \poemtitle{De viola}Parvula dum nascor, minor effecta senesco \\ Et cunctas praecedo maiori veste sorores. \\ Extremos ad brumae me primo confero mense \\ Et cunctis amoena verni iam tempora monstro. \\ Me reddit inlustrem parvo de corpore sumptus \\ Et viam quaerendi docet, qui nulli videtur. \\ 
      \end{verse}
  
            \subsection*{34}
      \begin{verse}
      \poemtitle{De rosa}Pulchra in angusto me mater concipit alvo \\ Et hirsuta barbis quinque conplectitur ulnis. \\ Quae licet parentum parvo sim genere sumpta, \\ Honor quoque mihi concessus fertur ubique. \\ 
        \pagebreak 
     \marginpar{[362]} \begin{center} \textbf{AENIGMATA} \end{center}Htero cum nascor, matri rependo decorem \\ Et parturienti nullum infligo dolorem. \\ 
      \end{verse}
  
            \subsection*{35}
      \begin{verse}
      \poemtitle{De lilio}Nos pater occultus conmendat patulae matri \\ Et mater honesta confixos porrigit hasta. \\ Vivere nec umquam valemus tempore longo, \\ Et leviter tactos incurvat aegra senectus. \\ Oscula si nobis causa figantur amoris, \\ Reddimus candentes signa flaventia labris. \\ 
      \end{verse}
  
            \subsection*{36}
      \begin{verse}
      \poemtitle{De croco}Parvulus aestivas latens abscondor in umbras \\ Et sepulto mihi membra sub †ellure vivunt. \\ Frigidas autumni libens adsuesco pruinas, \\ Et brumae propinqua miros sic profero flores. \\ Pulchra mihi domus manet, sed pulchrior infra \\ Modica in forma clausus aromata vinco. \\ 
      \end{verse}
  
            \subsection*{7}
      \begin{verse}
      \poemtitle{De pipere}Pereger externas vinctus perambulo terras \\ Frigidus et tactu praesto sumenti calorem. \\ Nulla mihi virtus, sospes si mansero semper: \\ Vigeo nam caesus, confractus valeo multum. \\ 
        \pagebreak 
    \begin{center} \textbf{CODICIS BERNENSIS 611.} \end{center} \marginpar{[363]} Mordeo mordentem, morsu nec vulnero dentem. \\ Lapis mihi finis simul, defectio lignum. \\ 
      \end{verse}
  
            \subsection*{38}
      \begin{verse}
      \poemtitle{De glaeie}Corpore formata pleno de parvulo patre \\ Nec a matre feror, nisi feratur et ipsa. \\ Nasci vetor ego sine genito patre. \\ Et creata rursus ego concipio matrem. \\ Hieme conceptos pendens cum servo parentes, \\ Rursus in aestivo coquendos ignibus apto. \\ 
      \end{verse}
  
            \subsection*{39}
      \begin{verse}
      \poemtitle{De hedera}Arbor mihi pater, nam et lapidea mater. \\ Corpore nam mollis duros disrumpo parentes. \\ Aestas me nulla nec ulla frigora vincunt. \\ Vnus bruma color vernoque simul et aestu. \\ Propriis erecta vetor consistere plantis, \\ Manibus sed alta peto cacumina tortis. \\ 
      \end{verse}
  
            \subsection*{40}
      \begin{verse}
      \poemtitle{De museipula}Vinculis extensa multos comprendo vagantes \\ Et soluta nullum queo comprendere pastum. \\ Venter nullus mibi, quo possint capta reponi, \\ Sed multa pro membris formantur ora tenendi. \\ Opes mihi non sunt, sursum sed pendor ad auras. \\ Nam fortuna mihi remanet, si tensa dimittor. \\ 
        \pagebreak 
    \begin{center} \textbf{AENIGMATA} \end{center} \marginpar{[364]} \poemtitle{De vento}Nascens curro velox grandi virtute sonorum. \\ Deprimo nam fortes, infirmos allevo sursum. \\ Os mihi est nullum, dente nec vulnero quemquam. \\ Mordeo sed cunctos silvis campisque morantes. \\ Cernere me quisquam, vinclis quoque neque tenere \\ Macedo nec Liber vincit nec Hercules umquam. \\ 
      \end{verse}
  
            \subsection*{42}
      \begin{verse}
      \poemtitle{De laeie}Arte mea nulla valet durescere quisquam. \\ Efficior dura, multos quae facio molles. \\ Cuncti me solutam cara per oscula gaudent \\ Et nemo constrictam manu vel tangere cupit. \\ Speciem mihi pulchram dat rigor et †auctor, \\ Qui saevos abire iubet, torpescere pulchros. \\ Der vermieulis serieis formatis \\ Concepi innumeros, de nido amitto volatus, \\ Corpus et inmensum parvis adsumo de membris. \\ Mollibus de plumis vestem contexo nitentem \\ Et texturae sonum nec auribus concipit ullus. \\ Si quis forte meo videatur vellere tectus, \\ Excussam vestem statim reicere temptat. \\ 
      \end{verse}
  
            \subsection*{4}
      \begin{verse}
      \poemtitle{De margaritae}Conspicuum corpus arte mirifica sumpsi. \\ Multis cava modis gemmarum ordine nector. \\ 
        \pagebreak 
    \begin{center} \textbf{CODICIS BERNENSIS 61.} \end{center} \marginpar{[365]} Publicis concepta locis in abdita nascor. \\ Confero sed lucrum vacua de luce referta. \\ Nullum mihi valet frigus nec bruma vigescit, \\ Sed calore semper molli sopita fatigor. \\ 
      \end{verse}
  
            \subsection*{4}
      \begin{verse}
      \poemtitle{De terra}Os est mihi patens, crebro qui tunditur ictu. \\ Reddo libens omnes escas, quas sumpsero lambens. \\ Nnlla mihi fames, sitim quoque sentio nullam, \\ Et ieiuna mihi semper praecordia restant. \\ Omnibus ad escam miros efficio sapores, \\ GCelidumque mihi durat per secula corpus. \\ 
      \end{verse}
  
            \subsection*{46}
      \begin{verse}
      \poemtitle{De malleo}Vna mihi toto cervix pro corpore constat \\ Et duo libenter nascuntur capita collo. \\ Versa mihi pedum vice dum capita currunt, \\ Lenes reddo vias, quas calle tero requenti. \\ Nullus mihi comam tondet nec pectine versat. \\ Vertice nitenti plures per oscula gaudent. \\ 
      \end{verse}
  
            \subsection*{7}
      \begin{verse}
      \poemtitle{De castanea}Aspera dum nascor, a matre cute producor \\ Et adulta vigens leni circumdor amictu. \\ ln tactu sonitum de ventre profero magnum \\ Et corrupta tacens vocem non profero ullam. \\ Nullus in amore certo me diligit unquam, \\ Nudam nisi tangat vestemque tulerit omnem. \\ 
        \pagebreak 
    . . \\ Quattuor en istas gerens ego clausa figuras, \\ Pandere quas paucis deposcit ratio brevis. \\ Humida sum sicca, subtili corpore crassa, \\ Dulcis et amara, duro gestamine mollis. \\ Dulcis esse nulli possum nec crescere iuste, \\ †Nisi amaro duroque carcere nascar. \\ 
      \end{verse}
  
            \subsection*{49}
      \begin{verse}
      \poemtitle{De pluvia}Mirantibus cunctis nascens infligo querelas, \\ Statim deficio, qui maior a patre nascor. \\ Me gaudere potest nullus, si terrae coaequor; \\ Me cuncti superas laetantur carpere vias. \\ Improbus amara diffundo pocula totis \\ Et videre volunt quanti tantique refutant. \\ 
      \end{verse}
  
            \subsection*{50}
      \begin{verse}
      \poemtitle{De vino}Innumeris ego nascor de matribus unus, \\ Et genitus nullum viventem relinquo parentem. \\ Multa me nascente subportant vulnera matres, \\ Quarum mors mihi est potestas data per omnes. \\ Laedere non possum, me si quis oderit, umquam, \\ Et iniqua meo reddo quoque satis amanti. \\ 5 . \\ Multiplici veste natus de matre producor, \\ Nec habere corpus possum, si vestem amitto. \\ 
        \pagebreak 
    \begin{center} \textbf{CODICIS BERNENSIS 611.} \end{center} \marginpar{[367]} Meo subito nascor in ventre, fero parentes. \\ Nam vivo sepultus, vitam et inde resumo. \\ Deductus superis nec umquam crescere possum, \\ Dum natura facit corpus succedere plantis. \\ Item de rosa \\ Mollis ego duros de corpore genero natos; \\ In conceptu numquam amplexu viri delector. \\ Sed dum infra meis concrescunt filii latebris, \\ Nascens quisque meum disrumpit vulnere corpus. \\ Postquam velantes decorato tegmine matrem \\ †Saepe religati frangunt commune . fortes. \\ 53 . \\ Venter mihi nullus, infra praecordia nulla. \\ Nam tenui feror semper in corpore sicco. \\ Cibum nulli quaero, ciborum milia servans; \\ Loco currens uno lucrum ac confero damnum. \\ Membra mihi duo tantum in corpore pendent, \\ Similemque gerunt caput et planta figuram. \\ 54 . \\ Duo generant mnltos sub numero fratres \\ Nomine sub uno, divisos quisque naturam. \\ Pauper ac dives pari labore premuntur. \\ Pauperes semper habet, dives quos saepe requiret. \\ 
        \pagebreak 
    \begin{center} \textbf{AENIGMATA} \end{center} \marginpar{[368]} Caput illis nullum, sed os cum corpore cingunt. \\ Nunc stantes et nunc iacentes plurima portant. \\ 
      \end{verse}
  
            \subsection*{55}
      \begin{verse}
      \poemtitle{De sole}Semine nec ullo patris creata renascor, \\ Vbera nec matris suxi, quo crescere possem, \\ Vberibusque meis ego saepe reficio multos. \\ Vestigia nulla figens perambulo terras. \\ Non anima nec caro mihi sunt nec cetera membra, \\ Attamen aligeras reddo temporibus umbras. \\ 
      \end{verse}
  
            \subsection*{56}
      \begin{verse}
      \poemtitle{De verbo}Vna mihi soror est, unus et ego sorori. \\ Coniux illa mihi, huius et ego maritus. \\ Nam numquam uno sed†multorum coniungimurambo, \\ Sed de longe meam praegnantem reddo sororem. \\ Quotquot illa suo gignit ex utero partus, \\ Cunctos uno reddo tectos de peplo nepotes. \\ 
      \end{verse}
  
            \subsection*{57}
      \begin{verse}
      \poemtitle{De igne}Prohibeor solus noctis videre tenebras \\ Et absconse ducor longa per avia fugiens. \\ Nulla mihi velox avis inventa volatu, \\ Cum videar nullas gestare corpore pinnas. \\ Vix auferre praedam me coram latro valebit. \\ Corpore defecta velox conprehendo senectam. \\ 
        \pagebreak 
    \begin{center} \textbf{CODICIS BERNENSIS 611.} \end{center} \marginpar{[369]} 
      \end{verse}
  
            \subsection*{58}
      \begin{verse}
      \poemtitle{De rota}Assiduo multas vias itinere currens \\ Versa vice rerum conpellor ire deorsum \\ Et ab ima redux trahor conscendere sursum. \\ Sed cum mei parvum cursus conplevero tempus, \\ Publica per diem dum semper conpita curro, \\ Infantia †pars simul est et curva senectus. \\ 
      \end{verse}
  
            \subsection*{59}
      \begin{verse}
      \poemtitle{De luna}Quo movear gressu, nullus cognoscere temptat, \\ Cernere nec vultus per diem signa valebit. \\ Quotidie currens vias perambulo multas \\ Et bis iterato cunctas recurro per annum. \\ Imber nix pruina glacies nec fulgura nocent, \\ Timeo nec ventum forti testudine tecta. \\ 
      \end{verse}
  
            \subsection*{60}
      \begin{verse}
      \poemtitle{De caelo}Promiscuo per diem vultu dum reddor amictus, \\ Pulcher saepe, sed turpis qui saepe babetur, \\ Innumeras ego res cunctis fero mirandas, \\ Pondere sub magno rerum nec gravor onustus. \\ Nullus mihi dorsum, faciem sed cuncti mirantur, \\ Et meo cum bonis malos recipio tecto. \\ 
        \pagebreak 
    \begin{center} \textbf{370 ANIGMATA CODICIS BERNENSI 61.} \end{center}\poemtitle{De stellis}Milia conclusae domo sub una sorores: \\ Minima non crescit, maior nec aevo senescit. \\ Et cum nulla parem conetur alloqui verbis, \\ Suos moderato servant in ordine cursus. \\ Pulchrior torpentem vultu non despicit ulla. \\ Odiuntque lucem; noctis secreta mirantur. \\ 
      \end{verse}
  
            \subsection*{62}
      \begin{verse}
      \poemtitle{De umbra}Humidis delector semper consistere locis \\ Et sine radice inmensos porrigo ramos. \\ Mecum iter agens nulla sub arte tenebit, \\ Comitem sed viae ego conprehendere possum. \\ Certum me videnti demonstro corpus a longe, \\ Positus et iuxta totum me numquam videbit. \\ 6B De vino \\ Pulchrior me nullus versatur in poculis umquam, \\ Ast ego primatum in omnibus teneo solus. \\ Viribus atque meis possum decipere multos. \\ Leges atque iura per me virtutes amittunt. \\ Vario me si quis haurire voluerit usu, \\ Stupebit ingenti mea percussus virtute. \\ 
        \pagebreak 
    \begin{center} \textbf{CARMINA} \end{center}
      \end{verse}
  
            \subsection*{}
      \begin{verse}
      \poemtitle{CODICVM QVORVNDAM}
      \end{verse}
  
            \subsection*{}
      \begin{verse}
      \poemtitle{SAECVL.0 NON0 ANTIQVIOVM}
        \pagebreak 
     \marginpar{[482]} V ide nunc fasc. I n. . \\ 
      \end{verse}
  
            \subsection*{483}
      \begin{verse}
      \poemtitle{SISEVI}regls lothorum eplstul mlss ad Ildorum de llbro \\ r0tarum \\ B. V 6. M. 388. \\ B. V p. 357. \\ Tu forte in lucis lentus vaga carmina gignis \\ Argutosque inter latices et musica flabra \\ Pierio liquidam perfundis nectare mentem. \\ At nos congeries obnubit turbida rerum \\ Ferrataeque premunt milleno milite curae, \\ Legicrepae tundunt, latrant fora, classica turbant; \\ Et trans Oceanum ferimur porro usque nivosus \\ Cum teneat Vasco nec parcat Cantaber horrens. \\ 
        \pagebreak 
    \begin{center} \textbf{C. 488, 929} \end{center}En quibus indicas, ut crinem frondea Foebi \\ Succingant hederave comas augustius umbrent! \\ En quos flammantem iubeas volitare per aethram! \\ Quin mage pernices aquilas vis pira elefantum \\ Praecurret volucremque pigens testudo molossum, \\ Quam nos rorilluam sectemur carmine lunam. \\ His tamen. incurvus per pondera terrea nitens, \\ Dicam, cur fesso liescat circulus orbe \\ Purpureumque iubar nivei cur tabeat oris. \\ Non illam (ut populi credunt) nigrantibus antris \\ Infernas ululans mulier praedira sub umbras \\ Detrahit altivago e speculo, nec carmine victa \\ Vel rore Stygis aut herbis terrae aericrepantem \\ Vincibilemque petit clangorem quippe per aethram, \\ Qua citimus limes dispescit turbida puris, \\ lnviolata meat); sed vasto corpore tellus \\ (Quae medium tenet ima polum dum lumina fratris 2s \\ Detinet umbriferis metis, tum sidere casso \\ Pllescit, teres umbra rotae dum transeat himen \\ ggerei velox cumuli speculoque rotanti \\ Fraternas reparet per caelum libera lammas. \\ 
        \pagebreak 
    \begin{center} \textbf{C. 43, 3055} \end{center}Sed quia mira putas, cur, cum vis maxima solis \\ 30 \\ Bis novies maior clueat quam terreus orbis, \\ Non circumcingat terrestres lumine metas, \\ Sume ratum rationis opus. Namque aspice Foebum, \\ Quam sublimis eat convexa per aurea mundi \\ Quamque lumilem terram conlustret cursibus altis: \\ Hic ingens, utcumque libet, cum desuper ignes \\ Sparserit, obliquo vel cum radiaverit axe, \\ In terram radii franguntur. cetera solis \\ Lumina, qua maior iaculis radiantibus exit, \\ 0 Nil obstante lobo tendunt per inania vasta, \\ Donec pyramidis peragat victa umbra cacumen. \\ Per quam cum oebe gelidos agit uda iugales, \\ Infima vicinis nonnumquam decolor umbris \\ Fratre caret vacuoque exsanguis delicit ore. \\ Cur autem sola spolietur lumine luna, \\ 45 \\ Nil vero mirum est. quippe illam lucis egentem \\ Lnx aliena fovet; quam cum pars proxima metae \\ Invidet, epectat radios male caerula fratris. \\ At chorus astrorum reliquus non tangitur umbris, \\ Et proprium cunctis iubar est, nec sole rubescunt. \\ Sed sudus radiis astralibus inpete celso \\ Porro ultra solem rapitur cum vertice caeli. \\ lamn cur semenstri non semper palleat orbe, \\ Inflexi praestant elico tramite cursus. \\ Namque vagans errore rato cum devia tortos \\ 
        \pagebreak 
    \begin{center} \textbf{C. 4, 58 61 C. 484, 12} \end{center}Conligit anfractus, metam sol eminus exit \\ Intorquetque peplum noctis radiatque sororem. \\ Haec eadem ratio est, subitis ubi frangitur umbris \\ Augusti solis rutilum iubar, indiga lucis \\ Quando inter terram et solem rota corporis almi \\ Luna meat, fratrem rectis obiectibus arcens. \\ 
      \end{verse}
  
            \subsection*{484}
      \begin{verse}
      B. V 14. M. 1056. \\ \poemtitle{De ventis}B. V 383. \\ Quattuor a quadro consurgunt limite venti. \\ Hos circum gemini dextra laevaque iugantur \\ 
        \pagebreak 
    \begin{center} \textbf{C. 484. 322} \end{center}Atque ita bis seno circumdant flamine mundum. \\ Primus Aparctias Arctoo spirat ab axe: \\ Huic nostra nomen lingua est Septentrio fictum. \\ Circius hinc dextro gelidus circumtonat antro; \\ Thrascian Graeci propria dixere loquella. \\ Huic laevus oreas glaciali turbine mugit; \\ Frigidus hic Aquilo nostris vocitatur in oris. \\ At Subsolanus medio flat rectus ab ortu; \\ Graecus Aphelioten apto quem nomine signat. \\ Huic Vulturnus adest, dextra qui parte levatur; \\ Attica Caecian tGrais quem littera signat. \\ Nubifero flatu laevum latus inrigat Eurus, \\ Dorica quem simili designat nomine lingua. \\ t Notus e medio solis dat flamina cursu: \\ Austrum rite vocant, quia nubila flatibus haurit. \\ Euronotus cui dexter adest, quem nomine mixto \\ Euroaustrum Latia dixerunt voce Latini. \\ Libonotus laevam calidis attaminat auris; \\ Aestibs inmensis ardens Austroafricus hir est. \\ Abscessum solis Eephyri tuba florea servat, \\ 
        \pagebreak 
    \begin{center} \textbf{C. 484, 2327. C. 484 . 18} \end{center}Italia nomen cui fixum est voce Favoni. \\ Huic dextram tangit dictus Lips Atthide lingua; \\ fricus hinc propria veniens regione vocatur. \\ At tu, Core, fremis Eephyri de parte sinistra; \\ Argesten Grai vocitant cognomine prisco. \\ 
      \end{verse}
  
            \subsection*{484a}
      \begin{verse}
      Versus sei AVSII episcopi \\ rAvrrr \\ Donatistarum crudeli caede peremptum \\ Infossum hic corpus pia est cum laude Nabori. \\ Ante aliquod tempus cum Donatista fuisset, \\ Conversus pavem, pro qua moreretur, amavit. \\ Optima purpure vestitur sanguine causa. \\ Non errore perit, non se ipse furore peremit, \\ Verum martyvrium vera est pietate probatum. \\ Suspice litterulas primas: ibi nomen honoris. \\ 
        \pagebreak 
    \begin{center} \textbf{C. 484, 1. C. 485, 1—10} \end{center}
      \end{verse}
  
            \subsection*{484b}
      \begin{verse}
      Versus CPERI hetoris \\ rp \\ Quisque gravas lacrimis Iilarini flebile marmor, \\ Fleto aviam potins duram vivacibus annis. \\ Ille deo meruit, tenero praelectus in aevo \\ Vivere tiro brevis, sed iam snb milite Christi . . \\ 
      \end{verse}
  
            \subsection*{185}
      \begin{verse}
      B. 4. \\ B. III 272. \\ \poemtitle{De flguris vel sehematibus}Collibitum est nobis, in lexi schemata quae sunt, \\ Trino ad te, Messi, perscribere singula versu, \\ Et prosa et versu pariter praeclare virorum! \\ op \\ Particulae membra efficiunt, haec circuitum omnem. \\ Particula est comma. ut versu tria commata in illo: \\ ‘Arcadiam petis, inmensum petis, haud tribuam istnd.’ \\ olo \\ Membra ea sunt, quae cola vocant; ea circuitum explent. \\ Nam qui eadem vult ac non vult’, colon facit unum. \\ Inic adinnge sequens: ‘ is demum est firmus amicus. \\ Ipooodo \\ Circuitus, periquam dicuntodos, orta duobus \\ 
        \pagebreak 
     \marginpar{[10]} \begin{center} \textbf{C. 485, 1128} \end{center}Membris, ut praedicta, venit tetracolon adusque. \\ Nam si plura itidem iungas, oratio fiet. \\ ilo \\ Est reflexio, cum contra reflectimu’ dicta. \\ Non expecto tuam mortem, pater’, inquit. at ille \\ ‘Immo’, ait, ‘expectes oro neve interimas me.’’ \\ fverol \\ Permutatio fit, vice cum convertimu’ verba. \\ ‘Sumere iam cretos, non sumptos cernere amicos.’ \\ ‘Quom queo, tempus abest; \lbrack quom tempus adest \rbrack ‘nequeo’ \\ inquit. \\ loioc \\ Dieritas fit, differre hoc ubi dicimus illi. \\ ‘Excitat hunc cantus galli, te bucina torva. \\ Te ciet armatus victus, huic otia cordi.’ \\ risco \\ Oppositum dico, contra cum opponimu quaedam. \\ ‘Doctor tute, ego discipulus.’ Tu scriba, ego censor.’ \\ ‘Histrio tu, spectator ego; adque ego sibilo, tu exis.’ \\ firoio \\ Redditio causae porro est, cum, cur ita, dico. \\ ‘Audi, etsi durum est; nam verum quod grave primo \\ Consilium acciderit, fit iucundum utilitate.’ \\ roopo \\ At si adversa mihi referam, relatio fiet. \\ 
        \pagebreak 
    \begin{center} \textbf{C. 485, 29—46} \end{center}‘‘Sed moveas te, lucifugus, sis in medio audax!’ \\ ‘Landes inductus cui pes malus optige ambos?t \\ Eipc \\ Fit responsio ad haec, quae contra fingimu’ dici. \\ ‘Irascetur: sperne. dabit damnum: reparabis. \\ Caedet: ne toleres. at sum minor: emorere, inquam.’ \\ Eroop \\ Est repetitio, cum verbo saepe incipio  \lbrack uno \rbrack . \\ ‘pse epulans, ipse exposcens laeta omnia nuptae, \\ lpse patrem prolemque canens, idem ipse peremit.’ \\ copo \\ Desitio contra, cum verbo desino in uno. \\ Vt possem, fecit fatum; dedit baec mihi fatum: \\ Si perdam, abstulerit fatum: regit omnia fatum.’ \\ oirc \\ HHaec duo coniunctim faciunt, communio uti sit. \\ Vis callere aliquid: discas. vis nobilitari \\ Ilngenio: discas. vis famam temnere: discas.’ \\ ELaolovg \\ Fit replicatio, si gemines iteramine quaedam. \\ ‘Ibo in eum, sit vel pollens ut fulmine dextra, \\ Pollens fulmine dextra, fero bis praedita ferro.’ \\ Bpelo \\ Est brevitas, raptim paucis cum dicimu’ multa. \\ 
        \pagebreak 
    \begin{center} \textbf{C. 48, 4763} \end{center} \marginpar{[10]} ‘Mentem contempla; nam consilio valuit: fors \\ Decepit.’ ‘Si peccat, homo est. concede; fatetur. \\ foop \\ Si verbum diverse iteres, distinctio fiet. \\ ‘Cuivis hoc homini dones: homo si modo, nolit.’ \\ 50 \\ ‘0 mulier, vere mulier! scelera omnia in hoc sunt.’ \\ Ioovidsro \\ Multiiuum dico, articulis quod pluribu’ iungo. \\ ‘ \lbrack lle hunc fallit, at hic gaudet, nos vero timemus \\ Praesertim in peregri ne fas abrumpere tentet.’ \\ feso \\ Abiunctum contra est, si nullis singula necto. \\ 55 \\ Cognoscas, qui sis, cnres te, vir sapiens sis, \\ Et prius verbis time illum quaelibet unum.’ \\ fgo \\ Disparsum reddo, quod passum non ordine reddo’ \\ ‘Ambo Iovis merito proles, verum ille equitando \\ Insignis Castor, catns hic pugilamine Pollux.’ \\ 60 \\ foo: \\ Fit percursio, percurro cum singula raptim. \\ ‘Vim maiorem haud inveniet, parilem similemve \\ Vincemus, non audebit certare minore.’ \\ 
        \pagebreak 
     \marginpar{[13]} \begin{center} \textbf{C. S, 64—s1} \end{center}Ezloi \\ Fit conexio, posterius si necto priri. \\ ‘Cum sensi, dixi; cum dixissem, inde suasi; \\ Cum suasissem, abii; simul atque abii, indupetravi.’ \\ rlt: \\ Illa resumptio lit, quaedam cum dicta resumo. \\ Cognitus est nobis, iam cognilus, ac bene novi.’ \\ Tu vero sapiens † cunctis, immo ipsa Minerva.’ \\ roa \\ o Fit concessio, cum quidvis concedimus optet. \\ Necivit vel non potuit vel noluit: ut vis, \\ Pone, tibi permitto; tamen non debuit uti.’ \\ oooupeo \\ Intersertio, cum inseritur sententia quaedam. \\ ‘Pollet enim forma, quod regnum aetatis habendum est, \\ s Fortuna, quae sola potest quemcunque beare.’ \\ Fit geminatio, vum sensus geminamus eosdem. \\ Thebae autem, Thebae, vicina urbs inclytaque olim.’ \\ Mi nate, o mi nate, meae spes sola senectae.’ \\ Easooc \\ Exclamatio ea est, quam ut motus reddo repente. \\ 0 A, postquam victum video me, tu improba et amens, \\ Fortuna, es, quos sublimas mox ipsa premendo.’ \\ 
        \pagebreak 
    \begin{center} \textbf{C. 485, 8299} \end{center} \marginpar{[14]} Iioo \\ Fit parimembre, ubi membra †aequa et circuitus sunt. \\ ‘Cui nec finis adest cupiendi nec modus extat \\ Vtendi, citus in dando est, celer in repetendo.’ \\ Mepoogc \\ Cum privis propria attribuas, fit distribuela. \\ ‘Huic furta in manibus, fuga plantis, ventre sagina.’ \\ ‘Tu sumptu pauper, dando dis, ingenio rex.’ \\ Mro \\ At remeatio fit, cum rursus me redigo ad rem. \\ ‘Verum longius excessi nec tempore in ipso \\ Fortasse indulgens animis: ergo redeo illuc.’ \\ fepoos \\ Fit variatio, cum simili re nomina muto. \\ ‘Regnavit Libyco generi, regnavit et Argis \\ Inachiis, dominatus item est apud Oebaliam arcem.’ \\ fsriv \\ Declinatio, cum verbo declino parumper. \\ ‘A primo puerum rertum est condiscere recte.’ \\ ‘Dinos digna manent, plerumque bonis bene vortit.’ \\ Opgi \\ definioprome.Detinitiofit,cumrem \\ ‘Diligere, hoc prorsum est velle id,  \lbrack quod prosiet illi \\ Nam qui ad se revocat quod vult, mihi sese amat ipse.’ \\ 
        \pagebreak 
    \begin{center} \textbf{C. 485, 100 117} \end{center} \marginpar{[15]} Oooeero \\ Confine est, simili fini cum claudimu’ quaedam. \\ ‘Quom minus indignatur, ibi magis insidiatur, \\ Vt metuas noxam, si non ostenderit iram.’ \\ Opoir ro \\ Aequeclinatum est, quod casu promimus uno. \\ ‘Auxilium, non consilium, rata, non cata verba, \\ Rem, non spem, factum, non dictum quaerit amicus.’ \\ Iolaoro \\ Multiclinatum contra, variantibu quod fit. \\ Tu solus sapiens, tibi cuncti cedere debent, \\ A te consilium petere et tua dicta probare.’ \\ Ipoof \\ Supparile est, alia aequisono si nomine dicas. \\ ‘Mobilitas, non nobilitas.’ ‘bona gens, mala mens est.’ \\ ‘Dividiae, non divitiae.’ ‘ibi villa favilla est.’ \\ lpooarodoo \\ Est subnexio, propositis subnectere quaeque. \\ ‘t nos non ut tu: nos simplicitate, tu arte.’ \\ ‘Hoc das, hoc adimis nobis: das spes, adimis res.’ \\ lpoadoro \\ Subdistinctio fit, cum rem distinguimus ab re. \\ ‘Dum fortem, quia sit vaecors, comemque vocat se, \\ Quod sit prodigus, et clarum, qui infamis habetur.’ \\ 
        \pagebreak 
     \marginpar{[16]} \begin{center} \textbf{C. 485. 11 —135} \end{center}Ipose \\ Interiectio, cum quaedam medio ordine famur. \\ ‘lluc ut venimus, interea nam tempus erat ver, \\ Et sacrum lorae et Cereri nemus imus ad aras.’ \\ lIpopoloi \\ Est suffessio, cum sensim pro parte fatemur. \\ Verum Academicus est.’ ‘Esto: tamen omnia nulli \\ In dubium revocant.’ ‘t quaedam; at pleraque, si vis.’ \\ lpois \\ Anticipatio fit, contraria cum occupo verba. \\ ‘Credo, ille et flebit multum et iurabit, amicos \\ 125 \\ Producet testes: sed vos rem quaerere par est.’ \\ Hapipoo \\ dsimula, momento cum simile hoc facio illi \\ ‘Nam plebeius homo, ut ferme tit libera in urbe, \\ eg \lbrack nat \rbrack  ibi, et puncto reguat suflragioloque.’ \\ lgpo \\ Inreticentia, cum verum reticere negamus. \\ 130 \\ ‘Dicere, quod res est, cogor; vos ista, Quirites, \\ Vos facitis, dum non dignis donatis honores.’ \\ Ipirc \\ Propositum, cum proponas, quod deinde repellas. \\ ‘Est ornanda domus spoliis: hic ornat amicam \\ Exuviis. leges discendum est: discit amores.’ \\ 135 \\ 
        \pagebreak 
    \begin{center} \textbf{C. 48, 136 15} \end{center}\begin{center} \textbf{1 4} \end{center}Hr apc a \\ Cuncta ad cuncta, ut: ‘Gens Graia,Afra, Hispanica servit; \\ Nam partim meritost ultus, partim insidiantes \\ Praevenit, partim victor virtute subegit.’ \\ eapoopi \\ Est conductio conquegregatio, cum adcumulo res. \\ ‘Multa hortantur me: res, aetas, tempus, amici, \\ Concilium tantae plebis, praenuntia vatum.’ \\ oaefo \\ Conciliatio, diversum si conciliamus. \\ ‘Prodigus  \lbrack est \rbrack  et parcus idem:  \lbrack nam \rbrack  nescit uterque \\ Vti opibus, peccant ambo; res deterit ambos. \\ po \\ Teriuga sunt, quae respondent secum ordine trino. \\ ‘Si neque divitiis polles neque corpore praestas \\ Nec corde exuperas, cur te dicam esse beatum?’ \\ pvpsopic \\ Fit depictio, cum verbis ut imagine pingo. \\ ‘Pocula, serta tenens flexa cervice iacebat, \\ Limodes, gravis optutu, madido ore renidens.’ \\ Errc \\ Est correctio, cum in quodam me corrigo dictu. \\ ‘Nam tarde tandem tarde dico? immo hodie, inquam’ \\ Vel sic: ‘non amor est, verum ardor vel furor iste.’ \\ 
        \pagebreak 
     \marginpar{[18]} \begin{center} \textbf{C. 4, 154 16} \end{center}Hpoapvc \\ Fit praeoccursio, si reddas priu’ posteriori. \\ V: ‘pluvias cernas nolle istos ac cupere illos: \\ 155 \\ Lirantes cupiunt imbrem noluntque viantes.’ \\ Efrpoe \\ Esse reversio et in prosa solet, ut fit in istis: \\ ‘Pauxillam ob culpam’ ‘male quod vult.’ ‘praecipiti \\ in re.’ \\ Troianos facit ire ut divus lHomerus ‘aves ut.’ \\ ‘eoari \\ Transcensus porro est, cum intersita pendula claudo. \\ ‘Atque ego, quod negat hic vivis, ius eripit ome, \\ Fas abolet, laedit leges, haec omnia mitto.’ \\ Mrsrfo \\ Exadversio fit, minimis si maxima monstres. \\ Non parva est res, qua de agitur’ pro ‘maxima res est,’ \\ V dictust Aiax uon infortissimu’ Graium.’ \\ 165 \\ eep \\ Nexum est, si varias res uno nectimu’ verbo. \\ ‘Oebalon ense ferit, Lycon hasta, Pedason arcu.’ \\ Nam mediost ‘ferit’ et fini pote principioque. \\ 
        \pagebreak 
    \begin{center} \textbf{C. 485, 169 186} \end{center} \marginpar{[19]} Msrdoi \\ Si verbum varie mutes, variatio fiet. \\ ‘Quis nos propter te dilexit? quando aliquem tu \\ Iuvisti? quas res gessisti? cur ita abundas?’ \\ Elofov aut ‘rali \\ Fit mutatio multimodis. eltto Africa flagrat,’ \\ Afros cum dicas bellare. et tempora quando \\ Et casus numerosque figurando variamus. \\ Els \\ Fit defectio, cum verbum, quod subtraho grate, \\ Detfit. ‘Curat enim nemo nec corrigit hanc rem, \\ Sed culpat.’ quippe hic ‘quisquamV’ subtraximu’ grate. \\ eoaopc \\ Exuperatio fit, cum causa appono decoris, \\ Quod vacat, ut: ‘quarta vix demum exponimur hora;’ \\ Saucius ille leo’ quia ‘vix’ pote tollere et ‘ilte? \\ Hpoiopoc \\ Est autem circum illa locutio: ‘bucera saecla’ \\ ‘Tae discas’ pro ‘disce’, et pro ‘di’ ‘dice loquendo.’ \\ lpovocoo \\ Si plenum cumules, adsigniticatio fiet. \\ Vt ‘mihi non placet hoc animo.’ quippe  \lbrack ‘hoc \rbrack  animo’ \\ aufer, \\ Et nihilo minus est plenum; verum auxerit illud. \\ 
        \pagebreak 
     \marginpar{[20]} \begin{center} \textbf{C. 48 , 1—1} \end{center}
      \end{verse}
  
            \subsection*{485a}
      \begin{verse}
      \poemtitle{LACCANVEII}A \\ B. M. \\ \poemtitle{De ave phoenioe}B. III 253. \\ Est locus in primo felix oriente remotus, \\ Qua patet aeterni maxima porta poli, \\ Nec tamen aestivos hiemisve propinquus ad ortus, \\ Sed qua sol verno fundit ab axe diem. \\ Illic planities tractus diffundit apertos, \\ Ne tumulus crescit nec cava vallis hiat; \\ Sed nostros montes, quorum iuga celsa putantur, \\ Per bis sex ulnas imminet ille locus. \\ Hi Solis nemus est et consitus arbore multa \\ Lucus, perpetuae frondis honore virens. \\ Cum Phaethonteis flagrasset ab ignibus axis, \\ lle locus flammis inviolatus erat, \\ cum diluvium mersisset fluctibus orbem, \\ 
      \end{verse}
  
            \subsection*{}
      \begin{verse}
      \poemtitle{E1}Dencalioneas exsuperavit aquas. \\ 
        \pagebreak 
    \begin{center} \textbf{C. 485 , 15—39} \end{center} \marginpar{[21]} Non huc exsangues morbi, non aegra senectus, \\ Nec mors crudelis nec metus asper adest; \\ Nec scelus infandum nec opum vesana cupido \\ Cernitur aut ardens caedis amore furor; \\ Luctus acerbus abest et egestas obsita pannis \\ Et curae insomnes et violenta fames. \\ Non ibi tempestas nec vis furit horrida venti \\ Nec gelido terram rore pruina tegit, \\ Nulla super campos tendit sua vellera nubes, \\ Nec cadit ex alto turbidus umor aquae. \\ Sed fons in medio  \lbrack est \rbrack , quem ‘vivum’ nomine dicunt, \\ Perspicuus, lenis, dulcibus uber aquis, \\ Qui semel erumpens per singula tempora mensum \\ Duodecies undis inrigat omne nemus. \\ Hic genus arboreum procero stipite surgens \\ Non lapsura solo mitia poma gerit. \\ Hoc nemus, hos lucos avis incolit upica Pboenix: \\ Vnica sed vivit morte refecta sua. \\ Paret et obsequitur Phoebo memoranda satelles: \\ Hoe Natura parens munus habere dedit. \\ Lutea cum primum surgens Aurora rubescit, \\ Cum primum rosea sidera luce fugat, \\ Ter quater illa pias inmergit corpus in undas, \\ Ter quater e vivo gurgite libat aquam. \\ Tollitur ac summo considit in arboris altae \\ 
        \pagebreak 
    \begin{center} \textbf{C. 485, 40—63} \end{center} \marginpar{[00]} Vertice, quae totum despicit una nemus, \\ conversa novos Phoebi nascentis ad ortus \\ Expectat radios et iubar exoriens. \\ Atque ubi Sol pepulit fulgentis limina portae \\ Et primi emicuit luminis aura levis, \\ Incipit illa sacri modulamina fundere cantus \\ Et mira lucem voce ciere novam, \\ Quam nec acdoniae voces nec tibia possit \\ Musica Cirrhaeis adsimulare modis, \\ Sed neque olor moriens imitari posse putatur \\ Nec Cylleneae tila canora lyrae. \\ Postquam Phoebus equos in aperta effudit lympi \\ Atque orbem totum protulit usque means, \\ Illa ter alarum repetito verbere plaudit \\ Igniferumque caput ter venerata silet. \\ Atque eadem celeres etiam discriminat horas \\ Innarrabilibus nocte dieque sonis, \\ Antistes luci nemorumque verenda sacerdos \\ Et sola arcanis conscia, Phoebe, tuis. \\ Quae postquam vitae iam mille peregerit annos \\ Ac si reddiderint tempora longa gravem, \\ Vt reparet lapsum spatiis vergentibus aevum, \\ Adsuetum nemoris dulce cubile fugit. \\ Cumque renascendi studio loca sancta reliquit, \\ 
        \pagebreak 
    \begin{center} \textbf{C. 485, 685} \end{center}Tunc petit lunc orbem, mors ubi regna tenet. \\ Dirigit in Syriam celeres longaeva volatus, \\ Phoenices nomen cui dedit ipsa vetus, \\ Secretosque petit deserta per avia lucos, \\ Sicubi per saltus silva remota latet. \\ Tum legit aerio sublimem vertice palmam, \\ Quae Graium phoenix ex ave nomen habet, \\ ln quam nulla nocens animans prorepere possit, \\ Lubricus aut serpens aut avis ulla rapax. \\ Tum ventos claudit pendentibus Aeolus antris, \\ Ne violent flabris ara purpureum \\ Neu concreta noto nubes per inania caeli \\ Submoveat radios solis et obsit avi. \\ Construit inde sibi seu nidum sive sepulchrum; \\ Nam perit,. ut vivat: se tamen ipsa creat. \\ Colligit hinc sucos et odores divite silva, \\ Quos legit Assyrius, quos opulentus Araps, \\ Quos aut Pygmaeae gentes aut lndia carpit \\ Aut molli generat terra Sabaea sinu. \\ Cinnamon hic auramque procul spirantis amomi \\ Congerit et mixto balsama cunt folio: \\ Non casiae mites nec olentis vimen acanthi \\ 
        \pagebreak 
    \begin{center} \textbf{C. 48, 86 102} \end{center} \marginpar{[24]} Nec turis lacrimae guttaque pinguis abest. \\ His addit teneras nardi pubentis aristas \\ Et sociat myrrae vim, panacea, tuam. \\ Protinus instructo corpus mutabile nido \\ Vitalique toro membra vieta locat. \\ Ore dehinc sucos membris circumque supraque \\ Inicit, exequiis inmoritura suis. \\ Tunc inter varios animam commendat odores, \\ Depositi tanti nec timet illa fidem. \\ Interea corpus genitali morte peremptum \\ Aestuat, et flammam parturit ipse calor, \\ Aetherioque procul de lumine concipit ignem: \\ Flagrat, et ambustum solvitur in cineres. \\ Quos velut in massam, generans in morte, coactos \\ Conflat, et effectum seminis instar habet. \\ 100 \\ Hinc animal primum sine membris fertur oriri, \\ Sed fertur vermi lacteus esse color. \\ 
        \pagebreak 
    \begin{center} \textbf{C. 485, 103 12} \end{center} \marginpar{[25]} Crescit, et emenso sopitur tempore certo, \\ Seque ovi teretis colligit in speciem. \\ Ac velut agrestes, cum filo ad saxa tenentur, \\ Mutari tineae papilione solent, \\ Inde reformatur qualis fuit ante figura, \\ Et Phoenix ruptis pullulat exuviis. \\ Non illi cibus est nostro concessus in orbe, \\ Nec cuiquam inplumem pascere cura subest. \\ Ambrosios libat caelesti nectare rores, \\ Stellifero tenues qui cecidere polo. \\ Hos legit, his alitur mediis iu odoribus ales, \\ Donec maturam proferat effigiem. \\ Ast ubi primaeva coepit florere iuventa, \\ Evolat, ad patrias iam reditura domus. \\ Ante tamen, proprio quidquid de corpore restat, \\ Ossaque vel cineres exuviasque suas \\ Vnguine balsameo myrraque et ture Sabaeo \\ Condit et in formam conglobat ore pio. \\ Quam pedibus gestans contendit Solis ad ortus \\ Inque ara residens ponit iu aede sacra. \\ Mirandam sese praestat praebetque verendam: \\ Tantus avi decor est, tantus abundat honor. \\ 
        \pagebreak 
     \marginpar{[26]} \begin{center} \textbf{C. 485, 125—145} \end{center}Primo qui color est malis sub sidere Cancri, \\ 125 \\ Cortice quae croceo Punica grana tegunt; \\ Qualis inest foliis, quae fert agreste papaver, \\ Cum pandit vestes lora rubente solo: \\ Hoc humeri pectusque decens velamine fulget; \\ Ioc caput, hoc cervix summaque terga nitent. \\ Caudaque porrigitur fulvo distincta metallo, \\ In cuius maculis purpura mixta rubet. \\ Alarum pennas insignit desuper iris, \\ Pingere ceu nubem desuper aura solet. \\ Albicat insignis mixto viridante zmaragdo \\ 135 \\ Et puro cornu gemmea cuspis hiat. \\ lngentes oculi: credas geminos hyacinthos, \\ Quorum de medio lucida flamma micat. \\ Arquatur cuncto capiti radiata corona, \\ Phoebei referens verticis alta decus. \\ 140 \\ Crura tegunt squamae fulvo distincta metallo; \\ Ast ungues roseo tinguit honore color. \\ f r er. \\ Cernitur et pictam Phasidis inter avem. \\ Magnitiem terris Arabum quae gignitur ales \\ 145 \\ 
        \pagebreak 
    \begin{center} \textbf{C. 48 , 146168} \end{center}Vix aequare potest, seu fera seu sit avis. \\ Non tamen est tarda ut volucres, quae corpore magno \\ Incessus pigros per grave pondus habent, \\ Sed levis ac velox, regali plena decore: \\ Talis in aspectu se tenet usque hominum. \\ HIuc venit Aegyptus tanti ad miracula visus \\ Et raram volucrem turba salutat ovans. \\ Protinus exculpunt sacrato in marmore formam \\ Et titulo signant remque diemque novo. \\ Contrahit in coetum sese genus omne volantum, \\ Nec praedae memor est ulla nec ulla metus. \\ Alituum stipata choro volat illa per altum \\ Turbaque prosequitur unere laeta pio. \\ Sed postqua puri pervenit ad aetheris auras, \\ Mox redit: illa suis conditur inde locis. \\ O0 fortunatae sortis felixque volucrum, \\ Cui de se nasvi praestitit ipse deus! \\ Femina seu  \lbrack sexu seu† masculus est seu neutrum \\ Felix, quae eneris foedera nulla colit! \\ Mor illi Venus est, sola est in morte voluptas: \\ Vt possit nasci, appetit ante mori. \\ Ipsa sibi proles, suus est pater et suus heres, \\ Nutrix ipsa sui, semper alumna sibi. \\ 
        \pagebreak 
     \marginpar{[28]} \begin{center} \textbf{C. 48, 160—170. C. 485, 12.} \end{center}psa quidem, sed non  \lbrack eadem est \rbrack , eademque nec ipsa est, \\ Aeternam vitam mortis adepta bono. \\ 170 \\ B. M. \\ 
      \end{verse}
  
            \subsection*{485}
      \begin{verse}
      B. III 169. \\ Iustius invidia nihil est, quae protinus ipsum \\ Auctorem rodit excruciatque animum. \\ 
        \pagebreak 
    \begin{center} \textbf{C. 4857,. 1. C. 486. 1} \end{center} \marginpar{[29]} 
      \end{verse}
  
            \subsection*{485c}
      \begin{verse}
      B. M. B. \\ Corduba me genuit, rapuit Nero, praelia dixi. \\ 
      \end{verse}
  
            \subsection*{486}
      \begin{verse}
      \poemtitle{REMI TAVINI}B. M. \\ \poemtitle{De ponderibus}B. V 71. \\ veterummemoratalibellisPonderaPaeoniis \\ 
        \pagebreak 
     \marginpar{[30]} \begin{center} \textbf{C. 480, 229} \end{center}Nosse iuvat. pondus rebus natura locavit \\ Corporeis: elementa suum regit omnia pondus. \\ Pondere terra manet: vacuus quoque ponderis aether \\ Indefessa rapit volventis sidera mundi. \\ Ordiar a minimis, post haec maiora sequentur. \\ Nam maius nihil est aliud quam multa minora. \\ Semioboli duplum est obolus, quem pondere duplo \\ Gramma vocant, scriplum nostri dixere priores. \\ Semina sex alii siliquis latitantia curvis \\ Attribuunt scriplo, lentis vel grana bis octo, \\ Aut totidem speltas numeranut tristesve lupinos \\ is duo; sed si par generatim his pondus inesset, \\ Servarent eadem diversae pondera gentes. \\ Nunc variant: etenim cuncta haec non foedere certo \\ Naturae, sed lege valent hominumque repertis. \\ Scripla tria dragmam vocitant, quo pondere doctis \\ Argenti facilis signatur nummus Athenis; \\ Olceque a dragma non re sed nomine differt. \\ Dragmam si gemines, erit is quem dicier audis \\ 20 \\ Sicilicum: dragmae scriplum si adiecero, fiet \\ Sextula quae fertur; nam sex his uncia constat. \\ Sextula cum dupla est, veteres dixere duellam. \\ Vncia fit dragmis bis quattuor; unde putandum \\ Grammata dicta, quod haec viginti et quattuor in se s \\ Vncia habet tot enim foris vox nostra notatur, \\ Horis quot mundus peragit noctemque diemque. \\ Vnciaque in libra pars est quae mensis in anno. \\ HaecmagoLatiolibraestgentiquetoatae: \\ 
        \pagebreak 
     \marginpar{[21]} \begin{center} \textbf{C. 486, 30— 54} \end{center}\begin{center} \textbf{50 .} \end{center}0 ttica nam minor est: ter quinque hanc denique dragmis \\ Et ter vicenis tradunt explerier unam. \\ Accipe praeterea, parvo quam nomine Grai \\ Mnam vocitant nostrique minam dixere priores. \\ Centum hae sunt dragmae; quod si decerpseris illi \\ Quattuor, efficies hanc nostram denique libram; \\ Attica quae fiet, si quartam dempseris unam. \\ Cecropium superest post hae vocitare talentum \\ Sexaginta minas, seu vis, sex milia dragmas, \\ Quod summum doctis perhibetnr pondus Athenis; \\ 0 Nam nihil his obolove minus maiusve talento. \\ Nunc dicam, solidae quae sit divisio librae \\ Sive assis (nam sic legum dixere periti), \\ Ex quo quod soli capimus perhibemur habere, \\ Dicimnr aut partis domini pro partibus huius. \\ Vncia si librae desit, dixere deuncem, \\ Ac si sextantem retrahas, erit ille decuncis. \\ Sed nullum reliquo nomen semuncia certum \\ Dempta dabit, neque  \lbrack quae \rbrack  est huius sescuncia triplex. \\ Dodrantem reliquum vocitant quadrante retracto; \\ Cnmque triens desit, bessem dixere priores. \\ dem septuncem dempto quincunce vocarunt. \\ Post haec semissis solidi pars maxima fertur; \\ Nam quae dimidium snperat, pars esse negatur, \\ Vt docuit tenui scribens in pulvere Musa. \\ 
        \pagebreak 
     \marginpar{[20]} \begin{center} \textbf{C. 46, 5578} \end{center}Cetera dicta prius, quibus est semuncia maior. \\ Haec de ponderibus: superest pars altera nobis \\ Vmida metiri, seu frugum semina malis. \\ Cuius principio nobis pandetur origo. \\ Pes longo in spatio latoque altoque notetur, \\ Angulus ut par sit quem claudit linea triplex, \\ Quattuor et medium quadris cingatur inane: \\ Amphora fit cybus hic, quam ne violare liceret, \\ Sacravere lovi Tarpeio in monte Quirites. \\ Huius dimidium fert urna, ut  \lbrack et \rbrack  ipsa medimni \\ Amphora, terque capit modium; sextarius istum \\ Sedecies haurit, quot solvitur in digitos pes. \\ At cotylas, quas si placeat dixisse licebit \\ Eminas, recipit geminas sextarius unus, \\ Quis quater adsumptis tit Graio nomine choenix. \\ Adde duos, chus tit, vulgo qui est congius idem, \\ E quo sextari nomen fecisse priores \\ Crediderim, quod eos recipit sex congius unus. \\ At cotyle cyathos bis ternos una receptat. \\ Sed cyatho nobis pondus quoque saepe notatur. \\ Bis quinae hunc faciunt dragmae, si adpendere malis: s \\ Oxybaphon fiet, si quinque addantur ad istas. \\ At mystrum cyathi quarta est; sed tertia mystri \\ Quam vocitant cbemen, capit haec coclearia bina. \\ 
        \pagebreak 
     \marginpar{[00]} \begin{center} \textbf{C. 486, 79105} \end{center}\begin{center} \textbf{0i0} \end{center}Quod si mensurae pondus conponere fas est, \\ s0 Sextari cyathus pars est quae est uncia librae; \\ Nec non oxybaphi similis sescuncia fiet, \\ Sicilicumque tibi mystro simulare licebit. \\ Coclear extremum est scripulique imitabitur instar. \\ Attica praeterea discenda est amphora nobis \\ Seu cadus: hanc facies, nostrae si adieceris urnam. \\ Est et bis decies quem conficit amphora nostra \\ Culleus: hac maior nulla est mensura liquoris. \\ Est etiam terris quas aduena Nilnus inundat \\ rtaba, cui superest modii pars tertia post tres, \\ Namque decem modiis explebitur artaba triplex. \\ Illud praeterea veteres perbihere memento, \\ Finitum pondus varios servare liquores. \\ Nam librae, ut memorant, bessem sextarius addit, \\ Seu puros pendas latices seu dona lLyaei. \\ Addunt semissem librae labentis olivae \\ Selibramque fernnt mellis superesse bilibri. \\ Ilaec tamen adsensu facili sunt credita nobis: \\ Namque nec errantes undis labentibs amnes \\ Nec mersi puteis latices aut fonte perenni \\ oo Manantes par pondus habent, non denique vina \\ Quae campi et colles nuperve aut ante tulere. \\ Quod tibi mechanica promptum est deprendere Musa. \\ Ducitur argento tenuive ex aere cylindrus, \\ Quantum inter nodos fragilis producit harundo, \\ Cui cono interius modico pars ima gravatur, \\ 
        \pagebreak 
     \marginpar{[34]} \begin{center} \textbf{C 486., 106—134} \end{center}Ne totus sedeat totusve supernatet undis. \\ Lineaque a snmmo tenuis descendit ad imam \\ Ducta superficiem, totidemque in frusta secatur, \\ Quot scriplis gravis est argenti aerisve cylinudrus. \\ Hoc cuiusque potes pondus spectare liquoris. \\ 110 \\ Nam si tenuis erit, maior pars mergitur unda; \\ Sin gravior, plures modulos superesse notabis. \\ Quod si tantundem laticis sumatur utrimque, \\ Pondere praestabit gravior; si pondera secum \\ Convenient, tum maior erit quae tenuior unda est; 1 \\ Ac si ter septem uumeros texisse cylindri \\ Hos videas latices, illos cepisse ter octo, \\ His dragma gravius fatearis pondus inesse. \\ Sed refert aequi tantum conferre liquoris, \\ Vt gravior superet dragma, quantum expulit undae \\ Illius aut huius teretis pars mersa cylindri. \\ Hlaec de mensuris. quarum si signa requires, \\ ipsis veterum poteris cognoscere chartis. \\ Nunc aliud partum ingenio trademus eodem. \\ Argentum fulvo si quis permisceat auro, \\ 125 \\ Quantum id sit quove hoc possis deprendere pacto, \\ Prima Syracosii mens prodidit alta magistri. \\ Regem namque ferunt Siculum quam voverat olim \\ Caelicolum regi ex auro statuisse coronam, \\ Conpertoque dehinc furto nam parte retenta \\ 130 \\ Argenti tantundem opifex inmiscuit auro \\ Orasse ingenium civis, qui mente sagaei, \\ Quis modus argenti fulvo latitaret in auro, \\ epperit inlaeso quod dis erat ante dicatum. \\ 
        \pagebreak 
    \begin{center} \textbf{C. 486, 135—163} \end{center} \marginpar{[35]} Quod te, quale siet, paucis (adverte) docebo. \\ Lancibus aequatis quibus haec perpendere mos est \\ Arenti atque auri quod edax purgaverit ignis \\ lmpones libras, neutra ut praeponderet, hasque \\ Summittes in aquam: quas pura ut ceperit unda, \\ Protinus inclinat pars haec, quae sustinet aurum; \\ Densius hoc namque est, simul aere crassior unda. \\ At tu siste iugum mediique a cardine centri \\ Intervalla nota, quantum discesserit illinc \\ Quotque notis distet suspenso pondere filum. \\ Fac dragmis distare tribus. cognoscimus ergo \\ Argenti atque auri discrimina; denique libram \\ Libra tribus dragmis snperat, cum mergitur unda. \\ Sume dehinc aurum cui pars argentea mixta est \\ Argentique meri par pondus, itemque sub unda \\ 0 Lancibus impositum specta: propensior auri \\ Materies sub aquis fiet furtumque docebit. \\ Nam si ter senis superabitur altera dragmis, \\ Sex solas auri libras dicemus inesse. \\ Argenti reliquum, quia nil in pondere ditfer \\ Arentum arento, liquidis cum mergitur undis. \\ Iaec eadem puro deprendere possumus auro, \\ Si par corrupti pondus pars altera gestet. \\ Nam quotiens ternis pars inlibata gravarit \\ Corruptam dragmis sub aqua, tot inesse notabis \\ Arenti libras, quas fraus permiscuit auro. \\ Pars etiam quaevis librae, si forte supersit, \\ Haec quoque dramarum simili tibi parte notetur. \\ Nec non et sine aquis eadem deprendere furtum \\ 
        \pagebreak 
    \begin{center} \textbf{C. 486, 164 194} \end{center} \marginpar{[36]} Ars docuit, quam tu mecum experiare licebit. \\ Ex auro finges librili pondere formam, \\ 165 \\ Parque ex argento moles siet; ergo duobus \\ Dispar erit pondus paribus, quod densius aurum est. \\ Post haec ad lancem rediges pondusque requires \\ Argenti, nam iam notum est quod diximus auri, \\ Idque fac argento gravius sextante repertum. \\ 170 \\ Tunc auro, cuius vitium furtumque requiris, \\ Finge parem argenti formam podusque notato: \\ Altera quo praestat leviorque est altera moles, \\ Sit semissis onus: potes ex hoc dicere, quantum \\ Argenti fulvo mixtum celetur in auro. \\ 175 \\ Nam quia semissem triplum sextantis habemus, \\ Tres inerunt auri librae, quodque amplius hoc est, \\ Quantumcumque siet, fraus id permiscuit auro. \\ Causa \lbrack que \rbrack  cur ita sit, prompta est, si discere verum \\ Non pigeat veterumque animos intendere chartis. \\ 180 \\ Nam si disparibus numeris accesserit idem, \\ Servat inaequales itidem, tantumque manebit \\ Discrimen quantum fuerat prius, idque notabis, \\ Sive in temporibus quaeras, seu pondera rerum \\ Sen moles spectare velis spatiumque locorum. \\ 185 \\ Quare diversis argenti aurique metallis, \\ Quis forma ac moles eadem est, par addito pondus: \\ Argento solum id crescit, nihil additur auro. \\ Sextantes igitur quot tum superesse videbis, \\ In totidem dices aurum cousistere libris, \\ 190 \\ Parsque itidem librae sextantis parte notetur. \\ Quodsiforteparemcorruptofingereformam \\ Argeuto neqeas, at mollem sumito ceram. \\ Atque brevis facilisque tibi formetur imago \\ 
        \pagebreak 
    \begin{center} \textbf{C. 486, 1295208. C. 487, 1—4} \end{center} \marginpar{[4]} Sive cybi seu semiglobi teretisve cylindri, \\ Parque ex argento simuletur forma nitenti, \\ Quarum pondus item nosces. fac denique dragmas \\ Bis sex argenti, cerae tres esse repertas: \\ Ergo in ponderibus cerae arentique liquebit, \\ Si par forma siet, quadrupli discrimen inesse. \\ Tum par efligies cera simuletur eadem \\ Corruptae, cuius fraudem cognoscere curas. \\ Sic iustum pondus, quod lance inveneris aequa, \\ l quadruplum duces; quadrupli nam ponderis esset, \\ Si foret argenti moles quae cerea nunc sit. \\ Cetera iam puto nota tibi nam diximus ante \\ Quo pacto furtum sine aquis deprendere possis. \\ Haec eadem in reliquis poteris spectare metallis. \\ 
      \end{verse}
  
            \subsection*{487}
      \begin{verse}
      B. M. B. fram. \\ \poemtitle{BILARII ARELATENSIS}I1 \\ p. L. 420. \\ Si vere exurunt ignes, cur vivitis, undae? \\ Si vere extinguunt undae, cur vivitis, ignes? \\ Lympharum in gremiis inimicos condidit ines: \\ Communes ortus imperat alta manus. \\ 
        \pagebreak 
    \begin{center} \textbf{C. 487a, 116} \end{center} \marginpar{[38]} 
      \end{verse}
  
            \subsection*{487a}
      \begin{verse}
      ium vs mii \\ B. III 245. \\ Quid tibi, Mors, faciam, quae nulli parcere nosti \\ Nescis laetitiam, nescis amare iocos. \\ His ego praevalui toto notissimus orbi, \\ Hinc mihi larga domus, hinc mihi census erat. \\ Gaudebam semper. quid enim, si gaudia desint, \\ Hic vagus ac fallax utile mundus habet? \\ Me viso rabidi subito cecidere furores; \\ Ridebat summus, me veniente, dolor. \\ Non licuit quemquam mordacibus urere curis \\ Nec rerum incerta mobilitate trahi. \\ Vincebat cunctos praesentia nostra timores \\ Et mecum felix quaelibet hora fuit. \\ Motibus ac dictis, tragica quoque voce placebam \\ Exhilarans variis tristia corda modis. \\ Fingebam vultus, habitus ac verba loquentum, \\ Vt plures uno crederes ore loqui. \\ 
        \pagebreak 
    \begin{center} \textbf{C. 487, 1726. C. 487b, 18} \end{center} \marginpar{[39]} Ipse etiam, quem nostra oculis gemiuabat imago, \\ Horruit iu vultus se †magis isse meos. \\ O quotiens imitata meo se femina gestu \\ Vidit et erubuit totaque mata fuit! \\ Ergo quot in nostro videbantur corpore formae, \\ Tot mecum raptas abstulit atra dies. \\ Quo vos iam tristi turbatus deprecor ore, \\ Qui titulum legitis cum pietate meum: \\ ‘O quam laetus eras, Vitalis’ dicite maesti, \\ ‘Sint tibi, Vitalis, sint tibi laeta modo.’ \\ 
      \end{verse}
  
            \subsection*{487b}
      \begin{verse}
      B. M . \\ Credite victuras anima remeante favillas \\ Rursus ad amissum posse redire diem. \\ Nam vaga bis quinos iam luna resumpserat orbes, \\ Nutabat dubia cum mihi morte salus. \\ Hrrita letiferos auxit medicina dolores, \\ Crevit et humana morbus ab arte meus. \\ O quantum Petro donavit Christus honorem: \\ lle dedit vitam, reddidit iste mihi. \\ 
        \pagebreak 
    \begin{center} \textbf{C. 487c,. 16. C. 487d, 1—2} \end{center} \marginpar{[40]} 
      \end{verse}
  
            \subsection*{487c}
      \begin{verse}
      II 220. \\ M. 45 \\ Epitafium Terentii \\ V 385. \\ Natus in excelsis tectis Xarthaginis altae \\ Romanis ducibus bellica praeda fui. \\ Descripsi mores hominum, iuvenumque senumque. \\ (Qualiter et servi decipiant dominos, \\ Quid meretrix, quid leno dolis confingat avarus \\ Haec quicunque leget. hic puto cautus erit. \\ 
      \end{verse}
  
            \subsection*{487d}
      \begin{verse}
      B. III 144. \\ M. 275. B. \\ Versus in mensa sancti Augustini \\ Quisquis amat dictis absentum rodere vitam, \\ lac mensa indignam noverit esse suam. \\ 
        \pagebreak 
    \poemtitle{CARMINA}\poemtitle{CODICVM SAECVLI NONI}
        \pagebreak 
     \marginpar{[488]} \begin{center} \textbf{B. V 88.} \end{center}M. 1054. \\ omina feriarum \\ B. V 353. \\ Prima dies Phoebi sacrato nomine fulget. \\ Vindicat et lucens feriam sibi Luna secundam. \\ Inde dies rutilat iam tertia Martis honore. \\ Mercurius quartam splendentem possidet altus. \\ luppiter ecce sequens quiutam sibi iure dicavit. \\ Concordat Veneris magnae cum nomine sexta. \\ Emicat alma dies Saturno septima summo. \\ 
      \end{verse}
  
            \subsection*{489}
      \begin{verse}
      \poemtitle{AVGVSTNI}\poemtitle{De anima}B. M. B. \\ Omnia sunt bona: sunt, quia tu, bonus, omnia condis. \\ Nil nostrum est in eis, nisi quod peccamus, amantes \\ 
        \pagebreak 
    \begin{center} \textbf{C. 489, 3 25} \end{center}Ordine neglecto pro te, quod conditur abs te. \\ Omnia nam, quae sunt, a te sunt, te sine nil  \lbrack est \rbrack . \\ llis sine tu, simul es pro cunctis his et in illis. \\ His sine  \lbrack tu \rbrack , quod es, es; non hi sunt te sine, quod sunt. \\ Ac nec id hi quod tu, nec tu quod hi, sed in illis \\ Totus ades: in te totus, totus et in ipsis. \\ linamnecsibinectibitoti,sedsibisunthoc \\ Quod sunt, in quantum sunt, in tantum sibi toti. \\ Vt natura docet ratio perceptaque monstrat, \\ Totus homo est anima (siquidem hic sibi totus habetur). \\ Quicquid abest, extra se nec in se sibi sentit. \\ Deficit in toto, cum totus †se sibi constat. \\ Dum stat corporeus, homo semper et hic et ubique \\ Non habet in sese aeternos et tempus et actum; \\ Sic habet hoc nec habet, est et non est etiamque \\ Tempus habet; vivit, cum corporealis et ipse est; \\ Tunc habet, est cum tempore et hoc ipso sine non est. \\ Cum non corporealis erit, ipsum neque tempus \\ Tunc habet et non est nec esse habet hoc, quod amisit: \\ En homo finit et est non iam homo; sic quoque tempus. \\ Non ita, vis animae ut careat semper, sed ubique \\ Cum nova nempe novo descendit corpore tota. \\ Semper habet et ubique, eque est et habet ubicumque. \\ 
        \pagebreak 
    \begin{center} \textbf{C. 489, 2653} \end{center} \marginpar{[45]} Ex quo constat ut est, naturam sumsit ut esset. \\ Semper habet, quoniam esse quod est non desinet esse. \\ Sunt eique modi, qui cuncta secuntur, adhaesi: \\ Sensus et ingenium, ratio, mens, perspicua quae \\ Et diffusa manet, cum sit in corpore toto; \\ Emigrat, ubicumque aciem porrexerit extra \\ Subtilique oculo, quoquo se verterit, adstat \\ Intuitu mentisque doletque cupit metuitque, \\ Gaudet, et ista gerit gestu sine corporis ullo, \\ Vt deus inmortalis et intrectabile lumen. \\ Permanet ipsa, deus minime, similis sed in istis, \\ Nonque dei pars sed similis per talia vivit. \\ Hoc vero esse quod est numquam desistit adesse. \\ Illius igneus est vigor, ex quo corporealem \\ 0 lspirando calet massam difusa per artus. \\ Corporeis licet oflicia gestis varientur, \\ Auditu visu olfactu tactu quoque motu, \\ Illa tamen spirando calet animatque replendo \\ Omnia nec † quidam †habet aut nec sumit ab ipsa. \\ Namque loco non corporeo concluditur ullo: \\ Est incorporea informis substantia quaedam. \\ At deus ‘esse’ habet, et fuit, et est semper in illo, \\ Dispar in hoc, quod cuncta tenet perlustrat et implet, \\ Totus ubique manetque patetque et regnat ubique. \\ lHaec aut lapsa chaos aut ad  \lbrack caelum \rbrack  alta volabit; \\ Haec loca sorte capit, sed dispar vivit in illis: \\ Si felix fuerit, hic tunc felicior extat, \\ Si infelix, etiam multo infelicior illic. \\ 
        \pagebreak 
     \marginpar{[46]} \begin{center} \textbf{C. 490, 1—18} \end{center}
      \end{verse}
  
            \subsection*{490}
      \begin{verse}
      \poemtitle{TIBERIANI}B. M. B. III 267. \\ Versus Platonis de Graeco in atinum \\ translati \\ Omnipotens, annosa poli quem suspicit aetas, \\ Quem sub millenis semper virtutibus unum \\ Ner numero quisquan poterit pensare nec aevo, \\ Nunc esto affatus, si quo te nomine dignum est, \\ Quo sacer ignoto gaudes, quo maxima tellus \\ Intremit et sistunt rapidos vaga sidera cursus. \\ Tu solus, tu multus item, tu primus et idem \\ Postremus mediusque simul mundique superstes. \\ Nam sine fine tui labentia tempora †finis. \\ Altus ab aeterno spectas fera turbine certo \\ Rerum fata rapi vitasque involvier aevo \\ Atque iterum reduces supera in convexa referri, \\ Scilicet ut mundo redeat, quod partubus †austrum \\ Perdiderit refluumque iterum per tempora fiat. \\ Tu, siquidem fas est in temet tendere sensum \\ Et speciem temptare sacram, qua sidera cingis \\ Inmensus longamque simul complecteris aethram, \\ Fulmineis forsan rapida sub imagine membris, \\ 
        \pagebreak 
    \begin{center} \textbf{C. 490, 19 32. C. 490a, 16} \end{center} \marginpar{[47]} Flammifluum quoddam iubar es, quo cuncta coruscans \\ lpse vides nostrumque premis solemque diemque. \\ Tu genus omne deum, tu rerum causa vigorque, \\ Tu natura omnis, deus innumerabilis unus, \\ Tu sexu plenus toto, tibi nascitur olim \\ lic deus, hic mundus, domus haec hominumque deumque, \\ Lucens, augusto stellatus flore iuventae. \\ Quem (precor, aspires), qua sit ratione creatus, \\ Quo genitus factusve modo, da nosse volenti. \\ Da, pater, augustas ut possim noscere causas, \\ Mundanas olim moles quo foedere rerum \\ 0 Sustuleris animamque levi quo maximus olim \\ Texueris numero, quo congrege dissimilique, \\ Quidque id sit vegetum, quod per cita corpora vivit. \\ 
      \end{verse}
  
            \subsection*{49a}
      \begin{verse}
      B. M. \\ Ofleila duodecim mensium \\ B. V 354. \\ Artatur niveus bruma lanuarius †arva. \\ Piscibus exultare solet Februarius altis. \\ Martius in vites curas extendit amicas. \\ Dat sucum pecori gratanter Aprilis et escam. \\ Maius hinc gliscens herbis generat †nigra bella. \\ Iunius auratis foliis iam pascua miscet. \\ 
        \pagebreak 
    \begin{center} \textbf{C. 490, 7 12. C. 491, 1—8} \end{center} \marginpar{[48]} Iulius educit falces per grata virecta. \\ Augustus Cererem pronus secat agmine longo. \\ Maturas munit September ab hostibus uvas. \\ Elicit Dctober pedibus dulcissima vina. \\ Baccha November ovans condit sub clave fideli. \\ More sues proprio mactat December adultas. \\ 
      \end{verse}
  
            \subsection*{491}
      \begin{verse}
      B. M. B. \\ Sedulius ed. \\ \poemtitle{TVRCII RVTI ASTERII}Huemer p. 307. \\ Sume, sacer meritis. veracis dicta poetae, \\ Quae sine ligmenti condita sunt vitio. \\ Quo caret alma fides, quo sancti gratia Christi, \\ Per quam iustus ait talia Sedulius. \\ Asteriique tui semper meminisse iubeto, \\ Cuius ope et cura edita sunt populis. \\ Quem quamvis summi celebrent per saecula fastus, \\ Plus tamen ad meritum est, si viget ore tuo. \\ 
        \pagebreak 
    \begin{center} \textbf{C. 4, 116. C. 493, 12} \end{center} \marginpar{[49]} 
      \end{verse}
  
            \subsection*{492}
      \begin{verse}
      B. M. B. \\ Vesus BELLESARIM seolastiei a \\ Sedulius Christi miracula versibus edenS \\ Emicat, invitans parvae ad solemnia mensaE \\ Dignum convivam: non hunc, †qui carperet illu, \\ Vix quod nobilium triplici fert aula para, \\ Laetum quod ponit sub aurea tecta tribuna., \\ In quo gemmiferi (totque aurea vasa) canistrl \\ Vivida pro modico portant sibi prandia victV, \\ Sed quod holus vile producit pauperis hortuS. \\ At post delicias properant qui sumere magn, \\ Nituntur parvum miserorum spernere germeN, \\ Tutum quod nihil est, dum nil cum ventre tumesci’ \\ Insidias membrisque movens animaeque ludent: \\ Si tamen his dapibus vesci dignatur egeni, \\ Temnat divitias animus paucisque quiesca’, \\ ExemploadsumptusDomini,quimiliaquinqu \\ Semotis cunuctis modicis saturavit ab esciS. \\ 
      \end{verse}
  
            \subsection*{493}
      \begin{verse}
      B. M. B. \\ Versus LIBERATI seolastiei \\ Sedulius Domini per culta novalia pergenS \\ En loca prospexit multo radiantia florE: \\ 
        \pagebreak 
    C. 49B, 316. C. 493 , \\ Discurrit per prata libens, quo gramine Davi9 \\ Vidit divino modulantem carmina cant’. \\ Laudabili psallente viro refluunt citharae meL. \\ Ille ubi randisoni captus dulcedine plectrl \\ Vritur et celeri graditur per lilia passV \\ Sacratosque iterum late conspexit amoeno, \\ Aeterna Christi fluvius quos abluit und, \\ Nec passus torpere diu doctoris acumeN \\ Tunc sua Daviticus delectus plectra poposci’. \\ Irrita polluti contemnens numina mundi \\ Signa crucis fronti ponit, breviterque triumphoS \\ Tangit, Christe, tuos numerosaque praemia liba’. \\ Ergo dum vario decorat sua rura color, \\ Stabunt hi garrula dicti testudine versuS. \\ 
      \end{verse}
  
            \subsection*{493}
      \begin{verse}
      B. M. B. \\ Haec Augustini ex sacris epigrammata dictis \\ Dulcisono rhetor componens carmine Prosper \\ Versibus hexametris depinxit pentametrisque, \\ Floribus ex variis cen fulget nexa corona. \\ Vnde ego te, lector, relegis qui haec sedulus, oro, \\ Intentas adhibere sonis caelestibus aures. \\ stic nam invenies, animum si cura subintrat, \\ 
        \pagebreak 
    \begin{center} \textbf{C. 493, 8 10. C. 493, 15. C. 494,. 1 . C. 494 , 1 2 5} \end{center}Maxime quid doceant sacrae modulamina legis \\ Observare homines, vel quid sibi maxime vitent, \\ Siderenm caeli cnupiunt qui scandere regnum. \\ 
      \end{verse}
  
            \subsection*{49}
      \begin{verse}
      B. M. B. \\ Augustine, tonans divino fulmine linguae \\ Impia daemonicae refutas ludibria sectae \\ Atque loquellosi confndens dogmata ritus \\ Pandis iter verum, quo cives tendere monstras \\ Erbis inocciduae, qua vita perennis habetur. \\ B M. \\ 
      \end{verse}
  
            \subsection*{494}
      \begin{verse}
      B V 363 \\ Discipulis cunctis domini praelatus amore, \\ Dignus apostolico primus honore coli, \\ Sancte, tuis, Petre, meritis haee munera supplex \\ Chintila rex ofert. Pande salutis opem! \\ 
      \end{verse}
  
            \subsection*{494a}
      \begin{verse}
      B. M. B. \\ CInudinnu \\ In Sirenas \\ e Birt p. 399. \\ Dulce malum pelago Siren volucresque puellae \\ Scyllaeos inter fremitus avidamque Charybdin \\ 
        \pagebreak 
    \begin{center} \textbf{C. 494 , 3 9. C. 49, 11} \end{center} \marginpar{[52]} Musica saxa fretis habitabant, dulcia monstra, \\ Blanda pericla maris, terror quoque gratus in undis. \\ Delatis licet huc incumberet aura carinis \\ Implessentque sinus venti de puppe ferentes, \\ Figebat vox una ratem. nec tendere certum \\ Delectabat iter reditus, otiumque iuvabat, \\ Ne dolor ullus erat; mortem dabat ipsa voluptas. \\ 
      \end{verse}
  
            \subsection*{494}
      \begin{verse}
      B. M. B. \\ Laus Herculis \\ Bir p. 399. \\ Pieridum columen, cuius Parnasia magno \\ Numine templa sonant, Phoebe, precor, huc age, laeto, \\ Tecum † cuncta choro; penetralia sancta sororum \\ Et nova Castalios latices per rura petentem \\ llippocrenaeon victorem sistere fontum \\ Me fac. namque tuam non nunc novus advena turbam \\ lngredior, laurusque gerens et florea sertis \\ Tempora vincta tuis, doctorum munera vatum, \\ Testor adhuc veteres quamvis desuetus honores. \\ Alcides mihi carmen erit, non vana Tonantis \\ Progenies, dignus credi post viscera numen, \\ Cuique per inmensos invicti roboris aestus \\ Nec nasci potuisse vacat. nam lucis in ipsis, \\ Inclite, primitiis tardo vix editus ortu \\ Fecisti de patre fidem. sed cur mihi lentis \\ ILudis adhuc, Cirrhaee, modis tenerumque resultans \\ 
        \pagebreak 
     \marginpar{[53]} \begin{center} \textbf{C. 494 , 1747} \end{center}Luxuriante leves impellis pollice chordas? \\ Pone habitum, quo molle canis, et frondis amatae \\ Linque nemus, mollique exutus tempora lauro \\ 0 Populea mecum carmen luctare sub umbra. \\ Iam grave plus etiam, quam ventris tempora vellent, \\ Alcmenam tendebat onus. sed regia lunc \\ Impedit et partus prohibet nascique vetabat, \\ Vt metus ipse deum monstret. nec vivida caeli \\ Semina mortales norunt sentire latebras, \\ Nec possunt sufferre moras. datur inde novercae \\ Materies, gravibusque odiis augmenta ministrat, \\ Quod vinci coepisse pudet. Mox inproba binos \\ In tua membra iubet, dum nasceris, ire dracones. \\ 0 Incumbunt celeres illi, squamosaque iussus \\ Armat colla furor, nec, quamvis maxima tractu, \\ Tardata spiris sequitur pars cetera pectus. \\ Tristis Tartarea vibratur sibilus aura. \\ Morte rubent oculi trifidisque horrentia lingnis \\ Ora sonant nigrumque fremens levat ira venenum. \\ Quid nunc invictis fraudes innectere fatis, \\ Caelicolum regina, iuvat? cur obicis angues? \\ Cur parvo geminos? ane unum posse necari \\ Iam strato Pythone times? licet omnia mundi \\ 0 Monstra voces ipsamque armes serpentibus hydram, \\ Defendet natura deum patremque probabit \\ (Quod non vis constare) lovem. lamque inrita taetri \\ lussa parant implere angues, miseroque furore \\ In sua fata tument. cernit tua membra petentes \\ Horrescitque parens numenque inara creasse \\ Mortali pietate timet. nil, sancta, superbae \\ Paelicis insidias, caelo modo freta, tremescas \\ 
        \pagebreak 
     \marginpar{[54]} \begin{center} \textbf{C. 494, 4872} \end{center}Neve haec monstra tibi faciant, Alcmena, pavorem! \\ Sic mater potes esse dei. iam tolle serenum \\ Laeta animum tantoque libens haec aspice vultu, \\ Vt deceat genuisse lovem. depone pavorem \\ Indignum partu, natumque exemplar habeto. \\ Cui metuis, nihil ipse timet. nam numine recto \\ Ridebas tu, dive, truces, animosque superbos \\ De genitore tenes, votisque aptissimus orbis \\ Gaudebas tantum iam tum meruisse novercam. \\ Corripis exiguis mox grandia guttura palmis, \\ Et manibus teneris cogens in brachia pondus \\ Constringis pressos, relevans tellure, dracones. \\ Eferat aetherium, quantum volet, orbis in axem \\ atoidas verosque probet sua fabula divos, \\ Quod Delo iam stare licet: non aequa laborum \\ Gloria, nec parili serpentes sorte necarunt: \\ Illi unum ferro, geminos et inermis et unus! \\ His coeptis non ulla parat cunabula partus, \\ Dive, tibi; sed cum totis iam bruma rigeret \\ Imbribus et solidis haererent flumina lymphis, \\ Nudum praegelidis durando firmat in undis. \\ Vtque rudes primo temptasti robore gressus, \\ Frondosae deserta vagus penetralia silvae \\ Secura iam matre petis telisque tremendis \\ Ludis et aerias adducto deicis arcu \\ 
        \pagebreak 
    \begin{center} \textbf{C. 494, 7396} \end{center} \marginpar{[55]} Aut funda violentus aves noctemque sub astris \\ Exigis et puram fractis bibis amnibus undam. \\ Immanem interea Nemeae per lustra leonem \\ Ipsa Chimaeraea cretum de gente noverca \\ In tua depastis armabat vota iuvencis \\ Augebatque fero vires rabiemque iuvabat, \\ Naturam minus esse putans.  \lbrack leu \rbrack  quanta virorum \\ Funera! quam multos stravit cum gentibus agros! \\ Non illum magnae viduatis moenibus urbes \\ Armorum fregere minis, Martique domando \\ Adsuetas morsu fudit graviore catervas. \\ Hunc gravis Euryvstheus nam te, quo cuncta levares, \\ Imperium duri voluit suferre tyranni \\ Sic mundo Fortuna favens hunc sternere leto \\ Imperat. at nullum virtus reticenda per aevum \\ Dignaque sidereos post membra intrare recessus \\ Posse mori quam vile putat! namque impiger ultro \\ Vadis  \lbrack et \rbrack  inmensae scrutatus devia silvae \\ In nova sangnineos armantem vulnera rictus \\ Admonita feritate † iubas visuque cruentus \\ Excussis movet arma toris dubiumque residens \\ Infremit. invadis trepidum solisque lacertis \\ Grandia corripiens eluso guttura morsu \\ Inbellem fractis prosternis faucibus hostem. \\ 
        \pagebreak 
    \begin{center} \textbf{C. 49, 97 124} \end{center} \marginpar{[56]} Quin et flavicomis radiantia tergora villis \\ Cen spolium fuso victor rapis. emicat omnis \\ In laudes mox turba tuas longoque relicta \\ Currit in arva metu. iuvat ire et libera rura \\ 100 \\ Defensosque videre locos silvamque iuvencis \\ Iam facilem et nullis resonantes fletibus agros. \\ Maenalium petis inde nemus fletamque colonis \\ Arcadiam et raro steriles iam robore silvas. \\ Namque hic inmensa membrorum mole cruentus \\ Indomitus regnabat aper soloque tremendus \\ Corpore lunatis fundebat dentibus ornos \\ Sternebatque suis lugentia rura colonis. \\ lorrebant rigidis nigrantia corpora saetis \\ Duratosque armos scopulis totosque per artus \\ 110 \\ Dificilis potuisse mori. non spicula in illum \\ Nodosumve rapis gravato pondere robur: \\ Armati viduatur honos; nec vulnera virtus \\ Exemplo tibi facta timet. iamque arripis ultro \\ Spumantem, cogisque diem sufferre tuendo, \\ 115 \\ Atque supinando mirantem lumina vinci \\ Argolici victor portas sub tecta tyranni. \\ Fama celer toto victorem sparserat orbe, \\ Auxiliumque dei poscebat Creta cruento \\ Victa malo. Taurus medio nam sidere lunae \\ 120 \\ Progenitus Dictaea Iovis possederat arva. \\ Fulmen ab ore venit, flammisque furentibus ardet \\ Spiritus; et terram non caeli flamma perurit, \\ Sed flatus monstri. dextro iam Siria cessent \\ 
        \pagebreak 
    \begin{center} \textbf{C. 494, 125137. C. 4947, 1—6} \end{center} \marginpar{[57]} Sidere, solque licet, glaciali frigore victus, \\ Abstrusum mundo claudat iubar, aurea condens \\ Lumina, et ignifluo stupefactus in orbe tepescat: \\ Aestus habet Cretam, pereunt silvaeque lacusque \\ Graminaque et fontes sacri, montesque perurit \\ Flamma ferox; Idam superis spectantibus ignis \\ Dissipat et magno cunabula grata Tonanti \\ Igne suo monstrum, si fas est dicere, vincit. \\ Tandem fama celer Dictaea ad litora magnum \\ Duxerat Alciden, cum taurum dira minantem \\ Accipit et saevum cornu flammasque vomentem \\ Corripit atque artus constringens fortibus ulnis \\ Ignifluos flatus animamque in pectore clausit. \\ 
      \end{verse}
  
            \subsection*{494}
      \begin{verse}
      ANDREAE oratoris \\ B. M. B. \\ Virgo parens hac luce deumque virumque creavit \\ Gnara puerperii, nescia coniugii. \\ Obtulit haec inssis uterum docuitque futuros, \\ Sola capax Christi quod queat esse fides. \\ Credidit et tumuit: verbum pro semine sumsit. \\ Clauserunt magnum parvula membra deum. \\ 
        \pagebreak 
     \marginpar{[58]} \begin{center} \textbf{C. 494V, 72} \end{center}Conditor extat opus, servi rex induit artus \\ Mortalemque domum vivificator habet. \\ lpse sator semenque sui matrisque creator, \\ Filius ipse hominis, qui deus est hominum. \\ Adfulsit partus, lucem lux nostra petivit, \\ Hospitii linquens ostia clausa sui. \\ Virginis et matris servatur gloria consors: \\ Mater das hominem noscere, virgo deum. \\ Vnius colitur duplex substantia nati: \\ Vir, deus, haec duo sunt; unus utrumque tamen. \\ Spiritus huic enitorque suus sine fine cohaerent, \\ Triplicitas simplex simplicitasque triplex. \\ Bis geuitus, sine matre opifex, sine patre redemptor, \\ Celsus utroque modo, celsior unde minor. \\ Sic voluit nasci, domuit qui crimina mundi, \\ Et mortem iussit mortuus ipse mori. \\ Nostras ille suo tueatur numine vitas; \\ Protegat ille tuum, usticiana, genus. \\ 
        \pagebreak 
    \begin{center} \textbf{C. 495} \end{center} \marginpar{[59]} 495 638 \\ Carmina duodecim sapientum \\ I Monosticha de ratione tabulae \\ ni verbi et litteris \\ 
      \end{verse}
  
            \subsection*{495}
      \begin{verse}
      \poemtitle{PALLADII}Sperne lucrum: versat mentes insana cupido. \\ 
        \pagebreak 
    \begin{center} \textbf{C. 496 500} \end{center} \marginpar{[60]} 
      \end{verse}
  
            \subsection*{496}
      \begin{verse}
      \poemtitle{ASCLEPIADII}Fraude carete graves, ignari cedite doctis. \\ 
      \end{verse}
  
            \subsection*{497}
      \begin{verse}
      \poemtitle{EVTIENII}Lusuri nummos animos quoque ponere debent. \\ 
      \end{verse}
  
            \subsection*{498}
      \begin{verse}
      \poemtitle{POMPILIANI}Irasci victos minime placet, optime frater. \\ 
      \end{verse}
  
            \subsection*{499}
      \begin{verse}
      \poemtitle{MAXIMMNI}Ludite securi, quibus aesest semper inarca! \\ 
      \end{verse}
  
            \subsection*{500}
      \begin{verse}
      \poemtitle{VTALIS}Siquis habens nummos venies, exibis inanis. \\ 
        \pagebreak 
    \begin{center} \textbf{C 501 506} \end{center}
      \end{verse}
  
            \subsection*{501}
      \begin{verse}
      \poemtitle{BASILII}Lusori cupido semper gravis exitus instat. \\ 
      \end{verse}
  
            \subsection*{502}
      \begin{verse}
      \poemtitle{ASMENII}Sancta probis paxest: irasci desine victus. \\ 
      \end{verse}
  
            \subsection*{503}
      \begin{verse}
      \poemtitle{VOMANII}Nullus ubique potest felici ludere dextra. \\ 
      \end{verse}
  
            \subsection*{504}
      \begin{verse}
      \poemtitle{EVPMORBII}Inicio Furias: egosum tribus addita quarta. \\ 
      \end{verse}
  
            \subsection*{505}
      \begin{verse}
      \poemtitle{IVLIANI}Flecte truces animos, utvere ludere possis. \\ 
      \end{verse}
  
            \subsection*{506}
      \begin{verse}
      \poemtitle{IILASI1}Ponite mature bellum, precor, iraque cesset. \\ 
        \pagebreak 
     \marginpar{[62]} \begin{center} \textbf{C. 507 510} \end{center}I Epitaphia . Vergilii Maronis distieha \\ Mantu me genuit, Calabri mpuere: tenet nunc \\ Parthenope. Cecini pascu, rur, duces. \\ 
      \end{verse}
  
            \subsection*{507}
      \begin{verse}
      B. II 198. \\ M. 433 444. \\ \poemtitle{ASCEPIADII}B. IV 120. \\ Tityron ac segetes cecini Maro et ‘arma virumque’ \\ Mantua me genuit, Parthenope sepelit. \\ 
      \end{verse}
  
            \subsection*{508}
      \begin{verse}
      \poemtitle{EVMENII}Verilius iacet hic, qui pascua versibus edit \\ Et ruris cultus et Phrygis arma viri. \\ 
      \end{verse}
  
            \subsection*{509}
      \begin{verse}
      \poemtitle{OMLIANI}Qui pecudes, qui rura canit, qui proelia vates, \\ In Calabris moriens hac requiescit humo. \\ 
      \end{verse}
  
            \subsection*{510}
      \begin{verse}
      \poemtitle{MAXIMNI}Carminibus pecudes et rus et bella canendo \\ Nomen inextinctum Vergilius merui. \\ 
        \pagebreak 
    \begin{center} \textbf{C. 511 516} \end{center} \marginpar{[63]} 
      \end{verse}
  
            \subsection*{511}
      \begin{verse}
      \poemtitle{VEALIS}r r \\ Mantua mi patria est, nomen Maro, carmina silvae \\ Ruraque cum bellis, Partheuope tumulus. \\ 
      \end{verse}
  
            \subsection*{512}
      \begin{verse}
      \poemtitle{BASILI}Qui silvas et agros et proelia versibus orat, \\ Mole sub hac situs est: ecce poeta Maro. \\ 
      \end{verse}
  
            \subsection*{513}
      \begin{verse}
      \poemtitle{ASMENII}Pastorum vates ego sum, cui rura ducesque \\ Carmina sunt. hic me pressit acerba quies. \\ 
      \end{verse}
  
            \subsection*{514}
      \begin{verse}
      \poemtitle{VOMANII}A silvis ad agros, ab agris ad proelia venit \\ Musa Maroneo nobilis ingenio. \\ 
      \end{verse}
  
            \subsection*{515}
      \begin{verse}
      \poemtitle{EVPRORBII}Bucolica expressi et ruris praecepta colendi, \\ Mox cecini pugnas. mortuus hbic habito. \\ 
      \end{verse}
  
            \subsection*{516}
      \begin{verse}
      \poemtitle{IVILINI}Hic data Vergilio requies, qui carmine dulci \\ Et Pana et segetes et fera bella canit. \\ 
        \pagebreak 
     \marginpar{[64]} \begin{center} \textbf{C. 517 521} \end{center}
      \end{verse}
  
            \subsection*{517}
      \begin{verse}
      \poemtitle{ILASII}Pastores cecini; docui, qui cultus in agris; \\ Proelia descripsi. contegor hoc tumulo. \\ 
      \end{verse}
  
            \subsection*{518}
      \begin{verse}
      \poemtitle{PALLADII}Conditus hic ego sum, cuius modo rustica Musa \\ Per silvas, per rus venit ad ‘arma virum.’ \\ I II Distieha de unda et speculo \\ 
      \end{verse}
  
            \subsection*{519}
      \begin{verse}
      IB. V 10t 112. \\ M. 517528. \\ \poemtitle{EVSTIENII}Redditur effigies liquida visentis in unda, \\ Qualis in adverso speculorum cernitur orbe. \\ 
      \end{verse}
  
            \subsection*{520}
      \begin{verse}
      \poemtitle{POMPLIANI}Formas pura refert oculis spectantibus unda, \\ Quales obiecto speculi fulgore videntur. \\ 
      \end{verse}
  
            \subsection*{521}
      \begin{verse}
      \poemtitle{MAXIMINI}Fontis aquae reddunt simulacra imitantia verum, \\ Qualia leve refert speculi, cum cernimus, aequor. \\ 
        \pagebreak 
    \begin{center} \textbf{C. 522 526} \end{center} \marginpar{[65]} 
      \end{verse}
  
            \subsection*{522}
      \begin{verse}
      \poemtitle{VITALIS}Exprimit oppositas immobilis unda figuras, \\ Levati quales speculi nitor ipse remittit. \\ 
      \end{verse}
  
            \subsection*{523}
      \begin{verse}
      \poemtitle{BASIII}Apparet mendax inlimi fonte figura, \\ Qualem reiectat speculi nitidissimus orbis. \\ 
      \end{verse}
  
            \subsection*{524}
      \begin{verse}
      \poemtitle{ASMENII}Vnda quieta refert alto de grgite formas \\ Ac veluti speculum nitido splendore coruscat. \\ 
      \end{verse}
  
            \subsection*{525}
      \begin{verse}
      \poemtitle{VOMANII}Spectantis faciem nitidissimus adsimulat fons, \\ Sicut in opposito speculi solet aequore cerni. \\ 
      \end{verse}
  
            \subsection*{526}
      \begin{verse}
      \poemtitle{EPOR3II}Forma repercussus liquidarum fingit aquarum, \\ Qualis purifico speculorum ex orbe relucet. \\ 
        \pagebreak 
    \begin{center} \textbf{C. 527 531} \end{center} \marginpar{[66]} 
      \end{verse}
  
            \subsection*{527}
      \begin{verse}
      \poemtitle{IVLIANI}Fontibus in liquidis simplex geminatur imago, \\ Vt solet a speculo facies splendente referri. \\ 
      \end{verse}
  
            \subsection*{528}
      \begin{verse}
      \poemtitle{IILASII}Effigies liquido resplendet ab aequore fontis, \\ Qualis et a speculo simulatrix umbra resultat. \\ 
      \end{verse}
  
            \subsection*{529}
      \begin{verse}
      \poemtitle{PALLADI}Effingit species purissimus umor aquarum, \\ Plana velut speculi vivas imitantia formas. \\ 
      \end{verse}
  
            \subsection*{530}
      \begin{verse}
      \poemtitle{ASCLEPIADII}Fonte repulsatur depicta tuentis imago, \\ Ceu levi in speculo solet apparere figura. \\ V Distieha de laceiali aqu \\ 
      \end{verse}
  
            \subsection*{31}
      \begin{verse}
      . V 89 100. \\ M. 505 516. \\ \poemtitle{POMPILIN1}B. IV 124. \\ Qua ratis egit iter, iuncto bove plaustra trahuntur, \\ Postquam tristis bhiems frigore iunxit aquas. \\ 
        \pagebreak 
    \begin{center} \textbf{C. 532 536} \end{center} \marginpar{[67]} 
      \end{verse}
  
            \subsection*{532}
      \begin{verse}
      7VIVr \\ \poemtitle{MAXIMINI}Sustinet unda rotam patulae modo pervia puppi \\ Et concreta gelu marmoris instar habet. \\ 
      \end{verse}
  
            \subsection*{533}
      \begin{verse}
      \poemtitle{VTEALI8}Quas modo plaustra premunt undas, ratis ante secabat, \\ Postquam brumali deriguere gelu. \\ 
      \end{verse}
  
            \subsection*{534}
      \begin{verse}
      \poemtitle{BASILI}Vnda rotam patitur celerem nunc passa carinam, \\ In glaciem solidam versus ut amnis abit. \\ 
      \end{verse}
  
            \subsection*{535}
      \begin{verse}
      \poemtitle{ASMENII}Quae solita est ferre unda rates, fit pervia plaustris, \\ Vt stetit in glaciem marmore versa novo. \\ 
      \end{verse}
  
            \subsection*{536}
      \begin{verse}
      \poemtitle{VOMANII}Semita fit plaustro, qu puppis adunca cucurrit, \\ Postquam frigoribus bruma coegit aquas. \\ 
        \pagebreak 
    \begin{center} \textbf{C. 537 541} \end{center} \marginpar{[68]} 
      \end{verse}
  
            \subsection*{537}
      \begin{verse}
      \poemtitle{EVPIOR3I}Orbita signat iter, modo qua cavus alveus ibat, \\ Strinxit aquas tenues ut glacialis hiems. \\ 
      \end{verse}
  
            \subsection*{538}
      \begin{verse}
      \poemtitle{IVLANI}Qua puppes ibant, hac ducunt plaustra iuvenci, \\ Pigrior ut cano constitit unda gelu. \\ 
      \end{verse}
  
            \subsection*{539}
      \begin{verse}
      \poemtitle{LASII}Amuis iter plaustro qui dat, dedit ante carinis. \\ Duruit ut ventis unda, fit apta rotis. \\ 
      \end{verse}
  
            \subsection*{540}
      \begin{verse}
      \poemtitle{PALLADII}Plaustra boves ducunt, qua remis acta carina est, \\ Postquam deriguit crassus in amne liquor. \\ 
      \end{verse}
  
            \subsection*{541}
      \begin{verse}
      \poemtitle{ASCLEPIADII}Vnda capax ratium plaustris iter algida praebet, \\ Frigoribus saevis ut stetit amnis iners. \\ 
        \pagebreak 
     \marginpar{[69]} \begin{center} \textbf{C. 542 545} \end{center}\poemtitle{EVSTIENII}Plaustra viam carpunt, qua puppes ire solebant, \\ Cum rigidus Boreas obstupefecit aquas. \\ V Tlsticha de arcu caeli \\ B. V 17 28. \\ M. 469 480. \\ 
      \end{verse}
  
            \subsection*{}
      \begin{verse}
      \poemtitle{MAXIMINI}B. IV 12. \\ Thaumantis proles varianti veste refulgens \\ Multicolor picto per nubila devolat arcu \\ Iris et insigni decorat curvamine caelum. \\ 
      \end{verse}
  
            \subsection*{}
      \begin{verse}
      \poemtitle{VTALS}Cum sol ardentis radios in nubila iecit \\ Cumque colorifico nimbos fulgore replevit, \\ Apparet variis distincta coloribus Iris. \\ 
      \end{verse}
  
            \subsection*{}
      \begin{verse}
      \poemtitle{BASLII}Cdara sub aetheriis fulget Thaumantia proles \\ Nubibus, ut radiis pluvium sol attigit imbrem, \\ Et picturato caelum velamine cingit. \\ 
        \pagebreak 
     \marginpar{[70]} \begin{center} \textbf{C. 546 549} \end{center}
      \end{verse}
  
            \subsection*{546}
      \begin{verse}
      \poemtitle{ASMENII}Discolor aetheriis apparet nubibus Iris, \\ Postquam flammiferi rapuerunt lumina solis, \\ Et caelum variis miranda coloribus ornat. \\ 
      \end{verse}
  
            \subsection*{547}
      \begin{verse}
      \poemtitle{VOMANII}Imbriferas nubes radiis ubi contigerit sol, \\ Luce sub adversa varios iacit unda colores. \\ Dicitur hbaec Iris, picto spectabilis arcu. \\ 
      \end{verse}
  
            \subsection*{548}
      \begin{verse}
      \poemtitle{EVPOR3II}Cum tetigit nubes radiis fulgentibus atras \\ Phoebus et adverso lumen resplenduit imbri, \\ Tunc Iris vario circumdat nubila cinctu. \\ 
      \end{verse}
  
            \subsection*{549}
      \begin{verse}
      \poemtitle{IVLIANI}Mirifico nubes ambit Thaumante creata, \\ Quas cum ex adverso tetigit rota fulgida solis, \\ Tum iacit insinis per nubila densa colores. \\ 
        \pagebreak 
    \begin{center} \textbf{C. 550 553} \end{center} \marginpar{[71]} 
      \end{verse}
  
            \subsection*{550}
      \begin{verse}
      \poemtitle{ILASII}Nuntia lunonis vario decorata colore \\ Aethera nubiferum conplectitur orbe decoro, \\ Cum Phoebus radios in nubem iecit aquosam. \\ 
      \end{verse}
  
            \subsection*{551}
      \begin{verse}
      \poemtitle{PALLADII}Nubila cum Phoebus perfudit lumine claro, \\ Tum flt ut umor aquae subfulgeat atque colores \\ Sub varia specie iaciat mirabilis arcus. \\ 
      \end{verse}
  
            \subsection*{552}
      \begin{verse}
      \poemtitle{ASCLEPIADII}Cum radiis imbres et aquarum pendulus umor \\ Tangitur, existit, quam Graecia nominat, lris, \\ Multorum insignis vario splendore colorum. \\ 
      \end{verse}
  
            \subsection*{553}
      \begin{verse}
      \poemtitle{EVSTENII}Iris habet varios subiecta luce colores, \\ Quam sol imbrifera fulgens de nnbe creavit, \\ Cum pepulit radiis obstantia nubila claris. \\ 
        \pagebreak 
    \begin{center} \textbf{C. 554 556} \end{center} \marginpar{[72]} 
      \end{verse}
  
            \subsection*{554}
      \begin{verse}
      \poemtitle{POMPLIANI}Luce repentina cum sol implevit aquosas \\ Adversus nubes, effulget protinus Hris, \\ Piceta veste decens et multicoloribus alis. \\ VI retrastiha de Vergilio \\ 
      \end{verse}
  
            \subsection*{555}
      \begin{verse}
      B. II 197. \\ M. 421 432. \\ \poemtitle{VTALIS}B. IV 128. \\ Prima mihi Musa est sub fagi Tityrus umbra. \\ Ad mea navus humum iussa colonus arat. \\ Proeliaque expertos ceciui Troiana Latinos, \\ Fertque meos cineres inclita Parthenope. \\ 
      \end{verse}
  
            \subsection*{556}
      \begin{verse}
      \poemtitle{BASLII}Hoc iacet in tumulo vates imitator Homeri, \\ Qui canit Ausonio carmine primus oves, \\ Ad cultos hinc transit agros, Aeneidis autem \\ Non emendatum morte reliquit opus. \\ 
        \pagebreak 
    \begin{center} \textbf{C. 557 560, 1—2} \end{center} \marginpar{[73]} 
      \end{verse}
  
            \subsection*{557}
      \begin{verse}
      \poemtitle{ASMENII}Bucolica Ausonio primus qui carmine feci, \\ Mox praecepta dedi versibus agricolae, \\ Idem cum Phrygibus Rutulorum bella peregi \\ Hunc mihi defuncto fata dedere locum. \\ 
      \end{verse}
  
            \subsection*{558}
      \begin{verse}
      \poemtitle{VOMANII}Tityre, te Latio cecinit mea fistula versu \\ Praeceptisque meis rusticus arva colit. \\ At ne Musa carens vitiis Aeneidos esset, \\ Invida me celeri fata tulere nece. \\ 
      \end{verse}
  
            \subsection*{559}
      \begin{verse}
      \poemtitle{EVOR3II}Romuleum Sicula qui fingit carmen avena \\ Ruricolasque docet, qua ratione serant, \\ Quique Latinorum memorat fera bella Phrygumque, \\ Hic cubat, hic meruit perpetuam requiem. \\ 
      \end{verse}
  
            \subsection*{560}
      \begin{verse}
      \poemtitle{IVLIANI}Qui pastorali peragravit Maenala Musa \\ Ruraque et Aeneae concinit arma Maro, \\ 
        \pagebreak 
    \begin{center} \textbf{C. 560. 3. C. 561 — 64, 12} \end{center} \marginpar{[74]} lle decem lustris geminos postquam addidit annos, \\ Concessit fatis et situs hoc tumulo est. \\ 
      \end{verse}
  
            \subsection*{561}
      \begin{verse}
      \poemtitle{IILASII}Haec tibi, Vergili, domus est aeterna sepulto, \\ Qui mortis tenebras effugis ingenio, \\ Maenalium carmen qui profers ore Latino \\ Et cultus segetum bellaque saeva ducum. \\ 
      \end{verse}
  
            \subsection*{562}
      \begin{verse}
      \poemtitle{PALLADII}Primus ego Ausonio pastorum carmina versu \\ Conposui et quo sint rura colenda modo; \\ Post, quibus Aeneas lutlos superaverit armis: \\ Vatis relliquias hic pia terra fovet. \\ 
      \end{verse}
  
            \subsection*{563}
      \begin{verse}
      \poemtitle{ASCLEPIADII}I 0 \\ Sicanius vates silvis, Ascraeus in arvis, \\ Maeonius bellis ipse poeta fui. \\ Mantua se vita praeclari iactat alumni, \\ Parthenope famam morte Maronis habet. \\ 
      \end{verse}
  
            \subsection*{564}
      \begin{verse}
      \poemtitle{EVSTNENII}Quisquis es, extremi titulum lege carminis, hospes. \\ lac ego Vergilius sum tumulatus humo, \\ 
        \pagebreak 
    \begin{center} \textbf{C. 564. 3. C. 565 566. VII, 1—2} \end{center} \marginpar{[75]} Qui pecudum pastus, qui cultum fertilis agri, \\ Mox Anchisiadae bella ducis cecini. \\ 
      \end{verse}
  
            \subsection*{565}
      \begin{verse}
      \poemtitle{POMPLIANI}Vergilius mihi nomen erat, quem Mantua felix \\ Edidit. hic cineres vatis et ossa iacent. \\ Cuius in aeternum pastoria fistula vivet, \\ Rustica mox, eadem Martia Calliope. \\ 
      \end{verse}
  
            \subsection*{566}
      \begin{verse}
      \poemtitle{MAXIMINI}Carmine bucolico nitui, cultoribus agri \\ Iura dedi, cecini bella Latina simul. \\ Iamque ad lustra decem Titan accesserat alter, \\ Cum tibi me rapuit, Mantua, Parthenope. \\ I Tetrasticha de quattuor temporibus anni \\ B. V 52 63. \\ Ovidius: \\ M. 493 504. \\ B. IV 131. \\ Verque novum stabat cinctum florente corona, \\ Stabat nuda Aestas et spicea serta gerebat, \\ 
        \pagebreak 
     \marginpar{[76]} \begin{center} \textbf{V I, 34. C. 567 569} \end{center}Stabat et Autumnus, calcatis sordidus uvis, \\ Et glacialis liems, canos hirsuta capillos’ \\ 
      \end{verse}
  
            \subsection*{567}
      \begin{verse}
      \poemtitle{BASILII}Vere sinum tellus aperit loresque ministrat. \\ Tempore solis ager messes fert pinguis opimas. \\ Fecundos, autumne, lacus de vitibus imples. \\ Vis hiemis glacie currentes alligat undas. \\ 
      \end{verse}
  
            \subsection*{568}
      \begin{verse}
      \poemtitle{ASMENII}Frigoribus pulsis nitidum ver aethera mulcet. \\ Scindit agros aestas Phoebeis ignibus ardens. \\ Autumno dat hiems mixtum vicina teporem. \\ Labentes haec durat aquas et flumina nectit. \\ 
      \end{verse}
  
            \subsection*{563}
      \begin{verse}
      \poemtitle{VOMANII}Ver pingit vario gemmantia prata colore. \\ Ignea vestit agros culmis Cerealibus aestas. \\ Vitibus autumus turgentes detrahit uvas. \\ Frigidus hiberno est gravibus nive nubibus aether. \\ 
        \pagebreak 
    \begin{center} \textbf{C. 570 573} \end{center}
      \end{verse}
  
            \subsection*{570}
      \begin{verse}
      \poemtitle{EVPOR3II}Vere Venus gaudet florentibus aurea sertis. \\ Flava Ceres aestatis habet sua tempore regna. \\ Evifero autumno summa est tibi, Bacche, potestas: \\ Imperinm saevis hiberno frigore ventis. \\ D 41 \\ \poemtitle{IVLIANI}Vere gravis fundit tellus cum floribus herbas. \\ Frugiferas arvis fert aestas torrida messes. \\ Pomifer autumnus tenero dat palmite fructus. \\ Mox humus hibernis albescit operta pruinis. \\ 
      \end{verse}
  
            \subsection*{572}
      \begin{verse}
      \poemtitle{LASII}Vere novis laeto decorantur floribus arva, \\ Et riget aestivis hirsutus campus aristis. \\ Labra per autumnum musto spumantia fervent, \\ Et ponunt frondes hiemali frigore silvae. \\ 4a0 \\ \poemtitle{PALLADII}Ver placidum vario nectit de flore coronas. \\ Spicea serta ligat calidissima solibus aestas, \\ Temporaque autumnus cingit tua, Bacche, racemis. \\ Tristis hiems montes niveo velamine vestit. \\ 
        \pagebreak 
    \begin{center} \textbf{C. 574 577, 12} \end{center}
      \end{verse}
  
            \subsection*{574}
      \begin{verse}
      \poemtitle{ASCEPIADII}m \\ Ver agros nitidum gemmis stellantibus ornat \\ Et fervens aestas pinguissima frugibus arva. \\ Mox autumnali redolet vindemia fetu. \\ Fronde nemus male nudat hiems amnesque rigescunt. \\ 
      \end{verse}
  
            \subsection*{575}
      \begin{verse}
      \poemtitle{EVSTIIENII}Purpureos flores humus efert vere comanti, \\ Et Cereris donis horrescunt arva per aestum. \\ Bacche, tuo tempus fluit autumnale liquore. \\ Obtegitur tellus per frigora veste nivali. \\ 
      \end{verse}
  
            \subsection*{576}
      \begin{verse}
      \poemtitle{POPLIANI}Vere tepet picto Eephyris spirantibus aer. \\ Decrescunt celeres aestivis ignibus amnes. \\ Temperies, autumne, fluit tua nectare dulci, \\ Perque hiemem lentus caelo nivis advolat imber. \\ D 4 \\ V 7 \\ \poemtitle{MAXIMINI}Veris honos tepidi lores, vere omuia rident. \\ Arva sub aestivis undant horrentia flabris. \\ 
        \pagebreak 
    \begin{center} \textbf{C. 577, 3. C. 578 580} \end{center} \marginpar{[79]} Vite coronatas autumnus degravat ulmos. \\ Decutit ipse rigor silvis hiemalis honorem. \\ 
      \end{verse}
  
            \subsection*{578}
      \begin{verse}
      \poemtitle{VIEALIS}Flore solum vario depingit odoriferum ver, \\ Faleiferamque deam messes remorantur in aestu. \\ Dat musto ravidas autumnus pomifer uvas. \\ Sithonia lacialis hiems nive cana senescit. \\ VIII  \lbrack Tetrastiehal de aurora et sole \\ 
      \end{verse}
  
            \subsection*{579}
      \begin{verse}
      IB, V 2t3. \\ M. 457 468. \\ \poemtitle{ASMENII}B. IV 134. \\ Aurora oceanum croceo velamine fulens \\ Liquerat et biiugis vecta rubebat equis. \\ Luce polum nitida perfudit candidus orbis \\ Et clarum emicuit sole oriente iubar. \\ 
      \end{verse}
  
            \subsection*{580}
      \begin{verse}
      \poemtitle{VOMANII}Roscida puniceo Pallantias exit amictu, \\ Astrigerum inficiens luce rubente polum. \\ Sol insigne caput radiorum ardente corona \\ Promit, ab aequoreis Tethyos ortus aquis. \\ 
        \pagebreak 
     \marginpar{[80]} \begin{center} \textbf{C. 581 5e, 1—2} \end{center}
      \end{verse}
  
            \subsection*{581}
      \begin{verse}
      \poemtitle{EVPOR3II}Extulit oceano caput aureus igniferum Sol: \\ Fugerunt toto protinus astra polo. \\ Concessere deo tenebrae, rebusque colores \\ Lux iterum cunctis reddidit alma suos. \\ 
      \end{verse}
  
            \subsection*{582}
      \begin{verse}
      \poemtitle{IVLIANI}V \\ ithoni coniunx roseo sublime rubore \\ Infecit caelum lutea sidereum, \\ Cum Sol igniferos currus e gurgite magno \\ Sustulit et claris astra fugavit equis. \\ 
      \end{verse}
  
            \subsection*{583}
      \begin{verse}
      \poemtitle{IILASII}Nox abit astrifero velamine cincta micanti \\ Et redigit stellas, exoriturque dies. \\ Emicat oceano Phoebi rota clara relicto \\ Inlustratque nitens lumine cuncta suo. \\ 
      \end{verse}
  
            \subsection*{584}
      \begin{verse}
      \poemtitle{PALLADII}AuroracapillisLuteafulgebatroseis \\ madebathumus.Etmatutinorore \\ 
        \pagebreak 
    \begin{center} \textbf{C. 584, 3. C. 585 587} \end{center} \marginpar{[81]} Tethyos undivagae tum prosilit aequore Titan, \\ Flammiferos vultus ore micante gerens. \\ 
      \end{verse}
  
            \subsection*{585}
      \begin{verse}
      \poemtitle{ASCLEPADII}IDI \\ Exoritur Phoebus perfundens luce nitente \\ Et maria et terras stelliferumque polum; \\ Astraque cesserunt fulgentia crinibus aureis \\ Et nox sidereas occulit atra faces. \\ 
      \end{verse}
  
            \subsection*{586}
      \begin{verse}
      \poemtitle{EVSTIENII}Sol oriens currusque suos e gurgite tollens \\ Oceani claro reddidit orbe diem \\ Flammiferumque iubar terraeque poloque reduxit \\ Et pepulit radiis astra repente suis. \\ \poemtitle{OMPILANI}Memnonis ut genetrix infecerat humida caelum \\ Et roseis manibus sidera dispulerat, \\ Phoebus Atlanteis e fluctibus aureus orbem \\ Sustulit igniferum, luxque diesque redit. \\ 
        \pagebreak 
     \marginpar{[82]} \begin{center} \textbf{C. 583 590} \end{center}
      \end{verse}
  
            \subsection*{588}
      \begin{verse}
      \poemtitle{MAXIMINI}Praevia flammiferi currus Aurora rubebat \\ Extuleratque alto gurgite Phoebus equos \\ Noctivagosque simul radiis flagrantibus ignes \\ Depulerat caelo reddideratque diem. \\ 
      \end{verse}
  
            \subsection*{589}
      \begin{verse}
      \poemtitle{VTAIIS}Vix Aurora suo rubefecerat aethera curru \\ Summaque canebat roribus herba novis: \\ Prosilit e mediis candens rota Tethyos undis \\ Et vaga cesserunt sidera Solis equis. \\ 
      \end{verse}
  
            \subsection*{590}
      \begin{verse}
      \poemtitle{BASILII}Surgit ab oceano Tithoni fulgida coniunx \\ Et veste ab rosea subrubet ipse polus, \\ Cum Phoebus radiis rutilum cingentibus orbem \\ Depellit tenebras noxque peracta fugit. \\ 
        \pagebreak 
    \begin{center} \textbf{C. 591 593} \end{center} \marginpar{[83]} I Pentsstieha de duodecim lbrls Aeneidos \\ B. I 195. \\ 
      \end{verse}
  
            \subsection*{591}
      \begin{verse}
      M. 409 420. \\ B. IV 136. \\ \poemtitle{VOMANII}Lier I \\ Aeolus inmittit ventum Iunone precante \\ Troianis, Libycasque vagos expellit in oras. \\ Solatur Venerem dictis pater ipse dolentem. \\ Aenean recipit pulchra Carthagine Dido, \\ Cui Venns Ascanii sub imagine mittit Amorem. \\ 
      \end{verse}
  
            \subsection*{592}
      \begin{verse}
      \poemtitle{EVPIORBII}\poemtitle{. 71}Cogitur Aeneas bellorum exponere casus \\ Graiorumque dolos et equum fraudemque Sinonis \\ Excisamque urbem Priamique miserrima fata, \\ Vtque patrem impositum forti cervice per igues \\ Extulerit caramque amiserit ipse Creusam. \\ 
      \end{verse}
  
            \subsection*{593}
      \begin{verse}
      \poemtitle{IVLIANI}Li. III \\ Post casum Troiae fabricata classe superstes \\ Vela dat Aeneas urbemque in litore Thraces, \\ Mox aliam pulsus Cretaeis condidit oris. \\ Cedit et hinc lelenumque videt praeceptaque sumit \\ Et caecum Cyclopa fugit sepelitque parentem. \\ 
        \pagebreak 
    \begin{center} \textbf{C. 594 597,1—2} \end{center} \marginpar{[84]} 
      \end{verse}
  
            \subsection*{594}
      \begin{verse}
      \poemtitle{ILASII}\poemtitle{L0. IIV}Ardet amore gravi Dido. soror Anna suadet \\ Nubere. iunguntur nimbo cogente sub antro. \\ Incusat precibus patrem contemptus larbas. \\ Navigat Aeneas iussu lovis; illa dolore \\ Inpatiens et amore necem sibi protinus infert. \\ 
      \end{verse}
  
            \subsection*{595}
      \begin{verse}
      \poemtitle{PALLADII}Li. \\ In Siculas iterum terras fortuna reducit \\ Aenean, tumuloque patris persolvit honorem. \\ Tum cogit naves incendere Troadas Hris. \\ Troes ibi linquunt socios. Venus anxia placat \\ Neptunum. somnus Palinurum mergit in undas. \\ 
      \end{verse}
  
            \subsection*{596}
      \begin{verse}
      \poemtitle{ASCLEPIADII}Libo. 7I \\ Sacratam Phoebo Cumarum fertur in urbem \\ Rex Phrygius vatisque petit responsa Sibyllae. \\ Misenum sepelit. post haec adit infera regna \\ Congressusque patri discit genus omne suorum, \\ Quoque modo casus valeat superare futuros. \\ 
      \end{verse}
  
            \subsection*{597}
      \begin{verse}
      \poemtitle{EVSTENII}Li. VII \\ Tandem deveniunt Laurentia Troes in arva \\ Et pace accepta laeti nova moenia condnt. \\ 
        \pagebreak 
    \begin{center} \textbf{C. 597, 35. C. 598 600} \end{center} \marginpar{[85]} Nocte satam luno Furiam evocat: illa Latinos \\ Inter et Aeneadas bellum serit et ciet arma. \\ Protinus auxiliis terra instruit Itala Turnum. \\ 
      \end{verse}
  
            \subsection*{598}
      \begin{verse}
      \poemtitle{POMPLIANI}\poemtitle{L. III}Vidit ut Aeneas summa vi bella parari, \\ Arcadas Euandrumque senem sibi foedere iungit \\ Dardanioque duci sociatur Etruria tota. \\ Arma petit genetrix, dat Mulciber, in clipeoque \\ Res fingit Latias et fortia facta nepotum. \\ 
      \end{verse}
  
            \subsection*{599}
      \begin{verse}
      \poemtitle{MAXIMINI}Lb. IX \\ Ad Turnum propere Iunoni mittitur Hris \\ Instigatque animos. aciem movet ille Phrygasque \\ Obsidet. in nymphas versa est Aeneia classis. \\ Euryalus Nisusque luunt nece proelia noctis. \\ V Turnus potitur castris, vi pellitur inde. \\ 
      \end{verse}
  
            \subsection*{600}
      \begin{verse}
      \poemtitle{VEALIS}Li0. X \\ Placat et uxoris dictis et iuria natae \\ Iuppiter. auxiliis instructus Troius heros \\ Advenit. occurrunt Rutuli atque in litore pugnant. \\ Occidit a Turno Pallas victorque superbus \\ Aeneae eripitur. Meaentius interit acer. \\ 
        \pagebreak 
    \begin{center} \textbf{C. 601 603, 1—2} \end{center} \marginpar{[86]} 
      \end{verse}
  
            \subsection*{601}
      \begin{verse}
      \poemtitle{BASLII}Li. XI \\ Occisis proprium pars utraque reddit honorem. \\ Supplicibus Calydone satus negat arma Latinis. \\ Cum Drance alterno iurgat certamine Turnus. \\ Aeneas equitem praemittit, et obvia virgo \\ Excipit. extincta Rutuli dant terga Camilla. \\ 
      \end{verse}
  
            \subsection*{602}
      \begin{verse}
      \poemtitle{ASMENII}Lb0. XII \\ Troianis lutulisque placet coniungere foedus. \\ Id Rutuli rumpunt. nato Venus alma medetur \\ Dictamno lRutulique luunt periuria victi. \\ Coitur Aeneae Dauni concurrere proles. \\ Pallantea necem misero dant cingnla Turno. \\ X  \lbrack Hexastichl de titulo Ciceroni \\ 
      \end{verse}
  
            \subsection*{603}
      \begin{verse}
      B. II 158 169 \\ M. 397 408. \\ \poemtitle{EVPMOR3II}B. IV 139. \\ Hic iacet Arpinas manibus tumulatus amici, \\ Qui fuit orator summus et eximius, \\ 
        \pagebreak 
    \begin{center} \textbf{C. 603, 3s. C. 604 605} \end{center} \marginpar{[87]} Quem nece crudeli mactavit civis et hostis. \\ Nil agis, Antoni: scripta diserta manent. \\ Vulnere nempe uno Ciceronem conficis, at te \\ Tullius aeteruis vulneribus lacerat. \\ 
      \end{verse}
  
            \subsection*{604}
      \begin{verse}
      \poemtitle{IVLIANI}Corpus in hoc tumulo magni Ciceronis humatum \\ Contegitur, claro qui fuit ingenio, \\ Quique malis gravis hostis erat tutorque bonorum, \\ Quo paene indigne consule oma perit. \\ Sed vigili cura deiectis hostibus urbe \\ Supplicioque datis, praestitit incolumem. \\ 
      \end{verse}
  
            \subsection*{605}
      \begin{verse}
      \poemtitle{BILASII}Vnicus orator, lumenque decusque senatus, \\ Servator patriae, conditor eloquii, \\ Cuius ab ingenio laude inlustrata perenui \\ Lumine praeclaro lingua latina viget, \\ Occidit indigne manibus laceratus iniquis \\ Tullius ac tumulo subditus exiguo est. \\ 
        \pagebreak 
    \begin{center} \textbf{C. 606 608} \end{center} \marginpar{[88]} 
      \end{verse}
  
            \subsection*{606}
      \begin{verse}
      \poemtitle{ALLAD11}Quicumque in libris nomen Ciceronis adoras, \\ Aspice, quo iaceat conditus ille loco. \\ Ille vel orator vel civis maximus; idem \\ Clarus erat factis, clarior eloquio; \\ Ac, ne quid Fortuua viro nocuisse putetur, \\ Vivus in aeternum docta per ora volat. \\ 
      \end{verse}
  
            \subsection*{607}
      \begin{verse}
      \poemtitle{ASCLEPIADII}Marcus eram Cicero toto notissimus orbe, \\ Cuius relliquias occulit urna brevis. \\ Dextera me patriae nuper civilis ademit, \\ Eripui patriam qui prius exitio. \\ Si quis in hoc saxo Tulli legis, advena, nomen, \\ Non dedigneris dicere: Marce, vale! \\ 
      \end{verse}
  
            \subsection*{60}
      \begin{verse}
      \poemtitle{EVTENII}Tullius Arpinas ex ordine natus equestri, \\ Sed virtute sua consul in urbe fuit. \\ Quem Catilina malus coniuratique nocentes \\ Senserunt vigilem civibus esse suis. \\ Hunc tamen (o pietas!) tres occidere tyranni; \\ At Lamia ille pio subposuit tumulo. \\ 
        \pagebreak 
     \marginpar{[89]} \begin{center} \textbf{C. 609 611} \end{center}
      \end{verse}
  
            \subsection*{609}
      \begin{verse}
      \poemtitle{POMPLIAN}Qui tenet eloquii fastigia summa Latini, \\ Qui consul patriam caedibus eripuit, \\ Quique trium saevo vitam dedit ense virorum, \\ Tullius en hac est ipse sepultus humo. \\ Sed vitae brevitas pensatur laude perenni; \\ Quod mors eripuit, gloria restituit. \\ 
      \end{verse}
  
            \subsection*{610}
      \begin{verse}
      \poemtitle{MAXIMNI}VV 1IV \\ Tullius hic situs est, venerabile nomen in aevum, \\ Clarus honore simul, clarus et ingenio, \\ Quem scelerata neci crudeliter arma dederunt, \\ Quod patriae vindex ille fidelis erat. \\ Sed nihil infanda profecit caede tyrannus: \\ Ingenium vivit; corpus inane perit. \\ 
      \end{verse}
  
            \subsection*{611}
      \begin{verse}
      \poemtitle{VTALS}lomani princeps populi, decus ordinis ampli, \\ Maximus orator, civis etPegregius, \\ Coniuratorum vindex hostisque malorum \\ Proscriptus periit a tribus ille viris. \\ Qui caesus graviter, qui detruncatus acerbe \\ Hoc Lamiae debet, quod iacet in tumulo. \\ 
        \pagebreak 
    \begin{center} \textbf{C. 612 614} \end{center} \marginpar{[90]} 
      \end{verse}
  
            \subsection*{612}
      \begin{verse}
      \poemtitle{BASILHI}Doctrinae antistes, rerum mirabilis auctor, \\ Tullius existens nobilis ex humili, \\ Cui dedit excellens ars oratoria nomen, \\ Virtute ingenii venit in astra sui. \\ Sed Fortuna nocens miserando funere raptum \\ Carpsit et hoc voluit membra iacere loco. \\ 
      \end{verse}
  
            \subsection*{613}
      \begin{verse}
      \poemtitle{ASMENII}Eloquio princeps, magnis memorabilis actis, \\ Tullius indigna caede peremptus obit. \\ Sed terras omnes implevit nomine claro; \\ Ingenium caeso corpore morte caret. \\ Vivit et ingenti pollet cum laude per orbem, \\ Cuinus in hoc tumulo membra sepulta iacent. \\ 
      \end{verse}
  
            \subsection*{614}
      \begin{verse}
      \poemtitle{VOMANII}Inclitus hic Cicero est Lamiae pietate sepultus, \\ Quem Fortuna nevi adidit immeritae. \\ Maximus eloquio, civis bonus, urbis amator, \\ Perniciesque malis perfugiumque bonis. \\ Qui sexaginta conpletis ac tribus annis \\ Servitio pressam destituit patriam. \\ 
        \pagebreak 
    \begin{center} \textbf{C. 615 617, 1—2} \end{center} \marginpar{[91]} XI  \lbrack extichl de duodecim sinis \\ 
      \end{verse}
  
            \subsection*{615}
      \begin{verse}
      B. V 29 39. \\ M. 431 492. \\ \poemtitle{IVLIAN1}B. IV 143. \\ Primus adest Aries Taurusque insignibus auro \\ Cornibus et fratres et Cancer, aquatile signum, \\ Tum Leo terribilis Nemeaeus et innuba Virgo, \\ Libra subit, caudaque animal quod dirigit ictum, \\ Armatusque arcu Chiron et corniger hircus, \\ Fusor aquae simul et fulgenti lumine Pisces. \\ 
      \end{verse}
  
            \subsection*{616}
      \begin{verse}
      \poemtitle{LASII}Proditor est llelles et proditor Europaeus \\ Et Gemini invenes et pressus ab lercule Cancer, \\ Horrendusque Leo sequitur cum Virgine sancta \\ Libraque lance pari et violentus acumine caudae, \\ Inde sagittiferi facies senis et Capricornus, \\ Et qui portat aquam puer urniger, et duo Pisces. \\ 
      \end{verse}
  
            \subsection*{517}
      \begin{verse}
      \poemtitle{PALLADII}Signorum princeps Aries et Taurus et una \\ Tyndaridae iuvenes et fervida brachia Cancri \\ 
        \pagebreak 
    \begin{center} \textbf{C. 617, 3. C. 618 619} \end{center} \marginpar{[92]} Herculeusque Leo, Nemeae pavor, almaque Virgo, \\ Libra iugo aequali pendens et Scorpius acer \\ Centaurusque senex Chiron et cornua Capri \\ Et iuvenis estator aquae Piscesque supremi. \\ 
      \end{verse}
  
            \subsection*{618}
      \begin{verse}
      \poemtitle{ASCLEPIADI}r m \\ Laniger astrorum ductor, Taurusque secundus, \\ Tum sidus geminum et Cancri fulgentis imago, \\ Truxque Leo et Virgo, quae spicea munera gestat, \\ Et Libram qui Caesar habet, chelaeque minaces \\ Atque arcu pollens et salsi gurgitis hircus \\ Vrnaque nimbiferi Piscesque, novissima forma. \\ 
      \end{verse}
  
            \subsection*{619}
      \begin{verse}
      \poemtitle{EVSTENII}Dux Aries et frons Tauri metuenda minacis \\ Et Ledae soboles et Cancri torridus ignis \\ Terribilisque Leo, species quoque Virginis almae, \\ Momentumque sequens, caudaque timendus obunca, \\ Hinc tendens arcum, liquidi Caper aequoris inde, \\ Troiadesque puer geminique sub aethere Pisces. \\ 
        \pagebreak 
    \begin{center} \textbf{C. 620 62, 12} \end{center} \marginpar{[93]} 
      \end{verse}
  
            \subsection*{620}
      \begin{verse}
      \poemtitle{POMPLIAI}
      \end{verse}
  
            \subsection*{621}
      \begin{verse}
      Vcr \\ \poemtitle{MAXIMINI}Nubigenae iuvenis vector Taurique trucis frons \\ Et proles duplex Hovis et nepa torrida flammis, \\ Aestifer inde Leo iusta cum irgine fulgens, \\ Quam sequitur Libra et violenta cuspide saevus, \\ Semifer rcitenens subit et Capricornus aquosus \\ Et cui nomen aquae faciunt, iscesque gemelli. \\ 
      \end{verse}
  
            \subsection*{4409}
      \begin{verse}
      \poemtitle{VEALIS}Corniger in primis Aries et corniger alter \\ Taurus, item Gemini, sequitur quos Cancer adustus, \\ Terribilisque ferae species et iusta puella, \\ ibra simul nigrumque gerens in acumine virus, \\ Centaurusque biformis adest pelagique capella \\ Atque amnem fundens et Pisces, sidus aquosum. \\ 
      \end{verse}
  
            \subsection*{623}
      \begin{verse}
      \poemtitle{BASILII}Lanigeri ductor gregis, Europae quoque vector \\ Et duo Tvndaridae. tum Cancer sole nerustus \\ 
        \pagebreak 
    \begin{center} \textbf{C. 623, 36. C. 624 626. 1} \end{center} \marginpar{[94]} Hlerculeaque manu pressus Leo et optima Virg, \\ linc trutinae species venit armatusque veneno \\ Scorpius atque sagittifer aequoreique Capri frons, \\ Quique nurnam gerit et Pisces, duo signa sub uno. \\ 
      \end{verse}
  
            \subsection*{624}
      \begin{verse}
      II1 \\ \poemtitle{ASMENII}Principium signis ovium pater, inde iuvencus, \\ Progenies duplex. et aquarum Cancer alumnus, \\ Pressa sub Herculeis manibus fera, iustaque Virgo. \\ ibra subest, caudaque gerens letale venenum. \\ Tum geminus tChiron et corniger aequoris alti \\ Dilectusque lovi puer et (duo sidera) Pisces. \\ 
      \end{verse}
  
            \subsection*{525}
      \begin{verse}
      \poemtitle{VOMANII}Dux gregis et placidum pandens subit aethera Taurus \\ Germanique pares et Cancro iam comes aestas \\ Atque Leo, primus labor Herculis. et pia Virgo. \\ Iibra comes sequitur minitans et Scorpius ictu \\ Et qui tela gerit Centaurus et aequoris hircus, \\ Deucalionis aquae et Pisces, postrema figura. \\ 
      \end{verse}
  
            \subsection*{626}
      \begin{verse}
      \poemtitle{EVPIIORBII}Velleris aurati fulget pecus aureaque lo, \\ 
        \pagebreak 
    \begin{center} \textbf{C. 626, 2s. C. 627, 1—10} \end{center} \marginpar{[95]} Eethus et Amphion Cancrique figura calentis. \\ Insequitur Leo saevus et almae Virginis astra, \\ Hlinc aequale iuum caudaque venenifer unca, \\ Centaurusque minax arcu et Neptunia capra \\ Quique refundit aquas, et Pisces, ultimus ordo. \\ XII Poystie \rbrack  \\ 
      \end{verse}
  
            \subsection*{627}
      \begin{verse}
      B. IV 14q. \\ B. I 42. \\ \poemtitle{ILASII}M. 598. \\ Dodecastieha de Iereule \\ Oppressit Nemeae primum virtute leonem. \\ Extincta est anguis, quae pullulat, llydra secundo. \\ Tertius enectus sus est Erymanthius ingens. \\ Cornibus auratis cervum necat ordine quarto. \\ Deicit horrisono quinto Stymphalidas arcu. \\ Abstulit llippolytae sexto sua cingula victae. \\ Septimus Anei stabulum labor egerit undis. \\ Octavo domuit magno luctamine taurum. \\ Tum Diomedis equos nono cum rege peremit. \\ Geryonem decimo triplici cum corpore vicit. \\ 
        \pagebreak 
    \begin{center} \textbf{C. 627. 1112. C. 628 629, 12} \end{center} \marginpar{[96]} Vndecimo extractus vidit nova Cerberus astra. \\ Postremolesperidum victor tulit aurea mala. \\ 
      \end{verse}
  
            \subsection*{628}
      \begin{verse}
      B. I 136. \\ \poemtitle{PALLAD}M. 263. \\ \poemtitle{De Orpheo}Threicius quondam vates fide creditur canora \\ Movisse sensus acrium ferarum \\ Atque amnes tenuisse vagos, sed et alites volantes, \\ Et surda cantu concitasse saxa, \\ Suavisonaeque moduos testudinis arbores secutae \\ Vmbram feruntur praebuisse vati. \\ Scilicet haud potuit, quae sunt sine, permovere, sensu \\ (Finxere doctam fabulam poetae), \\ Sed placidis hominum dictis fera corda mitigavit \\ Doctaque vitam voce temperavit; \\ Iustitiam docuit, coetu quoque congregavit uno \\ Moresque arestes expolivit tOrpheus. \\ 
      \end{verse}
  
            \subsection*{6293}
      \begin{verse}
      B. III 104. \\ \poemtitle{ASCLEPIADII}I \\ M. 540 \\ \poemtitle{De Fortuna}0 Fortuna potens et nimium levis, \\ Tantum iuris atrox quae tibi vindicas, \\ 
        \pagebreak 
    \begin{center} \textbf{C. 629, 3 15. C. 630, 17} \end{center} \marginpar{[97]} Evertisque bonos, erigis improbos, \\ Nec servare potes muneribus fidem! \\ Fortna immeritos auget honoribus, \\ Fortuna innocuos cladibus afficit. \\ Iustos illa viros pauperie gravat, \\ Indignos eadem divitiis beat. \\ Haec aufert iuvenes ac retinet senes, \\ Iniusto arbitrio tempora dividens. \\ Quod dignis adimit, transit ad impios. \\ Nec discrimen habet rectaque iudicat, \\ Inconstans fragilis perfida lubrica. \\ Nec quos clarificat, perpetuo fovet, \\ Nec quos deseruit, perpetuo premit. \\ 
      \end{verse}
  
            \subsection*{630}
      \begin{verse}
      \poemtitle{EVSTENII}B. I 98. \\ M. 1614. \\ \poemtitle{De Aehille}Pelides ego sum, Thetidis notissima proles, \\ Cui virtus clarum nomen habere dedit, \\ Qui stravi totiens armis victricibus hostes \\ Inque fugam solus milia multa dedi. \\ lHectore sed magno summa est mihi gloria caeso, \\ Qui saepe Argolicas debilitavit opes. \\ lle Menoetiadae solvit me vindice poenas; \\ 
        \pagebreak 
     \marginpar{[98]} \begin{center} \textbf{C. 60, 8 10. C. 631 32. 1—2} \end{center}Pergama tunc ferro procubuere meo. \\ Laudibus inmensis victor super astra ferebar, \\ Cum pressi hostilem fraude peremptus humum. \\ 
      \end{verse}
  
            \subsection*{63}
      \begin{verse}
      \poemtitle{POPLIANI}B. I 102. \\ M. 241. \\ \poemtitle{De Iectore}Defensor patriae, iuvenum fortissimus, lector, \\ Qui murus miseris civibus alter erat, \\ Occubuit telo violenti victus Achillis: \\ Occubuere simul spesque salusque Phrygm \\ Hunc ferus Aeacides circum sua moenia traxit, \\ Quae iuvenis manibus texerat ante suis. \\ O quantos Priamo lux attulit ista dolores! \\ Quos fletus Hecubae, quos dedit Andromachae! \\ Sed raptum pater infelix auroque repensum \\ Condidit et maerens hac tumulavit humo. \\ 
      \end{verse}
  
            \subsection*{632}
      \begin{verse}
      \poemtitle{MAXIMNI}B. V 140. \\ M. 1076. \\ \poemtitle{De V littera}Littera Pythagorae, discrimine secta bicorni, \\ HIumanae vitae speciem praeferre videtur. \\ 
        \pagebreak 
    \begin{center} \textbf{C. 632, 32. C. 633, 1 —10} \end{center} \marginpar{[99]} Nam via virtutis dextrum petit ardua callem \\ Difficilemque aditm primo spectantibus offert, \\ Sed requiem praebet fessis in vertice summo. \\ Molle ostentat iter via laeva, sed ultima meta \\ Praecipitat captos volvitque per aspera saxa. \\ Quisquis enim duros casus virtutis amore \\ Vicerit, ille sibi laudemque decusque parabit. \\ At qui desidiam luxumque sequetur inertem, \\ Dum fugit oppositos incauta mente labores, \\ Turpis inopsque simul miserabile transiget aevum. \\ 
      \end{verse}
  
            \subsection*{633}
      \begin{verse}
      B. III 85. \\ \poemtitle{VTALIS}M. 535. \\ \poemtitle{De libidine et vio}Ne Veneris nec tu vini tenearis amore; \\ Vno namque modo vina Venusque nocent. \\ V Venus enervat vires, sic copia Bacchi \\ Et temptat gressus debilitatque pedes. \\ Multos caecus amor cogit secreta fateri: \\ Arcanum demens detegit ebrietas. \\ Bellum saepe ciet ferus exitiale Cupido: \\ Saepe manus itidem lBacchus ad arma vocat. \\ Perdidit horrendo Troiam Venus improba bello: \\ At Lapithas bello perdis, Iacche, gravi. \\ 
        \pagebreak 
    \begin{center} \textbf{C. 6, I116. C. 634, 19} \end{center} \marginpar{[100]} Denique cum mentes hominum furiavit uterque, \\ Et pudor et probitas et metus omnis abest. \\ Conpedibus Venerem, vinclis constringe Lyaeum, \\ Ne te muneribus laedat uterque suis. \\ Vina sitim sedent, natis Venus alma creandis \\ Serviat: hos fines transiluisse nocet. \\ 
      \end{verse}
  
            \subsection*{634}
      \begin{verse}
      B. II 190. \\ \poemtitle{BASLII}M. 532. \\ \poemtitle{De XII libris Aeneidos}Primus habet, Libycam veniant ut Troes in urbem. \\ Edocet excidium Troiae clademque secundus. \\ Tertius a Troia vectos canit aequore Teucros. \\ Quartus item miserae duo vulnera narrat Elissae. \\ Manibus Anchisae quinto celebrantur honores. \\ Aenean memorat visentem Tartara sextus. \\ In Phrygas Italiam bello iam septimus armat. \\ Dat simul Aeneae socios octavus et arma. \\ Daunins expugnat nono nova moenia Troiae. \\ 
        \pagebreak 
    \begin{center} \textbf{C. 634, 10 12. C. 635, 120} \end{center} \marginpar{[101]} Exponit decimus Tuscorum in litore pugnas. \\ Vndecimo Rutuli superantur morte Camillae. \\ Vltimus imponit bello Turni nece finem. \\ 
      \end{verse}
  
            \subsection*{635}
      \begin{verse}
      B. III 51. \\ \poemtitle{ASMENII}M. 533 \\ \poemtitle{De laude horti}Adeste Musae, maximi proles Iovis, \\ Laudes feracis praedicemus hortuli. \\ Hortus salubres corpori praebet cibos \\ Variosque fructus saepe cultori refert: \\ Holus suave, multiplex herbae genus, \\ lvas nitentes atque fetus arborum. \\ Non detit hortis et voluptas maxima \\ Multisque mixta commodis iocunditas. \\ Aquae strepentis vitrens lambit liquor \\ Sulcoque ductus irrigat rivus sata. \\ Flores nitescunt discolore ermine \\ Pinguntque terram gemmeis honoribus. \\ Apes susurro murmurant gratae levi, \\ Cum summa florum vel novos rores legunt. \\ Fecunda vitis coniuges ulmos gravat \\ Textasve inumbrat pampinis harundines. \\ Opaca praebent arbores umbracula \\ Prohibentque densis fervidum solem comis. \\ Aves canorae garrnlos fundnnt sonos \\ Et semper aures cantibus mulcent suis. \\ 
        \pagebreak 
     \marginpar{[102]} \begin{center} \textbf{C. 63., 2125. C. 636, 1 16} \end{center}Oblectat hortus, avocat pascit tenet \\ Animoque maesto demit angores graves. \\ Membris vigorem reddit et visus capit. \\ Refert labori pleniorem gratiam, \\ Tribuit colenti multiforme gaudium. \\ 
      \end{verse}
  
            \subsection*{636}
      \begin{verse}
      B. III 92 \\ \poemtitle{VOMANII}M. 534. \\ \poemtitle{De interno livore}Livor, tabificum malis venenum, \\ Intactis vorat ossibus medullas \\ Et totum bibit artubus cruorem. \\ Quo quisquis furit invidetque sorti, \\ Vt debet, sibi poena semper ipse est. \\ Testatur gemitu graves dolores, \\ Suspirat fremit incutitque dentes; \\ Sudat frigidus, intuens quod odit. \\ Efundit mala lingua virus atrum, \\ Pallor terribilis genas colorat, \\ Infelix macies renudat ossa. \\ Non lux, non cibns est suavis illi; \\ Non potus iuvat aut sapor Lyaei, \\ Nec si pocula Iuppiter propinet \\ Atque haec porrigat et ministret llebe \\ Aut tradat Catamitus ipse nectar. \\ 
        \pagebreak 
    \begin{center} \textbf{C. 636, 1725. C. 637, 19} \end{center} \marginpar{[103]} Non somnum capit aut quiescit umquam: \\ Torquet viscera carnifex cruentus. \\ Vesanos tacite movet furores \\ Intentans animo faces Erinys; \\ Est ales Tityi usque vultur intus, \\ Qui semper lacerat comestque mentem. \\ Vivit pectore sub dolente vulnus, \\ Quod Chironia nec manus levarit \\ Nee Pboebus sobolesve clara Phoebi. \\ 
      \end{verse}
  
            \subsection*{637}
      \begin{verse}
      \poemtitle{EVPOR3II}B. I 169. M. 277. \\ B. IV 154. \\ AMenun ed. \\ \poemtitle{De Sirenis}Hoder p. 174. \\ Sirenes varios cantus, Acbheloia proles, \\ Et solitae miros ore ciere modos \\ (llarum voces, illarum Musa movebat \\ Omnia quae thymele carmina dulcis habet: \\ Qod tuba, quod litui, quod cornua rauca queruntur, \\ Quodque foraminibus tibia mille sonat, \\ Quod leves calami, quod suavis cantat aedon, \\ Quod lyra, quod citharae, quod moribundus olor) \\ Inlectos nautas dulci modulamine vocum \\ 
        \pagebreak 
     \marginpar{[104]} \begin{center} \textbf{C. 637. 10—18. C. 638} \end{center}Mergebant avidae fluctibus Honiis. \\ Sanguine Sisyphio generatus venit Vlixes \\ Et tutos solita praestitit arte suos. \\ Inlevit cera sociorum callidus aures \\ Atque suas vinclis praebuit ipse manus. \\ Transiluit scopulos et inhospita litora classis: \\ llae praecipites desiluere freto. \\ Sic blandas voces nocituraque carmina vicit, \\ Sic tandem exitio monstra canora dedit. \\ 
      \end{verse}
  
            \subsection*{638}
      \begin{verse}
      \poemtitle{IVLIANI}B. V 132 \\ M. 531 \\ \poemtitle{De die nutali}Clarus inoffenso procedat lumine Titan \\ Laetificusque dies eat omnibus aethere pure, \\ Vosque simul, iuvenes, animis ac voce faventes \\ Concelebrate diem votis felicibus almum, \\ Prosperus ut semper redeat vatique quotannis \\ Asmenidae referant alacres sua munera nati. \\ 
        \pagebreak 
     \marginpar{[105]} \begin{center} \textbf{C. 639, 1—2} \end{center}
      \end{verse}
  
            \subsection*{639}
      \begin{verse}
      \poemtitle{ \lbrack AVSONII}B. V 88 M. 1052. \\ ed. Petper \\ Tol.3r.Monostiehademensbusd.Schenxp.10. \\ Primus Rlomanas ordiris, Iane, lxalendas. \\ Februa vicino mense Numa instituit. \\ 
        \pagebreak 
    \begin{center} \textbf{C. 639. 2. C. 640, 1—3} \end{center} \marginpar{[106]} Martius antiqui primordia protulit anni. \\ Fetiferum Aprilem vindicat alma Venus. \\ Maiorum dictus patrum de nomine Mains. \\ Iunius aetatis proximus est titulo. \\ Nomine Caesareo Quintilem Ilulius auget. \\ Augustus nomen Caesareum sequitur. \\ Autumnum, Pomona, tuum September opimat. \\ Triticeo October fenore ditat agros. \\ Sidera praecipitas pelago, intempeste November. \\ Tu genialem hiemem, feste December, agis. \\ 
      \end{verse}
  
            \subsection*{640}
      \begin{verse}
      B. V 5. M. 1051. \\ B. P. p. 102. \\ \poemtitle{ \lbrack AvSONII}Sch. p. 13. \\ In quo mense quod signum sit ad \\ eurum olis \\ E ol. 4 r. \\ Principium lani sancit tropicus Capricornus. \\ Mense Numae (in medio solidi stat sidus quari. \\ Procedunt duplices in Martia tempora Pisces. \\ 
        \pagebreak 
    \begin{center} \textbf{C. 640, 12. C. 641, 1—2.} \end{center} \marginpar{[107]} Respicis Apriles, Aries Phrixee, lxalendas. \\ Maius Agenorei miratur cornua Tauri. \\ Iunius aequatos caelo videt ire Laconas. \\ Solstitio ardentis Cancri fert Hulius astrum. \\ Auustum mensem Leo fervidus ine perurit. \\ Sidere, Virgo, tuo Bacchum September opimat. \\ 0 Aequat et October sementis tempore ibram. \\ Scorpios hibernum praeceps iubet ire Novembrem. \\ Terminat Arquitenens medio sua signa Decembri. \\ 
      \end{verse}
  
            \subsection*{64}
      \begin{verse}
      B. I 4B. M. 589. \\ B. Petper 106. \\ \poemtitle{AVSONII)}. Schonxd \\ E fol. 4 . Monosticha de aerumnis Ierculis \\ Prima Cleonaei tolerata aerumna leonis. \\ Proxima Lernaeam ferro et face contudit hydram. \\ 
        \pagebreak 
    \begin{center} \textbf{C. 641, 31. C. 642,. 19} \end{center} \marginpar{[108]} Mox Erymantheum vis tertia perculit aprum. \\ Aeripedis quarto tulit aurea cornua cervi. \\ Stymphalidas pepulit volucres discrimine quinto. \\ Threiciam sexto spoliavit Amazona balteo. \\ Septima in Augei stabulis inpensa laboris. \\ Octava expulso numeratur adoria tauro. \\ In Diomedeis victoria nona quadrigis. \\ Geryone extincto decimam dat Hiberia palmam. \\ Vndecimo mala Hesperidum destricta triumpho. \\ Cerberus extremi suprema est meta laboris. \\ 
      \end{verse}
  
            \subsection*{642}
      \begin{verse}
      B. V. 41. M. 66. \\ B. frm p. . 315. \\ P. . 107. \\ \poemtitle{QVINTI CICERONIS}rr rum \\ Seh. p. 16. \\ Hi versu eo pertinent. ut. quod signum quo tempoe \\ illustre sit, noverimus; quod superius quoque nostris \\ versibus expeditur. \\ T ol. 4 c. \\ Flumina verna cient obscuro lumine Pisces \\ Curriculumque Aries aequat noctisque dieique, \\ Cornua quem condunt, florum praenuntia, Tauri. \\ Aridaque aestatis Gemini primordia pandunt \\ Longaque iam minuit praeclarus lumina Cancer \\ Languificosque Leo protlat ferus ore calores. \\ Post modium quatiens Virgo fugat orta vaporem. \\ Autumni reserat portas aequatque diurna \\ Tempora nocturnis dispenso sidere ibra. \\ 
        \pagebreak 
    \begin{center} \textbf{C. 642. 100. C. 643} \end{center} \marginpar{[109]} o Efetos ramos denudat flamma Nepai. \\ Pigra Sagittipotens iaculatur frigora terris. \\ Bruma gelu glacians iubar est spirans Capricorni, \\ Quam sequitur nebulas rorans liquor altus Aquari. \\ Tanta supra circaque vigent cum lumina mundi, \\ A dextra laevaque ciet rota fulgida Solis. \\ Mobile curriculum, et Lunae simulacra feruntur. \\ Squama sub aeterno conspectu torta Draconis \\ Eminet; hunc infra fulgentes Arcera septem \\ Magna quatit stellas, quam servans serus in alta \\ 0 Conditur Oceani ripa cum luce Bootes. \\ 
      \end{verse}
  
            \subsection*{643}
      \begin{verse}
      B. M. B. \\ 2. 108. eb. 17. \\ Hie verus ine uetore e. \\ Eol. , uo die quid demi de corpore oporteat \\ Vngues Mercurio, barbam love, Cypride crines. \\ 
        \pagebreak 
     \marginpar{[110]} \begin{center} \textbf{C. 644, 1—15} \end{center}
      \end{verse}
  
            \subsection*{644}
      \begin{verse}
      \poemtitle{ \lbrack AVSONII}B. V 141. \\ M. 111. B. \\ \poemtitle{De viro bono}. 0. Scb. 149. \\ ayopvi iavooc \\ E ol. 15 o. \\ Vir bonus et sapiens, qualem vix repperit unum \\ Milibus e cunctis hominum consultus Apollo, \\ Iudex ipse sui totum se explorat ad unguem. \\ Quid proceres vanique levis quid opinio volgi \\ Securus, mundi instar habens, teres atque rotundus, \\ Externae ne quid labis per levia sidat. \\ Ille diem, quam longus erit sub sidere Cancri \\ Quantaque nox tropico se porrigit in Capricorno, \\ Cogitat et iusto trutinae se examine pendit, \\ Ne quid hiet, ne quid protuberet, anulus aequis \\ Partibus ut coeat, nil ut deliret amussis; \\ Sit solidum quodcumque subest, nec inania subter \\ Indicet admotus digitis pellentibus ictus; \\ Non prius in dulcem declinans lumina somnum, \\ Omnia quam longi reputaverit acta diei: \\ 
        \pagebreak 
    \begin{center} \textbf{C. 644, 1326. C. 645, 15} \end{center} \marginpar{[111]} ‘Qua praetergressus, quid gestum in tempore, quid non? \\ Cur isti facto decus afuit aut ratio illi? \\ Quid mihi praeteritum? cur haec sententia sedit, \\ Quam melius mutare fuit? miseratus egentem \\ 0 Cur aliquem fracta persensi mente dolorem? \\ Quid volui quod nolle bonum foret? utile honesto \\ Cur malus antetuli? num dicto aut denique voltu \\ Perstrictus quisquam? cur me natura magis quam \\ Disciplina trahit’ sic dicta et facta per omnia \\ Ingrediens ortoque a vespere cuncta revolvens \\ Ofensus pravis dat palmam et praemia rectis. \\ 
      \end{verse}
  
            \subsection*{645}
      \begin{verse}
      B. V 139. \\ \poemtitle{ \lbrack AVSONII)}M. 285. B. \\ P. 9. SelM. 150. \\ ot. 16r. Val al Oc. Heyopi. \\ Est et Non cuncti monosyllaba nota frequentant. \\ His demptis nil est, hominum quod sermo volutet. \\ Omnia in his et ab his sunt omnia, sive negoti \\ Sive oti quicquam est, seu turbida sive quieta. \\ Alterutro pariter non numqua, saepe seorsis \\ 
        \pagebreak 
     \marginpar{[112]} \begin{center} \textbf{C. 64, 6 25} \end{center}Obsistunt studiis, ut mores ingeniumque, \\ Et facilis vel difficilis contentio nata est. \\ Si consentitur, mora nulla, intervenit ‘Est, est’; \\ Sin controversum, dissensio subiciet ‘Non’. \\ Hinc fora dissultant clamoribus, hinc furiosi \\ Iurgia sunt circi, cuneati hinc †laeta theatri \\ Seditio, et tales agitat quoque curia lites. \\ Coniugia et nati cum patribus ista quietis \\ Verba serunt studiis salva pietate loquentes. \\ Hinc etiam placitis schola consona disciplinis \\ Dogmaticas agitat †placido certamine lites. \\ Hinc omnis certat dialectica turba sophorum: \\ ‘Si lux est, est ergo dies? non convenit istic. \\ Nam facibus multis aut fulguribus quotiens lux \\ Est nocturna homini, non est lux ista diei.’ \\ Est et Non igitur, quotiens lucem ‘esse’ flatendum est, \\ Sed ‘non esse’ diem. mille hinc certamina surgunt; \\ Hinc pauci, multi quoque, talia commeditantes \\ Murmure concluso rabiosa silentia rodunt. \\ Qualis vita hominum, duo quam monosyllaba versant! \\ 
        \pagebreak 
    \begin{center} \textbf{C. 646, 1—20} \end{center}\begin{center} \textbf{1 1} \end{center}
      \end{verse}
  
            \subsection*{646}
      \begin{verse}
      B. III 292. \\ M 102B. B. \\ . 19. Sebh. 243. \\ \poemtitle{De rosis nasecentibus}Ver erat et blando mordentia frigora sensu \\ Spirabat cruoceo mane revecta dies. \\ Strictior Eoos praecesserat aura iugales, \\ Aestiferum suadens anticipare diem. \\ Errabam riguis per quadrua compita in hortis, \\ Maturo cupiens me vegetare die. \\ Vidi concretas per gramina lexa pruinas \\ Pendere aut olerum stare cacuminibus, \\ Caulibus et patulis teretes conludere guttas \\ Vidi Paestano gaudere rosaria cultu \\ Exoriente novo roscida Lucifero. \\ lara pruinosis canebat gemma frutectis \\ Ad primi radios interitura die. \\ Ambigeres, raperetne rosis Aurora ruborem \\ An daret et flores tingueret orta dies. \\ Ros unus, color unus, ct unum mane duorum: \\ Sideris et floris nam domina una Venus. \\ Forsan et unus odor: sed celsior ille per auras \\ Diflatur, spirat proximus iste magis. \\ 
        \pagebreak 
    \begin{center} \textbf{C. 640, 21 46} \end{center} \marginpar{[114]} CommunisPaphiedeasiderisetdeafloris \\ Praecipit, unius muricis esse habitum. \\ Momentum intererat, quo se nascentia florum \\ Germina comparibus dividerent spatiis. \\ Haec viret angusto foliorum tecta galero, \\ IHanc tenui folio purpura rubra notat, \\ Haec aperit primi fastigia celsa obelisci, \\ Mucronem absolvens purpurei capitis. \\ Vertice collectos illa exsinuabat amictus, \\ Iam meditans foliis se numerare suis. \\ Nec mora: ridentis calathi patefecit honorem \\ Prodens inclusi semina densa croci. \\ Haec, modo quae toto rutilaverat igne comarum, \\ Pallida conlapsis deseritur foliis. \\ Mirabar celerem fugitiva aetate rapinam, \\ Et, dum nascuntur, consenuisse rosas. \\ Ecce et defluxit rutili coma punica loris, \\ Dum loquor, et tellus tecta rubore micat. \\ Tot species tantosque ortus variosque novatus \\ Vna dies aperit, conficit una dies. \\ Conquerimur, Natura, brevis quod gratia  \lbrack talis \rbrack : \\ Ostentata oculis ilico dona rapis. \\ Quam longa una dies, aetas tam longa rosarum: \\ Cum pubescenti iuncta senecta brevis. \\ Quam modo nascentem rutilus conspexit Eous, \\ Hanc rediens sero vespere vidit anum. \\ 
        \pagebreak 
    \begin{center} \textbf{C. 646, 1750. C. 647, 110} \end{center} \marginpar{[115]} Sod bene quod, paucis licet interitura diebus, \\ Succedens aevum prorogat ipsa suum. \\ Collige,virgo,rosas,dumlosnovusetnovapubes, \\ Et memor esto, aevum sic properare tuum. \\ 
      \end{verse}
  
            \subsection*{64}
      \begin{verse}
      B. V. 142. M. 1078. \\ \poemtitle{ \lbrack AVSONI \rbrack }V \\ B. P. p. 93. \\ 8en. p. 152. \\ \poemtitle{De aetatibus. Hesiodion.}E ol. 14 e. \\ Ter binos deciesque novem super exit in annos \\ \poemtitle{....}Iusta senescentum quos implet vita virorum. \\ Hos novies superat vivendo garrula cornix \\ Et quater egreditur cornicis saecula cervus. \\ Alipedem cervum ter vincit corvus, et illum \\ Multiplicat novies Phoenix, reparabilis ales. \\ Quem nos perpetuo decies praevertimus aevo \\ Nymphae Hamadryades, quarum longissima vita est. \\ Haec cohibet finis vivacia fata animantum. \\ Cetera secreti novit deus arbiter aevi: \\ 
        \pagebreak 
    \begin{center} \textbf{C. 647. 1117. C. 648, 112} \end{center} \marginpar{[116]} Tempora quae Stilbon volvat, quae saecula Phaenon, \\ Quos Pyrois habeat, quos luppiter igne benigno \\ Circuitus, quali properet Venus alma recursu, \\ Qui Phoeben, quanti maneant Titana labores, \\ Donec consumpto, magnus qui dicitur, anuo \\ Rursus in anticum veniant vaga sidera cursum, \\ Quali dispositi steterant ab origine mundi. \\ B. III 97. M. 532 \\ 
      \end{verse}
  
            \subsection*{648}
      \begin{verse}
      B. IV 107. \\ \poemtitle{SVLPICII LVPERCI SERVASII IVNI0RIS}\poemtitle{V I.IDI3II VI1}r \\ \poemtitle{I IIV}II \\ E7o.37 c. \\ Omne quod Natura parens creavit, \\ Quamlibet firmum videas, labascit, \\ Tempore ac longo fragile et caducum \\ Slvitur usu. \\ Amnis insueta solet ire valle \\ Mutat et rectos via certa cursus, \\ Rupta cum cedit male pertinaci \\ ipa fluento. \\ Decidens scabrum cavat unda tofum, \\ Ferreus vomis tenuatur agris, \\ Splendet adtrito digitos honoraus \\ Anulus auro. \\ 
        \pagebreak 
    \begin{center} \textbf{C. 649. 123} \end{center} \marginpar{[117]} 
      \end{verse}
  
            \subsection*{649}
      \begin{verse}
      VmI7 \\ AEL E31 \\ B. III 7. M. 543. \\ \poemtitle{De cupiditate}E ol. 37 e. \\ B. IV 107. \\ Heu misera in nimios hominum petulantia census! \\ Caecus inutilium quo ruit ardor opum, \\ Auri dira fames et non expleta libido, \\ Ferali pretio vendat ut omne nefas! \\ iclatebrasEriphylaviripatefecit,ubiaurum \\ Accepit, turpis materiam sceleris; \\ SicquondamAcrisiaeingremiumperclaustrapuellae \\ Corruptore auro fluxit adulterium. \\ quam mendose votum insaturabile habendi \\ Inbuit infami pectora nostra malo! \\ Quamlibet inmenso dives vigil incubet auro, \\ Aestnat augendae dira cupido rei. \\ Henu mala paupertas nunquam locupletis avari: \\ Dum struere inmodice quod tenet optat, eget. \\ Quis metus hic legum quaeve est reverentia ver, \\ Crescenti nummo si mage cura subest? \\ Cognatorum animas promtum est lratrumque cruorem \\ Fundier: afectus vincit avara fames. \\ Divitis est, semper lragiles male quaerere gazas: \\ Nulla huic in lucro cura pudoris erit. \\ stud templorum damno excidioque requirit; \\ Hoc caelo iubeas ut petat: inde petet. \\ Mirum ni pulcras artes omana iuvenutus \\ 
        \pagebreak 
     \marginpar{[118]} \begin{center} \textbf{C. 649, 24—42} \end{center}Discat et egregio sudet in eloquio, \\ 
      \end{verse}
  
            \subsection*{}
      \begin{verse}
      \poemtitle{V1}post iurisonae famosa stipendia linguae \\ Barbaricae ingeniis anteferantur opes? \\ qui sunt, quos propter honestum rumpere foedus \\ 
      \end{verse}
  
            \subsection*{}
      \begin{verse}
      \poemtitle{A1}Audeat inlicite pallida avaritia? \\ Bomani sermonis egent ridendaque verba \\ Frangit ad horriticos turbida lingua sonos. \\ Sed tamen ex cultu adpetitur spes grata nepotum? \\ Saltem istud nostri forsan honoris habent? \\ Ambusti torris species exesaque saeclo \\ btantur priscis corpora de tumulis! \\ Perplexi crines, frons improba, tempora pressa, \\ Extantes malae deficiente gena, \\ Simataeque iacent pando sinuamine nares, \\ Territat os nudum caesaque labra tument. \\ Defossum in ventrem propulso pondere tergum \\ Frangitur et vacuo crure tument genua. \\ Discolor in manibus species, ac turpius illud, \\ Quod cutis obscure pallet in invidiam. \\ 
        \pagebreak 
     \marginpar{[119]} \begin{center} \textbf{C. 650. C. 651, 18} \end{center}B. III 131. \\ 
      \end{verse}
  
            \subsection*{650}
      \begin{verse}
      M. 180. B. IV 109. \\ Petron. od. Bu \\ \poemtitle{ETRONII}E ol. 38 r. \\ cheler f. 29. \\ Fallunt nos oculi vagique sensus \\ Oppressa ratione mentiuntur. \\ Nam turris, prope quae quadrata snrgit, \\ Detritis procul angulis rotatur. \\ Hyblaeum refugit satur liquorem \\ Et naris casiam frequenter odit. \\ Hoc illo magis aut minus placere \\ Non posset, nisi lite destinata \\ Pugnarent dubio tenore sensus. \\ 
      \end{verse}
  
            \subsection*{651}
      \begin{verse}
      \poemtitle{EIVSDEM}B. VI 0 \\ M. 170. B. IV 110 \\ \poemtitle{De somniis}E ol. 38 r. \\ Petron.ed.l3.fg.3t \\ Somnia, quae mentes ludunt volitantibus umbris, \\ Non delubra deum nec ab aethere numina mittunt, \\ Sed sibi quisque facit. nam cum prostrata sopore \\ Vrguet membra quies et mens sine pondere ludit, \\ Quidquid luce fuit, tenebris agit. oppida bello \\ Qui quatit et flammis miserandas eruit urbes, \\ Tela videt versasque acies et funera regum \\ Atque exundantes profuso sanguine campos. \\ 
        \pagebreak 
    \begin{center} \textbf{C. 651, 16. C. 652} \end{center} \marginpar{[120]} Qui causas orare solent, legesque forumque \\ Et pavida cernunt inclusum chorte tribunal. \\ Condit avarus opes defossumque invenit aurum. \\ Venator saltus canibus quatit. eripit undis \\ Aut premit eversam periturus navita puppem. \\ Scribit amatori meretrix. dat adultera munus. \\ Et canis in somnis leporis vestigia lustrat. \\ In noctis spatium miserorum vulnera durant. \\ B. nd VI 90. \\ M. B. Clu \\ 
      \end{verse}
  
            \subsection*{}
      \begin{verse}
      \poemtitle{CLAVDIANI}V \\ dian. ed. ir. \\ 234. \\ 
      \end{verse}
  
            \subsection*{}
      \begin{verse}
      \poemtitle{De eadem re}E olt. 3 . \\ Omnia quae sensu volvuntur vota diurno, \\ Pectore sopito reddit amica quies. \\ Venator defessa toro cum membra reponit, \\ Mens tamen ad silvas et sua lustra redit. \\ Indicibus lites, aurigae somnia currus, \\ Vanaque nocturnis meta cavetur equis. \\ Furto gaudet amans, permutat navita merces, \\ Et vigil elapsas quaerit avarus opes. \\ Blandaque largitur frustra sitientibus aegris \\ Inriguns gelido pocula fonte sopor. \\ 
        \pagebreak 
    \begin{center} \textbf{C. 65 , 19} \end{center}\begin{center} \textbf{10 1} \end{center}
      \end{verse}
  
            \subsection*{653}
      \begin{verse}
      \poemtitle{SVLPICII CATIAGNIENSIS E .39e.}Hexasticha in Aeneidis libris. \\ B. II 175. . 286. \\ Praefatio \\ B. IV 169. \\ Carmina Vergilius Phrygium prodentia Martem \\ Secum fatali iusserat igne mori. \\ Tucca negat, Varius prohibet, superaddite Caesar \\ Nomen in Aenea non sinis esse nefas. \\ O quam paene iterum geminasti funere funus, \\ Troia, bis interitus causa futura tui. \\ B. II I9. M. 223. \\ Arma virumque canlit vates lunonis ob iram \\ Et totum Aeoliis turbatum flatibus aequor, \\ Disiectas classes submersaque corpora ponto, \\ 
        \pagebreak 
    \begin{center} \textbf{C. 653, 10—28} \end{center} \marginpar{[122]} HIospitium Didus, casus cogentis et annos \\ Aenean Troiae fatumque et bella referre. \\ Quorum pars terra, pelago pars addita famae est. \\ Conticuere omnes intentique or†e loquentis \\ Ora tenent. ac tum dolus introducitur hostis \\ Et fallacis equi damnosum munus, inermis \\ Perfidia notusque Sinon, amissaque coniunx, \\ Per medios ignes ablatus saevaque tela \\ Anchises umero facilis pietate ferentis. \\ Postquam res Asiae \rbrack  disiectaque moenia Troiae, \\ Dant pelago classes. Polydori funera discunt \\ Atque Heleni praecepta, tfidem cui fecerat usus, \\ Quae non accessu via sit temptanda: Cyclopes \\ Et Scyllae rabies et iueluctabilis unda \\ Erroresque maris; labor in contraria versus. \\ Iv \\ At regina gravi \rbrack  pectus succensa dolore \\ Ardet amore viri. clausum †‘veneratur amorem \\ Dumque capit, capitur: sentit, quos praebuit, ignes. \\ Aeneas altum sociis et classe petivit. \\ 
        \pagebreak 
    \begin{center} \textbf{C. 653, 2946} \end{center}\begin{center} \textbf{1 0} \end{center}Extructa regina pyra penetralibus instat \\ Morte fugam †praestare morae nec defuit hora. \\ Interea medium Aeneas \rbrack  tendebat in aequor, \\ Moenia respiciens causa flagrantia amoris. \\ Mo Siculae tenet arva domus manesque parentis \\ Ludorum exequiis celebrat. quibus additur Iris \\ Hn faciem Beroes classem flammare iubentis \\ Iunonis mutata dolo: namque illa monebat. \\ 
      \end{verse}
  
            \subsection*{VI}
      \begin{verse}
      Sic fatur lacrimans. \rbrack  Cumarum adlahitur oris, \\ Descensusque parans adiit praecepta Sibyllae, \\ Qua duce non fastum mortali limen aditur. \\ 0 Hic primum maestos videt inter cetera Troas. \\ Tum patrem agnoscit, discit reditura snb ortus \\ Corpora lomanosque duces seriemque nepotum. \\ vmI \\ Tu quoque litoribus nostris, \rbrack  Caieta, manebis. \\ Servat honos nomen pietas testatur honorem. \\ Causam opus insequitur; bellum namque incipit esse, \\ Tyrrhidae iuvenum quod tunc conflaverat ira. \\ 
        \pagebreak 
    \begin{center} \textbf{C 65B. 47—62} \end{center} \marginpar{[124]} o Turnus adest, monet, arma sibi contraria sumat. \\ r. Tum gentes socia arma ferunt, fremit arma iuventus. \\ Vt belli signum \rbrack  cecinit, sociosque vocavit \\ Turnus et in varias turbatus pectora mentes \\ Aeneas Euandron adit; facit hospitis usum \\ Atque operum causas, urbis cognoscit honores. \\ Arma virum Cytherea rogat, mox accipit heros. \\ Tuque opere ars piget, facti lavor efficit artem.’ \\ Atque ea diversa penitus dum parte geruntur, \\ Iri \rbrack s adest, monet, arma sibi contraria sumat. \\ Caeduntur vigiles et mutua corpora fratrum. \\ Nisus et Euryalus morte et pietate tideles. \\ Dumque petunt laudem, vincunt contraria fata. \\ Audacem Remulum leto dat pnlcher Iulus. \\ Panditur interea \lbrack  caelum sedesque Tonantis; \\ Alternos questus Venus ac Saturnia promunt. \\ 
        \pagebreak 
    \begin{center} \textbf{C. 65. 6378. C. 654, 12} \end{center} \marginpar{[125]} lla dolos, haec bella movet, sed vincitur ira. \\ Mater at Aeneas bello non segnior instat. \\ Vulnere Meaenti Lauso nam fecerat iram. \\ Quem tetigit virtus, superat per fata periclum. \\ Oceanum interea \rbrack  Phoebus superaverat ortu. \\ Extruit Aeneas dextra quaesita tropaea. \\ Condit humi socios, fatum quos condidit ante. \\ r0 Legati responsa ferunt, veniamque petitam \\ Non negat. et contra pugnat secura Camilla, \\ Femina caede potens, casu temeraria tanto. \\ Turnus ut infractos \rbrack  vidit cessisse Latinos, \\ Instat atrox punaelque vices sibi percipit hostis. \\ Pectore secreto violatur vulnere teli \\ Aeneas. causa est illi luturna; nec ante \\ dfuit obsequio, quam mors finiverat iram. \\ Sed sua fata virum traxerunt quaerere mortem. \\ 
      \end{verse}
  
            \subsection*{654}
      \begin{verse}
      Tetrastieha in libris Vergilii n. I 3. 5 . \\ M. 86B. 5 . \\ In Georgicis \\ 
      \end{verse}
  
            \subsection*{7}
      \begin{verse}
      B. IV 173. \\ Sidera deinde canit, segetes et dona Lyaei \\ Et pecorum cultus, lyblaei mella saporis. \\ 
        \pagebreak 
    \begin{center} \textbf{C. 64, 320} \end{center} \marginpar{[126]} Principio breviter ventura volumina dixit. \\ Intercidit opus coepitque referre secunda. \\ In Aeneide. \\ Arma virumque canlit mira virtute potentem, \\ Iunonis \lbrack que \rbrack  odio disiectas aequore puppes, \\ lospitium Didus, classem sociosque receptos, \\ Vtque epulas inter casus regina requirat. \\ Conticuere omnes. Inlfandos ille labores \\ Deceptamque dolis Troiam patriaeque ruinas \\ Et casus Priami docet et flagrantia regna, \\ Ignibus e mediis raptum deque hoste parentem. \\ Postquam res Asiae \rbrack  deceptaque Pergama dixit, \\ Tum, Polydore, tuos tumulos, tum Gnosia regna, \\ Andromachen lelenumque et vasta mole Cyclopas \\ Amissumque patrem Siculis narravit in oris. \\ At regina gravi Veneris iam carpitur igni \rbrack  \\ Venatusque petit. capitur venatibus ipsa \\ Et taedas, lymenaee, tuas ad funera vertit, \\ Postquam Anchisiades fatorum est iussa secutus. \\ 
        \pagebreak 
    \begin{center} \textbf{C. 654, 2138} \end{center} \marginpar{[127]} Interea Aeneas pelagus iam classe tenebat \rbrack  \\ Ludorumque patris tumulum celebrabat honore. \\ Puppibus ambustis fundavit moenia Acestae \\ Destititque ratem media Palinurus in unda. \\ Sic lacrimans tandem Cumarum adlabitur oris \rbrack  \\ ol. \\ Descenditque domus Ditis comitante Sibylla. \\ 40 . \\ Agnoscit Troas caesos, agnoscit Acbivos, \\ Et docet Anchises venturam ad sidera prolem. \\ 
      \end{verse}
  
            \subsection*{VI}
      \begin{verse}
      Tu quoque litoribus famam, Caieta, dedisti. \\ o lmpetrat Aeneas Latium regnumque Latini, \\ Foedus agens. saevit luno bellumque lacessit \\ Finitimosque viros Turnumque in proelia mittit. \\ 
      \end{verse}
  
            \subsection*{VIII}
      \begin{verse}
      Vt belli signum Turnus \rbrack  Meentiaque arma \\ Concivitque duces, tum moenia Pallantea \\ s5 Aeneas adit Euandri socia agmina quaerens. \\ Arma Venus portat proprio Vulcania nato. \\ \poemtitle{vIIII}Atque ea diversa dum parte. \rbrack  hic diva Cybebe \\ Puppes esse suas Nympharum numina iussit. \\ 
        \pagebreak 
    \begin{center} \textbf{C. 654, 39 52. C. 655} \end{center} \marginpar{[128]} Euryali et Nisi caedes et fata canuntur, \\ Fecerit inclusus castris quae funera Turnus. \\ Panditur interea \rbrack  caelum coetusque deorum. \\ Iam redit Aeneas et Pallas sternitur acer. \\ Eripuit luno Turnum, Lausoque parentem \\ Adiecit comitem mortis Cythereia proles. \\ Oceano interea surgens Aurora \rbrack  videbat \\ Meenti ducis exuvias caesosque sodales \\ Et lLatium proceres Diomedis dicta referre, \\ Tum qualis pugnae succedat Etrusca Camilla. \\ Turnus ut infractos \rbrack   \lbrack hos \rbrack  vidit et undique caesos, \\ Vltro Anchisiaden bello per foedera poscit; \\ Quae luturna parat convellere; set tamen armis \\ Occidit et pactum liquit cum coniuge regnum. \\ 
      \end{verse}
  
            \subsection*{655}
      \begin{verse}
      piramma \\ B. II 185. \\ M. 289. \\ \poemtitle{AVGVTI CAESARIS}ct. B. IV 179. \\ Nescio quid fugiente anima, non sponte sed altis sqq. \\ 
        \pagebreak 
    \begin{center} \textbf{C. 656 657c} \end{center} \marginpar{[129]} Aenigmata. \\ B. M. B. \\ 
      \end{verse}
  
            \subsection*{656}
      \begin{verse}
      Si me retro legis, potui quae vivere numquam, \\ Continuo vivam, sumens de nomine vitam. \\ 
      \end{verse}
  
            \subsection*{657}
      \begin{verse}
      me retro legis, faciam de nomine verbum. \\ Femina cum fuerim, imperativus ero. \\ e \\ me retro legis, dicam tibi semper id ipsum; \\ Vna mihi facies ante retroque manet. \\ 
      \end{verse}
  
            \subsection*{6572}
      \begin{verse}
      Mollior in tactu, sed durior omnibus actu \\ Ille ego qui rabiem possum superare ferinam. \\ 
      \end{verse}
  
            \subsection*{657}
      \begin{verse}
      me retro legis, facere qui vulnera novi, \\ Ex me confestim noscis adesse deum. \\ 
        \pagebreak 
    \begin{center} \textbf{C. 658, 1—12} \end{center} \marginpar{[130]} 
      \end{verse}
  
            \subsection*{658}
      \begin{verse}
      B. V 149. M. 39. \\ B. V 36. Me \\ Anct. nntiquis. \\ \poemtitle{De iloel}XIV 253. \\ Distichon filomelaicum \\ Sum noctis socia, sum cantus dulcis amica: \\ Nomen ab ambiguo sic filomela gero. \\ Item \\ Insomnem filomela trahit dum carmine noctem, \\ Nos dormire facit, se vigilare docet. \\ tem dialogon tetrastichon \\ Dic, filomela, velis cur noctem vincere cantu? \\ Ve noceat ovis vis inimica meis.’ \\ Dic age, num cantu poteris depellere pestem? \\ ‘Aut possim aut nequeam, me vigilare iuvat.’ \\ Item carmen filomelaicum \\ Vox, filomela, tua cantus edicere cogit, \\ Inde tui laudem rustica lingua canit. \\ Vox, filomela, tua citharas in carmine vincit \\ Et superat miris musica flabra modis. \\ 
        \pagebreak 
    \begin{center} \textbf{C. 6, 1328} \end{center} \marginpar{[1]} Vo, filomela, tua curarum semina pellit, \\ Recreat et blandis anxia corda sonis’ \\ lorea rura colis, herboso cespite gaudes, \\ Frondibus arboreis pignera parva foves. \\ Cantibus ecce tuis recrepant arbusta canoris, \\ Consonat ipsa suis frondea silva comis. \\ 777. \\ 2 Cedat et inlustri psittacus ore tibi, \\ Nulla tuos umquam cantus imitabitur ales, \\ Murmure namque tuo dulcia mella fluunt. \\ Dic ergo tremulos lingua vibrante susnrros \\ Et suavi liquidum gutture pange melos. \\ Porrige dulcisonas adtentis auribus escas; \\ Nolo tacere velis, nolo tacere velis! \\ Gloria summa tibi, laus et benedictio, Christe, \\ Qui praestas famulis haec bona grata tuis. \\ 
        \pagebreak 
    \begin{center} \textbf{C. 659 661,1—4} \end{center} \marginpar{[132]} 
      \end{verse}
  
            \subsection*{659}
      \begin{verse}
      V de nunc c. \\ 
      \end{verse}
  
            \subsection*{660}
      \begin{verse}
      B IM 118. M. 209 3. \\ AL epir. 427. \\ In memoria cuiusdam militi \\ lle ego Pannoniis quondam notissimus oris, \\ Inter mille viros primus fortisque Batavos, \\ Hadriano potui qui iudice vasta profundi \\ Aequora Danuvii cunctis transnare sub armis, \\ Emissumque arcu dum pendet in aere telum \\ Ac redit, ex alia tixi fregique sagitta; \\ Quem neque lomanus potuit nec barbarus umquam, \\ Non iaculo miles, non arcu vincere Parthus, \\ lic situs hoc memori saxo mea facta sacravi. \\ Viderit, anne aliquis post me mea gesta sequatur! \\ Exemplo mihi sum, primus qui talia gessi. \\ 
      \end{verse}
  
            \subsection*{661}
      \begin{verse}
      B. M. \\ Iugduni in memoria eminorum \\ Hic gemini fratres iuncti dant membra sepulchris, \\ Quos iunxit meritum, consociavit humus. \\ Germine barharico nati, sed fonte renati \\ Dant animas caelo, dant sua membra solo. \\ 
        \pagebreak 
     \marginpar{[133]} \begin{center} \textbf{C. 66t, 8. C. 662 663} \end{center}Advenit Sagilae patri cum coniuge luctus, \\ Defungi haud dubie qui voluere prius. \\ Sed dolor est nimius Christo moderante ferendus. \\ Orbati non sunt: dona dedere deo. \\ 
      \end{verse}
  
            \subsection*{662}
      \begin{verse}
      B . M. \\ In tumulo cuiudam medici \\ Praeteriens hominum sortem miserere, viator, \\ Deque meis, restent quae tibi fata, vide. \\ En mihi terra domnm praebet cinisque sepulchrum, \\ Vermis et exiguus membra caduca vorat. \\ Conditor omnipotens paradysi cum esse colonum \\ Iusserat, hanc tribuit culpa nefanda vicem. \\ Nomine Felicem me olim dixere parentes, \\ Vita dicata mihi hic, ars medicina fuit. \\ Aegros multorum poti relevare dolores, \\ Morbum non potui vincere ab arte menum. \\ B. II 265. M. 853. \\ 
      \end{verse}
  
            \subsection*{663}
      \begin{verse}
      B. IV 161. \\ Littera rem gestam loquitur; res ipsa medullam \\ Verbi, quam vivax mens videt, intus habet. \\ 
        \pagebreak 
    \begin{center} \textbf{C. 664} \end{center} \marginpar{[134]} 
      \end{verse}
  
            \subsection*{664}
      \begin{verse}
      (B. I 74) \\ \poemtitle{CATONIS}M. 618 (617. 619. \\ B III 4B. Auon. \\ Momina Musrum \\ ed. Peiporp. 4t2. \\  \lbrack Clio historias, Euterpe tibias, Thalia comoedias, Mel \\ pomene tragoedias, Terpsichore psalterium, Erato geo \\ metricam, Polymnia rhetoricam, Vrania astrologiam, \\ Calliope litteras. \rbrack  \\ Clio gesta canens transactis tempora reddit. \\ Dulciloquis calamos Euterpe flatibus urget. \\ Comica lascivo gaudet sermoue Thalia. \\ Melpomene traico proclamat maesta boatu. \\ Terpsichore affectus citharis movet imperat auget. \\ Plectra gerens Erato saltat pede carmine vultu. \\ Signat cuncta manu loquiturque Polymia gestu. \\ Erania poli motus scrutatur et astra. \\ Carmina Calliope libris heroica mandat. \\ Mentis Apollineae is has movet undique Musas. \\ In medio residens complectitur omnia Phoebus. \\ 
        \pagebreak 
    \begin{center} \textbf{C. 664. C. 665, 18} \end{center} \marginpar{[135]} 
      \end{verse}
  
            \subsection*{664a}
      \begin{verse}
      \poemtitle{IB. M. B.}Ineipiunt versus earundem usarum \\ Historias primo rerum canit ordine Clio. \\ Dulce melos calamis Euterpe datqne secunda. \\ Tertia Melpomene traicos fert flendo boatus. \\ Quartaque comedis dat lusum fando Thalia. \\ Rhetoricos profert at quinta Polimnia sensus. \\ Sexta canens Erato geometras carmine pangit. \\ Organa Terpsicore . dat septima cunctis. \\ Drania polos octavo limine scandit. \\ Poemate Calliope perlustrat nona libellos. \\ 
      \end{verse}
  
            \subsection*{66}
      \begin{verse}
      B. M. \\ B. I 210. \\ Monostieha de mensius \\ Primus, lane, tibi sacratur nomine mensis, \\ Vndiqne cui semper cuncta videre licet. \\ Vmbrarnm est alter, quo mense putatur honore \\ Pervia terra dato Manibus esse vagis. \\ Condita Mavortis magno sub numine lRoma \\ Non habet errorem: lomulus auctor erit. \\ Caesareae est Veneris mensis, quo floribus arva \\ Compta virent, avibus quo sonat omne nemus. \\ 
        \pagebreak 
     \marginpar{[136]} \begin{center} \textbf{C. 665, 92} \end{center}Hos sequitur laetus toto iam corpore Maius, \\ Mercurio et Maiae quem tribuisse iuvat. \\ Iunius ipse sui causam tibi nominis edit, \\ Praegravida attollens fertilitate sata. \\ Quam bene, Quintilis, mutasti nomen: honori \\ Caesareo, Iuli, te pia causa dedit. \\ Tu quoque, Sextilis, venerabilis omnibus annis \\ Numinis Augusti nomine notus eris. \\ Tempora maturis, September, vincta racemis \\ Vela tegant: numero nosceris ipse tuo. \\ Octobri laetus portat vindemitor uvas: \\ Omnis ager Bacchi munera voce sonat. \\ Frondibus amissis repetunt sua frigora mensem, \\ Cum iuga Centaurus celsa retorquet eques. \\ Argumenta tuis festis concludo, December, \\ Quis quemvis annum claudere  \lbrack iure potes. \\ 
        \pagebreak 
    \begin{center} \textbf{C. 666, 1—18} \end{center} \marginpar{[137]} Rescriptum lONORI scholstiei \\ Contra epistolas Seneeae \\ B. M. B. \\ Ad Iordanem episcopum \\ fontis brevis unda latens demersa tenetur, \\ Ignotae et viles esse putantur aquae. \\ Quas cum docta manus produxerit arte magistra, \\ Pura fit exiliens lympha vocata manu: \\ Tunc praegnantis humi laxantur viscera partu \\ Et subito sterilis flumina terra creat. \\ Non aliter validum genuino robore lignum \\ Imbutis digitis dextra domare solet, \\ Arboris et speciem humanis non usibus aptam \\ Cogit in externum crescere factor opus. \\ Sed cum te potior, Seneca meliore magistro \\ Quique monens lucem cordis habere facis, \\ Non dubitare queam, Lucillo clarius illo \\ Aeternas Christi sumere dantis opes. \\ Cedat opus priscum vera nec luce coruscans \\ Nec de catholici dogmatis ore fluens. \\ Ille mihi commenta dedit te vera docente; \\ Nec dedit infida quae sibi nente tulit. \\ 
        \pagebreak 
     \marginpar{[138]} \begin{center} \textbf{C. 666, 198. C. 667} \end{center}Nam cum de pretio mortis regnante perenni \\ ucillum inbueret, hac sine morte perit. \\ At tu cum doceas homines superesse beatos \\ Ex obitu Christum morte sequendo pia, \\ Erigis et Senecam dominus verusque magister \\ Ingeniis fidei me superare facis. \\ Vnde precor: Lucillum alium nec pectore talem, \\ Quae me nosse cupis, scire precando iube; \\ Discipulumque tuum, prius isto nomine ditans, \\ Conforta, revoca, corripe, duce, mone! \\ 
      \end{verse}
  
            \subsection*{667}
      \begin{verse}
      B. II 22s. M. 38. \\ B. V 386. \\ Epitaphium Senecae \\ Cura labor, meritumn, sumpti pro munere honores, \\ Ite, alias posthac sollicitate animas! \\ Me procul a vobis deus evocat. ilicet actis \\ Rebus terrenis, hospita terra, vale. \\ Corpus, avara, tamen sollemnibus accipe saxis: \\ Namque animam caelo reddimus, ossa tibi. \\ 
        \pagebreak 
    \begin{center} \textbf{C. 668 669} \end{center} \marginpar{[139]} 
      \end{verse}
  
            \subsection*{668}
      \begin{verse}
      B. II 229. M. 839 \\ B. V 386. \\ ucnu ed. H \\ Epitaphium Lucani \\ iu p. 338. \\ Corduba me genuit, rapuit Nero, praelia dixi, \\ Quaegesserepareshincsocer,indegener. \\ Continuonumquamdirexicarminaductu, \\ Quaeractimserpant:plusmihicommaplacet. \\ Fulminisinmorem,quaesintmiranda,citentur: \\ Haec vere sapiet dictio, quae feriet! \\ 
      \end{verse}
  
            \subsection*{669}
      \begin{verse}
      B. M B . \\ MGh Auct. anti \\ \poemtitle{DOMNI EVANTII}uis. XIV 253. \\ Nobilis et magno virtutum culmine cels, \\ Ingens consiliis et dextrae belliger actV, \\ Care mihi genitor et vita carior ips! \\ Hoc nati pietas offert post funera carmeN, \\ Oferre incolumi quod mors infanda vetaviT. \\ Lux tibi summa Dei nec non et gratia Christl \\ Adsit perpetuo nec desit temporis usV, \\ Omnipotensque tuis non reddat debita culpiS. \\ 
        \pagebreak 
     \marginpar{[140]} \begin{center} \textbf{C. 670} \end{center}
      \end{verse}
  
            \subsection*{670}
      \begin{verse}
      \poemtitle{BASSI}In tumulo Monicae \\ . M. 3. \\ Hic posuit cineres genetrix castissima prolis, \\ Augustine, tui altera lux meriti, \\ Qui servans pacis caelestia iura sacerdos \\ Conmissos populos moribus instituis. \\ Gloria vos maior gestorum laude coronat \\ Virtutum mater felicior subolis. \\ 
        \pagebreak 
    \begin{center} \textbf{C 671, 124} \end{center} \marginpar{[141]} 
      \end{verse}
  
            \subsection*{671}
      \begin{verse}
      \poemtitle{PM0CAE}gammatici urbis omae \\ Vca Vergilii. \\ B. II 186. M. 288. \\ Praefatio \\ B. V 85 \\ vetustatis memoranda custos, \\ Regios actus simul et fugaces \\ Temporum cursus docilis referre, \\ Aurea Clio, \\ Tu nihil magnum sinis interire, \\ Nil mori clarum pateris, reservans \\ Posteris prisci monumenta saecli \\ Condita libris. \\ Sola fucatis variare dictis \\ Paginas nescis, set aperta quicquid \\ Veritas prodit, recinis per aevum \\ Simplice lingua. \\ Tu senescentes titulos avorum \\ Flore durantis reparas iuventae. \\ Militat virtus tibi; te notante \\ Crimina pallent. \\ Tu fori turbas strepitusque litis \\ Efugis dulci moderata cantu, \\ Nec retardari pateris loquellas \\ Compede metri. \\ His fave dictis! retegenda vita est \\ Vatis Etrusci modo, qui perenne \\ Romulae voci decus adrogavit \\ Carmine sacro: \\ 
        \pagebreak 
    \begin{center} \textbf{C. 671, 2554} \end{center} \marginpar{[142]} V 7t V ergilii \\ Maeonii specimen vatis veneranda laronem \\ Mantua, omuleae generavit flumina linguae. \\ Quis, facunda, tuos toleraret, Graecia, fastus, \\ Quis tantum eloquii potuisset ferre tumorem, \\ Aemula Vergilinm tellus nisi Tusca dedisset? \\ Huic genitor figulus Maro nomine, cultor agelli, \\ Vt referunt alii, tenui mercede locatus, \\ Sed plures figulum. quis non miracula rerum \\ Haec stupeat? dives partus de paupere vena \\ Enituit, figuli suboles nova carmina finxit! \\ Mater Polla fuit, Magii non infuma proles, \\ Quem socerum probitas fecit laudata Maroni. \\ Iaec cum maturo premeretur pondere ventris, \\ Vt solet in somnis animus ventura repingens \\ Anxius e vigili praesumere audia cura, \\ Phoebei nemoris ramum fudisse putavit. \\ O sopor indicium veri! nil certius umquam \\ Cornea porta tulit. facta est interprete lauro \\ Certa parens onerisque sui cognoverat artem. \\ Consule Pompeio vitalibus editus auris \\ Et Crasso tetigit terras, quo tempore Chelas \\ Iam mitis Phaethon post Virginis ora receptat. \\ Infantem vagisse negant; nam fronte serena \\ Conspexit mundum, cui commoda tanta ferebat. \\ lpse puerperiis adrisit laetior orbis; \\ Terra ministravit flores et munere verno \\ lerbida supposuit puero fulmenta virescens. \\ Praeterea si vera fides (set vera probatur), \\ Lata cohors apium subito per rura iacentis \\ Labra favis texit dulces fusura loquellas. \\ 
        \pagebreak 
    \begin{center} \textbf{C. 67t, 55—84} \end{center} \marginpar{[143]} loc quondam in sacro tantum mirata Platone \\ Indicium linguae memorat famosa vetustas. \\ Set Natura parens properans extollere lomam \\ Et Latio dedit hoc, ne quid concederet uni. \\ Iusuper his genitor, nati dum fata requirit, \\ o Populeam sterili virgam mandavit harenae, \\ Tempore quae nutrita brevi, dum crescit in omen, \\ Altior emicuit cunctis, quas auxerat aetas. \\ Hlaec propter placuit puerum committere Musis \\ Et monstrare viam victurae in saecula famae. \\ cs Tum Ballista rudem lingua titubante receptum \\ Instituit primus, quem nox armabat in umbris \\ Grassari solitum. crimen doctrina tegebat; \\ Mox patefacta viri pressa est audacia saxis. \\ Incidit titulum iuvenis, quo pignera vatis \\ o Edidit. auspiciis suffecit poena magistri. \\  \lbrack ‘Monte sub boc lapidum tegitur Ballista sepultus; \\ Nocte die tutum carpe viator iter’ \rbrack  \\ Nos tamen hoc brevius, si fas simulare Maronem: \\ ‘Bllistam sua poena tegit; via tuta per oras.’ \\  \lbrack ‘Ilice Ballista iacet: certo pede perge viator.’ \rbrack  \\  \lbrack ‘Carcere montoso clausus Ballista tenetur: \\ Securi fraudis pergite nocte, viri.’ \rbrack  \\  \lbrack ‘Quid trepidas tandem gressu pavitante, viator? \\ Nocturnum furem saxeus imber habet.’ \rbrack  \\  \lbrack ‘Ballistae vitam rapuit lapis, †ipse sepulcrum \\ Intulit; umbra nocens pendula saxa tremit.’ \rbrack  \\ ‘Crimina latronis dignissima poena coercet: \\ Duritiam mentis damnat ubique lapis.’ \rbrack  \\ IHinc Culicis tenui praelusit funera versu. \\ 
        \pagebreak 
     \marginpar{[144]} \begin{center} \textbf{C. 671, 85—113} \end{center}‘Parve culex, pecudum custos tibi tale merenti \\ Funeris offlcium vitae pro munere reddit.’ \\ Tum tibi Siroem, Maro, contulit ipsa magistrum \\ Roma potens proceresque suos tibi iunxit amicos. \\ Pollio Maecenas Varus Cornelius ardent; \\ Te sibi quisque rapit, per te victurus in aevum. \\ Musa refer: quae causa fuit componere libros \\ Sumpserat Augustus rerum moderamina princeps. \\ Iam necis ultor erat patriae, iam caede priorum \\ Perfusos acies legitur visura Phbilippos. \\ Cassius hic Magni vindex et Brutus in armis \\ Intereunt. victor nondum contentus, opimis \\ Emeritas belli spoliis ditasse cohortes, \\ Proscripsit miserae florentia rura Cremonae, \\ Totaque militibus pretium concessa laborum \\ Praeda fuit. violenta manus bacchata per agros. \\ Non flatus, non tela Iovis, non spumeus amnis, \\ Non imbres rapidi, quantum manus impia, vastant. \\ Mantua, tu coniuncta loco, sociata periclis; \\ Non tamen ob meritum misera  \lbrack es: vicinia fecit. \\ Iam Maro pulsus erat, set viribus obvius ibat \\ 105 \\ Fretus amicorum clipeo, cum paene nefando \\ Ense perit. quid, dextra, furis quid viscera Romae \\ Sacrilego mucrone petis? tua bella tacebit \\ Posteritas ipsumque ducem, nisi Mantua dicat! \\ Non tulit hanc rabiem doctissima turba potentum. \\ Itur ad auctorem rerum: quid Martius horror \\ Egerit, ostendunt. qui tam miseranda tulisset, \\ Caesaris huic placido nutu repetuntur agelli. \\ 
        \pagebreak 
    \begin{center} \textbf{C. 671, 114131. C. 672, 1—2} \end{center} \marginpar{[145]} Iis auctus meritis cum digna rependere vellet, \\ lnvenit carmen, quo munera vincere posset. \\ Praedia dat Caesar, quorum brevis usus habendi: \\ Obtulit hic laudes, quas saecula nulla silebunt. \\ Pastores cecinit primos. hoc carmine consul \\ Pollio laudatur ter se revocantibus aunis \\ Conposito. post haec ruris praecepta colendi \\ Quattuor exposuit libris et commoda terrae \\ Edocuit geminis anno minus omnia lustris. \\ Inde cothurnato Teucrorum praelia versu \\ Et Rutulum tonuit. bis sena volumina sacro \\ s ormavit donanda duci trieteride quarta. \\ Sed loca, quae vulgi memoravit tradita fama, \\ Aequoris et terrae statuit percurrere vates, \\ Certius ut libris oculo dictante notaret. \\ Pergitur. ut Calabros tetigit, livore nocenti \\ Parcarum vebhemens laxavit corpora morbus. \\ Hic ubi languores et fata minacia sensit, \\ 
      \end{verse}
  
            \subsection*{672}
      \begin{verse}
      B. lI 18 \\ M. 858. \\ esari usto brbutm \\ B. IV 179. \\ Ergone supremis potuit vox inproba verbis \\ Tam dirum mandare nefas? ergo ibit in ignes \\ 
        \pagebreak 
     \marginpar{[146]} \begin{center} \textbf{C. 672, 317} \end{center}Magnaque doctiloqui morietur Musa Maronis? \\ A scelus indignum! solvetur littera dives \\ Et poterunt spectare oculi, nec parcere honori \\ Flamma suo? †ductumque operi servabit amorem? \\ Pulcher Apollo, veta! Musae prohibete Latinae! \\ iber et alma Ceres, succurrite! vester in armis \\ Miles erat, vester docilis per rura colonus. \\ Nam docuit, quid ver ageret, quid cogeret aestas, \\ Quid pater autumnus, quid bruma novissima ferret. \\ Munera telluris larga ratione notavit, \\ Arbuta formavit, sociavit vitibus ulmos, \\ Curavit pecudes, apibus sua castra dicavit. \\ Illum, illum Aenean nesciret fama perennis, \\ Docta Maroneo caneret nisi pagina versu! \\ Haec dedit, ut pereant, ipsum si dicere fas est! \\ lcceso 5 Nec . . parcet hiatu 6 pio 6 honore \\ 1 novavit 15 Illum ipsum Aenean nescisset lom p. \\ 
        \pagebreak 
    \begin{center} \textbf{C. 672, 18—30} \end{center} \marginpar{[147]} ‘Sed legum est servanda fides; suprema voluntas \\ Quod mandat fierique iubet, parere necesse est.’ \\ Frangatur potius legum reverenda potestas, \\ Quam tot congestos noctesque diesque labores \\ Auferat una dies, supremaque verba parentis \\ Amittant vigilasse suum. si forte supremum \\ Erravit iam morte piger, si lingua locuta est \\ Nescio quid titubante animo, non sponte sed altis \\ Expugnata malis odio languoris iniqui, \\ Si mens caeca fuit: iterum sentire ruinas \\ Troia suas, iterum cogetur reddere †voces? \\ Ardebit miserae (narratrix fama Creusae? \\ Sentiet appositos Cumana Sibylla vapores? \\ 
        \pagebreak 
    \begin{center} \textbf{C. 672, 314} \end{center} \marginpar{[148]} Vretur Tyriae post funera vulnus Elissae \\ (Et iurata mori, ne cingula reddat, Amaon? \\ Tam sacrum solvetur opus? tot bella, tot enses \\ In cineres dabit hora nocens et perfidus error? \\ Huc huc, Pierides, date flumina cuncta, sorores; \\ Exspirent ignes, vivat Maro ductus ubique \\ ngratusque sui studiorumque invidus orbi \\ Et factus post fata nocens. quod iusserat ille \\ S7vetuissemeumsatisestposttemporavitae, \\ Immo sit aeternum tota resonante Camena \\ Carmen, et in populo divi sub numine nomen \\ Laudetur vigeat placeat relegatur ametur! \\ 
        \pagebreak 
    \begin{center} \textbf{C. 672 674. 1} \end{center} \marginpar{[149]} 
      \end{verse}
  
            \subsection*{672}
      \begin{verse}
      Ocio ributm \\ B. M. \\ \poemtitle{De XI libris Aeneidarum}B. IV 178. \\ Primus habet pelagi \lbrack que \rbrack  minas, terraeque secundus, \\ Tertius errores, et amores quartus Elissae. \\ Quintus habet ludos, sextus deducit ad umbras. \\ Septimus Ausonios, Aenean proximus armat. \\ Nonus Iyrtaciden, decimus Pallanta peremit. \\ Vndecimus Drancem damnat, pars ultima Turnum. \\ 
      \end{verse}
  
            \subsection*{673}
      \begin{verse}
      rilio fibut \\ B. M. \\ B. IV 11. \\ \poemtitle{De uodam cum eruribus obliuis nato}En dat aperturam crurum flexura recurvam \\ Et patet oblicus inter utrumque locus, \\ Quo praegnantis equae calcaribus urgeat alvum \\ Curvato et tutum crure sit intus onus. \\ 
      \end{verse}
  
            \subsection*{674}
      \begin{verse}
      eryilio e Ooilio ribt \\ B. M. \\ \poemtitle{De imagine et somno}B. IV 118. \\ Pulchra comis annisque decens et candida vultu \\ 
        \pagebreak 
     \marginpar{[150]} \begin{center} \textbf{C. 674, 2. C. 67a} \end{center}Dulce quiescenti basia blanda dabas. \\ te iam vigilans non unquam cernere possum, \\ Somne, precor, iugiter lumina nostra tene! \\ 
      \end{verse}
  
            \subsection*{674}
      \begin{verse}
      B.II 17. M. 254. \\ Vigilius \\ B. IV 18. \\ Maeonium quisquis Romanus nescit Homerum, \\ Me legat, et lectum credat utrumque sibi. \\ llius immensos miratur Graecia campos; \\ At minor est nobis, sed bene cultus ager. \\ Hic tibi nec pastor nec curvus deerit arator. \\ Haec Grais constant singula, trina mihi. \\ 
        \pagebreak 
    \begin{center} \textbf{C. 672 676, 12} \end{center} \marginpar{[151]} 
      \end{verse}
  
            \subsection*{67}
      \begin{verse}
      B. M. 239. \\ B. III 169. \\ Nereides freta sic verrentes caerula tranant, \\ Flamine confidens ut notus Icarium. \\ Icarium notus ut confidens flamine tranant \\ Caerula verrentes sic frata Nereides. \\ 
      \end{verse}
  
            \subsection*{675}
      \begin{verse}
      Hic adee ibet eyics ab Aldlelo e eris \\ (cf p. 284) Gil. Vgilio adsriptos: \\ \poemtitle{VIRLII}B. M. \\ B. IV 161. \\ Virgilius item libro, quem Paedagogum praetitulavit, \\ cuius principium est \\ Carmina si fuerint te iudice digna favore, \\ Reddatur titulus purpureusque nitor. \\ Sin minus, aestivas poteris convolvere sardas \\ Aut piper aut calvas hinc operire nuces, \\ syllabam elisit dicens: \\ Durum iter et vitae magnns labor.’ \\ 
      \end{verse}
  
            \subsection*{676}
      \begin{verse}
      B. V 4B. M. 1030. \\ B. V 349. \\ Me legat, annales cupiat qui noscere menses \\ Tempora dinumerans aevi vitaeque caducae. \\ 
        \pagebreak 
    \begin{center} \textbf{C. 676, 310} \end{center}Omnia tempus agit, cum tempore cuncta trahuntur. \\ Alternant elementa vices et tempora mutant. \\ Accipiunt augmenta dies noctesque vicissim. \\ Tempora sunt florum, retinet sua tempora messis, \\ Sic iterum spisso vestitur gramine campus. \\ Tempora gaudendi, sunt tempora certa dolendi. \\ Tempora sunt vitae, sunt tristia tempora mortis. \\ Tempus et hora volat. momentis labitur aetas. \\ 
        \pagebreak 
    \begin{center} \textbf{C. 676, 1113. C. 677 678, 16} \end{center} \marginpar{[153]} Omnia dat tollit minuitque volatile tempus. \\ Ver aestas autumnus hiems: redit annus in annum. \\ Omnia cum redeant, homini sua non redit aetas. \\ B. V 44. M. 1031 \\ 4 4 \\ B. V 350. \\ Exsurgens Chelas Aries demergit in ima. \\ Scorpion aurati submergunt cornua Tauri. \\ Tum snbit Arcitenens, Geminis surgentibus, aequor. \\ Dum surgit Cancer, Capricornus mergitur undis. \\ Portitor urceoli formidat signa Leonis. \\ Virgo fugat Pisces. redit et victoria viectis. \\ B. V 45. M. 1032. \\ 
      \end{verse}
  
            \subsection*{678}
      \begin{verse}
      B. V 350. \\ Bis sex signiferae numerantur sidera sphaerae, \\ Per quae planetae dicuntur currere septem. \\ Pollucis proles ter denis volvitur annis. \\ Fulmina dispergens duodenis lustrat aristis. \\ Bellipotens genitor . \\ . mensum pensare bilibri. \\ In medio mundi fertur Phaethontia flamma \\ 
        \pagebreak 
    \begin{center} \textbf{C. 678,. 7 1B. C. 679. 1—4} \end{center} \marginpar{[154]} Ter centum soles, sex denos, quinque, quadrantem. \\ Ter senas partes his  \lbrack tu \rbrack  Cytherea retorques \\ Lustrando totum praeclaro lumine mundum. \\ Terque dies ternos puro de vespere tollens \\ Semonis divi completur circulus anni. \\ Ioras octo, dies ternos servato novenos, \\ Proxima telluri dum curris, candida Phoebe. \\ 
      \end{verse}
  
            \subsection*{679}
      \begin{verse}
      A r \\ \poemtitle{PRISCIANI GRAMMATICI}B. V 7. M. 284. \\ \poemtitle{De sideribu}B. V 351. \\ Ad Boreae partes Arctoi vertuntur et Anguis, \\ Post has Arctophylax pariterque Corona, genuque \\ Prolapsus, Lyra, Avis Cepheus et Cassiopea, \\ Auriga et Perseus, Deltoton et Andromedae astrum, \\ 
        \pagebreak 
    \begin{center} \textbf{C. 679. 52. C. 680, 1—7} \end{center} \marginpar{[155]} Pegasus et Delpbin Telumque, Aquila Anguitenensque. \\ Signifer inde subest, bis sex et sidera complent \\ Hunc: Aries Taurus Gemini Cancer Leo Virgo \\ Libra Scorpius Arcitenens Capricornus et urnam \\ Qui tenet, et Pisces. post sunt in partibus Austri \\ 0 Orion, Procyon, Lepus, ardens Sirius, Argo, \\ Hydrus, Chiron, Turibulum quoque, Piscis et ingens. \\ linc sequitur Pistrix simul Eridanique fluenta. \\ B. V 68. M. 390. \\ 
      \end{verse}
  
            \subsection*{680}
      \begin{verse}
      B. V 352. \\ Bis sena mensum vertigine volvitur annus, \\ Septimanis decies quinis simul atque duabus, \\ Ter centenis bisque tricenis quinque diebus, \\ Quos ternis gaudet divisos stare columnis, \\ Scilicet Idibus et Nonis simul atque alendis. \\ Nam quadris constat Nonis concurrere menses \\ Omnes excepto Marte et Maio, sequitnr quos \\ 
        \pagebreak 
     \marginpar{[156]} \begin{center} \textbf{C. 680, 817. C. 680, 12} \end{center}Iulius et October: senis soli hi moderauntur. \\ Septenis patet hos denis quadrare alendis. \\ Octonisque pares menses sunt Idibus omnes. \\ Ianus et Angustus tantum mensisque December \\ Volvuntur denis semper nonisque Xalendis; \\ At contra currunt bis nonis rite quaterni \\ Iunius Aprilis September et ipse November. \\ Sedenis Februus cito solus ab omnibus errat. \\ Bis senis sic namque rotatur mensibus annus \\ Pe Nonas ldusque decurrens atque alendas. \\ 
      \end{verse}
  
            \subsection*{680a}
      \begin{verse}
      B. M. \\ B. V 354. \\ Versus de diebus Aegptiacis \\ Bis deni binique dies scribuntur in anno, \\ In quibus una solet mortalibus hora timeri. \\ 
        \pagebreak 
    \begin{center} \textbf{C. 680a, 322} \end{center} \marginpar{[157]} Mensis quisque duos captivos possidet horum \\ Nec simul hos iunctos, homines ne peste trucident. \\ Si tenebrae Aegyptus Graeco sermone vocantur, \\ Inde dies mortis tenebrosos iure vocamus. \\ I lani prima dies et septima fine timetur. \\ 
      \end{verse}
  
            \subsection*{}
      \begin{verse}
      \poemtitle{VI}HI st Februi quarta est; praecedit tertia finem. \\ I Martis prima necat, cuius sub cuspide quarta est. \\ X Aprilis decima est, undeno et fine minatur. \\ 
      \end{verse}
  
            \subsection*{}
      \begin{verse}
      \poemtitle{XI} \lbrack I Tertius in Maio lupus est et septimus anguis. \\ 
      \end{verse}
  
            \subsection*{}
      \begin{verse}
      \poemtitle{VI}X Iunius in decimo quindenum a fine salutat. \\ 
      \end{verse}
  
            \subsection*{}
      \begin{verse}
      \poemtitle{XV}X Tredecimus lulii decimo inuuit ante xalendas. \\ I Augusti nepa prima fugat de fine secundam. \\  \lbrack  I Tertia Septembris vulpis ferit a pede denam. \\ I Tertius Octobris pullus decimum ordine nectit. \\ VQuintaNovembrisacus,vixtertiamansitinurna.11t \\ VI Dat duodena cohors septem inde decemque Decembris. \\ His caveas, ne quid proprio de sanguine demas. \\ Nullum opus incipias, nisi forte ad gaudia tendat. o \\ Et caput et finem mensis in corde teneto, \\ Ne in media ima ruas, sed clara per aethera vivas. \\ 
        \pagebreak 
     \marginpar{[158]} \begin{center} \textbf{C. 681 682} \end{center}B. . \\ 
      \end{verse}
  
            \subsection*{681}
      \begin{verse}
      B. IV 441. \\ Basia coniugibus, sed et oscula dantur amicis; \\ Savia lascivis miscentur grata labellis. \\ 
      \end{verse}
  
            \subsection*{682}
      \begin{verse}
      B. M. \\ Ooio Vasoni briufm \\ B. II 170. \\ Rustice lustrivage capripes cornute bimembris \\ Cinyphie hirpigena pernix caudite petulce \\ Saetiger indocilis agrestis barbare dure \\ Semicaper villose fugax periure biformis \\ Audax brute ferox pellite incondite mute \\ Silvicola instabilis saltator perdite mendax \\ Lubrice ventisonax inflator stridule anhele \\ Hirte hirsute biceps fallax niger hispide sime \\ Scrans aridus iolae spurce bruciole Fatucle! \\ 
        \pagebreak 
    \begin{center} \textbf{C. 683 686, 1—2} \end{center} \marginpar{[159]} 
      \end{verse}
  
            \subsection*{683}
      \begin{verse}
      ide nunc c. \\ B.IV 88. M. 1225. \\ 
      \end{verse}
  
            \subsection*{684}
      \begin{verse}
      B. V 385. \\ Luciola,effigiesillorumiudicequovis, \\ Quos peperit quondam morum fecunda vetustas, \\ Hic tegitur, laudis monumento exstante superstes. \\ Nobilitas patriae Treveri praeclara, patrumque \\ Arvernum hospitium civem se adscisse superbit. \\ Pignorat haec tellus terrae felicis alumnam. \\ Fundorum spatiis cinerum est possessio maior. \\ Coniuge fortunata equitum peditumque magistro \\ Omni humilem officio dum se gerit, auxit honorem: \\ Altior invidia, qui non subcumbit honori, est. \\ . M. \\ 
      \end{verse}
  
            \subsection*{685}
      \begin{verse}
      B. III 170. \\ Collis sum collisque fi collisque manebo; \\ Tertia si pereat littera, sexta manet. \\ B. M. \\ 
      \end{verse}
  
            \subsection*{686}
      \begin{verse}
      B. IV 442 \\ Vrbs, quae tantum alias inter caput extulit urbes, \\ Quantum lenta solent inter viburna cupressi, \\ 
        \pagebreak 
     \marginpar{[160]} \begin{center} \textbf{C. 686, 325} \end{center}Mantua nostra aliis tantum concedit honoris, \\ Puniceis humilis quantum saliunca rosetis \\ Aut oleae spinus caris aut vitibus alnus, \\ Elleborus nardo, piperi faba, tofus et auro. \\ Qua nullus princeps, nullus quoque verna moratur, \\ Aequales totum retinent vel cuncta tyranni. \\ Pax abiit tristis, civilia bella geruntur, \\ Friget amoris honos, odiorum semina pollent, \\ Frumentum premitur, lolium sine nomine surgit; \\ Moerorum lapides et propugnacula vendunt, \\ Excubias qui sorte gerunt. cultura deorum \\ Virtutumque cadit; fingit sibi quisque colendum, \\ Mens vaga quod suadet. magnae vicina ruinae \\ Mantua vae miserae, quam barbarus incola replet, \\ Quam sermone secant, venter quos protulit unus, \\ Frater et ad fratrem verbis non haeret eisdem! \\ Tityrus admouit cives quam saepe cavere, \\ Ne lupus in stabulis ovium misceret acervos! \\ Dissona sed cunctam vetuit discordia plebem, \\ Ne saltim excubiis vel saepe ambronibus obstet. \\ Stertit enim upilio. casus heu cerno propinquos: \\ Ei mihi iamque nefas heu, pro dolor! ei mihi \\ tandem, \\ Si qua tuae nunc matris habet te cura, faveto! \\ 
        \pagebreak 
    \begin{center} \textbf{C. 687 689} \end{center} \marginpar{[161]} 
      \end{verse}
  
            \subsection*{687}
      \begin{verse}
      B. V 70. M. 39. \\ Conveniunt subito cuncti de montibus altis sqq. \\ B. M. \\ 
      \end{verse}
  
            \subsection*{688}
      \begin{verse}
      B. III 171. \\ Fonte lavat genitor, quem crimine polluet uxor, \\ Et puerum refovet, quem iuvenem perimat. \\ Ante suum gremium portat portatus alumnum; \\ Vnum gestat equus, sed duo terga premunt. \\ Mergitur Hippolytus mersurus amore novercam, \\ Quem quia fata iuvant, flumina nulla nocent. \\ causa Iippolyti versa est natura parentum: \\ Saeva noverca fovet, quem pater ipse necat. \\ 
      \end{verse}
  
            \subsection*{689}
      \begin{verse}
      B. M. B. \\ Conlatum vitae destruxit femina culmen, \\ Femina sed vitae gaudia longa dedit. \\ 
        \pagebreak 
    \begin{center} \textbf{C. 689a} \end{center} \marginpar{[162]} 
      \end{verse}
  
            \subsection*{689a}
      \begin{verse}
      Versus SLVII \\ B. M. B. \\ \poemtitle{De cognomentis Salvatoris}(Omnipotens, vis trina, deus, pater, optima rerum, \\ Quo generante satus sine tempore semine matre, \\ Ortus †sine loco vel membris, post caro natus, \\ Permittens cerni, multo quoque nomine dictus: \\ Spes, ratio, via, vita, salus, sapientia, mens, mons, \\ Iudex, porta, gigas, rex, gemma, propheta, sacerdos, \\ Messias, Sabaoth, rabbi, sponsus, mediator, \\ Virga, columba, manus, petra, filius, Emmanuel, lux, \\ Vinea, pastor, ovis, pax, radix, vitis, oliva, \\ Fons, haedus, panis, agnus, vitulus, leo, Iesus,’ \\ Verbum, homo, rete, lapis, dominus, deus: omiaChristus. \\ 
        \pagebreak 
    \begin{center} \textbf{C. 689b, 1—21} \end{center} \marginpar{[163]} 
      \end{verse}
  
            \subsection*{689a}
      \begin{verse}
      \poemtitle{VERSVS CYPRIANI}B. M. B. \\ Ad uendam asenatorem ex christiana religione \\ ad idolorum servitutem eonversum \\ Cum te diversis iterum vanisque viderem. \\ Inservire sacris priscoque errore teneri, \\ Obstipui motus. Quia carmina semper amasti, \\ Carmine respondens properavi scribere versus, \\ Vt te corriperem tenebras praeponere luci. \\ Quis patiatur enim te Matrem credere magnam \\ Posse deam dici rursusque putare colendam, \\ Cuius cultores infamia turpis inurit? \\ Namque sacerdotes tunicis muliebribus idem \\ lnterius vitium cultu exteriore fatentur, \\ Idque licere putant, quod non licet; unde per urbem \\ Leniter incedunt mollita voce loquentes, \\ Laxatosque tenent extenso pollice lumbos, \\ Et proprium mutant vulgato crimine sexum. \\ Cumque suos celebrant ritus, his esse diebus \\ Se castos memorant: ut, si tantummodo tunc sunt, \\ Vt perhibent, casti, reliquo iam tempore quid sunt? \\ Sed quia coguntur saltim semel esse pudici, \\ Mente fremunt, lacerant corpus funduntque cruorem. \\ Quale sacrum est, vero quod fertur nomine sanguis? \\ Nunc etiam didici, quod te non fecerit aetas, \\ 
        \pagebreak 
     \marginpar{[164]} \begin{center} \textbf{C. 689, 248} \end{center}Sed tua religio calvum, caligaque remota \\ Callica sit pedibus molli redimita papyro: \\ Res miranda satis deiectaque culmine summo. \\ Si quis ab siaco consul, procedat in orbem, \\ Risus orbis erit; quis te non rideat autem, \\ Qui fueris consul, nunc sidis esse ministrum? \\ Quodque pudet primo, te non pudet esse secundo, \\ Ingeniumque tuum turpes damnare per hymos \\ Respondente tibi vulgo et †lacerante senatu, \\ Teque domo propria pictum cum fascibus ante, \\ Nunc quoque cum sistro faciem portare caninam? \\ Haec tua humilitas et humilitatis imago est! \\ Aedibus illa tuis semper monumenta manehunt. \\ Rumor et ad nostros pervenit publicus aures, \\ Te dixisse: ‘Dea, erravi; ignosce! redivi.’ \\ Dic mihi, si valeas: cum talia saepe rogares \\ Et veniam peteres, quae tecum verba locuta est? \\ Vera mente cares, sequeris qui mente carentes. \\ Iaec iterum repetis nec te delinquere sentis? \\ Quid mereare, vide. minus esses forte notandus, \\ Si tantum hoc scires et in hoc errore maneres; \\ At cum vericolae penetraveris ostia legis, \\ Et tibi nosse Deum paucis provenerit annis, \\ Cur linquenda tenes aut cur retinenda relinquis? \\ Nilque colis, dum cuncta colis, nec corde retractas, \\ Vera quid a falsis, quid ab umbris lumina distent. \\ Philosophum fingis, cum te sententia mutet: \\ 
        \pagebreak 
     \marginpar{[165]} \begin{center} \textbf{C. 68, 79} \end{center}Nam tibi si stomachum popularis moverit ira, \\ Et Iudaeus eris totusque incertus ageris. \\ Indulge dictis: sapientia non placet alta. \\ Omne quod est nimium, contra cadit. Vnum operantur \\ Et calor et frigus; sic hoc, sic illud adurit. \\ Sic tenebrae visum, sic sol contrarius aufert, \\ Et pariter laedunt †tepidum fervensque lavacrum. \\ Esca alitur corpus, corpus consumitur esca, \\ Vimque suam minuit, si quid protenditur ultra. \\ Denique si sedeas, requies est magna laboris; \\ Si multum sedeas, labor est. Maro namque poeta \\ Pro poena posuit: ‘sedet aeternumque sedebit \\ Infelix Theseuns.’ semper nocet utile longum: \\ Prandia longa nocent, ieiunia longa fatigant. \\ Sic nimium sapere stultum facit. ‘Improba secta,’ \\ Me dea sic docuit; ‘moderamen amabile’ dixit. \\ Sed tu nec sectam modo nec moderamina curas. \\ Mens antem stabilis nullo pervertitur aestu, \\ psaque simplicitas numquam mala cogitat ulla. \\ Hinc sincera fides aeterna sede fruetur \\ Et dolus e contra longo cruciabitur igni. \\ Elige quid velis, ut digna piacula vites. \\ Sic tamen hanc veniam mereatur †creditor inquam. \\ Vt leve crimen erit, si nolis noscere vera, \\ Non leve crimen erit, si cognita vera relinquas. \\ Sed te correctum forsan matura senectus \\ In melius revocat satiatum erroribus istis; \\ Tempus enim mutat, mala digerit omnia tempus. \\ Tunc igitur, cum te consulta reduxerit aetas, \\ Disce Deo servare fidem, ne forte bis unum \\ Incurras lapsum. quia vere dicitur illud: \\ 
        \pagebreak 
     \marginpar{[166]} \begin{center} \textbf{C. 689, 8o s5. C. 689} \end{center}Qui pedis offensi lapidem vitare secundo \\ Nescit et incautus iterum vexaverit artus, \\ Imputet ipse sibi nec casibus imputet ullis. \\ Corrige delictum fidamine, corrige mentem: \\ Sufecit peccare semel. desiste vereri: \\ Non erit in culpa, quem paenitet ante fuisse. \\ 
      \end{verse}
  
            \subsection*{689}
      \begin{verse}
      B. M. B. \\ AMdere libet er farci onac Casinesis de ’. \\ endicto carmnne uemad ntiquitate specania \\ Caeca profanatas coleret dum turba figuras, \\ Et manibus factos crederet esse deos, \\ Templa ruinosis haec olim struxerat aris, \\ Quis dabat obsceno sacra cruenta Iovi. \\ Sed iussus veniens eremoque vocatus ab alta \\ Purgavit sanctus hanc Benedictus humum \\ Sculptaque confractis deiecit marmora signis \\ Et templum vivo praebuit esse Deo . . . \\ Iunc plebs stulta locum quondam vocitaverat Arcem, \\ Marmoreisque sacrum fecerat esse deis . . . \\ Ad quem caecatis errantes mentibus ibant \\ Improba mortifero reddere vota Iovi . . . \\ Ast huc perducto scopuli cessere rubique, \\ Siccaque mirandas terra retexit aquas . . . \\ 
        \pagebreak 
    \begin{center} \textbf{C. 690 691, 1—3} \end{center} \marginpar{[167]} 
      \end{verse}
  
            \subsection*{690}
      \begin{verse}
      B. . Ptrou. \\ I r e. Buee. . 26. \\ \poemtitle{PETRONII ARBTRI}B. IV 95. \\ Sic contra rerum naturae munera nota \\ Corvus maturis frugibus ova refert. \\ Sic format lingua fetum cum protulit ursa \\ Et piscis nullo iunctus amore parit. \\ Sic Phoebea chelys vinclo resoluta parentis \\ Lucinae tepidis naribus ova fovet. \\ Sic sine concubitu textis apis excita ceris \\ Fervet et audaci milite castra replet. \\ Non uno contenta valet natura tenore, \\ Sed permutatas gandet habere vices. \\ 
      \end{verse}
  
            \subsection*{691}
      \begin{verse}
      B.V.148. M.1084. \\ Buech. . \\ \poemtitle{PETRONII}B. IV 95. \\ Indica purpureo genuit me litore tellus, \\ Candidus accenso qua redit orbe dies. \\ Hic ego divinos inter generatus honores \\ 
        \pagebreak 
    \begin{center} \textbf{C. 691a, 4 8. C. 692 693} \end{center} \marginpar{[168]} Mutavi Latio barbara verba sono. \\ Iam dimitte tuos, Paean o Delphice, cycnos: \\ Dignior haec vox est, quae tua templa colat. \\ 
      \end{verse}
  
            \subsection*{692}
      \begin{verse}
      B. M. \\ B. IV 6. \\ \poemtitle{PETRONII}Buech. 42. \\ Naufragus eiecta nudus rate quaerit eodem \\ Percussum telo, cui sua fata fleat; \\ Grandine qui segetes et totum perdidit annum, \\ In simili deflet tristia fata sinu. \\ Funera conciliant miseros, orbique parentes \\ Coniungunt gemitus, et facit hora pares. \\ Nos quoque confusis feriemus sidera verbis, \\ Et fama est, iunctas fortius ire preces. \\ 
      \end{verse}
  
            \subsection*{693}
      \begin{verse}
      B.III 21B. M. 983. \\ B. IV 97. \\ \poemtitle{PETRONII}Buech. 44. \\ Si Phoebi soror es, mando tibi, Delia, causam, \\ Scilicet ut fratri quae peto verba feras: \\ ‘Marmore Sicanio struxi tibi, Delphice, templum \\ Et levibus calamis candida verba dedi. \\ Nunc si nos audis atque es divinus Apollo, \\ Dic mihi, qui nummos non habet, unde petat’ \\ 
        \pagebreak 
     \marginpar{[169]} \begin{center} \textbf{C. 694 696, 1} \end{center}
      \end{verse}
  
            \subsection*{694}
      \begin{verse}
      B. M. \\ B. IV 97. \\ \poemtitle{PERONII}Buecb. 35. \\ Omnia quae miseras possunt finire querellas, \\ In promptu voluit candidus esse deus. \\ Vile olus et duris haerentia mora rubetis \\ Pungentis stomachi composuere famem. \\ Flumine vicino, stultus sitit. †effugit Euro, \\ Cum calidus tepido consonat igne focus. \\ Lex armata sedet circum fera limina nuptae; \\ Nil metuit licito fusa puella toro. \\ Quod satiare potest, dives natura ministrat; \\ Quod docet infrenis gloria, fine caret. \\ 
      \end{verse}
  
            \subsection*{695}
      \begin{verse}
      B. III 238. \\ . 1001. B. IV97. \\ \poemtitle{TRONII}Buecl. 36. \\ Mlitis in galea nidum fecere columbae: \\ Adparet, Marti quam sit amica Venus. \\ 
      \end{verse}
  
            \subsection*{636}
      \begin{verse}
      B. M. \\ B. IV 93. \\ \poemtitle{TRONII}uecb. 3. \\ Iudaeus licet et porcinum numen adoret \\ Et caeli summas advocet auriculas, \\ Ni tamen et ferro succiderit inguinis oram \\ Et nisi nodatum solverit arte caput, \\ 
        \pagebreak 
     \marginpar{[170]} \begin{center} \textbf{C. 696, 5. C. 697 699, 1—3} \end{center}Exemptus populo Graia migrabit ab urbe \\ Et non ieiuna sabbata lege premet. \\ 
      \end{verse}
  
            \subsection*{697}
      \begin{verse}
      B. M. \\ B. IV 98. \\ \poemtitle{PRoI}Buech. ad 37. \\ Vna est nobilitas argumentumque coloris \\ Ingenui, timidas non habuisse manus. \\ 
      \end{verse}
  
            \subsection*{698}
      \begin{verse}
      B. III 205. \\ M. 977. B. IV 98. \\ \poemtitle{PTRONII}Buch. 38. \\ Lecto compositus vix prima silentia noctis \\ Carpebam et somno lumina victa dabam: \\ Cum me saevus Amor prensat sursumque capillis \\ Excitat et lacerum pervigilare iubet. \\ ‘Tu famulus meus,’ inquit, ‘ames cum mille puellas, \\ Solus, io, solus, dure, iacere potes’ \\ Exsilio et pedibus nudis tunicaque soluta \\ Omne iter impedio, nullum iter expedio. \\ Nunc propero, nunc ire piget, rursumque redire \\ Poenitet, et pudor est stare via media. \\ Ecce tacent voces hominum strepitusque viarum \\ Et volucrum cantus turbaque fida canum: \\ Solus ego ex cunctis paveo somnumque torumque \\ Et sequor imperium, magne Cupido, tuum. \\ 
      \end{verse}
  
            \subsection*{699}
      \begin{verse}
      B. III 203. \\ . 975. B. IV 99. \\ \poemtitle{PETRONII}Buech. 39. \\ Sit nox illa diu nobis dilecta, Nealce, \\ Quae te prima meo pectore composuit; \\ 
        \pagebreak 
    \begin{center} \textbf{C. 699, 3 8. C. 700 701} \end{center} \marginpar{[171]} Sit torus et lecti genius secretaque †longa, \\ Queis tenera in nostrum veneris arbitrium. \\ Ergo age duremus, quamvis adoleverit aetas, \\ Vtamurque annis, quos mora parva tenet. \\ Fas et inra sinunt veteres extendere amores: \\ Fac, cito quod coeptum est, non cito desinere. \\ 
      \end{verse}
  
            \subsection*{700}
      \begin{verse}
      B.III220. M.992. \\  \lbrack Item \rbrack  \\ B. IV 99. \\ Foeda est in coitu et brevis voluptas \\ Et taedet Veneris statim peractae. \\ Non ergo ut pecudes libidinosae \\ Caeci protinus irruamus illuc: \\ Nam languescit amor peritque flamma. \\ Sed sic sic sine fine feriati \\ Et tecum iaceamus osculantes. \\ Hic nullus labor est ruborque nullus: \\ Hoc iuvit, iuvat et diu iuvabit. \\ Hoc non deficit incipitque semper. \\ 
      \end{verse}
  
            \subsection*{701}
      \begin{verse}
      B.III221. M.993. \\ Iem \\ B. IV 99. \\ Accusare et amare tempore uno \\ Ipsi vix fuit Herculi ferendum. \\ 
        \pagebreak 
    \begin{center} \textbf{C. 702 705, 12} \end{center} \marginpar{[172]} 
      \end{verse}
  
            \subsection*{702}
      \begin{verse}
      B.I 206. M.978. \\ Item \\ . IV 100. \\ Te vigilans oculis, animo te nocte require, \\ Victa iacent solo cum mea membra toro. \\ Vidi ego me tecum falsa sub imagine somni. \\ Somnia tu vinces, si mihi vera venis. \\ 
      \end{verse}
  
            \subsection*{703}
      \begin{verse}
      B.II214. M. 984. \\ B. IV 100. \\ Hoc sibi lusit opus de stamine floricolore \\ Hesperie, teneras officiosa manus. \\ Et pulcbro pulchras strophio producta papillas \\ Gaudet utrumque sui pectorls esse decus. \\ 
      \end{verse}
  
            \subsection*{704}
      \begin{verse}
      B. III215. .986. \\  \lbrack Item \rbrack  \\ B. IV 100. \\ Hesperie lateri redimicula nectit eburno \\ Facta suis manibus, pectore digna suo. \\ Iam veteres iras Venus et Tritonia ponit: \\ Pectora nam Veneris Palladis ambit opus. \\ 
      \end{verse}
  
            \subsection*{705}
      \begin{verse}
      B.I216. M. 986. \\  \lbrack Item \rbrack  \\ B. IV 100. \\ Intertexta rosa Tyrii subtemine fuci \\ Involvet quoties mobile zona latus, \\ 
        \pagebreak 
    \begin{center} \textbf{C. 705, 3. C. 706 707} \end{center} \marginpar{[173]} Ambrosium gemino potabit ab ubere rorem \\ Et vere roseo fiet odore rosa. \\ 
      \end{verse}
  
            \subsection*{706}
      \begin{verse}
      B.III213. M.988. \\ Item \\ B. IV 101. \\ Me nive candenti petiit modo Iulia. rebar \\ Igne carere nivem: nix tamen ignis erat. \\ Quid nive frigidius? nostrum tamen urere pectus \\ Nix potuit manibus, Iulia, missa tuis. \\ Quis locus insidiis dabitur mihi tutus amoris, \\ Frigore concreta si latet ignis aqua? \\ Ilia sola potes nostras extinguere flammas: \\ Non nive, non glacie, sed potes igne pari. \\ 
      \end{verse}
  
            \subsection*{707}
      \begin{verse}
      B. III 7. . 175. \\ \poemtitle{De Delo}B. IV 101. \\ Delos iam stabili revincta terra \\ Olim purpureo mari natabat \\ Et moto levis hinc et inde vento \\ bat fluctibus inquieta summis. \\ Mox illam geminis deus catenis \\ Hac alta Gyaro ligavit, illac \\ Constanti Mycono dedit tenendam. \\ 
        \pagebreak 
     \marginpar{[174]} \begin{center} \textbf{C. 708 709, 12} \end{center}
      \end{verse}
  
            \subsection*{708}
      \begin{verse}
      \poemtitle{ERMANICI CAESARIS}B. I 103. M. 117. \\ Ad Iectoris tumulnm \\ B. IV 102. \\ Martia progenies, lector, tellure sub ima \\ (Fas audire tamen si mea verba tibi) \\ lespira, quouiam vindex tibi contigit heres, \\ Qui patriae famam proferat usque tuae. \\ lios en surgit rursum inclita, gens colit illam \\ Te Marte inferior, Martis amica tamen. \\ Myrmidonas periisse omnes dic Hector Acbilli, \\ Thessaliam et magnis esse sub Aeneadis. \\ 
      \end{verse}
  
            \subsection*{709}
      \begin{verse}
      \poemtitle{EIVSDEM GERMAN1CI}B. IV 2. M. 69. \\ \poemtitle{De puero glaeie perempto}B. IV 103. \\ Thrax puer adstricto glacie cum luderet lebro, \\ Frigore frenatas pondere rupit aquas, \\ Anth. Pal X 3s7 toroe Mivov oi I epoo \\ Eropo. foio olp, v orb ro oie \\ ape l prsoo ob ?p pido. \\ o oirog. s ic, gopoco \\ om p ioeooo, ’ ’ onoiyoy \\ Meopopoies 0’ ooro avoivrgvo ’ plei, \\ Osoaip sivo vo ’ Ltisida. \\ . prefert corr. Phoeus proferet olim scrpsi. \\ Seuitur ‘Caesaris de libris Lucani’ (3), defnde \\ . Eiusdem Germanici de sqgq. . Et autem Paul \\ Piacont a quo in comemoratonc nstitutionis linguae \\ Graecae puer usns erat aertur ( oetae aeei Carolini \\ ed. Duemmler . sc: \\ Sed omnino ne linuguarum dicar(am) esse nescius, \\ Pauca, mihi quae fuerunt trdita puerulo, \\ Dicam; cetera fugerunt iam gravante senio: \\ De puero sq. C Atnn. phitol. 189, 764. \\ D Paulus. Parisius 2s s. X . M Musei \\ 
        \pagebreak 
    \begin{center} \textbf{C. 709, 38. C. 710} \end{center} \marginpar{[175]} Cumque imae partes fundo raperentur ab imo, \\ Abscidit a iugulo lubrica testa caput. \\ Quod mox inventum mater dum conderet igni, \\ ‘Hoc peperi flammis, cetera’ dixit ‘aquis. \\ Me miseram! plus amnis habet solumque reliquit, \\ Quo nati mater nosceret interitum.’ \\ B.I 23. M. 194. \\ 
      \end{verse}
  
            \subsection*{710}
      \begin{verse}
      B. IV 13. \\ \poemtitle{C. CAECILII PLINII SECVN0I}I I \\ Huc mihi vos, largo spumantia pocula vino, \\ Vt calefactus Amor pervigilare velit. \\ Ardenti Baccho succenditur ignis Amoris, \\ Nam sunt unanimi Bacchus Amorque dei. \\ 
        \pagebreak 
     \marginpar{[176]} \begin{center} \textbf{C. 711 712. 1—12} \end{center}B.III25. M.232. \\ 
      \end{verse}
  
            \subsection*{}
      \begin{verse}
      \poemtitle{GALLIEN}B. IV 104. \\ Ite agite, o iuvenes, et desudate medullis \\ Omnibus inter vos! non murmura vestra columbae, \\ Brachia non hederae, non vincant oscula conchae. \\ Ludite: sed vigiles nolite extinguere lychnos. \\ Omnia nocte vident, nil cras meminere lucernae. \\ 
      \end{verse}
  
            \subsection*{712}
      \begin{verse}
      \poemtitle{. APVLEI}B.III231. M. 230. \\ B. IV 104. \\ Mepeo. Ex Menandro \\ Amare liceat, si potiri non licet. \\ Fruantur alii: non moror, non sum invidus; \\ Nam sese excruciat, qui beatis invidet. \\ Quos Venus amavit, facit amoris compotes: \\ Nobis Cupido velle dat, posse abnegat. \\ Odli purpurea delibantes oscula \\ Clemente morsu rosea labia vellicent, \\ Candentes dentes efligient suavio, \\ Malas odorent ore et ingenuas genas \\ Et pupularum nitidas geminas gemmulas. \\ Quin et cum tenera membra molli lectulo \\ Compactiora adhaerent Veneris glutino, \\ 
        \pagebreak 
    \begin{center} \textbf{C. 712, 132. C. 713 714, 12} \end{center} \marginpar{[79]} ibido cum lascivo instinctu suscitet \\ Sinuare ad veneris usum femina feminae \\ Inter gannitus et subantis voculas, \\ Carpant papillas atque amplexus intiment \\ Thrysumque pangant hortulo Cupidinis \\ 1 Arentque sulcos molles arvo Venerio, \\ Dent crebros ictus †conhibente †lnmine \\ Trepidante cursu venae et anima fessula \\ Eiaculent tepidum rorem niveis laticibus. \\ Haec illi faciant, queis Venus non invidet; \\ At nobis casso saltem delectamine \\ Amare liceat, si potiri non licet! \\ 
      \end{verse}
  
            \subsection*{1.0}
      \begin{verse}
      \poemtitle{ACIMI}B. I 177. M. 255. \\ B. IV 105. \\ de Vergilio et Homero \\ Maeonio vati qui par aut proximus esset, \\ Consultus Paean risit et haec cecinit: \\ Si potuit nasci, quem tu sequereris, lHomere, \\ Nascetur, qui te possit, lomere. sequi. \\ 
      \end{verse}
  
            \subsection*{714}
      \begin{verse}
      B.III212. M.258. \\ \poemtitle{EIVSDEM}B. IV 105. \\ O blandos oculos et †infacetos \\ Et quadam propria nota loquaces! \\ 
        \pagebreak 
    C. 714. 3. C. 715 \\ Illic et Venus et leves Amores \\ Atque ipsa in medio sedet Voluptas. \\ B.III211. . 257. \\ \poemtitle{EIVSDEM}B. IV 105. \\ Lux mea puniceum misit mihi Lesbia malum: \\ Iam sordent animo cetera poma meo. \\ Sordent velleribus vestita cydonia canis, \\ Sordent hirsutae munera castaneae; \\ Nolo nuces, Amarylli, tuas nec cerea pruna: \\ Rusticus haec Corydon munera magna putet. \\ Horreo sanguineo male mora rubentia suco: \\ Heu grave funesti crimen amoris habent! \\ Misi dente levi paulo libata placenta \\ Nectare de labris dulcia †membra suis. \\ Nescio quid plus melle sapit, quod contigit ipsa, \\ Spirans Cecropium dulcis odorethymum. \\ 
        \pagebreak 
     \marginpar{[179]} \begin{center} \textbf{C. 716, 1—19} \end{center}B. III110. M.938. \\ 
      \end{verse}
  
            \subsection*{716}
      \begin{verse}
      B. III 23q. \\ Vtilibus monitis prudens adcommodet aurem. \\ Non laeta extollant animum, non tristia frangant. \\ Dispar vivendi ratio est, mors omnibus una. \\ Grande aliquid caveas timido conmittere cordi. \\ Numquam sanantur deformis vulnera famae. \\ Naufragium rerum est mulier male fida marito. \\ Tu si animo regeris, rex es; si corpore, servus. \\ Proximus esto bonis, si non potes optimus esse. \\ Nullus tam parcus, quin prodigus ex alieno. \\ Audit quod non vult, qui pergit dicere quod vult. \\ Non placet ille mihi, quisquis placuit sibi multum. \\ Iuri servitium defer, si liber haberis. \\ Vel bona contemni docet usus, vel mala ferri. \\ Ex igne ut fumus, sic fama ex crimine surgit. \\ Paulisper laxatus amor decedere †potest. \\ Splendor opum sordes vitae non abluit umquam. \\ Inprobus officium scit poscere, reddere nescit. \\ Inridens miserum dubium sciat omne futurum. \\ Mortis imago invat somnus, mors ipsa timetur. \\ 
        \pagebreak 
     \marginpar{[180]} \begin{center} \textbf{C. 716, 20—45} \end{center}Quanto maior eris, tanto moderatior esto. \\ Alta cadunt odiis, parva extolluntur amore. \\ Criminis indultu secura audacia crescit. \\ Quemlibet ignavum facit indignatio fortem. \\ Divitiae trepidant, paupertas, libera res est. \\ Iaud homo culpandus, quando est in crimine casus. \\ Fac quod te par sit, non alter quod mereatur. \\ Dissimilis cunctis vox vultus vita voluntas. \\ Ipsum se cruciat, te vindicat invidus in se. \\ Semper pauperies quaestum praedivitis auget. \\ Magno conficitur discrimine res memoranda. \\ Terra omnis patria est, qua nascimur et tumulamur. \\ Aspera perpcssu fiunt iucunda relatu. \\ Acrius adpetimus nova, quam iam parta tenemus. \\ Labitur ex animo benefactum, iniuria durat. \\ Tolle mali testes: levius mala nostra feremus. \\ Vir constans quicquid coepit complere laborat. \\ Saepe labor siccat lacrimas et gaudia fundit. \\ Iniustus, qui sola putat proba, quae facit ipse. \\ Tristibus adficiar gravius, si laeta recordor. \\ Omne manu factum consumit longa vetustas. \\ Quid cautus cavea, aliena exempla docebunt. \\ Hud ullum tempus vanitas simulata manebit. \\ Ne crede amissum, quicquid reparare licebit. \\ Condit fercla fames; plenis insuavia cuncta. \\ Doctrina est fructus dulcis radicis amarae. \\ 
        \pagebreak 
     \marginpar{[181]} \begin{center} \textbf{4 716, 46—70} \end{center}Vti ut divitiis bonitas, sic luxus abuti. \\ Non pecces tum, cum peccare impune licebit. \\ Spes facit inlecebram visuque libido movetur. \\ Ver libens dices, quamquam sint aspera dictu. \\ Tristis adest maeror, si cesset laeta voluntas. \\ Non facit ipse aeger, quod sanus suaserit aegro. \\ Absentum causas contra maledicta tuere. \\ psos absentes inimicos laedere noli. \\ Vlcus proserpit, quod stulta silentia celant. \\ Qui vinci sese patitur pro tempore, vincit. \\ Nemo ita despectus, quin possit laedere laesus. \\ Cum accusas alium, propriam prius inspice vitam. \\ Nemo reum faciet, qui vult dici sibi verum. \\ A deo expectemus longevam ducere vitam. \\ Votis concessam scelus est odisse senectam, \\ Quicquid inoptatum cadit, hoc homo corrigat arte. \\ Consilii regimen virtuti corporis adde. \\ Vincere velle tuos perquam victoria turpis. \\ Cum vitia alterius satis acri lumine cernas \\ Nec tua prospicias, fis verus crimine caecus. \\ Nonnumquam vultu tegitur mens taetra sereno. \\ Durum etiam facilem facit adsuetudo laborem. \\ Quisque miser casu alterius solatia sumat. \\ Si piget admissi, committere parce pigenda. \\ Robur confirmat labor, at longa otia solvunt. \\ 
        \pagebreak 
     \marginpar{[182]} \begin{center} \textbf{C. 716, 71 81} \end{center}Vt niteat virtus, absit rubigo quietis. \\ Mitte arcana deo caelumque requirere quid sit. \\ Quod nocet, interdum si prodest, ferre memento; \\ Dulcis enim labor est, cum fructu ferre laborem. \\ Laetandum est vita, nullius morte dolendum est: 7 \\ Cur etenim doleas, a quo dolor ipse recessit? \\ Dum speras, servis; cum sint data praemia, †saevis. \\ Ille nocet gravius, quem tu contemnere possis. \\ Quod metuis, cumulas, si velas crimine crimen. \\ Suffragium laudis quod fert malus, hoc bonus odit. 0 \\ Magni magna parant, modici breviora laborant. \\ Haec uquc sut n E, utc HiEfdeberti sut in \\ Paris. \\ Formula vivendi praesto est tibi: pauca loquaris, \\ Plurima fac. sit utrisque comes modus, utile, pulchrum; \\ Obsequiis instes: ea pro te munera poscent. \\ Sobrius a mensis, a lecto surge pudicus. \\ Stans casum metuas, speres prostratus; et illum, \\ Quem colis in titulis, miserum abiectumque tuere. \\ Vt decet et prodest, et amabis et oderis idem. \\ 
        \pagebreak 
     \marginpar{[183]} \begin{center} \textbf{C. 717 718, 18} \end{center}B. . \\ 
      \end{verse}
  
            \subsection*{717}
      \begin{verse}
      B. IV 178. \\ Doctiloqui carmen ructatum fonte Maronis \\ Bis senis numero florens se milibus explet \\ Et super hos octingentis septem quadraginta \\ Versibus adiunctis concluditur omne volumen, \\ Quod cecinit quondam variato fulmine linguae: \\ Pastores Cererem Blacchum pecus et bona mellis, \\ Naufragium flammas errores vulnera ludos, \\ Tartara, post Latium, sic Teucros bella frementes, \\ Hostibus Ascanium lutulis in castra relictum, \\ o Proelia post reditum, devictam Marte Camillam, \\ Et sua cedentem profugo conubia Turnum. \\ 
      \end{verse}
  
            \subsection*{718}
      \begin{verse}
      B. V 11. . 1055. \\ Ad Oceanum \\ B. III 165. \\ Vndarum rector, genitor maris, arbiter orbis, \\ Oceane o placido conplectens omnia  \lbrack gyro \rbrack , \\ Tu legem terris moderato limite signas, \\ Tu pelagus quodcumque facis fontesque lacusque. \\ Flumina quin etiam te norunt omnia patrem; \\ Te potant nubes, ut reddant frugibus imbres. \\ Cyaneoque sinu caeli tu diceris oras \\ Prtibus ex cunctis inmenso cingere nexu. \\ 
        \pagebreak 
    \begin{center} \textbf{C. 718, 92s. C. 719, 12} \end{center} \marginpar{[184]} Tu fessos Phoebi reficis si gurgite currus \\ Exhaustisque die radiis alimenta ministras, \\ Gentibus ut clarum referat lux aurea solem, \\ Si mare, si terras caelum mundumque gubernas, \\ Me quoque cunctorum partem, venerabilis, audi. \\ Alme parens rerum, supplex precor. ergo carinam \\ Conserves, ubicumque tuo committere ponto \\ Hanc animam, transire fretum, discurrere cursus \\ Aequoris horrisoni Sortis fera iussa iubebunt; \\ Tende favens glaucum per levia dorsa profundum, \\ Ac tantum tremulo crispentur caerula motu, \\ Quantum vela ferant, quantum sinat otia remis. \\ Sint fluctus, celerem valeant qui pellere puppem, \\ Quos numerare libens possim, quos cernere laetus: \\ Servet inoffensam laterum par linea libram, \\ Et sulcante viam rostro submurmuret unda. \\ Da pater, ut tute liceat transmittere cursum, \\ Perfer ad optatos securo in litore portus \\ Me comitesque meos. quod cum permiseris esse, \\ Reddam quas potero pleno pro mu \lbrack nere \rbrack  grates. \\ B. M. B. \\ C. Scr. eccl. V. \\ 
      \end{verse}
  
            \subsection*{7193}
      \begin{verse}
      X 310. XVI 315. \\ Omnipotens genitor tandem miseratus ab alto, \\ Postquam cuncta dedit caelo constare sereno, \\ 
        \pagebreak 
    \begin{center} \textbf{C. 719, 325} \end{center} \marginpar{[185]} Omnibus in terris divinum aspirat amorem, \\ Semper honore pio nomen Natique Patrisque \\ Ornare et canere paribusque in regna vocari \\ Auspiciis; huic progeniem virtute futuram \\ Egregiam, et totum quae legibus occupet orbem. \\ Ne tamen in terris mortalia pectora turbet \\ Ignotum numen, Deus aethere missus ab alto \\ 0 Mortalis visus potuit quantusque videri. \\ Virgo matura fuit iam plenis nubilis annis, \\ Cui genus a proavis ingens nomenque decusque. \\ Intemerata toris talem se laeta ferebat \\ Casta pudicitiam miro servabat amore. \\ Hluic se forma dei (caelo demissus ab alto \\ Spiritus intus alit) et casto se corpore miscet. \\ Ante tamen dubiam dictis solatur amicis: \\ ‘Alma parens, mundi dominum paritura potentem \\ (Nam te digna manent generis cunabula sancti), \\ Vade’ ait ‘o felix nati pietate, quocumque vocaris \\ Auspiciis manifesta novisl hic vertitur ordo \\ Iuius in adventu: fides et fama perennis.’ \\ Dixerat: illa pavens, oculos suffusa nitentis, \\ Suspirans imoque trahens a pectore vocem \\ Virgo refert: ‘haud equidem tali me dignor honore; \\ 
        \pagebreak 
     \marginpar{[186]} \begin{center} \textbf{C. 719, 26 55} \end{center}Non opis est nostrae nec fas, nec coniugis umquam \\ Praetendi taedas aut haec in foedera veni. \\ Sed post iussa Deum nihil est, quod dicta recusem. \\ Accipio agnoscoque libens: sequor omina tanta \\ Promissisque Patris exsequar caelestia dona, \\ Admiranda Dei tantarum munera laudum.’ \\ Panditur interea domus omnipotentis Olympi \\ Sideream in sedem, terras unde arduus omnis \\ Aspicit et natum verbis conpellat amicis: \\ ‘Nate, meae vires, mea magna potentia solus, \\ Nate, mihi quem nulla dies ab origine rerum \\ Dissimilem arguerit, comitem complector in omnis. \\ Te sine nil altum mens inchoat: omnia mecum \\ Aeternis regis imperiis; et quidquid ubique est, \\ Nulla meis sine te quaeretur gloria rebus: \\ Omnia sub pedibus, qua sol utrumque recurrens \\ Aspicit oceanum, vertique regique videbmnt. \\ Quae tibi polliceor (neque est te fallere quicquam), \\ Haee tibi semper erunt vatum praedicta priorum, \\ Nec mea iam mutata loco sententia cedit. \\ Nascere praeque diem veniens age, lucifer, almum; \\ Nascere, quo toto surgat gens aurea mndo. \\ Vnde etiam magnus caelorum nascitur ordo. \\ Nascere, ut incipiant magni procedere menses, \\ Ne maneant terris priscae vestigia fraudis, \\ Prospera venturo laetentur ut omnia saeclo. \\ Adgredere o magnos (aderit iam tempus) honores: \\ Aspera tum positis mitescent saecula bellis, \\ Pacatumque reges patriis virtutibus orbem.’ \\ Haut mora: continuo Patris praecepta facessit, \\ 
        \pagebreak 
    \begin{center} \textbf{C. 179, 56 83} \end{center} \marginpar{[187]} Aethere se mittit figitque in virgine vultus, \\ Nec mortale tuens, afflata est lumine quando \\ Iam propiore Dei . . nam tempore eodem \\ Matri longa decem tulerunt fastidia menses, \\ Et nova progenies mox clara in luce refulsit. \\ Mox etiam magni processit numinis astrum, \\ Stella facem ducens multa cum luce cucurrit. \\ ‘Ile dies primus lei primsque salutis \\ Monstrat iter vobis ad eum; quem semper acerbum, \\ 6s Semper honoratum cuncti celebrate faventes. \\ Annua vota tamen noctem non amplius unam \\ Haut segnes vigilate, viri, dapibusque futuris \\ Luce palam cumulate piis altaria donis. \\ Hac vestri maneant in religione nepotes. \\ 0 Iamque egomet Patris sedes arcemque reviso. \\ Accipite ergo animis atque haec mea figite dicta, \\ Ore favete omnes et huc advertite mentem. \\  \lbrack Hanc \rbrack  e diverso sedem quotiens venietis in unam, \\ Vndique collecti pacem laudate frequentes. \\ Cogite concilium, coeant in foedera dextrae, \\ Qua datur pacis solum inviolabile pignus. \\ Discite iustitiam, aeterna in pace futurae \\ Concordes animae, si non inrita dicta putatis. \\ Nulla dies usquam memori vos eximet aevo: \\ o Mortalem eripiam formam, et praemia reddam \\ Fortunatorum nemorum sedesque beatas. \\ Non eritis regno indecores, nec vestra feretur \\ Fama levis, mecum pariter considere regnis. \\ 
        \pagebreak 
     \marginpar{[188]} \begin{center} \textbf{, 719, 4 —111} \end{center}Vrbem quam statuo, vestra est: intrare licebit. \\ Nusquam abero, et tutos patrio vos limine sistam: s \\ Idem venturos tollemus in astra nepotes. \\ Quae vero nunc quoque vobis, dum vita manebit, \\ Praemia digna feram? non vobis numine nostro \\ Divitis uber agri rerumque opulentia deerit. \\ Fundit humo facilem victum iustissima tellus \\ Proventuque onerat sulcos atque horrea vincit, \\ Floret ager, spumat plenis vindemia labris, \\ Exuberant fets ramos frondentis olivae, \\ Quotque in flore novo pomis se fertilis arbor \\ Induerit, totidem autumno matura tenebit. \\ Non liquidi gregibus fontes, non gramina deerunt, \\ Et quantum longis carpent armenta diebus, \\ Exigua tantum gelidus ros nocte reponet. \\ Iaec sunt, quae nostra deceat vos voce moneri. \\ Vivite felices et condita mente tenete.’ \\ 100 \\ laec ubi dicta dedit, mox sese attollit in auras. \\ Suspiciens caelum caput inter nubila condit. \\ Atque ita discedens terris animisque suorum \\ Concretam exemit labem purumque relinquit \\ Aetherium sensum atque aurai simplicis ignem. \\ 105 \\ Ex illo celebratus houos, laetique minores \\ Servavere diem, atque haec pia sacra quotannis \\ Matres atque viri, pueri innuptaeque puellae \\ Carminibus celebrant paterisque altaria libant. \\ Ast ego qui cecini magnum et mirabile numen, \\ Haec eadem gentique meae generique manebunt. \\ 
        \pagebreak 
    \begin{center} \textbf{C. 719a, 12} \end{center} \marginpar{[189]} ta \\ B. M. B. \\ C cr. eeel. V. \\ 
      \end{verse}
  
            \subsection*{}
      \begin{verse}
      \poemtitle{POMONII}XVI 609. \\ Versus ad gratiam domini \\ Inducit duas personas, Meliboeum \\ et Tityrum. \\ Tityre, tu patulae recubans sub tegmine fagi, \\ Nescio qua praeter solitum dulcedine laetus, \\ Fortunate senex! hic inter flumina nota \\ Et fontis sacros deductos dicere versus \\ Et cantare paras divino carmine, pastor, \\ Formonsi pecoris custos, formonsior ipse. \\ rr, Non incerta cano vatum praedicta priorum. \\ An quicquam nobis tali sit munere maius? \\ 0 Meliboee, deus haec nobis otia fecit. \\ o Namque erit ille mihi semper deus atque homiuum rex, \\ Omnipotens genitor, rerum cui summa potestas; \\ Quem qui scire velit, divinum aspiret amorem. \\ Haut ignota loquor, totum quae sparsa per orbem. \\ Ipsi laetitia voces ad sidera iactant \\ Intonsi montes, ipsae iam carmina rupes, \\ Ipsa sonant arbusta: deum namque ire per omnes \\ Terrasque tractusque maris caelumque profundum \\ Ne dubita nam vera vides qui foedere certo \\ Aeternis regit imperiis et temperat iras. \\ Ni faciat, maria ac terras nox incubat atra. \\ MEL. Felix, qui potuit rerum cognoscere causas. \\ 
        \pagebreak 
    \begin{center} \textbf{C 719a, 2252} \end{center} \marginpar{[190]} Namque fatebor enim genus a quo principe \\ nostrum, \\ Audierat Stimicon; laudavit carmina nobis. \\ Sis felix! nam te maioribus ire per altum \\ Auspiciis manifesta fides pro landibus istis. \\ TT. Accipe daque fidem: neque est ignobile carmen. \\ Maior agit deus atque opera ad maiora remittit. \\ Vnus qui nobis cunctando restituit rem, \\ Ille operum custos, hominum rerumque repertor, \\ Quo sine nil altum mens inchoat, ipse volutat \\ Quae sint, quae fuerint, quae mox ventura trahantur. \\ His etenim signis atque haec exempla secuti \\ Esse animas partem divinae mentis et haustus \\ Aetherios dixere, quia sit divinitus illis \\ Ingenium. Quamvis angusti terminus aevi \\ Terrenique hebetant artus moribundaque membra, \\ At genus inmortale manet: ne quaere doceri. \\ Igneus est ollis vigor et caelestis origo, \\ Et cum frigida mors anima seduxerit artus, \\ Has omnis, ubi mille rotam volvere per annos, \\ Tempora dinumerans deus evocat agmine magno. \\ Reddunt se totidem facies terraeque dehiscunt; \\ Sed revocare gradum superasque evadere ad auras, \\ Hoc virtutis opus, terras temptare repostas \\ Sideream in sedem atque alto succedere caelo. \\ ME. Tityre, tamne aliquas ad caelum hincire putandum est \\ Sublimis animas iterumque ad tarda reverti \\ Corpora? nos alia ex aliis in fata vocamur? \\ Inmortalis ego? pertemptant gaudia pectus, \\ Si modo quod memoras factum contingere possit. \\ . Ne dubita, nulla fati quod lege tenetur: \\ Crede deo nam vera vides ; sine posse parentem, \\ 
        \pagebreak 
     \marginpar{[191]} \begin{center} \textbf{C. 719a, 5383} \end{center}Quod minime reris! fato prudentia maior. \\ MEL. Credo equidem, nec vana fides. Quis talia demens \\ Abnuat? et me victusque volensque remitto. \\ Stultus ego parvis componere magna solebam; \\ Nec mea iam mutata loco sententia cedit. \\ Vnum oro: doceas iter et sacra ostia pandas, \\ Quidve sequens tantos possim superare labores. \\  \lbrack TT Dicam equidem nec te suspensum, nate, tenebo, \\ Et quo quemque modo fugiasque ferasque laborem. \\ Aude, hospes, contemnere opes: via prima salutis. \\ Intemerata fides et mens sibi conscia recti \\ Praemia digna ferunt; freti pietate per ignem \\ s Invenere viam: requies ea certa laborum. \\ Invitant croceis halantes floribus horti \\ Fortunatorum nemorum sedesque beatae \\ Semper erunt, quorum melior sententia menti: \\ His locus urbis erit, divini gloria ruris. \\ o Nam qui divitiis soli incubuere repertis, \\ Distulit in seram commissa piacula mortem. \\ Ausi omnes immane nefas ausoque potiti \\ Vrgentur poenis. Quam vellent aethere in alto \\ Omnia et superas caeli venisse sub auras! \\ MEL. Quae tibi, quae tali reddam pro carmine dona? \\ c Non opis est nostrae; nomen tollemus ad astra, \\ Tityre. Discussae umbrae et lux reddita menti. \\  \lbrack TT Non haec humanis opibus, non arte magistra \\ Proveniunt: quae sit poteris cognoscere virtus. \\ Ni refugis, prima repetens ab origine pergam. \\ MEL. Immo age et a prima dic, hospes, origine nobis: \\ Accipio agnoscoque libens ut verba parentis. \\ r Accipe: prisca fides facto, sed fama perennis. \\ 
        \pagebreak 
     \marginpar{[192]} \begin{center} \textbf{C. 719a, 84—116} \end{center}Nunc canere incipiam, quoniam convenimus ambo \\ Montibus in nostris: referunt ad sidera valles. \\ Magnus ab integro saeclorum nascitur ordo. \\ Maius opus moveo: laudes et facta parentis. \\ Nam neque erant astrorum ignes nec lucidus aethra \\ Siderea polus, et nox obscura tenebat. \\ Tum pater omnipotens, rebus iam luce retectis, \\ Aera dimovit tenebrosum et dispulit umbras. \\ Principio caelum et terras solemque cadentem \\ Lucentemque globum lunae camposque liquentes, \\ Noctis iter, stellis numeros et nomina fecit, \\ Vnde hominum pecudumque genus vitaeque volantum s \\ Et quae marmoreo fert monstra sub aequore pontus. \\ Et medium luci atque umbris iam dividit orbem \\ Temporibusque parem diversis quattuor annum. \\ Nec torpere gravi passus sua regna veterno \\ Movit agros, curis acuens mortalia corda, \\ 100 \\ Vt varias usus meditando extunderet artes. \\ Hic genus indocile ac dispersum montibus altis \\ Composuit legemque dedit. dicione tenebat. \\ Hinc populum late regem aevoque superbum \\ Venturum excidio docuit post exitus ingens \\ 105 \\ Victor ab Aurorae populis et litore rubro. \\ Tunc victu revocant vires, caelestia dona, \\ Deterior donec paulatim ac decolor aetas \\ Et belli rabies et amor successit habendi. \\ Regnorum inmemores turpique cupidine capti! \\ 110 \\ Tunc variae inludunt pestes: malus abstulit error \\ Aegyptum viresque Orientis: miranda videntur \\ Omnigenumque deum monstra et latrator Anubis. \\ Quid delubra iuvant simulacraque luce carentum? \\ Non tali auxilio nec defensoribus istis \\ 115 \\ Tempus eet cum vestra dies volventibus annis \\ 
        \pagebreak 
    \begin{center} \textbf{C. 719, 11732. C. 719, 1—2} \end{center} \marginpar{[193]} Verba redarguerit, poena commissa luetis. \\ Quin potius pacem aeternam †et tanti muneris \\ Cuncti obtestemur? Haec ara tuebitur omnis. \\ o His actis aliud genitor secum ipse volutat. \\ Quo vitam dedit aeternam, quo mortis adempta est \\ Condicio, et caelo tandem miseratus ab alto est. \\ Respicit humanos pietas antiqua labores \\ Exitiis positura modum, responsa dabantur \\ 1s Fida satis: manifesta fides secreta parentis. \\ pse haec ferre iubet celeris mandata per auras \\ Interpres divum; spirantemque adfore verbis \\ Seraque terrifici cecinerunt omina vates; \\ Namque fore inlustrem dictis factisque canebant. \\ O quam te memorem, virgo cui mentem animumque \\ Semine ab aetherio superis concessit ab oris \\ Omnipotens.’ \\ 
      \end{verse}
  
            \subsection*{719}
      \begin{verse}
      B. M. \\ \poemtitle{rBERIANI \rbrack }B. III 265. \\ Incipit versus Soeratis philosophi. \\ Aurum quod nigri manes, quod turbida versant \\ Flumina, quod duris extorsit poena metallis, \\ 
        \pagebreak 
    \begin{center} \textbf{C. 719 326} \end{center} \marginpar{[194]} Aurum, quo pretio reserantur limina Ditis, \\ Quo Stygii regina poli Proserpina gaudet, \\ Aurum, quod penetrat thalamos rumpitque pudorem, s \\ Qua ductus saepe inlecebra micat impius ensis . . . \\ In gremium Danaes non auro fluxit adulter \\ Mentitus pretio faciem fulvoque veneno? \\ Non Polydorum hospes saevo necat incitus auro? \\ Altrix infelix, sub quo custode pericli \\ Commendas natum cui regia pignora credis? \\ Fit tutor pueri, ft custos sanguinis aurum! \\ Inmitis nidos coluber custodiet ante \\ Et catulos fetae poterunt servare leaenae. \\ Sic etiam ut Troiam popularet Dorica pubes, \\ Aurum causa fuit, pretium dignissim merces. \\ Infami probro palmam convendit adulter. \\ Denique cernamus, quos aurum servit in usus. \\ Auro emitur facinus, pudor almus venditur auro, \\ Tum patria  \lbrack atlque parens, leges pietasque fidesque: o \\ Omne nefas auro tegitur, fas proditur auro. \\ Porro hinc Pactolus, porro fluat et niger Hermus! \\ Aurum, res gladii, furor amens, ardor avarus, \\ Te celent semper vada turbida, te vada nigra, \\ Te tellus mersum premat infera, te sibi nasci \\ Tartareus cupiat Phlegethon Styvgiaeque paludes. \\ 
        \pagebreak 
     \marginpar{[195]} \begin{center} \textbf{C. 71, 272s. C. 719c, 1—16} \end{center}Inter liventes pereat tibi fulgor arenas, \\ Ne post ad superos redeat famis aurea puros! \\ tc \\ B. M. \\ SIDONI ubdiconi \\  \lbrack Lucani Belli eivilis \rbrack  \\ Arumentum libri seeundi \\ Anxiam  \lbrack at \rbrack  interea plebem luctusque dolorque \\ Inpellunt maestas in caelum fundere voces. \\ Consilium capiunt una lrutusque Catoque, \\ Caesaris an partes an Magni signa sequantur. \\ Sed Caesar veniens adversos proterit armis. \\ Interea Magnus Campanam tendit ad urbem \\ lortaturque suos veneranda voce cobhortes, \\ Destinat et natum, reges ut cogat in arma. \\ Brundusii portum Caesar munimine claudit, \\ 0 Pompeiusque fgit perruptis nocte catenis. \\ Argumentum libri \\ B. M. \\ Lentulus affatur maerentem voce senatum. \\ Consulitur dubio Foebi de Marte sacerdos. \\ nterea multo fessus iam membra labore \\ Miles deposcit modicae sibi tempora vitae, \\ Quem solita spernit mentis constantia Caesar. \\ Inde petit tutus desertae moenia lomae \\ 
        \pagebreak 
    \begin{center} \textbf{C. 719c, 1720. C. 719d. 115} \end{center} \marginpar{[196]} Brundusiumque celer tenuitque Ceraunia velis. \\ Hinc fatis fretus parvam conscendere puppem \\ Ausus per noctem solo comitatus Amicla. \\ Corneliam Magno tutatur insula Lesbos. \\ B. M. B. \\ 
      \end{verse}
  
            \subsection*{719}
      \begin{verse}
      C. E. V. XVI568. \\ Romulidum ductor, clari lux altera solis, \\ Eoa qui regna regis moderamine iusto, \\ Spes orbis fratrisque decus: dignare Maronem \\ Mutatum in melius divino agnoscere sensu, \\ Scribendum famulo qui iusseras. hic tibi mundi \\ Principium formamque poli hominemque creatum \\ Expediet limo, hic Christi proferet ortum, \\ Insidias regis, magorum praemia, doctos \\ Discipulos pelagique minas gressumque per aequor: \\ Hic fractum famulare iuum vitamque reductam \\ Vninus crucis auxilio, reditumque sepultae \\ Mortis et ascensum pariter sua regna petentis. \\ laec relegas servesque diu tradasque minori \\ Arcadio, haec ille suo eneri; haec tua semper \\ Accipiat doceatque suos augusta propago. \\ 
        \pagebreak 
     \marginpar{[197]} \begin{center} \textbf{C 719, 122} \end{center}
      \end{verse}
  
            \subsection*{719}
      \begin{verse}
      B. M. B. \\ Quod natum Phoebus docuit, quod Chiron Achillem, \\ Quod didicere olim Podalirius atque Machaon \\ A genitore suo, qui quondam versus in anguem \\ Templa Palatinae subiit sublimia Romae: \\ Quod Cous docuit senior quodque Abdera suasit, \\ Quod logos aut methodos simplexque empirica pangit \\ Hc liber iste tenet diverso e dogmate sumptum. \\ Namque salutiferas disponit pagina curas. \\ Istic repperies per nomina perque medellas \\ o Descriptas species et pondera mensurarum \\ Congrua, quae sapiens sumes moderamine certo. \\ Ne fallare, vide, neu quae sunt parta saluti \\ Vertat in exitium non sollers cura medentis. \\ Sume igitur medicos pro tempore proque labore \\ Aetatisque habitu summa ratione paratos, \\ Gramine seu malis aegro praestare medellam \\ Carmine seu potius namque est res certa saluti \\ Carmen ab occultis tribuens miracula verbis. \\ Quae curis hominnm physicorum inventa pararunt, \\ o Quaeque suis natura bonis terraque marique \\ Edidit, illa suis altrix simul atque creatrix \\ Fetibus omnigenis, quos parturit, ergo salubres \\ 
        \pagebreak 
    \begin{center} \textbf{C. 7 19, 23—46} \end{center} \marginpar{[198]} Suggerit inpensas ponto et tellure creatas, \\ Angue, fera, pecude et fruge, alite, murice, pisce, \\ Lacte, mero, pomis, lymphis, sale, melle et olivo, \\ Sucis, unguinibns, taedis, pice, sulfure, cera, \\ Polline, farre, fabis, lino, scobe, vellere, cornu, \\ Bacis et balanis, lignis, carbone, favilla, \\ Floribus et variis herbis, holere atque metallis, \\ Sandyce et creta, spimitho, pumice, gpso, \\ Cadmia, chalciti, chalcautho, calce, camino, \\ Cassitero † molli, lepide, cypro atque atramento. \\ Prome etiam (seu tunde prius seu contere gyro). \\ Quod viride hortus habet, vel quod carnaria siccum, \\ Alia serpyllumque, herbas thymbramque salubrem \\ Brassicaque et raphanos ac longis intiba fibris \\ Et mentam et sinapi coriandrum prototomumque, \\ Erucam atque apium, malvam betamque salubrem \\ Rutamque et nasturcum et amara absinthia misce, \\ Puleiumque potens nec non et lene cyminum. \\ Palmula nec desint ldumes nec pruna Damasci, \\ Quae cum multiplici contriveris orbe terendo, \\ I patinis excocta dabis aut grandibus ollis \\ (Verum adoperta coques, ne fumida iura saporem \\ Corruptum reddant, quae mox fastidiat aeger. \\ Adde et aromaticas species, quas mittit Eous, \\ 
        \pagebreak 
     \marginpar{[199]} \begin{center} \textbf{C. 719, 789} \end{center}Vel quae ludaicis fragrant bene condita capsis, \\ Tus, costum, folium, myrram, styracem, crocomagma, \\ Aspalathum, gallam, elleborum nigrumque bitumen \\ Et nardum et casias et amoma et cinnama rara, \\ Balsama, peucedanum, spicam, crocum atque bidellam, \\ Irim, castoreum, scillam, opium, pnaceam, \\ Resinam, lepidum, euforbium, git atque pyretrum \\ Eingiber et calidum, mordax piper, et laser algens, \\ Agaricum, asarumque potens, aloen, aconitum, \\ Galbana, sandaracam, samsucum, sporon, alumen, \\ Acaciam, propolen, adarchen, cnicon, acanthum, \\ Andrachnen, acoron, opopanaca, pompholygemque, \\ Cyperum, ladanum, sagapenon et tragacanthon, \\ o Scammoniam, †cypen, malabathron, ammoniacon. \\ Denique repperies istic, quod lucis in ortu \\ Indus Arabs Serus Perses divesque Sabaeus \\ Vicino sub sole legunt, quod praebet Orontes, \\ Eximium inoto mittit quod Nilus ab ortu, \\ 6s Decerptum foliis, ramo, cute, cortice, virga, \\ Quodque ab ldumaeis vectum seplasia vendunt \\ Ft quidquid confert medicis Lagea cataplus. \\ Iaec quicumque leges, poteris discernere tecum \\ Agnoscenda magis sive exercenda rearis. \\ 
        \pagebreak 
     \marginpar{[200]} \begin{center} \textbf{C. 719, 7078} \end{center}Quisque tame nostrum hoc studium dignabere, quaeso o \\ Praestes inudicium purum mentemque benignam. \\ Sic tua perpetuo vegetentur membra vigore \\ Et peragas placidam per multa decennia vitam. \\ Sic non incuses validam placidamque senectam \\ Nec tibi sit medicis opus umquam nec tibi casus \\ Aut morbus pariant ullum quandoque dolorem, \\ Sed procul a curis et sano corpore vivas, \\ Quotque hic sunt versus, tot agant tua tempora Ilanos. \\ 
        \pagebreak 
    
      \end{verse}
  
            \subsection*{}
      \begin{verse}
      \poemtitle{CARMINA}
      \end{verse}
  
            \subsection*{}
      \begin{verse}
      \poemtitle{CODICVM SAECVLI X}
        \pagebreak 
     \marginpar{[419]} B. M. \\ 
      \end{verse}
  
            \subsection*{}
      \begin{verse}
      \poemtitle{OCTAVANI AVGVSTI}B. IV 111. \\ Convivae, tetricas hodie secludite curas: \\ Ne maculent niveum nubila corda diem. \\ Omnia sollicitae vertantur murmura mentis, \\ Vt vacet indomitum pectus amicitiae. \\ Non semper gaudere licet: fugit hora: iocemur. \\ Difficile est fatis subripuisse diem. \\ 
      \end{verse}
  
            \subsection*{720}
      \begin{verse}
      B. V. 1I1B. M. 234. \\ Pontieon \\ B. III 172. \\ Tethya marmoreo fecundam pandere ponto \\ Et salis aequoreas spirantis mole catervas, \\ Quaeque sub aestifluis Thetis umida continet antris, \\ Coeptantem, Venus alma, fove, quae semine caeli \\ 
        \pagebreak 
    C. 720. 522. C. 720a, \\ Parturiente salo divini germinis aestu, \\ Spumea purpureis dum sanguinat unda profundis, \\ Nasceris e pelago, placido dea prosata mundo! \\ Nam cum prima foret rebus natura creandis \\ In foedus conexa suum, ne staret inerti \\ Machina mole vacans, tibi primum candidus aether o \\ Astrigeram faciem nitido gemmavit Olympo. \\ Te fecunda sinu Tellus amplexa resedit \\ Ponderibus fundata suis, elementaque iussa \\ Aeternas servare vices. tu fetibus auges \\ Cuncta suis, totus pariter tibi parturit orbis. \\ Quare, diva, precor, quoniam tua munera parvo \\ Ausus calle sequor, vitreo de gurgite vultus \\ Dextera prome pios et numine laeta sereno \\ Pierias age pande vias. da Nerea molli \\ Pacatum gandere freto votisque litata \\ Fac saltem primas pelai libemus harenas. \\ Vos quoque, qui resono colitis cava Tempea coetu \\ B. II 1. M. 859 \\ 
      \end{verse}
  
            \subsection*{720a}
      \begin{verse}
      B. IV 177 \\ Pastorum Musam vario certamine promit. \\ Ruris item docili culturam carmine monstrat. \\ 
        \pagebreak 
     \marginpar{[205]} \begin{center} \textbf{C. 720a. 3—17. C. 720b, 1—2} \end{center}Arboribus vites, prolem et iungit olivae \\ Pastorumque Palen et curam tradit equorum; \\ Tunc apium seriem, mellis et doua recenset. \\ Aeneas profugus intrat Carthaginis oras. \\ Continno series narratur Troica belli. \\ Tertius et complet narrantis †ordine gesta. \\ Ardet amans Dido fatum sortita supremum. \\ Quintus habet tumuli varia spectacula patris. \\ Infernos Manes et Ditis regna pererrat. \\ Aeneas Latium, Italas simul intrat in oras. \\ Intonat hic bellum tecti de culmine Turnus. \\ Euryalum et Nisum deflet cum matre iuventus. \\ Pllantis exitium et Turni deluditur orsus \\ Euanderque simul multorum funera deflet. \\ Turni vita fugit infernas maesta sub umbras. \\ 
      \end{verse}
  
            \subsection*{720b}
      \begin{verse}
      Versu DAMASI papae \\ . B M. B. \\ Ad quendam fratrem corriptendum \\ Tityre, tu fido recubans sub tegmine Christi \\ Divinos apices sacro modularis in ore, \\ 
        \pagebreak 
     \marginpar{[206]} \begin{center} \textbf{C. 720, 30o. C. 721 722, 18} \end{center}Non falsas fabulas studio meditaris inani. \\ Illis nam capitur felicis gloria vitae, \\ Istis succedent poenae sine fine perennes. \\ Vnde cave, frater, vanis te subdere curis, \\ Inferni rapiant miserum ne tartara taetri. \\ Quin potius sacras animo spirare memento \\ Scripturas, dapibus satiant quae pectora castis. \\ Te Domini salvum conservet gratia semper! \\ B. IV206.M.1163. \\ 1. \\ B. III 270. \\ Vivere post obitum vatem vis nosse, viator? \\ Quod legis, ecce loquor: vox tua nempe mea est. \\ B.IV206.M.131. \\ 
      \end{verse}
  
            \subsection*{00}
      \begin{verse}
      B. III 270. \\ Nymphius aeterno devinctus membra sopore \\ Hic situs est, caelo mens pia perfruitur. \\ Mens videt astra, quies tmuli conplectitur artus, \\ Calcavit tristes sancta fldes tenebras. \\ Te tua pro meritis virtutis ad astra vehebat \\ Intuleratque alto debita fama polo. \\ 
        \pagebreak 
    \begin{center} \textbf{C. 722. 9 2. C. 723, 1—3} \end{center} \marginpar{[207]} Immortalis eris, nam multa laude vigebit \\ Vivax venturos gloria per populos. \\ Te coluit proprium provincia cuncta parentem, \\ I0 Optabant vitam publica vota tuam. \\ Exeruere tuo quondam data munera sumptu \\ Plaudentis populi gaudia per cuneos. \\ Concilium procerum per te patria alma vocavit \\ Seque tuo duxit sanctius ore loqui. \\ 1 Publicus orbatas modo luctus conficit urbes \\ Confusique sedent, anxia turba, patres, \\ Vt capite erepto torpentia membra rigescunt, \\ Vt grex amisso principe maeret iners. \\ Parva tibi, coniux, magni solacia luctus \\ Hunc tumuli titulum maesta Serena dicat. \\ Haec individui semper comes addita fulcri \\ Vnanimam tibi se lustra per octo dedit. \\ Dulcis vita fuit tecum: comes anxia lucem \\ Aeternam sperans hanc cupit esse brevem. \\ 
      \end{verse}
  
            \subsection*{723}
      \begin{verse}
      \poemtitle{CLAVDII}\poemtitle{De luna}B. V 15. M. 554. \\ B. III 163. \\ Luna decus mundi, magni pars maxima caeli, \\ Luna iugum Solis, splendor vagus, ignis et humor, \\ Luna parens mensum numerosa prole renascens! \\ 
        \pagebreak 
     \marginpar{[208]} \begin{center} \textbf{C. 723, 16. C. 72a, 15.} \end{center}Tu biiugos stellante polo sub Sole gubernas, \\ Te redeunte dies fraternas colligit boras, \\ Te pater Oceanus renovato respicit axe, \\ Te spirant terrae, tu vinclis Tartara cingis, \\ Tu sistro renovas brumam, tu cymbala quassas, \\ Isis, Luna, Choris, Caelestis Iuno, Cybebe! \\ Alternis tu nomen agis sub mense diebus \\ Et rursum renovas alterni lumina mensis. \\ Tunc minor es, cum plena venis; tunc plena resurgens, \\ Cum minor es: crescis semper, cum deficis orbe. \\ Iuc ades et nostris precibus dea blandior esto \\ Luciferisque iugis concordes siste iuvencas, \\ Vt volvat ortuna rotam, qua prospera currant. \\ B. M. B. \\ Dalmatiane, iugi, Caesar, quem terra triumpho \\ Excolit imperium cuncta tremendo tuum, \\ Auribus hunc audi sacris oculisque beatis \\ Aspice gestantem pessima facta librum. \\ Hoe tibi sit votum: laus perpes, sacra potestas, \\ 
        \pagebreak 
     \marginpar{[209]} \begin{center} \textbf{C. 72a, 6— 34} \end{center}Ecclesiae sanctae iura tenendo sacra. \\ Ipsius Augusto sulcando vomere sulcos \\ Constantinus eris maior in imperio. \\ Ast ubi vel quando syuodos complectere sacras \\ Malebant, proprio hoc erat in libitu. \\ Tempora praeteritis retro defluxa diebus \\ Testantur, divos has peregisse solos. \\ Ad quas papa †suos currebat et ipse Romanus, \\ Transmiserat quorum iussa tenendo sacra. \\ Hoc Nicena canit quoque Constantinus in ipsa, \\ Efulsit divus papa simul Iulius. \\ Arrius hac cecidit tanto librante magistro, \\ A qua sancta caput traxit amore fides. \\ Calcedona sequens hanc Marcianus et auctor \\ Testatur feriens Eutychiana labra. \\ Non tacet hoc divo fulgens Ephesena tropheo, \\ Theodosii calcans dogmata Nestorii. \\ . . mnana † deinde hoc pandunt iura per orbem \\ Legibus extensa prorsus ntrisque sacra. \\  \lbrack Quod \rbrack si cuncta canunt synodi per dogmata sacrae, \\ Condecet illarum vos retinere ritum. \\ Non minor Augustus, minor est tibi neque potestas, \\ Nec ibi dissimilis imperialis apex. \\ Bissona nec absunt vobis librata talenta, \\ Velle sit aut posse alterius proprium. \\ Sentiat inde dei gaudens plebs incrementum, \\ Polleat et vestro tempore sancta fides. \\ His valitura bonis vestro libramine, Caesar, \\ Saecula perpetuum dent tibi cuncta melos. \\ 
        \pagebreak 
     \marginpar{[210]} \begin{center} \textbf{C. 724 725. 1—3} \end{center}B V I15. M. 274. \\ 
      \end{verse}
  
            \subsection*{724}
      \begin{verse}
      IB. V 84 \\ Hoc opus egregium, quo mundi summa tenetur, \\ Aequora quo, montes, fluvii, portus, freta et urbes \\ Signantur, cunctis ut sit cognoscere promptum, \\ Quicquid ubique latet, clemens genus, inclita proles, \\ Ac per saecla pius, totus quem vix capit orbis, \\ Theodosius princeps venerando iussit ab ore \\ Confici, ter quinis aperit cum fascibus annum. \\ Supplices hoc famuli, dum scribit, pingit et alter, \\ Mensibus exiguis, veterum monimenta secuti, \\ ln melius reparamus opus culpamque priorem \\ Tollimus ac totum breviter comprendimus orbem: \\ Sed tamen hoc tua nos docuit sapientia, princeps. \\ 40 \\ B M. B.III 60. TMAMIRA. LADA8. MIDA. \\ Te, formose Mida, iamdudum nostra requirunt \\ 
      \end{verse}
  
            \subsection*{}
      \begin{verse}
      \poemtitle{I1.}Iurgia: da vacuam pueris certantibus aurem. \\ 
      \end{verse}
  
            \subsection*{}
      \begin{verse}
      \poemtitle{MM.}lHaud moror, et casti nemoris secreta voluptas \\ 
        \pagebreak 
    \begin{center} \textbf{C. 725, 426} \end{center}\begin{center} \textbf{0 1 1} \end{center}Invitat calamos: imponite lusibus artem. \\ 
      \end{verse}
  
            \subsection*{}
      \begin{verse}
      \poemtitle{TI.}Praemia si cessant, artis fiducia muta est. \\ Sed nostram durare fidem duo pignora cogent: \\ 
      \end{verse}
  
            \subsection*{}
      \begin{verse}
      \poemtitle{L.}Vel caper ille, nota frontem qui pingitur abba, \\ Vel levis haec et mobilibus circumdata bullis \\ Fistula, silvicolae munus memorabile Fauni. \\ SivecaprummavisvelFauniponeremunus,5oTII. \\ Elige utrum perdas; et erit, puto, certius omen \\ Fistula: damnato iam nunc pro pignore certas. \\ Quid iuvat insanis lucem consumere verbis? \\ Iudicis e gremio victoris gloria surgat. \\ Praeda mea est, quia Caesareas me dicere laudes \\ 5TM. \\ Mens iubet: huic semper debetur palma labori. \\ Et me sidereo †corrumpit Cynthius ore \\ 
      \end{verse}
  
            \subsection*{}
      \begin{verse}
      \poemtitle{LA.}Laudatamque chelyn iussit variare canendo, \\ Carmine ceu virgo furit et canit ore coacto. \\ 
      \end{verse}
  
            \subsection*{}
      \begin{verse}
      \poemtitle{1 MM.}Pergite io, pueri, promissum reddere carmen. \\ Sic vos cantantes deus adiuvet! Incipe, Lada, \\ Tu prior, alternus Thamyras imponet honorem. \\ 
      \end{verse}
  
            \subsection*{}
      \begin{verse}
      \poemtitle{1.}Maxime divorum caelique aeterna potestas, \\ Seu tibi, Phoebe, placet temptare loquentia fila \\ Et citharae modulis primordia iungere mundi ... \\ (Fas mihi sit vidisse deos, fas prodere † mundum): \\ 
        \pagebreak 
     \marginpar{[212]} \begin{center} \textbf{C. 725, 27 48} \end{center}Seu caeli mens illa fuit sen solis imago, \\ Digus utroque stetit  \lbrack deus \rbrack  ostro clarus et auro \\ Intonuitque manu. Talis divina potestas, \\ Quae genuit mndum septemque intexuit orbis \\ rtificis onas et totum miscet amore: \\ Tlis Phoebus erat, cum laetus caede draconis \\ Docta repercusso generavit carmina plectro. \\ Caelestes ulli si sunt, hac voce locuntur! \\ Venerat ad modulos doctarum turba sororum \\ Huc huc Pierides volucri concedite saltu: \\ rI. \\ Hic Heliconis opes florent, hic vester Apollo est! \\ Tu quoque, Troia, sacros cineres ad sidera tolle \\ Atque gamemnoniis opus hoc ostende Mycenis. \\ Iam tanti cecidisse fuit! gaudete ruinae \\ Et laudate rogos: vester vos tollit alumnus. \\ . plurima barba \\ Albaquecaesariesplenoradiabathonore. \\ Ergo ut divinis implevit vocibus auras, \\ Candida faveti distinxit tempora vitta \\ Caesarenmque caput merito velavit amictu. \\ Haud procul liaco quondam non segnior ore \\ Stabat et ipsa suas delebat Mantua cartas. . . . \\ 
        \pagebreak 
    \begin{center} \textbf{C. 726,. 12} \end{center}\begin{center} \textbf{0 1 2} \end{center}
      \end{verse}
  
            \subsection*{726}
      \begin{verse}
      B. M. \\ LCE2ANVSB. M7SrEs. \\ B. III 63. \\ Quid tacitusMystes t. Curae mea gaudia turbant, \\ 61. \\ Cura dapes sequitur, magis inter pocula surgit \\ Et gravis anxietas laetis incumbere gaudet. \\ \poemtitle{.}M. Nec me iuvat omnia fari. \\ Forsitan imposuit pecori lupus? Mi. Haud timet \\ 5 Gl.. \\ hostes \\ Turba canum vigilans. C . Vigiles quoque somnus \\ obumbrat. \\ M. Altius est, Glycerane, aliquid, non hoc, pater; \\ erras. \\ C. Atquin turbari sine ventis non solet aequor. \\ M. Quod minime reris, satias mea gaudia vexat. \\ G. Deliciae somnusque solent adamare querellas. \\  \lbrack , Ergo si causas curarum scire laboras \\ 
      \end{verse}
  
            \subsection*{}
      \begin{verse}
      \poemtitle{1.}Quae spargit ramos, tremula nos vestiet umbra \\ Vlmus, et in tenero corpus submittere prato \\ lerba iubet: tu dic, quae sit tibi causa tacendi. \\ 
      \end{verse}
  
            \subsection*{}
      \begin{verse}
      \poemtitle{M.}Cernis ut adtrito difusus caespite pagus \\ Annua vota ferat sollemnisque imbuat aras? \\ Spirant templa mero, resonant cava tympana palmis, \\ Maenalides teneras ducunt per sacra choreas, \\ Tibia laeta canit, pendet sacer hircus ab ulmo \\ Et iam nudatis cervicibus exuit exta. \\ Ergo non dubio pugnant discrimine nati: \\ 
        \pagebreak 
     \marginpar{[214]}  \marginpar{[01]} \begin{center} \textbf{C. 726. 22s. C. 727} \end{center}Et negat huic aevo stolidum pecus aurea regna? \\ Saturni rediere dies Astraeaque virgo, \\ Totaque in antiquos redierunt saecula mores. \\ Condit securus tuta spe messor aristas, \\ Languescit senio Bacchus, pecus errat in herba. \\ Nec gladio metimus nec clausis oppida muris \\ Bella tacenda parant, nullo iam noxia partu \\ Femina quaecumque est hostem parit. arva iuventus \\ Nuda fodit tardoque puer domifactus aratro \\ Miratur patriis pendentem sedibus ensem. \\ Sed procul a nobis infelix gloria Sullae \\ Trinaque tempestas, moriens cum Roma supremas \\ Desperavit  \lbrack opes \rbrack  et Martia vendidit arma. \\ Nunc tellus inculta novos parit ubere fetus, \\ Nunc ratibus tutis fera non irascitur unda, \\ Mordent frena tigres, subeunt iuga saeva leones. \\ Casta fave Lucina, tuus iam regnat Apollo! \\ B. V 21. M.1061. \\ 
      \end{verse}
  
            \subsection*{72}
      \begin{verse}
      B. V 370. \\  \lbrack Quadam nocte Niger dux nomine, Candidus alter \\ Forte subintrarunt unica tecta simul sq \rbrack  \\ 
        \pagebreak 
    \begin{center} \textbf{C. 728 729} \end{center} \marginpar{[215]} 
      \end{verse}
  
            \subsection*{728}
      \begin{verse}
      B III272.M. 1009 \\ Versus ad puellam \\ B 1V 443 \\ Candida iam nostris aptentur colla lacertis, \\ Suspenso maneat poplite noster amor, \\ Lusibus optatis noctis luctemur in umbris, \\ Pervigiles laudes, o rubicunda dies. \\ Fulgidus ardenti iungatur saphirus auro, \\ loribus in thalamis cincta cupressus eat, \\ Exultent nostro magnae certamine nymphae, \\ Tactibus exultes tuque puella meis! \\ 
      \end{verse}
  
            \subsection*{729}
      \begin{verse}
      B . III273 M. 1010 \\ Responsum puellae \\ B. 1V 444 \\ Conspicua primum specie quam fata bearunt, \\ . Desine pompifero tu violare toro. \\ Absit ut albiplumem valeat calcare columbam \\ Inter tot niveas rustica milvus avis, \\ Nec rubeis cardus succrescat iure rosetis, \\ Lilia nec campis vana cicuta premat, \\ Nec miser eximiae cervae iungatur asellus, \\ Quem stimulis crebris sarcina saeva domat. \\ 
        \pagebreak 
    \begin{center} \textbf{C. 730 732, 1—2} \end{center} \marginpar{[216]} B. M. \\ 
      \end{verse}
  
            \subsection*{730}
      \begin{verse}
      B. V 368. \\ Mb Auct. nti \\ GFV S De voee hominis absona \\ Dissona vox hominis rugitum signat aselli \\ Grunnitumque suis et rancae murmura mulae. \\ Quod bos mugitu fingit blateratque camelus, \\ Quodque lupus ululat vel quod vulpecula gannit, \\ Quod pardus felit, quod raccat pessima tigris, \\ Quod glatit catulus, quod miccit setiger hircus, \\ Absona cuncta sonat et dulcia nulla repingit \\ Estque feris socia, non nostrae vocis amica! \\ Desine iam talis incassum pandere labra, \\ Desine iam frustra pulmonum rumpere fibras, \\ Desine postremo miserum discerpere guttur! \\ Non deus hoc recipit, quod homuncio sanus abhorret! \\ 0 \\ ITde unc c. \\ 40 \\ . III 232.M. 997 \\ Pasiphaes fabula \\ B. V 108. \\ ila  \lbrack  Solis \rbrack  \\ Aestuat igne novo  \rbrack  \\ 
        \pagebreak 
    \begin{center} \textbf{C. 733,. 322} \end{center}\begin{center} \textbf{0 1} \end{center}Et per  \lbrack  prata iu vencum  \rbrack  \\ Mentem  \lbrack  perdita l quaeritat.  \rbrack  \\ Non il \lbrack lam thalami pudor  \rbrack  arcet, \\ Non re galis ho†nor, non † magni  \lbrack  cura mariti. \\ Optat in  \lbrack  formam  \lbrack  bovis \\ Convertier vultus  \rbrack  suos  \rbrack  \\ Et Proetidas  \lbrack  dicit beatas  \rbrack  \\ Ioque laudat, non  \lbrack  quod Hsis alta est, \\ 10’ \\ Sed quod  \lbrack  iuven cae cornibus  \lbrack  frontem † levat. \rbrack  \\ Si quan \lbrack do miserae copia † suppetit, \\ Brachi is ambit fera  \lbrack  colla  \lbrack  tauri \\ Floresque vernos  \rbrack  cornibus illigat  \rbrack  \\ Oraque iungere quaerit ori.  \rbrack  \\ Auda ces animos efficiunt tela Cupi \lbrack dinis, \\ Inlicitisque gaudent.  \rbrack  \\ Corpus inclu†dit tabulis efficiens iuvencam,  \rbrack  \\ Et amoris  \rbrack  pudibundi malesuadis \\ Obsequitur votis et  \lbrack  procreat † (heu ne fas) bi mem \\ brem, \rbrack  \\ Cecropi†des iuvenis quem perculit † fractum  \lbrack  manu, \\ Filo  \lbrack  resol†vens Gnosiae tristia tecta domus.  \rbrack  \\ 
        \pagebreak 
     \marginpar{[218]} \begin{center} \textbf{C. 733, 117} \end{center}4Da \\ B V 143. . 1079. \\ 
      \end{verse}
  
            \subsection*{}
      \begin{verse}
      \poemtitle{De eantibus avium}B. V 367. \\ Quis volucrum species numeret, quis nomina discat? \\ Mille avium cantus, vocum discrimina mille. \\ Nec nostrum (fateor) tantas discernere voces. \\ Hinc titiare cupit diversa per avia passer, \\ ,Garrula versifico tignis mihi trissat hirundo, \\  \lbrack Accipitres pipant, longo \lbrack que \rbrack  ciconia collo \\ Glottorat et ranas grandi rapit improba rostro. \\ Haec inter merulae dulei modulamine cantus \\ Einzilat et laetis parrus nunc tinnipat arvis. \\ Faccilat hinc volitans turdus, gallina cacillat. \\ Dum miluus iugilat, trinnit tunc improbus anser. \\ Interea perdix cacabat nidumque revisit. \\ Nunc cuculus cantans †scottos iter ire perurget. \\ Nec minus interea pecudum genus omne ferarum \\ Musitat, et proprias norunt animalia voces. \\ Sic ululare lupos certum est hircareque lynces. \\ Blatterat ut aries, nunc raccat ut Indica tigris, \\ 
        \pagebreak 
    \begin{center} \textbf{C. 73, 18 20. C. 734 788, 1—2} \end{center} \marginpar{[219]} Hinc latrare canes, timidos vagitare lepores, \\ Et miccire caprum, murem mintrire videbis. \\ Nec non mustelae dindrant ranaeque coaxent. \\ 
      \end{verse}
  
            \subsection*{734}
      \begin{verse}
      ide nunc c. \\ Vide nunc c. \\ 
      \end{verse}
  
            \subsection*{736}
      \begin{verse}
      ide nunc c. \\ B. M. B. \\ 40 4 \\ Me legat, antiquas qui vult proferre loquelas; \\ Me qui non sequitur, vult sine lege loqui. \\ 
      \end{verse}
  
            \subsection*{738}
      \begin{verse}
      B. M. B. \\ Qui modica pelagus transcurris lintre Maronis, \\ Bis senos Scyllae vulgo cave scopulos. \\ 
        \pagebreak 
     \marginpar{[220]} \begin{center} \textbf{C. 738, 3s. C. 73a .738} \end{center}Sed si more cupis nautae contingere portum, \\ Carbasus ut ephyris †desine detr ovans. \\ Tumque salis lustra reliquos ope remigis ames: \\ Sic demum cymbam portus habebit opis. \\ 
      \end{verse}
  
            \subsection*{73a}
      \begin{verse}
      M. B. \\ Prima sonat quartae, respondet quinta secundae, \\ Tertia cum sexta: nomen habebit avis. \\ 
      \end{verse}
  
            \subsection*{73}
      \begin{verse}
      B. M. B \\ Quod cernis, dicor. tollatur littera prima: \\ Scando polum calidum, curro solum gelidum. \\ 
        \pagebreak 
    \poemtitle{CARMINA}\poemtitle{CODICVM SAECVLI XI}
        \pagebreak 
    B M. \\ 
      \end{verse}
  
            \subsection*{739}
      \begin{verse}
      B. IV 438. \\ ‘Rauca sonora \\ Languida voce \\ Tibia nostra \\ Est, pater’ inquam, \\ ‘Ast gerit ora \\ Fusca colore, \\ Tristis habunde, \\ lens modo serta \\ Forte dirempta. \\ Nam rosa mollis, \\ Fragmina lanae, \\ Lilia clara \\ (Singula quaeque \\ Quid memorem nunc?), \\ Nectara mixta \\ Plurima sunt hic. \\ Perfice, belle \\ Vt queat illud \\ Pallere voto’! \\ 
        \pagebreak 
    C. 739, 172. C. 740 \\ ‘Sed mibi’ Bacchus \\ Inquit ‘abest, heu! \\ Confiiatur \\ Vnde phonascus, \\ Quo medicata \\ Vivida passim \\ Carmina fingam, \\ Larga potestas.. . . \\ 
      \end{verse}
  
            \subsection*{740}
      \begin{verse}
      \poemtitle{ALCIMI}B. II 17. M. 256 \\ de Vergilio \\ B. IV 187. \\ De numero vatum si quis seponat lomerum, \\ Proximus a primo tunc Maro primus erit. \\ Si post primum Maro seponatur Homerum, \\ Longe erit a primo, quisque secundus erit. \\ 
        \pagebreak 
    \begin{center} \textbf{C. 741 742. 13} \end{center} \marginpar{[225]} 
      \end{verse}
  
            \subsection*{741}
      \begin{verse}
      B. V 125 128. \\ M. 106p \\ B . \\  \lbrack Libra vel as ex unciolis constat duodenis s \\ B. VI 87. M. 1144. \\ 
      \end{verse}
  
            \subsection*{742}
      \begin{verse}
      B. III 293. \\ CInudinu ed. \\ Epithalamium Laurenti \\ Ieep. II 183. \\ ed. Birt. Auct. \\ . ntiquins. 404. \\ In primis te, sponse, precor: patiare canentem, \\ Teque, puella, magis: tacito mihi crimine parcas. \\ Scimus enim, scimus vobis nunc carmina nostra \\ Doctiloquique etiam linguam sordere Maronis. \\ Sed breviter strictimque duos dicemus amantes, \\ Materiesque licet plus quaerat, pauca loquemur. \\ Principio generis simili vos stirpe creatos \\ 
        \pagebreak 
     \marginpar{[226]} \begin{center} \textbf{C. 742, 83} \end{center}Florenti Florique patris sat nomina produnt. \\ Matribus et pariter respondet fetus uterque. \\ Nam decuit Mariam sapientem fundere  \lbrack natam \rbrack  \\ Calliopenque simul iuvenem proferre to \lbrack gatum \rbrack . \\ O similes multumque pares! te prima iuventus \\ Insignem vegetumque tenet. nam nuper  \lbrack inumbrans \rbrack  \\ Flore genas plenaque decens lanugine malas \\ Mollia votifero dempsisti vellera ferro. \\ Egregio fulges cultu totusque decorus, \\ Ex facie mores patriamque ex nomine pandens. \\ Nam quae primates quondam genuere Latinos \\ Antiquaeque urbi proprium tribuere vocamen, \\ Dant tibi, Laurenti, Laurentes nomina nymphae. \\ Quid memorem mores iuvenili in corde seniles \\ Atque Italum ingenium lomana fervere lingua? \\ Tu fora, tu leges celebras sanctumque tribunal, \\ Promptaque impavidus tu suetus dicere dextra  \lbrack es \rbrack . \\ Te palmam insontes semper tenuere patrono, \\ Te contra stantem semper timuere nocentes: \\ Prorsus habes iuvenis totum, quod Tullius auctor \\ Causidicos retinere iubet. nam fultus utroque \\ Vir bonus es nimium, fandi pariterque peritus. \\ Ad te nunc breviter (nam sic te velle putamus), so \\ Verba, puella, feram. pulchro formosa colore \\ 
        \pagebreak 
     \marginpar{[227]} \begin{center} \textbf{C. 742, 3255} \end{center}ilia ceu niteant rutilis commixta rosetis, \\ Sic rubor et candor pingunt tibi, Florida, vultus. \\ Denique miramur, quod colla monilia gestant: \\ Ex umeris frustra phaleras inponis eburnis. \\ Nam tibi non gemmae, sed tu das lumina gemmis, \\ Atque alias comit, per te quod comitur, aurum. \\  \lbrack Doct \rbrack a loqui scriptique tenax veloxque legendi  \lbrack es \rbrack , \\  \lbrack Et tamquam talis fueris praesaga mariti, \\ o  \lbrack Sic) \rbrack  Musea tuis insedit cura medullis. \\ Nec minus in propriis studium: nam vellera lanae \\ Textilibus calathis semper tractare perita \\ Inque globos teretes coeuntia volvere pensa \\ Compositas tenui suspendis stamine telas; \\ Quas cum multiplici frenarint licia gressu \\ Traxeris et diitis cum mollia fila gemellis, \\ Serica Arachneo densentur pectine texta \\ Subtilisque seges radio stridente resultat. \\ Sed iam sufficiat de magnis pauca locutum, \\ Nec sinit hoc tempus totas nunc pandere laudes. \\ Quin magis, o pueri, vosque exaudite, puellae, \\ Quas optare pares thalamos taedasque iugales \\ Sensibus ex imis suspiria ducta fatentur! \\ Consertas prensate manus magnumque per orbem \\ Dextra liget laevam; festos celebrate hymenaeos \\ 
        \pagebreak 
     \marginpar{[228]} \begin{center} \textbf{C. 742, 56 76} \end{center}Ac modulate melos pariter, quassoque pavito \\ Cum pede vox resonet! persultent atria longa, \\ Quae virides cingunt hederae laurique coronant \\ Votigerique ignes stellanti lumine complent! \\ Tympana, chorda simul, symphonia, tibia, buxus. \\ Cymbala, bambilium, cornus et fistula, sistrum, \\ Quaeque per aeratas inspirant carmina fauces, \\ Humida folligenis exclament organa votis! \\ Surge age iam, iuvenis, dextram conplectere sponsae, \\ Tuque puella, caput niveo velamine tecta, \\ Non cunctante gradu gressum comitare mariti. \\ Teque etiam paucis moneamus, pronuba, verbis: \\ Cum fuerit ventum ad thalamos primumque cubile, \\ Sit tibi cura vigens innoxia reddere membra \\ Virginis, ut totum, quod possit laedere, demas. \\ Nullum sit capiti, quo crinis comitur, aurum, \\ Nec collo maneant nisi quae sunt laevia fila, \\ digitistollaturmollibusasperAnuluset \\ Ac niveos auro propera spoliare lacertos, \\ Ne, dum proludunt atque oscula dulcia iactant \\ Exercentque toris veneris luctamen anhelum, \\ 
        \pagebreak 
    \begin{center} \textbf{C. 742,. 7787. C. 743 744, 1—2} \end{center} \marginpar{[229]} luncta per amplexus foedentur membra mariti \\ Atque invita viri maculet, quae diligit, ora. \\ Ite pares tandemque toro recubate potito. \\ 0 Mellea tunc roseis haerescant basia labris \\ Et compressa suis insudent pectora membris \\ Per \rbrack  niveosque umeros collumque, per os . . . \\ Dextera cervicem roseam subiecta retentet, \\ Turentesque simul constringat laeva papillas. \\ Vivite felices quam longaque carpite saecla; \\ Vivite concordes, donec premat una senectus, \\ Donec \rbrack  vestra habeant natorum vota nepotes. \\ 40 \\ B.M . B. III300. \\ cIL. I 412 \\ 
      \end{verse}
  
            \subsection*{}
      \begin{verse}
      \poemtitle{De Isidis navigio}3irt p. 402. \\ si, o fruge nova quae nunc dignata videri \\ Plena nec ad Cereris munera poscis opem \\ (Nam tu nostra dea es nec te dens ipse tacendi \\ Abnegat, expertus quis tua vela ferat: \\ Namque tibi ephyrus favet ac Cyllenins ales), \\ Ne nostra referas de regione pedem. \\ 
      \end{verse}
  
            \subsection*{744}
      \begin{verse}
      B. III 50. M. 927. \\ B. III 300. \\ \poemtitle{De lavacro}B1rt p. 410. \\ Qui splendere cupis claro tenuique lavacro, \\ Pontica succedas in balnea nobilis undae, \\ 
        \pagebreak 
     \marginpar{[230]} \begin{center} \textbf{C. 474, 32. C. 745} \end{center}Quam nec Alexandri mater sub sole cadenti \\ Emeruit; non si varia se aspergine Gai \\ Effundat per aperta latex e sedibus inis \\ Cum Syrio unguento, cui semper roscidus humor. \\ Iic femora et suras et brachia molliter ambit \\ Et rigat in pluviam, sensimque ad colla resultans \\ Tangit odore levi et grato spiramine nares \\ Lenis et externas admittere nescius artes. \\ Iuc ades, o Florens, et festa luce relaxa \\ Mentis onus nebulasque fuga, quae frontis honorem \\ 
      \end{verse}
  
            \subsection*{745}
      \begin{verse}
      B. I 25. M. 576. \\ B. III 301. \\ \poemtitle{De vinalibus}B1rt 410. \\ Non tibi vina placent, o insanabilis bospes, \\ Nec mens est Thebana tibi, licet aggere celso \\ Dircaeae rupis dicas fluxisse parentes \\ Vertice de Nysae per rura et nostra Lyaeus \\ Transiit implevitque vias nigrantibus uvis. \\ Musta sibi posuit pater, et non tempore ab ill \\ Desierant haec sacra coli, vatumque sonoro \\ Carmine fincius et strepuit circumsita ripa \\ luminis Etrusci, quem non aequabit Orontes. \\ 
        \pagebreak 
    \begin{center} \textbf{C. 746 747} \end{center} \marginpar{[01]} 
      \end{verse}
  
            \subsection*{746}
      \begin{verse}
      B.III276 . 926. \\ B. III 301. \\ \poemtitle{De Cythera}irt 4tt0. \\ Forte erat Aurorae tempus Solisque quadriga \\ Fecerat et ventum et sonitum per nobile marmor \\ Adstantis pueri, cum te, mea bella Cythere, \\ Aspicio venientem et tu mea limina grato \\ Introitu dignata rosas et lenis amomi \\ Delicias miras tecum allicis, unde secutae \\ Palladis et frondes nulliusque inscia laurus. — \\ Atria nostra virent et agunt in limine primo \\ Radicem platani, ad portam venit arbutus ipsam. \\ Felix multa arbos, imitans miracula Pindi, \\ Quam non delebit livor nec sera vetustas! \\ O iucunda nimis, tenui quae visa poetae, \\ Dum credis vitium non auscultare Camenis. \\ 4 t \\ V 190 M. 1120. \\ B. III 302. \\ 
      \end{verse}
  
            \subsection*{}
      \begin{verse}
      \poemtitle{De eereo}B5rt 411 \\ Flora venit. quae Flora? dea an de gente Latina? \\ Non reor; at Chloris dicta per arva fuit. \\ Iuius in adventum radiant de nocte lucernae; \\ Nam nitet atque hilarat lumine cuncta suo. \\ Cerea materies apibus debetur amicis, \\ Floribus atque horti sit precor aequa mei, \\ Non ut mel rapiam, cuius non tangor amore, \\ Sed cera in talem fiat ut alba diem. \\ 
        \pagebreak 
     \marginpar{[232]} \begin{center} \textbf{C. 748 749} \end{center}B.V 17. M. 1083. \\ B. III 303. \\ 
      \end{verse}
  
            \subsection*{748}
      \begin{verse}
      Birt 409. \\ De aquila, uae in mensa de sardonyche \\ Iapide erat \\ Mensa coloratis aquilae sinuatur in alis, \\ Quam floris distinguit honos, similisque tigura \\ Texitur: inplumem mentitur gemma volatum. \\ 
      \end{verse}
  
            \subsection*{749}
      \begin{verse}
      B. I 26. M. 585 \\ B. III 303. \\ ILaus Martis \\ Btrt 40. \\ Mars, pater armorum, fortissime belligerator: \\ Esto volens, mitis, facilis deus, esto benignus. \\ Sic tibi post pugnas et pastos sanguine campos \\ Amplexus tribuat vincli secura Cythere. \\ Tu crista galeaque rubes, tu pulcher in aere \\ Incutis e vultu radiantia lumina ferro. \\ Te thorax galeaque tegunt, non quo tibi terror \\ Hostilis subeat, sed quod decor exit ab armis. \\ Tu cum pulsatum clipei concusseris orbem, \\ Inmugit mundus, tellus tremit, aequora cedunt. \\ Da nobis reditum, patriam repetamus ovantes. \\ Sic tibi lascivae celebrentur in urbe lxalendae. \\ 
        \pagebreak 
    \begin{center} \textbf{C. 760 752} \end{center}
      \end{verse}
  
            \subsection*{750}
      \begin{verse}
      B. M. \\ B. III 304. \\ \poemtitle{De Innonalibus}B1r 49. \\  \lbrack Sancta \rbrack  poli domina, cui viucla iugalia curae, \\  \lbrack Suprlemi caeli regis coniunxque sororque, \\  \lbrack Da re \rbrack ditum nobis. sic regnum transeat orbis \\ 
      \end{verse}
  
            \subsection*{751}
      \begin{verse}
      B. I 21. M. 574. \\ B. III 304. \\ \poemtitle{De Liberalibus}Birt 408. \\ Lenaee vitisator Rromie Semeleie lacche \\ Thyrsitenens bimater trieterice Nysie iber, \\ Flos Ariadnaee †coriatice, laete Thyoneu, \\ Da reditum nobis. sic totis dulcia rivis \\ Musta fluant, spumetque cavis vindemia labris. \\ 
      \end{verse}
  
            \subsection*{752}
      \begin{verse}
      I. V 146. M.102. \\ B. III 305. \\ \poemtitle{De hippopotamo  \lbrack et cerocodilo \rbrack  BiB 09.}Vtraque fecundo nutritur belua Nilo: \\ Quaeque vorat morsu quaeque sub ore fremit. \\ 
        \pagebreak 
     \marginpar{[234]} \begin{center} \textbf{C. 753 759, 13} \end{center}
      \end{verse}
  
            \subsection*{753}
      \begin{verse}
      B. V 192. M. 112. \\ B. III 305. \\ \poemtitle{De duleio}Btrt 404. \\ iectareo muro dulces cinguntur harenae. \\ 
      \end{verse}
  
            \subsection*{753}
      \begin{verse}
      Bit 44 adnot. \\ Suave tibi nomen; sed si te talia tangunt, \\ Moribus atque animo post ea dulcis eris. \\ Panevrieus Aniciorum \\ 
      \end{verse}
  
            \subsection*{755}
      \begin{verse}
      \poemtitle{De hirundine}e e mero \\ 
      \end{verse}
  
            \subsection*{756}
      \begin{verse}
      \poemtitle{De vitulis marinis}\poemtitle{De paupere singnlari}\poemtitle{De ape}B.M B.III305. \\ 
      \end{verse}
  
            \subsection*{759}
      \begin{verse}
      Birt 404. \\ De ona missa ab eadem Arcadio Auusto \\ Stamine resplendens et mira textilis arte \\ Balteus alipedis regia terga liget, \\ Quem decus Eoo fratri pignusque propinqui \\ 
        \pagebreak 
    \begin{center} \textbf{0.} \end{center}\begin{center} \textbf{C. 759, —6. C.} \end{center}\begin{center} \textbf{760 760a, 1—2} \end{center}Sanguinis Hesperio misit ab orbe soror. \\ Hoc latus adstringi velox optaret Arion, \\ Ioc proprium vellet cingere Castor equum. \\ B.III 23B. M. 998. \\ 
      \end{verse}
  
            \subsection*{760}
      \begin{verse}
      B. III 306. \\ Bir 413. \\ Marcus amans puerum natum mentitur amare \\ Vultque pater dici nescius esse pater \\ Et pietate nefas et amorem velat amore: \\ Se pietas umbram criminis esse dolet. \\ ‘Nate’ dies andit, nox et torus audit ‘amice’, \\ Et pro temporibus nomina mutat ei. \\ Stulte quid ignaro non dicit Cyntbia frtri? \\ Ne credas nocti digna latere diem! \\ Qui ‘puer’ est, hic ‘filius’ est: a lumine primo \\ ‘Filius’, a thalamis incipit esse ‘puer’. \\ 
      \end{verse}
  
            \subsection*{760a}
      \begin{verse}
      B. II 119. M. 109. \\ Maeeena \\ B. I 125. \\ Defleram iuvenis tristi modo carmine fata: \\ Sunt etiam merito carmina danda seni. \\ 
        \pagebreak 
    \begin{center} \textbf{C. 760a, 32} \end{center} \marginpar{[236]}  \marginpar{[0]} Vt iuvenis deflendus enim tam candidus et tam \\ Longius annoso vivere dignus avo . . . \\ Inreligata ratis numquam defessa carina \\ It redit in vastos semper onusta lacus. \\ lla rapit iuvenes prima florente iuventa, \\ Non oblita tamen sera petitque senes. \\ Nec mihi, Maecenas, tecum fuit usus amici: \\ Lollius hoc aegro conciliavit opus. \\ Foedus erat vobis nam propter Caesaris arma \\ Caesaris et similem propter in arma fidem. \\ Regis eras, Etrusce, genus; tu Caesaris almi \\ Dextera, lomanae tu vigil urbis eras. \\ Omnia cum posses tanto tam carus amico, \\ Te sensit nemo posse nocere tamen. \\ Pallade cum docta Phoebus donaverat artes: \\ Tu decus et laudes huius et buius eras, \\ Sicut vulgares vincit beryllus harenas, \\ Litore in extremo quas simul unda movet. \\ Quod discinctus eras nimio (quod carpitur unum), \\ Diluis hoc animi simplicitate tui. \\ 
        \pagebreak 
     \marginpar{[00]} \begin{center} \textbf{C. 760, 2346} \end{center}Sie illi vixere, quibus fuit aurea virgo, \\ Quae bene praecinctos postmodo pulsa fugit. \\ ivide, quid tandem tunicae nocuere solutae, \\ Aut tibi ventosi quid nocuere sinus? \\ Num minus urbis erat custos et Caesaris obses? \\ Num tibi non tutas fecit in urbe vias? \\ Nocte sub obscura quis te spoliavit amantem? \\ Quis tetigit ferro, durior ipse, latus \\ Maius erat potuisse tamen nec velle triumphos; \\ Maior res magnis abstinuisse fuit. \\ Mluit umbrosam quercum nymphasque cadentes \\ Paucaque pomosi iugera certa soli. \\ Pieridas Phoebumque colens in mollibus hortis \\ Sederat argutas garrulus inter aves. \\ Marmora temnentur, vincent monimenta libelli: \\ Vivitur ingenio, cetera mortis erunt. \\ Quid facerett discinctus erat comes integer idem \\ 0 Miles et Augusti fortiter usque pius! \\ llum piscosi viderunut saxa Pelori \\ Ignibus hostilis reddere lina ratis; \\ Pulvere in Emathio forem videre Philippi: \\ Tam tunc ille tener tam gravis hostis erat. \\ Cum freta Niliacae texerunt laeta carinae, \\ Fortis erat circa, fortis et ante ducem, \\ 
        \pagebreak 
     \marginpar{[238]} \begin{center} \textbf{C. 760a, 769} \end{center}Miliis Eoi fugientis terga secutus, \\ Territus ad Nili dum fugit ille caput. \\ Pax erat: haec illos laxarunt otia cultus. \\ Somnia victores Marte sedente decent. \\ Actius ipse lyram plectro percussit eburno, \\ Paostquam victrices conticuere tubae. \\ Hic modo miles erat, ne posset femina lomam \\ Dotalem stupri turpis habere sui. \\ ic tela in profugos (tantum curvaverat arcuml \\ Misit ad extremos exorientis equos. \\ Bacche, coloratos postquam devicimus lndos, \\ Potasti galea dulce iuvante merum. \\ Et tibi securo tunicae fluxere solutae: \\ Te puto purpureas tunc habuisse duas. \\ Sum memor et certe memini sic ducere thyrsos \\ Brachia vel pura candidiora nive. \\ tibi thyrsus erat gemmis ornatus et auro: \\ Serpentes hederae vix habuere locum. \\ Argentata tuos etiam talaria talos \\ Vinxerunt certe; nec puto, Bacche, negas. \\ Mollius es solito mecum tum multa locutus \\ Et tibi consulto verba fuere nova. \\ Inpiger Alcide, multo defuucte labore, \\ 
        \pagebreak 
    \begin{center} \textbf{C. 760a, 70 ——25} \end{center} \marginpar{[239]} o Sic memorant curas te posuisse tuas; \\ Sic te cum tenera multum lusisse puella \\ Oblitum Nemeae iamque, Erymanthe, tui. \\ Vltra numquid erat torsisti pollice fusos, \\ Lenisti morsu levia tiia parum. \\ Percussit crebros te propter Lydia nodos, \\ Te propter dura stamina rupta manu. \\ Lydia te tunicas iussit lasciva fluentes \\ Inter lanificas ducere saepe suas. \\ Clava torosa tua pariter cum pelle iacebat, \\ 0 Quam pede suspenso percutiebat Amor. \\ Quis fore credebat, cum rumperet impiger infans \\ Iydros ingentes vix capiente manu, \\ Cumve renascentem †terret velociter hydram, \\ Frangeret inmanes vel Diomedis equos, \\ Vel tribus adversis communem fratribus alvum \\ Et sex adversas solus in arma manus? \\ Fudit Aloidas postquam dominator Olympi, \\ Dicitur in nitidum percubuisse diem \\ Atque aquilam misisse snam, quae quaereret, ecquid \\ Posset amaturo digna referre lovi. \\ Valle sub ldaea tum te, formose sacerdos, \\ Invenit et presso molliter ungue rapit. \\ Sic est: victor amet, victor potiatur in umbra, \\ Victor odorata dormiat inque rosa. \\ Victus aret victusque metat, metus imperet illi, \\ 
        \pagebreak 
    \begin{center} \textbf{C 760, 96 120} \end{center} \marginpar{[240]} Membra nec in strata sternere discat humo. \\ Tempora dispensant usus et tempora cultus: \\ laec homines, pecudes, haec moderantur aves. \\ Lnux est, taurus arat; nox est, requiescit arator \\ Liberat et merito fervida colla bovi. \\ 100 \\ Conglaciantur aquae, scopulis se condit hirundo: \\ Verberat egelidos garrula vere lacus. \\ Caesar amicus erat: poterat vixisse solute, \\ Cum iam Caesar idem quod cupiebat erat. \\ Indulsit merito. non est temerarins ille. \\ 105 \\ Vicimus: Augusto iudice dignus erat. \\ Argo saxa pavens postquam Scyllaceia legit \\ Cyaneosque metus iam religanda ratis, \\ Viscera dissecti mutaverat arietis agno \\ Aeetis sucis omniperita suis. \\ 110 \\ His te, Maecenas, iuvenescere posse decebat. \\ Haec utinam nobis Colchidos herba foret! \\ Redditur arboribus florens revirentibus aetas: \\ Aegro non homini, quod fuit ante, redit? \\ Vivacesque magis cervos decet esse paventes, \\ 115 \\ Si quorum in torva cornua fronte rigent? \\ Vivere cornices multos dicuntur in annos: \\ Cur nos angusta condicione sumus? \\ Pascitur Aurorae Tithonus nectare coniunx, \\ nullasenectanocet.Atqueitaiamtremulo \\ 120 \\ 
        \pagebreak 
    \begin{center} \textbf{C. 760a, 121—12} \end{center} \marginpar{[04]}  \marginpar{[241]} Vt tibi vita foret semper medicamine sacro, \\ Te vellem Aurorae conplacuisse virum. \\ Illius aptus eras croceo recubare cubili \\ Et modo poeniceum rore lavante torum; \\ s lllius aptus eras roseas adiungere higas \\ Et dare purpurea lora regenda manu, \\ Tum mulcere iubam, cum iam torsisset habenas \\ Procedente die respicientis ecqui. — \\ Quaesivere chori iuvenem sic HIesperon illum, \\ Quem nexum medio solvit in igne Venus: \\ Quem unc infuscis placida sub nocte nitentem \\ Luciferum contra currere cernis equis. \\ Hie tibi Corycium, casias bic donat olentis, \\ lic c palmiferis balsama missa iugis. \\ s unc pretium candoris habes, nunc, redditus umbris: \\ Te sumus obliti decubuisse senem. \\ Et Pylium flevere sui ter Nestora canum: \\ Dicebant tamen, hunc non satis esse senem. \\ Nestoris annosi vicisses saecula, si me \\ Dispensata tibi stamina nente forent. \\ Nunc ego quid possum? tellus, levis ossa teneto, \\ Pendula librato pondus et ipsa tuum! \\ 
        \pagebreak 
    \begin{center} \textbf{C. 760a, 14314 . C. 760b, 129} \end{center} \marginpar{[242]} Semper serta tibi dabimus, tibi semper honores: \\ Non umquam sitiens, florida semper eris. \\ B. II 120. M. 110. \\ 
      \end{verse}
  
            \subsection*{760}
      \begin{verse}
      B. I 134. \\ Sie est Maecenas fato veniente locutus, \\ Frigidus et iamiam cum moriturus erat. \\ ‘Mene’ inquit ‘iuvenis primaevi, luppiter, ante \\ Angustum Drnsi non cecidisse diem! \\ Pectore maturo fuerat puer, integer aevo, \\ Et magnum magni Caesaris illud opus. \\ Discidio vellemque prius’ non omnia dixit, \\ Inciditque pudor, quae prope dixit amor. \\ Set manifestus erat: moriens quaerebat amatae \\ Coniugis amplexus oscula verba manus. \\ ‘Set tamen hoc satis est: vixi te, Caesar, amico \\ Et morior’ dixit; ‘dum moriorque, sat est. \\ Mollibus ex oculis aliquis tibi procidet humor, \\ Cum dicar subito voce ‘fuisse’ tibi. \\ Hoc mihi contingat: iaceam tellure sub aequa. \\ Nec tamen hoc ultra te doluisse velim. \\ Set meminisse velim. vivam sermonibus illic: \\ Semper ero, semper si meminisse voles. \\ Hoc decet: et certe vivam tibi semper amicus, \\ Nec tibi qui moritur desinit esse tuus. \\ 
        \pagebreak 
    \begin{center} \textbf{C. 760, 21 3. C. 761, 1 6} \end{center} \marginpar{[243]} lpse ego, quicquid ero cineres interque favillas, \\ Tunc quoque non potero non memor esse tui. \\ Exemplum vixi te propter molle beati, \\ Vnctus Maecenas teque ego propter eram. \\ 2s Arbiter ipse fui; volui quod, contigit esse: \\ Pectus eram vere pectoris ipse tui. \\ Vive diu, mi care, senex pete sidera sero! \\ Est opus hoc terris: te quoque velle decet. \\ Sit secura tibi quam primum Livia coniunx, \\ 3 Expleat amissi munera rupta gener; \\ e Et tibi succrescant iuvenes bis Caesare digni \\ Et tradant porro Caesaris usque genus. \\ Tum deus intersis divis insignis avitis: \\ Te Venus in proprio collocet ipsa sinu.’ \\ B. . \\ 
      \end{verse}
  
            \subsection*{761}
      \begin{verse}
      B. V 380. \\ \poemtitle{De sphaera caeli}laec pictura docet quicquid recitavit lyginus \\ ln septem quinis describens sidera signis \\ Ad caeli terraeque globos in mole rotundos. \\ Mallem prorsus opus solidis insigne figuris, \\ Quas nequit in plano similes expendere quivis, \\ Cum lateant intus quaedam curvisque profundis. \\ 
        \pagebreak 
     \marginpar{[244]} \begin{center} \textbf{C. 761, 735} \end{center}Nam borealis apex arctos conplexus et anguem \\ Arctophylaca tegit nec non simulacra coronae, \\ Engonasinque, lyram, cygnum seu Cassiepiam, \\ Cuius adest pedibus coniunx et filia dextris. \\ Perseus inde gener, tunc est caprarius, inde \\ Deltoton, equus ac delpbin, aquila atque sagitta, \\ Anguitenens, aries, taurus, cum Castore Pollux \\ Et cancer, leo, virgo, suis cum scorpio chelis, \\ Arcitenens taudem, capricornus et urnifer inde; \\ Piscibus extremus locus est quem signifer explet. \\ Primus in austrinis Orion partibus exit, \\ Tum lepus est et utrique canes Argoque triremis, \\ Hydrus, Centaurus, sed et ara et piscis enormis, \\ Pistrix, Hleridanus: sic sphaerae finis habetur, \\ Quam gemini findunt aequa sub sorte coluri, \\ Se tangendo polis dum onas quinque perercant. \\ Has hinc inde sibi diversa a parte coaequat \\ Linea quae scindit medios utrosque coluros. \\ Torrida ona duas circa se a frigore servat; \\ Nam zonas similes aequales dicimus esse \\ In caeli terraeque modo Cicerone magistro. \\ Vertex alteruter terdenis partibus a se \\ Semper abest circumque facit sex undique sumptis; \\ Tum quinas utrimque feret habitabilis ora. \\ Aequidies capit octonas hinc, inde quaternas. \\ Corpora signorum circis resecantur eisdem. \\ IIis super esse ferunt caelo cuicumque notandos, \\ Quorum primus is est. qui candidus extat in astris \\ Obliquo caeli portas discrimine tangens; \\ 
        \pagebreak 
    \begin{center} \textbf{C. 761, 3665} \end{center} \marginpar{[245]} \begin{center} \textbf{0L4} \end{center}Alter ubique vagus graece vocitatur horion. \\ Solus eget terrae spatiis ut limes in astris \\ Dimidium sphaerae momentis omnibus abdens. \\ Ergo decem circis totus variatur Olympus, \\ 0 Ex quibus ille latet semper qui dicitur austri, \\ Cum nobis numquam lateat qui continet arctos. \\ Inter utrosque tamen quod hinc levat, occidit illic. \\ Arcticus his signis finitur circulus: extra \\ Laeva Bootis inest dextro cum poplite flexo \\ llerculis innixi pedibus, umeris quoque Cephei; \\ Tum siliquastrensis tangit confinia basis. \\ Solstitialis et hoc signorum limite constat: \\ Arcturus lapsusquc genu, Cepheia coniunx, \\ Anguiger oblongus, curvi quoque sinciput anguis \\ 0 A superis tanguntur eo cum coniuge Persei \\  \lbrack Cui tamen arctophylax est in contraria versus \\ Pegaseo iunctae medio) pedibusque marito \\ Qui sectus laevo cubito cum crure sinistro \\ IHeniocbi caput ut currens ex pulvere foedat; \\ llle tamen quasi lora tenens pede cornua tauri \\ Deprimit ac geminis traiectis denique collis \\ Inter aselliferi consurgit lumina cancri, \\ Cnurrens per pectus, ventrem lumbosque leonis \\ Perque caput dextramque alam volitantis oloris. \\ Qui lucis noctisque pares dat circulus horas \\ Arietis ima pedum recipit vestigia primum \\ Semibovisque genu praecidit et inguinis eius, \\ Vltima quem fudit putens urina deorum; \\ Sustinet et geminos flexus ex ore draconis, \\ 6 Exit et a genibus longo serpente ligati, \\ 
        \pagebreak 
    \begin{center} \textbf{C. 761, 68 76. C. 761a 762. 12} \end{center} \marginpar{[246]} Postquam chelarum longissima brachia pressit; \\ Tum Ganymedeae raptricis transilit alam \\ Pegaseamque iubam dirimens ex ordine pisces. \\ Quid hiemalis agat signorum corpora scindens \\ Decollatus eo novit qui spicula mittit \\ Piscinusque caper, nec non lymphaticus auspex \\ Et pistrix, fluvius, lepus et leporarius adsunt, \\ Finditur et puppis, Centauro terminat orbis. \\ Vltimus, aversus boreae, †sua dindima solis \\ Manibus ostendit fluvio finitus et Argo, \\ Centaurique pedes postremos tangit et aram. \\ B. V 87. M. 1053. \\ 
      \end{verse}
  
            \subsection*{761a}
      \begin{verse}
      B V 356. \\ Si novus a lani sacris numerabitur annus, \\ Quintilis falso nomine dictus erit. \\ Si facis, ut fuerant, primas a Marte Nalendas, \\ Tempora constabunt ordine ducta suo. \\ 
      \end{verse}
  
            \subsection*{762}
      \begin{verse}
       \lbrack  2ss. De volucribus et iumentis. \\ B. V 363. \\ Do flomela \\ Dulcis amica veni, noctis solatia praestans; \\ Inter aves etenim nulla tui similis. \\ 
        \pagebreak 
    \begin{center} \textbf{C. 762, 3—19} \end{center} \marginpar{[4]} Tu, filomela, potes vocum discrimina mille, \\ Mille vales varios rite referre modos. \\ Nam quamvis aliae volucres modulamina temptent, \\ Nulla potest modulos aequiperare tuos. \\ Insuper est avium, spatiis garrire diurnis: \\ Tu cantare simul nocte dieque sole, \\ Parrus enim quamquam per noctem tinnipet omnem, \\ Sed sua vox nulli iure placere potest. \\ Dulce pelora sonat, dicunt quam nomine droscam, \\ Sed fugiente die illa quieta silet. \\ Et merulus modulans tam pulchris initat odis, \\ Nocte ruente tamen cantica nulla canit. \\ Vere calente novos componit acredula cantus \\ Matutinali tempore rurirulans, \\ Dum turdus trucilat, sturnus tunc pusitat ore: \\ Sed quod mane canunt, vespere non recolunt. \\ Caccabat hinc perdix ct graccitat improbus anser, \\ 
        \pagebreak 
    \begin{center} \textbf{C. 762, 20—41} \end{center} \marginpar{[248]} Et castus turtur atque columba gemunt. \\ Pausitat arborea clamans de fronde palumbes \\ In fluviisque natans forte tetrinnit anas. \\ Grus gruit in gronnis, cygni prope flumina drensant, \\ Accipitres pipant milvus hiansque lupit. \\ Cucurrire solet gallus, gallina cacillat, \\ Pulpulat et pavo, trissat hirundo vaga. \\ Dum clangunt aquilae, vultur pulpare probatur, \\ Et crocitat corvus, fringulit et graculus. \\ c c c \\ Pessimus et passer sons titiare solet. \\ Pittacus humanas depromit voce loquelas \\ Atque suo domino ‘chaere’ sonat vel ‘ave’. \\ Pica loquax varias concinnat gutture voces, \\ Scurrili strepitu omne quod audit ait. \\ Et cuculi cuculant et rauca cicada fritinit. \\ Bombilat ore legens munera mellis apis. \\ Rubilat horrendum ferali murmure bubo \\ Iumano generi tristia fata ferens. \\ Strix nocturna sonans et vespertilio stridunt, \\ Noctua lucifuga cucubit in tenebris. \\ Ast ululant ululae lugubri voce canentes \\ 
        \pagebreak 
    \begin{center} \textbf{C. 762, 264} \end{center} \marginpar{[249]} Inque paludiferis butio butit aquis. \\ legulus atque merops et rubro pectore progne \\ Consimili modulo zinizulare sciunt. \\ Scribere me voces avium filomela coegit, \\ Quae cantu cunctas exsuperat volucres. \\ Sed iam quadrupedum fari discrimina vocum \\ Nemine cogente nunc ego sponte sequar. \\ Tigrides indomitae rancant rugiuntque leones, \\ Panther caurit amans, pardus hiando felit. \\ Dum lynces urcando fremunt, ursus ferus uncat, \\ Atque lupus ululat, frendit agrestis aper. \\ Et barrus barrit, cervi crocitant, mugilant et onagri; \\ Ac taurus mugit, et celer hinnit equus. \\ Quirritat et verres setosus et oncat asellus, \\ Bratterat hinc aries et pia balat ovis. \\ Sordida sus subiens ruris per gramina grunnit, \\ At miccire caprae, hirce petulce, soles. \\ Rite canes latrant, fallax vulpecula gannit, \\ Glaucitat et catulus ac lepores vagiunt. \\ Mus avidus mintrit, velox mustelaque drindat, \\ Et grillus grillat, desticat inde sorex. \\ Ecce venenosus serpendo sibilat anguis, \\ Garrula limosis rana coaxat aquis. \\ 
        \pagebreak 
    \begin{center} \textbf{C. 762. 667o. C. 763b, 1—10} \end{center} \marginpar{[250]} Has volucrum voces describens quadrupedumque \\ Paucas, discrimen cuique suum dederam. \\ Sed cunctas species animantum nemo notavit, \\ Atque ideo sonitus dicere quis poterit? \\ Cuncta tamen domino depromunt munera laudis, \\ Seu semper sileant sive sonare queant. \\ 
      \end{verse}
  
            \subsection*{763}
      \begin{verse}
      Vfle nunc c. 490e \\ 
      \end{verse}
  
            \subsection*{763}
      \begin{verse}
      B. M. \\ EL . e PATENII preorter \\ . . . tempore quo medio peragunt suum sidera cursum \\ Noctis et obscuris densantur cuncta tenebris, \\ Cum refugit somnus fallaxque recedit imago, \\ Teque tenax lolum nostris memoria reddit \\ Sensibus, intuemur cordisque amplexu tenemus \\ Et fruimur mutuis tacito sic ore loquellis. \\ Ast ego cernere sic te absentem gratulor absens: \\ Credo equidem, quod docta talem nec Graecia misit \\ Neque Larissa potens similem procreavit Achillem, \\ Nostris qualem  \lbrack te \rbrack  armipotens tam fertilis Africa o \\ frugum \\ 
        \pagebreak 
    \begin{center} \textbf{C. 76a, I11. C. 764 764a, 13} \end{center} \marginpar{[01]} Vexit ad astra virum. Quem claro lumine fulgens \\ Sol ictu placido nostro de pectore tollit. \\ Te clipeo loricaque et galea caelitus armet \\ Omnipotens, et sit vita beata tibi \\ B. M. B. \\ Boethlu C. p. \\ 
      \end{verse}
  
            \subsection*{764}
      \begin{verse}
      ed. Peper \\ . xxXVIII. \\ Invictus constans Anicius, ortus ab urbe, \\ Torquati genus, exconsul famosus et exul \\ Patriciusque bonus mage civibus atque oprbog \\ Vsque caput iugulis genero comitante servavit. \\ Graviter exaegit furiis monimenta vetustis \\ Flebilibus scribendo modis tot carmina felix \\ Et, quod libertas multo sermone bilingui \\ Artificale tulit, dum post interprete lingua \\ Quicquid Aristoteles docuit transferre sategit: \\ Pandens conditionis opes hinc inde coactas, \\ Quas ab Atheneis rapuit bibliopola gais, \\ Ne Romana fames epulas nesciret Achivas . . . \\ 
      \end{verse}
  
            \subsection*{76a}
      \begin{verse}
      B. M. B. \\ Versibus egregiis decnrsum clarus Arator \\ Carmen apostolicis cecinit insigne coronis, \\ Historiamque nrins raeonens cautus ubiue \\ 
        \pagebreak 
    \begin{center} \textbf{C. 764a, 4 . C. 767 768, 1—2} \end{center} \marginpar{[252]} Substituit typice sensatim verba figurae. \\ Lingua canora bonum testatur iure poetam, \\ Mysticus ingenium sic indicat ordo profundum. \\ 
      \end{verse}
  
            \subsection*{765}
      \begin{verse}
      ide nnc c. \\ 
      \end{verse}
  
            \subsection*{766}
      \begin{verse}
      Tle ne c. 494. \\ 
      \end{verse}
  
            \subsection*{767}
      \begin{verse}
      B. M. B \\ ILaus domnae Eunomiae saerae virginis \\ Plena deo, moderata animo, miranda decore, \\ Larga manu Eunomia provida virgo pia. \\ Alta sapis, praecelsa petis, profunda rimaris, \\ Angelicos motus pectore sancto geris. \\ Vive, caput vivum, generis veneranda propago, \\ Et meritis caeli culmina celsa pete! \\ Vnici iam desunt solacia congrua fratris: \\ Sola deo vivis, vivis et imperio. \\ Largior extensa sit dextera, longior aetas, \\ Nestoreos superes annos et imperium. \\ 
      \end{verse}
  
            \subsection*{768}
      \begin{verse}
      Item laus Eunomiae \\ Fulens Ennomia decensque virgo, \\ Pollens nobilis et fecunda libris \\ 
        \pagebreak 
    \begin{center} \textbf{C 768, 325} \end{center} \marginpar{[02]} Et totum venerabilis per orbem \\ Atque in culmine constituta celso, \\ Subter cuncta videns, beata clemens \\ Mitis blanda gravis quieta vivis. \\ Sic es Christo parens talisque, priscis \\ Qualis rustica Veritas capillis. \\ Augustum caput atque consecratum \\ Constans erigis et promittis omnes \\ Castam vivere te deo per annos. \\ Adsit cuncta regens pater volenti; \\ Dextram filius ille Nazarenus \\ Succurrens tibi tradat imploranti; \\ Sanctus spiritus influens medullis \\ Sensus inriget et fomenta donet, \\ Gressus dirigat et viam procuret, \\ Confirmet pedes et fidem propaget, \\ Enses proterat et dolos latentes \\ Prodat magnificus protector, et te \\ Annis praegravem et corona laetam \\ Sanctam collocet angelus in urbem. \\ Hic te perpetua quiete donet. \\ Tunc quaeso meminisse te clientis, \\ Cum Christum fide videris serena. \\ 
      \end{verse}
  
            \subsection*{769}
      \begin{verse}
      Sde nunec c. 48e \\ \poemtitle{0. 4}de nunc c. \\ 
        \pagebreak 
     \marginpar{[254]} \begin{center} \textbf{C. 772,. 128} \end{center}B. M. \\ 4 47 \\ B. IV 189. \\ Cum sua cornua Luna \\ Condit in aere prima, \\ Signat abunde, sequentes \\ Imbre madescere soles. \\ Si dabit ore ruborem \\ Virgineo aurea Phoebe, \\ Monstrat in orbe futuros \\ Flamina fundere ventos. \\ Puraque si sit in ortu \\ Cynthia vespere quarto, \\ Totius haec erit index \\ Mensis arescere soles. \\ Iam maculis ubi mane \\ Sol variaverit ortum, \\ Fit notus imbre ruenti \\ Arboribusque sinister. \\ Ast ubi pallidus ipse est, \\ Crastina vel rubet hora, \\ Hinc fluit horrida tectis \\ Et ferit omnia grando \\ Igneus hic notat euros, \\ Caeruleus docet imbres, \\ Et rutilans maculosus \\ Fundere dat mare ventos. \\ Denique si occidet auro \\ ucidus aut micet ortu, \\ Permanet aether amoenus: \\ Cede pavescere nimbos. \\ 
        \pagebreak 
     \marginpar{[255]} \begin{center} \textbf{C. 772, 29—57} \end{center}Etsi aquilone moveri \\ Cernis in arbore frondes, \\ Te levis aura docebit, \\ Tempora credere clara. \\ Dumque polum astra relinquunt \\ Flammea terga sinentes, \\ Iam male flamina curvis \\ Temperat unda carinis . \\ Hinc super ardea nubes \\ Mergus et aequora scandit, \\ Iinc fulicaeque marinae \\ Litore ludere perg, \\ Si tonat en domus euri, \\ Si ephyrique resultant, \\ Si boreasque coruscat \\ (Maxima turbinis arma haec): \\ Vallibus effugit binc grus \\ Et lacus ambit birundo, \\ Corvus et agmina confert, \\ Garrit et improba cornix, \\ ucula hinc bibit auras \\ Naribus ardua spectans, \\ Hinc canit horrida rana, \\ Vermis et efferit ova. \\ Ast †potus imbre notaret,’ \\ Aethera quando serenamt, \\ Cum bene sidera splendent \\ Et micat ore Selena. \\ Haec tibi signa manebunt \\ 
        \pagebreak 
     \marginpar{[256]} \begin{center} \textbf{C. 772, 58 60. C. 772. 72a} \end{center}Omnia tempore tuta; \\ Ast tua cura valebit \\ Noscere iura polorum. \\ 
      \end{verse}
  
            \subsection*{772}
      \begin{verse}
      B. M. B. \\ IN. Campanianus. PA. Oybrio \\ Maiorum similis, nostrorum maior, Olybri, \\ Stemma poetarum, regula dogmatibus, \\ Trade notas, quis quaeque nitent bene dicta priorum; \\ Dux bonus audentes prisca tropaea doce. \\ Clarius auctorum pateant quae pollice laudes \\ Scis bene cunctorum, conscius ipse tuis. \\ P Olybrius. N. Campanino \\ Stigmata cur spectas maiorum infigere dictis, \\ Cuius iudicium sufficit ad titulos? \\ Censuram spernunt, quae per te lauta patescunt; \\ Sit satis ad laudem complacuisse tibi, \\ Omnia doctorum quem sic cinxere tropaea, \\ Vt cedat titulis lingua diserta tuis. \\ 4 \\ \poemtitle{rVLLI}B. M. B. \\ Fontibus in liquidis parvum requiesce viator \\ Atque tuum rursus carpe refectus iter. \\ 
        \pagebreak 
    
      \end{verse}
  
            \subsection*{}
      \begin{verse}
      \poemtitle{CARMINA}
      \end{verse}
  
            \subsection*{}
      \begin{verse}
      \poemtitle{CODICVM SAECVLI IXIV}
        \pagebreak 
    \begin{center} \textbf{B. VI 8. . 1697.} \end{center} \marginpar{[40]} \begin{center} \textbf{B. II 158.} \end{center}Vere rosa, pomis autumno, aestate frequentor \\ Spicis: una mihi est horrida pestis hiems. \\ Nam frigus metuo et vereor, ne ligneus ignem \\ lic deus ignavis praebeat agricolis. \\ B. VI 85. M. 6981. \\ 4 t \\ B. II 158. \\ Ego haec, ego arte fabricata rustica, \\ Ego arida, o viator, ecce populus \\ Agellulum bunc, sinistra et ante quem vides, \\ Erique villulam hortulumque pauperis \\ Tueor malaque furis arceo manu. \\ Mihi corolla picta vere ponitur, \\ Mihi rubens arista sole fervido, \\ Mbivirentedulcisnvapampino, \\ Mihi glauca duro oliva cocta frigore. \\ Meis capella delicata pascuis \\ In urbem adulta lacte portat ubera, \\ Meisque pinuis agnus ex ovilibus \\ Gravem domum remittit aere dexteram, \\ 
        \pagebreak 
    \begin{center} \textbf{C. 774, 12. C. 775, 1—14} \end{center} \marginpar{[260]} Teneraque matre mugiente vaccula \\ Deum profundit ante templa sanguinem. \\ Proin, viator, hunc deum vereberis \\ Manumque sursum habebis: hoc tibi expedit. \\ Parata namque trux stat  \lbrack †ecce \rbrack  mentula. \\ rVelim pol’ inquis? at pol ecce vilicus \\ Venit, valente cui revulsa brachio \\ Fit ista mentula apta clava dexterae. \\ B. VI 86. M. 1699. \\ 
      \end{verse}
  
            \subsection*{40}
      \begin{verse}
      B. I 160 \\ Hunc ego,  \lbrack o \rbrack  iuvenes, locum villulamque palustrem \\ Tectam vimine iunceo caricisque maniplis \\ Quercus arida rustica fabricata securi \\ Nutrior: magis et magis sit beata quotannis! \\ Huius nam domini colunt me deumque salutant \\ Pauperis tuguri pater filiusque adulescens, \\ Alter assidua cavens diligentia, ut herbae \\ Asper aut rubus a meo sint remota sacello, \\ Alter parva manu ferens saepe munera larga. \\ Florido mihi ponitur picta vere corolla, \\ Primitus tenera virens spica mollis arista, \\ Luteae violae mihi lacteumque papaver \\ Pallentesque cucurbitae et suave olentia mala, \\ I va pampinea rubens educata sub umbra . . . \\ 
        \pagebreak 
    \begin{center} \textbf{C. 775,152. C. 776 777} \end{center} \marginpar{[261]} Sanguine  \lbrack haec \rbrack  etiam mihi (sed tacebitis) arma \\ Barbatus linit hirculus cornupesque capella. \\ Pro quis omnia honoribus huc necesse Priapo est \\ Praestare et domini hortulum vineamque tueri. \\ Quare hinc, o pueri, malas abstinete rapinas: \\ o Vicinus prope dives est neglegensque Priapus. \\ Inde sumite: semita haec deinde vos feret ipsa. \\ B. M. \\ 40 \\ B. II 172. \\ Pallida mole sub hac celavit membra Secu \lbrack ndus, \\ Antiquis sospes non minor ingeniis, \\ 
      \end{verse}
  
            \subsection*{}
      \begin{verse}
      \poemtitle{E1}quo Roma viro doctis certaret Athenis. \\ Ferrea sed nulli vincere fata datur. \\ B. II 182. M. 834. \\ 4 t \\ B. II 177. \\ Vate Syracosio qui dulcior lesiodoque \\ Maior, lomereo non minor ore fuit, \\ llius haec quoque sunt divini elementa poetae \\ Et rudis in vario carmine Calliope. \\ 
        \pagebreak 
     \marginpar{[262]} \begin{center} \textbf{C. 778 78B, 18} \end{center}B. II 202. M. 865 \\ 
      \end{verse}
  
            \subsection*{778}
      \begin{verse}
      B. IV 187. \\ Tristia fata tui dum fes in Daphnide Flacci, \\ Docte Maro, fratrem dis inmortalibus aequas. \\ 779. 780 \\ Vde nunc c. \\ 781. 782 \\ ide nunc c. 812. \\ 
      \end{verse}
  
            \subsection*{783}
      \begin{verse}
      B. M. \\ \poemtitle{pPR0O}B. V 83. \\ Vade, liber, nostri fato meliore memento; \\ Cum leget haec dominus, te sciat esse meum. \\ Nec metuas fulvo strictos diademate crines, \\ Ridentes blandum vel pietate oculos. \\ Communis cunctis, hominem se regna tenere \\ Si meminit, vincit hinc magis ille homines. \\ Ornentur steriles fragili tectura libelli \\ Theudosio et doctis carmina nuda placent. \\ 
        \pagebreak 
    \begin{center} \textbf{C. 783. 9 12. C. 784 785} \end{center} \marginpar{[263]} Si rogat auctorem, paulatim detege nostrum \\ Tunc domino nomen: me sciat esse Probum. \\ Corpore in hoc manus est genitoris avique meaque: \\ Felices, dominum quae meruere, manus! \\ 
      \end{verse}
  
            \subsection*{784}
      \begin{verse}
      B. M. B. \\ Tllius lesperios cpiens componere ores \\ Eldit os libros ppellans iciorm. \\ Quo solo fers ecticfs fror est Cactiliae. \\ Coslio sper custos directus a aurbem, \\ ALuc orbs prieue sls, ens totc seatsr \\ Hic pls sole mct, crcius propter honestm. \\ 
      \end{verse}
  
            \subsection*{785}
      \begin{verse}
      ustio ribut \\ B. M. B. \\ Ercedt cnctos li libros pilosopor \\ Lbri quos feecit tres Tllius Oficiorm. \\ 
        \pagebreak 
     \marginpar{[264]} \begin{center} \textbf{C. 785e 785a} \end{center}
      \end{verse}
  
            \subsection*{785}
      \begin{verse}
      Augustino tributum \\ B. M. B. \\ Duleia non meri, qui non ustaeit amar. \\ EE i no sfuduit, sunt illi gaudic rar. \\ 
      \end{verse}
  
            \subsection*{785b}
      \begin{verse}
      Auustino trutm \\ 1B. M. B. \\ Prcigim ci sola fit diei colntas, \\ Actas non fnit emoliturue cetustas. \\ Dissoleit teps uigpui producit acdesse, \\ Si on prcsens, constt unoue necessc. \\ rg sper tali ui let conditione, \\ Aut niil acut minim clarct rtionis hbere. \\ 
      \end{verse}
  
            \subsection*{785}
      \begin{verse}
      \poemtitle{2VSIcI}IB. M. . \\ Ter quinos animo suadente per ardua libros, \\ Augustine, trahens nobile condis opus, \\ Et quamvis dederis numerosa volumina mundo, \\ Hlaec tamen ingenii est maxima palma tui. \\ Ena trium virtus deus est, quem divite verbo \\ Auribus infestum credula verba bibunt. \\ lle pium tinxit calamum quem lingua locuta est, \\ Descripsitque tua se deus ipse manu. \\ 
        \pagebreak 
     \marginpar{[265]} \begin{center} \textbf{86 786, 1} \end{center}
      \end{verse}
  
            \subsection*{786}
      \begin{verse}
      B.III17. M.1538. \\ ferphoditu \\ B. IV 114. \\ Cm me me mactcr reida gestret in alro, \\ Quid parerct, fertur cosluisse deos. \\ Ploebus act ‘per est’, Mars ‘emin’, Io ‘netrm’: \\ cm, gquc sum nfus, lermaproditus erm. \\ Quereti let de sic ait ‘ocitct arms,’ \\ Mars‘crce’, Pboebs‘agua’. sorsrtaqueiit. \\ Arbor obumbrat aqus conscedo, labitur ensis \\ Quem tulerm, csu, labor ect pse super. \\ Pes aesit rcnis, cgp iucidit ame, fuge \\ ir ulier neurum flutic telac cucem. \\ Vesco quem sem ii sors ecrem religui!; \\ lir, si sciero, cr trisue fii? \\ 
      \end{verse}
  
            \subsection*{786}
      \begin{verse}
      \poemtitle{DYNAMII}\poemtitle{De Lerine insula}B. M. B. \\ lnter praecipuas quas cingunt aequora terras \\ Nil simile in mundo est, sancte Lerine, tibi. \\ 
        \pagebreak 
    \begin{center} \textbf{C. 786, 326. C. 786b, 12} \end{center} \marginpar{[266]} Optima quae vivo fundata est insula saxo \\ Et super ornato tegmine plana viret. \\ Dives multiplici laetatur silva colore; \\ Arboribus mixtis fert coronata comas. \\ Per nemus umbrosum ventorum flamina vitat \\ Et portum sanctis praeparat illa viris. \\ Vt caret haec nunquam foliis nec tempora mutat, \\ Sic meritis semper pectora  \lbrack viva \rbrack  tenet \\ Prisca redivivo quo constat regula cultu, \\ Quo grex agnorum non timet ora lpi. \\ Culmine lHonoratus meritis et nomine dictus \\ Floruit hic primus incola, Christe, tuus. \\ Postquam sancta viri perrexit fama per orbem, \\ Vix  \lbrack alium \rbrack  meruit dives habere patrem. \\ Hic novus antiquum iecit ad leta draconem, \\ Nec post hic rabidus horrida fauce nocet. \\ Quod si vel casu veniat in litore serpens, \\ Vivere non ultra noxius ore potest. \\ Iustorum hoc opus est, ut nostri funeris auctor \\ Pellatur victus, vita iubente mori. \\ Sic electa deo praecellit insula saeclo, \\ Quae tot perfectis gaudet amoena viris. \\ Temnere mundauas optat qui mente procellas, \\ Invenit hic valvas iam, paradise, tuas. \\ 
      \end{verse}
  
            \subsection*{786b}
      \begin{verse}
      \poemtitle{ALEXANDRI}1 \\ B. V 42. M. 1029. \\ B. Pram. p. l. \\ \poemtitle{De ordine planetarum}p. 409. \\ Sortitos celsis replicant amfractibus orbes. \\ Vicinum terris circumvolat aurea Luna, \\ 
        \pagebreak 
    \begin{center} \textbf{C. 786, 310. C. 787} \end{center} \marginpar{[267]} Quam super invehitur Cyllenius. alma superne \\ Nectareum ridens late splendet Cytherea. \\ Quadriiugis invectus equis Sol ignens ambit \\ Quartus et aethereas metas, quem deinde superne \\ Despicit Armipotens. sextus Pbaetbontius ardor \\ Suspicit excelsum brumali frigore sidus. \\ Plectricanae citharae septem discriminibus quos \\ Assimilans genitor concordi examine iunxit. \\ B. II 172. M. 833. \\ 5 \\ B. IV 443. \\ Cum foderet gladio castum Lucretia pectus, \\ Sanguinis et torrens egereretur, ait: \\ ‘Testes procedant, me non favisse tyranno, \\ Sanguis apud Manes, spiritus ante deos.’ \\ 
        \pagebreak 
    \begin{center} \textbf{C. 788 790a} \end{center} \marginpar{[268]} 
      \end{verse}
  
            \subsection*{788}
      \begin{verse}
      Vde nunc c. 674a. \\ 
      \end{verse}
  
            \subsection*{789}
      \begin{verse}
      \poemtitle{EVCLERHM comi}B. M. B. \\ O pater omnipotens, celsi dominator Olympi, \\ O terrae pelagique sator, qui sedibus olim \\ Missus ab aethereis, humano corpore nasci \\ Non indignatus, caedis cruciatibus atrae \\ Mortales avidi rapuisti e faucibus Orci, \\ Dirige vela precor; curvo diducere rectum \\ Densaque Romulei dignoscere iura senatus \\ Ingenio permitte meo. qua luce reperta \\ Fas mihi sit populis reserata resolvere iura \\ Atque inter nebulas legum dignoscere causas. \\ B. III 145. \\ 
      \end{verse}
  
            \subsection*{790}
      \begin{verse}
      M 939. B. \\ Lingua brecris, breritate lcris, lertate moetr, \\  \lbrack obltate og, arrulitate occs. \\ 
        \pagebreak 
    \begin{center} \textbf{C. 791, 122} \end{center} \marginpar{[269]} 
      \end{verse}
  
            \subsection*{791}
      \begin{verse}
      \poemtitle{PATRICII}B. M. B. \\ Plurima mira malum signantia signa futurum \\ Sive bonum dederat clemens deus, arbiter orbis, \\ Vt terreret eos, quos illa videre volebat. \\ Omnia paene loca, quibus haec iam facta fuerunt, \\ , Tempora cuncta simul brevitas intacta reliquit. \\ Tres simul in caelo visi sunt currere soles; \\ Terribilem quaedam tellus efuderat inem; \\ Maxima pars noctis fulgebat luce diei; \\ Ecce lapis cecidit de caelo magnus in amnem; \\ Circulus et solem circumdedit aureus altum. \\ Agnus in Aegypto mirum fuit ore locutus; \\ Bos loquitur Romae simulanti voce prophetam: \\ ‘Copia farris erit nobis hominesque peribunt.’ \\ Spicas turba hominum iam vidit in arbore natas; \\ Panibus abscisis sanguis quoque fluxit habunde \\ Coram convivis, quos signum terruit illud; \\ Bos peperit dudum in † lectis conventibus agnum; \\ Armatas multis acies equitesque diebus \\ Aere pugnantes crudeliter arma movere \\ o Ante quidem cives viderunt tempora belli; \\ Natus equus fuerat totus homo tempore nostro \\ Aque homine hinnitum faciens quoque moris equini, \\ 
        \pagebreak 
    \begin{center} \textbf{C. 791. 2381. C. 792 793} \end{center} \marginpar{[270]} Tam comedens fenum, quam panem et cetera edebat; \\ Natus erat duplex homo vivens tempore longo, \\ Quadrimanus, bipes atque biceps et pectore bino, \\ Atque duas animas unum ventremque gerebat. \\ Quorundam pars posterior nova verba sonabat. \\ Tunc mirabiliter cantans modulamina quaedam \\ Vox avis audita est dicentis talia verba: \\ ‘Mane novo surgens dominum laudabo potentem.’ \\ His ita prodigiis signisque per omnia dictis \\ B. M. \\ 
      \end{verse}
  
            \subsection*{792}
      \begin{verse}
      B. V 388. \\ rs sutc fatacles quae lucunt ila sorors: \\ Clotho colm bailt, Laccsis trlit, Ltropos occat. \\ 
      \end{verse}
  
            \subsection*{793}
      \begin{verse}
      B. I B. M. 561. \\ B. V 88. \\ Neptuno et Plutone \\ \poemtitle{De ove et}Iuppiter astra, fretum Neptunus, Tartara Pluto \\ Regna paterna tenent, tres tria, quisque suum. \\ 
        \pagebreak 
    \begin{center} \textbf{C. 794, 1—25} \end{center} \marginpar{[07]} B. M. \\ 
      \end{verse}
  
            \subsection*{794}
      \begin{verse}
      B. V 388. \\ Conquerar an sileam? monstrabo crimen amicae \\ An quasi iam sanus vulnera nostra tegam? \\ Non queror ant molles oculos aut aspera crura \\ Nec vitio quovis exteriora premo: \\ Quod queror, est animi! laudaret cetera livor: \\ Verba fide, vitiis lubrica forma caret. \\ lla decem menses mecum feliciter egit \\ Gratis in amplexus docta venire meos: \\ Aemulus ecce meus gemmis male fisus et auro \\ llanc petit, ingeminat munera, flectit eam. \\ Muneribus vicit, quoniam natura vel usus \\ Praeter flagitium nulla dedere sibi. \\ Thersiten gerit in facie, gerit intus Oresten: \\ Pulcrior iste tamen, mitior ille fuit. \\ Non prins incurrit leviores ille reatus \\ Nec gradibus certis destitit esse bonus, \\ Set simul omne nefas auso puerilibus annis \\ Iam praeter facinns nulla licere putat. \\ Tnurpis atrox exlex, naturae crimen et hostis \\ In luctu ridet, flet nisi flenda videt. \\ Sufficit exemplis totum corrumpere mundum, \\ Sufficiunt sceleri nomina nulla suo, \\ Quippe tot illicitis famam lacerare laborat, \\ Vt nulla redimi laude vel arte queat. \\ Cur igitur placuit? quid honesti vidit in illo, \\ 
        \pagebreak 
    \begin{center} \textbf{C. 794, 26 54} \end{center}Quem iam nulla sequi praeter honesta pudet? \\ Cur, inquam, placuit? dignusne placere puellis, \\ Qui non exilio sed cruce dignus erat, \\ Lapsus in excessus tantos, ut nulla putaret \\ Deteriora fide vel potiora dolis? \\ Cur placuit letale nefas, cur dedecus orbis, \\ Cur tam terribilis larva pudorque patris, \\ Qui non tam locuples rebus quam pauper honesto \\ Et minus infamis quam vitiosus erat? \\ Crimen opes redimunt, reus est crucis omnis egenus, s \\ Et laudes hominum pensat acervus opum. \\ Hic quoque nec vita nec nobilitate parentum \\ Nec specie placuit, sed quia dives erat. \\ Divitiis animnm tenerae turbavit amicae \\ Divitiisque patent oscula crura sinus. \\ Iam nec pura fides nec largi gloria sensus \\ Nec probitas morum nec bona fama invat: \\ Aurum sinceras solitum praevertere mentes \\ Mortales animos in scelus omne vocat. \\ Aurum dum speret, nil iam negat Iectoris uxor, \\ Iam populo iungi sustinet asse dato. \\ Dona truces animos et verba severa relaxant: \\ Penelope donis altera Thais erit. \\ Sed iam Thais erit lunone severior ipsa, \\ Si nullas habeat pulcher amator opes. \\ Vos igitur iuvenes, quos nondum fervor amoris \\ Attigit, illarum laudo cavere dolos. \\ Nam licet uratur, tamen in complexibus ipsis \\ Quaeque salictores quaerit habere novos. \\ 
        \pagebreak 
    \begin{center} \textbf{C. 794, 555s. C. 795 797, 1—2} \end{center} \marginpar{[273]} Protea multiplicem solet ars retinere, sed illas \\ Quin elabantur, nil retinere nequit. \\ B. M. \\ 
      \end{verse}
  
            \subsection*{795}
      \begin{verse}
      B. V 390. \\ Lumina, colla, genae, flavi fexura capilli \\ In Ganymede suo flamma fuere Iovi. \\ Iuppiter in puerum quaerens sibi pauca licere \\ In puero statuit cuncta licere deus. \\ Oblitusque poli curas et murmura divum \\ Et linguam laesae coniugis atque lovem \\ Iliacum tulit ad superos, ad sidera sidus, \\ Et se tunc tandem credidit esse deum. \\ Vtque puer pelex visu tactuque liceret, \\ Oscula nocte Iovi, pocula luce dabat. \\ B. M. \\ 
      \end{verse}
  
            \subsection*{796}
      \begin{verse}
      B. V 390. \\ Ad cenam Varus me nuper forte vocavit: \\ Ornatus dives, parvula cena fuit. \\ Auro, non dapibus decoratur mensa; ministri \\ Apponunt oculis plurima, pauca gulae. \\ Tunc ego ‘non oculos sed ventrem pascere veni: \\ Vel tu pone dapes, Vare, vel aufer opes.’ \\ B. M. \\ 49 \\ B. V 390. \\ Graecinum virgo, puerum Graecinus amabat \\ Et puer in sola virgine captus erat. \\ 
        \pagebreak 
    \begin{center} \textbf{C. 797, 3 4. C. 798 798 , 16} \end{center} \marginpar{[074]} Tradidit hanc puero Graecinus, se puer illi, \\ Et fruitur voto virque puerque suo. \\ B. M. \\ 
      \end{verse}
  
            \subsection*{798}
      \begin{verse}
      B. V 382. \\ Signifer aethereus, mundus quo cingitur omnis, \\ Astra tenet tantum se sede moventia septem; \\ Caetera nam proprio stant semper in ordine fixa. \\ Saturni sidus summa concurrit in arce \\ Ter denoque suus completur tempore cursus. \\ Inde Iovis cursus bis senis volvitur annis; \\ Sic Mars quingentis †rubeus quadraginta diebus. \\ Ast uno Solis completur circulus anno, \\ Trecentis Venus octo et quadraginta diebus, \\ Mercurius centum triginta novemque diebus. \\ Bis denis septemque diebus Luna peragrans \\ Octo horisque simul proprium sic conticit orbem. \\ B. V 117. \\ 
      \end{verse}
  
            \subsection*{798a}
      \begin{verse}
      M. 1057. B. \\ Est ubi non imber nec ros dilabitur umquam, \\ Est ubi nec nix est nec glacialis hiems, \\ Est sine vite solum, quaedam  \lbrack est \rbrack  sine matre propago. \\ Est sine rege tribus, est sine nave fretum. \\ Non sine mors gemitu, non partus absque dolore, \\ Non nix absque gelu, non notus absque sono. \\ 
        \pagebreak 
    \begin{center} \textbf{C. 79a, 728. C. 799, 1 2} \end{center} \marginpar{[275]} Absque calore focus non est nec amore puella, \\ Non sine carne pilus, non sine pelle caro, \\ Non sine matre puer, non est sine vite Lyaeus, \\ Non sine pisce lacus, non sine sorde palus, \\ Non sine laude pius, non est sine crimine latro, \\ Non sine fraude forum, non sine mure penus. \\ Non urbs absque malo, non scortis absque lupanar, \\ Non sine voce sonus, non sine luce dies. \\ s Dulce sopor fessis et terris flumina siccis, \\ Dulce patri proles divitibusque gaa. \\ Tela viro decus est peplumque columque puellis. \\ Laus sine lite domus, laus sine fure locus. \\ Scire aliquid laus est, et nil nescire verendum: \\ Nam sine doctrina quid nisi turpis homo? \\ Visus, auditus, tactus, olfactus, hiatus: \\ Vnum quinque duces sub vice corpus alit. \\ Dant oculi lacrimas, auditus concipit auris, \\ Cor dolet, os loquitur, nasus odore sapit. \\ Armeniae tigres, Libyae fert terra leones, \\ Est elefas lndis, tura Sabaeus habet. \\ Grifes hyperboreae septem subiecta trioni \\ Genti non absunt, aemula pestis edax. \\ B.III I18. M.133. \\ (lim \\ 898) \\ . \\  \lbrack AMoroquidliusiasps.quidiasidesenss. \\ uidsensratoqirctioneods \\ 
        \pagebreak 
    \begin{center} \textbf{C. 799, 3. C. 800 801, 12} \end{center} \marginpar{[276]} Quid levius fama? fulmen. quid fulmine ventus. \\ Quid vento? mulier. quid muliere? nihil. \\ 
      \end{verse}
  
            \subsection*{800}
      \begin{verse}
      e sithium suer verum \\ Pastor arator eques pavi colui superavi \\ Capras rus hostes fronde ligone manu. \\ caprispastis,deruresato,hostesubactoDe \\ Nec lac nec segetes nec spolia ulla tuli. \\ 
      \end{verse}
  
            \subsection*{801}
      \begin{verse}
      B. M. B. \\ \poemtitle{De adventu cuiusdam novi maistri}Lucifer exoritur, emittunt sidera lumen; \\ Quom latuere diu lumina, stella nitet. \\ 
        \pagebreak 
    \begin{center} \textbf{C. 801, 30. C. 802 804, 12} \end{center} \marginpar{[277]} Nube prius latuit lux non extincta sed absens: \\ Non sibi sed mundo perdita stella nitet. \\ Nube carens, depulsa die, dans lumen Olympo \\ Mundus ovat, fugiunt nubila, stella nitet. \\ Quam gallus totiens cantu praedixerat, ecce \\ Lnx oritur, mundo reddita stella nitet. \\ Per gallum formam, per lucem signo magistrum; \\ lic canit, illa refert; haec nitet, ille docet. \\ 
      \end{verse}
  
            \subsection*{802}
      \begin{verse}
      B. M. B. \\ Fupe saclitr oris, dum densis eceprbs haeret: \\ ac Lgres genitos fabla stire reert. \\ pliciti sunt sc etiisr a cepribs um, \\ A eoerrcce do, ceterac cnpis abet. \\ Gcs c ceprc ten, oc spper, cellere mollis, \\ Gens e pfre so caua, olos, pens. \\ B III275. M.262. \\ 
      \end{verse}
  
            \subsection*{803}
      \begin{verse}
      B. V 391 \\ arcc, prccor, cirgo, toties ii elt rileri sgp \\ 
      \end{verse}
  
            \subsection*{804}
      \begin{verse}
      B. M. \\ \poemtitle{De quieta vita}B. IV 57. \\ Phoebe, fave coeptis nil grande petentibus aut quod \\ A te transferri turba maligna velit. \\ 
        \pagebreak 
    \begin{center} \textbf{C. 804, 314. C. 805 806. 1} \end{center} \marginpar{[278]} Divitias averte; alios praetura sequatur \\ Optantes, alios gratia magna iuvet. \\ Hic praefectus agat classes alienaque castra \\ Laetus sollicita †sollicitate roget, \\ Bis senos huius metuat provincia fasces; \\ Audiat hic plausus ter geminante manu. \\ Pauperis arva soli securaque \rbrack  carmina curem, \\ Nec sine fratre mihi transeat una dies. \\ Otia contingant pigrae non sordida vitae, \\ Nec timeat quidquam mens mea nec cupiat; \\ Ignotumque diu solvat non aegra senectus \\ Ossaque compositi frater uterque legat. \\ B. M. \\ 
      \end{verse}
  
            \subsection*{805}
      \begin{verse}
      B. IV 189 \\ Aeneas et Amor, pariter Iocus atque Cupido \\ Sunt nati Veneris diversis patribus orti. \\ Anchises primum genuit Mavorsque secundum, \\ Vulcanus quartum; love natus tertius astat. \\ Eloquio primus dulcis, dat amara secundus, \\ Tertius illecebras, fervorem denique quartus. \\ 
      \end{verse}
  
            \subsection*{806}
      \begin{verse}
      B. M. \\ Argumenta ucani \\ B. V 413. \\ Proponit primus liber, invehit, invocat atque \\ 
        \pagebreak 
     \marginpar{[279]} \begin{center} \textbf{C. 806, 2 22} \end{center}Exponit causas, cursus properantis ad urbem \\ Caesaris et nimios hic narrat in urbe timores. \\ Quadruplices questus libri pars prima secundi \\ Continet; eiusdem pars proxima verba Catonis \\ Et Bruti. dicit, quo foedere Martia nupsit. \\ Hostis in occursum ducit pars tertia; Magnnm \\ Opposuisse manus notat et quod Caesaris ira \\ Cuncta ruunt. arcesque capit, cedentibus instat. \\ Ast uni vitam tribuit qui nuntius hosti, \\ Exemplumque fuit; quo viso Magnus ad omnes \\ Turmas ipse suas hortandas magna minatur. \\ Hinc pars quarta notat Pompeium tunc properasse \\ Brundusium; tandemque videns maris ostia claudi \\ lesperiam ptppesque duas in parte reliquit. \\ 
      \end{verse}
  
            \subsection*{III}
      \begin{verse}
      Tertius exponit primo quid Hulia dixit, \\ Quid Magnus fecit, audax quo Curio missus. \\ Altera pars libri dicit, quod Caesar in urbem \\ Ivit opesque dedit Romae nolente Metello \\ Militibus, lagnique notat qui signa sequuntur. \\ Vltima quod tendens lispanas Caesar ad oras \\ Massiliae stetit: hanc sed vicit in aequore Brutus. \\ 
        \pagebreak 
    \begin{center} \textbf{C. 806., 2345} \end{center} \marginpar{[280]} 
      \end{verse}
  
            \subsection*{IV}
      \begin{verse}
      At quarti libri narrat pars prima, quod ivit \\ Caesar in llispanos ad iussa ducesque reversos. \\ Mortem Vultei cum multis altera pars dat. \\ Vltima, quod Varum pepulit campoque fugavit \\ Curio, fraude Iubae cecidit qui strage suorum. \\ In prima quinti Pompeio Roma regenda \\ Est data. multa timens pro se responsa recepit \\ Appius; exponit pars proxima seditionem \\ Sedatam poena. mare transiit urbe relicta \\ Caesar, qui queslus, quod non Antonius ultra \\ Iverat, expertus fuit ipse pericula ponti. \\ Vltima, quod posita mansit Cornelia Lesbo. \\ 
      \end{verse}
  
            \subsection*{VI}
      \begin{verse}
      ‘Postquam castra’ notat, quod Caesar victus ab hoste s \\ Fugit in Emathiam, quamvis clausisset is ipsum. \\ Hinc et Tbessaliam quae sit gentemque profanam \\ Describit. damnat Sextum non dina petentem. \\ 
      \end{verse}
  
            \subsection*{VII}
      \begin{verse}
      ‘Segnior Oceano’ casu quo bella geruntur \\ Ostendit primo, sic et quae dixit uterque. \\ Proxima pars bellum describit, et ultima, Magnum \\ Devictum cepisse fugam. Sed Caesar habendas \\ Militibus monstravit opes castrisque recedit. \\ 
        \pagebreak 
    \begin{center} \textbf{C. 8060, 4 60. C. 807, 1—4} \end{center} \marginpar{[281]} 
      \end{verse}
  
            \subsection*{VIII}
      \begin{verse}
      ‘Iam super lHerculeas’ quo fugit, denotat atque \\ Quid dixit  \lbrack Magnus . . quando quaerere Parthos \\ Consuluit: sed cassa fuit sententia Magni. \\ Parsque secunda notat Pompeium morte peremptum \\ Indigna; Phariis pars ultima datque sepulcrum. \\ ‘At non in Pharia’ dicit, quod bella Catoni \\ Libertate placent, qui Sextum multa minantem \\ Corripuit, postquam scivit de funere Magni. \\ Altera pars multos correptos voce Catonis \\ Dicit per Syrtes fore multa pericula passos. \\ Tertia quod Caesar simulavit ferre dolorem \\ Nec doluit saevus generi cervice recisa. \\ ‘Vt primumi primo notat ut perrexit ad urbem \\ Aegypti Caesar et ut est Cleopatra locuta. \\ Et dapibus sumptis Nili disquiritur ortus. \\ Parsque secunda refert famulos qui fata parabant \\ o Prava duci caesos adversa nefandaque passos. \\ 
      \end{verse}
  
            \subsection*{807}
      \begin{verse}
      B. M. B. \\ Ad fora pictoris dum numina lignea gestans \\ Venalesque deos tendit asellus iners, \\ Vertice detecto vel flexo poplite turba \\ Obvia cum precibus numina sancta colit. \\ 
        \pagebreak 
     \marginpar{[282]} \begin{center} \textbf{C. 807, 516} \end{center}Gaudet iners tanto secum sublimis honore, \\ Virtutes. mores, gesta, genus numerans. \\ Sors vaga cuncta rotat: olitorem nactus asellus \\ Allia cum porris fertque refertque forum. \\ Horret turba procul et olenti cedit asello. \\ Hic stupet ignarus et gemebundus ait: \\ ‘Cur sic contempnor? cur tanti cessit honoris \\ Cloria? quae coluit me modo, turba fugit. \\ Haec est lex mundi: genitum consistere quicquam \\ Non valet. Hic factus †portheor alter ego.’ \\ Seputatindignemutatumfunctushonore, \\ Post positos fasces vilis ad ima cadens. \\ 
        \pagebreak 
    
      \end{verse}
  
            \subsection*{}
      \begin{verse}
      \poemtitle{CARMINA}
      \end{verse}
  
            \subsection*{}
      \begin{verse}
      \poemtitle{CODICVM RECETIVM}
        \pagebreak 
    \begin{center} \textbf{Olim 799 830 ed. pr.} \end{center}Tile anot. crit. \\ 
      \end{verse}
  
            \subsection*{808}
      \begin{verse}
      B M. \\ Aegritudo Perdicae \\ V 112. \\ ie mihi, parve puer, numquam tua tela quiescant? \\ Non sat erant frondes, non nndae nec fera nec fons? \\ Non Satyrus, non taurus amans, uon ales et imber? \\ Non tristes epulae, post quas petit aera Tereus? \\ HIoc tibi restabat postremum, saeve Cupido: \\ Ad dirum matris iuvenem conpellis amorem! \\ Mutai, precor, flammas alioque intende sagittas: \\ Quid possit nosti pietas. ah, perlida mater, \\ 
        \pagebreak 
     \marginpar{[286]} \begin{center} \textbf{C. 808, 931} \end{center}Et Paphiae quam triste decus †arcere furorem’ \\ Claudite nunc animos miserandaque pectora, matres, \\ Ne scelus hoc vestras iteratum polluat aures \\ Neu vos sollicitas temptet dolor iste nefandus \\ Viderit ac simili poena commissa recuset. \\ Namque omnes superos et cetera templa deorum \\ Ture pio sacroque mero votisque colebat, \\ Oblitus Veneris puerique oblitus Amoris. \\ Hinc offensa dea est, haec diri causa furoris, \\ Iinc quoque partus amor redeunti ad tecta parentum, \\ Infelix Perdica, tibi nam nuper Athenas \\ Venerat et studiis animos praebebat et aures ; \\ Hinc quoque regreditur matris periturus amore. \\ Infelix qui Cecropias nunc deserit arces: \\ Iam praeda est Veneris; iam flammis atque sagittis \\ Armatus tenuit servans iter omne Cupido. \\ Lucus erat variis in frondibus, undique saeptus, \\ Quem Pboebi Dapbne foliis diffusa tenebat \\ Et myrtus Paphies, speciosi testis Adonis, \\ Egrediturque solo fundens sua brachia pinus \\ Iae Phrygius pastor spernens in amore Cybeben \\ Desertusque viro per tympana plangitur Attis \\ Fonsque regit medio nota per gramina lapsum: \\ lllic dispersi flores mixtique colores \\ Ostendunt, Veneris quid amor; nam candidus illic \\ Flos Narcissus amat veteris vestigia fontis \\ 
        \pagebreak 
     \marginpar{[287]} \begin{center} \textbf{C. 808, 3561} \end{center}Et rosa purpureum spargens per prata ruborem; \\ Seu Veneris cruor est seu flamma Cupidinis ista \\ Nescio, sed gratus memini, quia servit amori. \\ lunc lucum Filomela tenet: circumvolat alis \\ Et dulcis queritur fetus sesaque rmo,. \\ o Lucus Amoris: erat delapsus ab aethere pinnis. \\ Namque illi conquesta Venus mandaverat ignis. \\ Paruit imperio matris pharetramque sagittis \\ Plenam fundit humi tollitque e pluribus unam \\ ‘Hoc telum est’ dicens ‘olim quo luppiter auro \\ A Decidit et Danaen fulvo compressit amore. \\ Est’ (aliud tolli) lLedam hoc quo cygnus amavit, \\ AntiopamSatyrnstenuit.iamfessasagittast. \\ Quo, Perdica, tibi calamo frmemus amorem? \\ Vulnera iam nostrae veteres fecere sagittae, \\ Nunc nova visenda est. dixit rivumque secutus \\ Quaerit arundineas scrutatus limite silvas. \\ Nec mora nata deo est. namque obvia venit arundo, \\ Quam puer excissam totis radicibus aufert. \\ Et primo mollis eradit pumice libros, \\ Post volucri cupiens missu librare sagittam, \\ Pinnam de propriis ardentibus abscidit alis \\ Et religat cera, possit quo certa tenere \\ Quod temptabat opus. et Amoris pluma tenebat. \\ Iam sol emenso radios libraverat ortu \\ Atque diem sexta magnum discreverat hora: \\ Omnia per terras animalia fessa calore \\ 
        \pagebreak 
     \marginpar{[288]} \begin{center} \textbf{C. 808, 62—83} \end{center}Sideris aestiferi frondis sub tecta subibant. \\ Ad lucum Perdica venit fessusque labore \\ Inlimes respexit aquas lymphasque recentes \\ Vmbriferumque nemus, mixtos per gramina flores. s \\ Ingressus postquam est lucos Perdica rigentes, \\ Talibus est verbis socios ac voce secutus: \\ ‘O socii, vestro iustum si corde videtur, \\ Defessos artus ac membra calore gravata \\ Hic poterit relevare locus. nam frigida fontis \\ Vena fluit, flores sunt hic,  \lbrack bic \rbrack  dulcia prata’ \\ (Heu, Perdica, gravis aestus radiosque micantes \\ Solis te fugisse putas lucosque petisse \\ lgnoras: intus gravior tibi flamma paratur!) \\ Sic postquam fatus, fusi per gramina terrae \\ Accipiunt epulas et dulcia dona Lyaei, \\ Post somno reparant vires. tunc aliger ille \\ Paruit officio mutatusque ore Cupido \\ Perdicae reddit Castaliam nomine matrem, \\ Complexusque dedit per somnia tristis image \\ Et saevo iuvenem confodit pectora telo. \\ Matris enim miserae caros dinoscere vultus \\ Non poterat, quam parvus adhuc dimiserat olim, \\ Cum peteret divae doctissima templa Minervae. \\ Qui postquam somno miser est deceptus acerbo, \\ Ardet in incestum, verum simulante ligura: \\ 
        \pagebreak 
    \begin{center} \textbf{C. 808, 84— 11} \end{center} \marginpar{[289]} 8 lngrediturque suae regalia limina matris. \\ Continuo natum famulae venisse parenti \\ Castaliae dixere suum: pietatis honore \\ o lla memor nati venienti est obvia facta \\ Osculaque aequa dedit materni plena decoris. \\ Quam miser ut vidit, suscepit vulnere curas; \\ o Et quotiens iuvenis mutata est mente figura \\ Vel quotiens pulsante deo nova forma secuta est, \\ s Haesit et insano obstipuit deceptus amore. \\ ‘Heu, eo quam vidi per somnia tristia demens, \\ Mater eras? aut ista tibi par fertur imago? \\ Est sed caeca . \\ Nam fari scelus est,  \lbrack est \rbrack  admissi quoque crimen.’ \\ Talia constanter secum Perdica locutus. \\ Sed nox umbriferis per caelum roscida pinnis \\ Presserat arios fugientis solis bonores \\ Cunctaque per terras animalia pressa sopore: \\ Solus ibi dulci numquam Perdica quieti \\ s Tradidit assiduis ardentia lumina flammis. \\ Nox ipsi maesta  \lbrack est \rbrack : vigilat metuitque tepetque, \\ Suspirat numquam requiem daturus amori. \\ Omnia fessa domat caelestia sidera somnus \\ Fluminaque  \lbrack ille tenet nec non maris imperat undis; \\ I1o Corpora vel modicam conpellit adire quietem. \\ Pro dolor! hoc scelus est soli vigilantis amori. \\ Tunc quoque Perdicam  \lbrack saevo \rbrack  premit igne Cupido, \\ 
        \pagebreak 
    \begin{center} \textbf{C. 808, I113 137} \end{center} \marginpar{[290]} Vt possit vix ferre vicem. nam fulmine tactus \\ Ardebat miser;  \lbrack et \rbrack  ducens suspiria cordis, \\ Quae puer edocuit mortales cire Cupido, \\ 115 \\ Tales †triste feras reddit de pectore voces: \\ ‘Nox sceleris secreta mei, nox conscia cladis, \\ Soli me commendo tibi nostrumque furorem. \\ Tu nosti quid possit amor, sine te nihil ille, \\ Seu Veneris pars es tu seu Venus aut Venus in te est: \\ Des requiem miserando precor, et posse fateri. \\ ‘t matri narrabo nefas’ Tamen ibo coactus? \\ ‘Credamus!’ Quibus hoc poteris conponere verbis \\ Aut vox qualis erit? adgressus namque parentem \\ 125 \\ ‘Mater, ave’ dicturus ero. quid deinde? tacebo! \\ 123 \\ Oedipodem thalamos matris vult fama subisse \\ 126 \\ Incestosque toros: satis est quod nescius iste \\ Commisit, culpamque tulit liecet ille nefandam, \\ Exegit, sese privat dum lumine, poenam.’ \\ Talis Perdicam per noctem cura premebat \\ 130 \\ Et proprium miserando nefas fit causa laboris. \\ Iamque dies ortus † clarior nudaverat orbem \\ Et radiis Titan noctis disperserat umbras: \\ Deficiunt iuveni paulatim fortia membra \\ Decoquiturque umor, cunctos qui continet artus. \\ 135 \\ Namque undas cereremque negat victumque ciborum. \\ Tunc quoque sollicitam monuit maestamque parentem \\ 
        \pagebreak 
    \begin{center} \textbf{C. 808, 138 162} \end{center} \marginpar{[291]} Maternae pietatis honos, famulosque vocavit \\ Ad sese iussitque artis medicae  \lbrack venerandos \rbrack  \\ Primores qui forte forent, adducere secum. \\ Iussa citi peragunt: vitae venere magistri \\ Ingressique fores atque abdita tecta cubantis \\ Inveniunt iuvenem postrema clade gravatum. \\ Et primum quaerunt, quae causa laboris inesset, \\ 1s Post vena  \lbrack est \rbrack  temptata: sed haec pulsusque quietus. \\ Esse negant causas vitiati corporis illic, \\ Sed iecur et splenis temptanda cubilia et atri \\ Fellis: quae metuenda † domus, sunt omnia sana \\ Per proprium digesta larem, sunt cuncta quieta \\ so Et vitae devota suae; sed dira procella \\ Mente latens caecos urguebat pectore coeptus. \\ Hippocrates illic fuerat qui forte vetustus \\ Ac vitae spatio lonum qui ceperat usum, \\ Restitit ac secum docto sermone locutus \\ ‘Quid, medicina, tacest rationem redde petenti! \\ Non isti calor est pulsus nec vena minatur; \\ Non sacrae partes, quibus omnis vita tenetur, \\ Discordare parant, †cum mox elementa resolvant, \\ Quae faciunt hominem, dum quattuor ista ligantur. \\ Stridenti gremio vivaces inpedit auras. \\ Non omenta suas per mollia viscera sedes \\  \lbrack Deficiunt \rbrack , non corda vagi pulmonis anhelant \\ 
        \pagebreak 
     \marginpar{[292]} \begin{center} \textbf{C. 808, 163188} \end{center}Intersaepta sero, non ilia concita coxis \\ Incutiunt saevos †iaculata saepe dolores: \\ Displicet os solum, quod sunt suspiria longa.’ \\ 165 \\ Sic fatus fessae scrutatur conscia venae. \\ Ingreditur mater. tum, quae fuit ante tenenti \\ Mitis et in lentos motus aequaliter acta, \\ Inprobiter digitos quatiens pulsatibus urguet, \\ Sc mentis confessa nefas. magnusque virorum \\ 10 \\ Invenit Iippocrates, quae pectore clausa fuere. \\ Et tali sequitur miserandam voce parentem: \\ ‘Causas mater habes: medicinae munera cessent. \\ HIic animi labor est: abeo; iam cetera di dant.’ \\ Talia fatus abit. matrem nova cura premebat \\ 175 \\ Per varios divisa metus, natumque cubantem \\ Adgreditur redditque pio de pectore voces: \\ ‘Nate, precor, miserere mei, miserere tuorum! \\ Lumina tu partus, tu me facis esse parentem. \\ Inclita si virgo est, hymenaeos iungere possum, \\ I80 \\ Sive suo matrona fovet viduata marito. \\ Ne dubites.  \lbrack haec \rbrack  cura mea est, hoc maesta verebar, \\ Inlicitos ne forte toros temptare mariti \\ Cogeret acer amor matrisque gravaret honorem.’ \\ lle silet solumque trahit suspiria longa \\ 185 \\ Avertens faciem, nec matrem cernere rectis \\ Luminibus poterat sacro prohibente pudore. \\ ‘Mater’ ait, ‘discede, precor: plus uris amantem.’ \\ 
        \pagebreak 
    \begin{center} \textbf{C. 808, 19 214} \end{center} \marginpar{[293]} Roscida post radios aeternaque lumina solis \\ o Nox tenebris diffusa suis conpresserat omnes. \\ At iam te, Perdica, puer numquam ille Cupido \\ Vel partem minimam patitur decerpere  \lbrack somni \rbrack : \\ Sed solus tenet et tenebrosa tectus in umbra \\ Continuus tollit pharetras ac tela furoris \\ s Et tecum vigilat per noectis tempora longa, \\ Intorquens dira assiduis incendia flammis. \\ Et Pudor buc aderat, proprio comitante vigore. \\ Stant duo diversis pugnantia numina telis \\ Ante toros, Perdica, tuos: Amor hinc, Pudor inde. \\ Inde Cupido monet secreta referre furoris, \\ Inde Pudor probibet vocisque \rbrack  exordia rumpit \\ Famamque surgentem revocit.† negat alter, at alter \\ Ire iubet propriumque nefas exponere mentis, \\ Verbaque multa docet; quae vix e pectore lapsa \\ os Perdicae miseri moriuntur in ore pudico. \\ Sed postquam calor inmensus per pectora currens \\ Vsserat exesas ardenti corde medullas, \\ Talia dimittit reserato pectore verba: \\ ‘Saeve puer, semper lacrimis et funere gaudes! \\ O scelerate, tuas si tu paterere sagittas! \\ Sique tuos ignes in te convertere discas, \\ Et credas, quid possit Amor! sed parce, Cupido \\ lnprobe! quae mandas, non possum dicere matri. \\ Tormentis adfige tuis, constringe catenis: \\ 
        \pagebreak 
    \begin{center} \textbf{C. 808, 21522} \end{center} \marginpar{[294]} Non fatear. totas in me consume sagittas, \\ 215 \\ Quotquot amoris habes, et si tibi tela furoris \\ Defuerint, etiam tibi de Hove fulmina sumas: \\ Vincere non poteris sanctum, scelerate, pudorem!’ \\ Talia per noctem iuvenis miserandus agebat. \\ Interea matrem nati nova cura premebat, \\ 220 \\ Multaque quaerenti placuit sententia talis, \\ Matronas omnes totis e moenibus urbis \\ Ad propriam concire domum, si quis vigor illis \\ Aut species inlustris erat vel forma superba, \\ Quae proprio iuvenem statuisset amore †gravare. \\ 225 \\ Hoc visum placitum matri. Non distulit ultra. \\ Iam \lbrack que \rbrack  dies ortus †clarior nudaverat orbem: \\ Matronae veniunt forma cultuque micantes. \\ li erat Andromeda, hic altera Laudamia, \\ Ditior haec Danae, fulgentior altera Glauce, \\ 230 \\ Candidior Chione venit  \lbrack altera et altera Dirce. \\ Iuc etiam tenerae sanctae venere puellae, \\ Virgineum florem servantes lege maritis. \\ Has tristis Perdica videns et lumina lectens \\ In matrem traxit dura suspiria corde \\ 235 \\ Et tali secum miser est sermone locutus. \\ ‘Pro dolor, o superi defecerat altera forma: \\ Mater amanda fuit. sed vincere certo furorem \\ Quaerendo vultus, liceat quos iure tenere. \\ Hoc etiam voluisse nefas. sed respice, quales \\ 240 \\ Vituperas! cernis, niteat quae gratia formae? \\ Sunt niveae, sunt hic procero corpore pulchrae, \\ 
        \pagebreak 
    \begin{center} \textbf{C. 8O08, 243 238} \end{center} \marginpar{[295]} Virgineo \lbrack que \rbrack  nitent gratae de flore puellae. \\ Nulla tamen matri similisl’ fatusque coercens \\ Detorsit fessos artus et languida membra. \\ Nunc, o Calliope, nostro succurre labori: \\ Non possum tantam maciem describere solus, \\ Tu nisi das animos viresque in carmina fundis. \\  \lbrack At quod \rbrack  mandasti, iam possum expromere, Musa. \\ Tristis languentes pallor perfuderat artus, \\ Tempora demersis intus cecidere latebris \\ Et gracili cecidere modo per acumina nares, \\ Concava luminibus macies circumdata sedit \\ Longaque testantur ieiunia viscera fame, \\ Arida nudati distendunt brachia nervi, \\ Ordine digestae consumpto tegmine costae \\ Produnt, quidquid homo est vel quod celare sepulcbris \\ Mors secreta solet. satis id tibi, saeve Cupido? \\ Materia est iam nulla, atrox ubi flamma moretur: \\ co Denique defessos artus ac membra calore \\ Mlitur estare . . victusque virorum’ \\ Solvitur infelix per tota cubilia fusus \\ Miratusque suos artus haec verba remisit: \\ ‘Quid dicis, Paphie? retulisti nempe triumpbum: \\ Ad tantam maciem deducimur. haec tibi virtus, \\ Si dea mortalem propriis superaveris armis? \\ Cerne, precor, quid agas: flammis absumis et ossa, \\ Quae semper servata rogis. miserere rogantis, \\ 
        \pagebreak 
    \begin{center} \textbf{C. 8O8, 269290. C. 809, 1—2} \end{center} \marginpar{[296]} Alma Venus! nosti, quae sint tormenta caloris \\ Et quid possit amor: nam mater Amoris amasti. \\ 270 \\ Nunc finem, Perdica, vides: nam spes puto nulla est. \\ Quod superest, moriamur’  \lbrack ait. \rbrack  ‘letumne bibamus? \\ Cur, miserande, petis frnstra potare venena? \\ lam fauces clausere viam dirosque recusant \\ In mortem latices. ferro reseremus amorem? \\ 275 \\ O demens! gladio quibus armis quove vigore \\ Haec manus ecce valet librare in vulnera mortem? \\ Praecipitem iactare libet? fors poena placebit, \\ Sed vereor ne forte leve et sine pondere corpus \\ Vento gestatum rursum servetur Amori. \\ 2s0 \\ Stringamus laqueum? sic finis detur amanti. \\ Quid turbaris, Amor? puto, vicimus! omnia leti \\ Praedixi tormenta mei, nec te pavor ullus \\ Terruit. et laqueum metuis mihi redde tenebris! \\ Iam scio quid fugias: ne te mea vincula prodant! \\ Da laqueum collo: vel sic cum corpore nostro \\ Inclusus morieris, Amor. solatia fati \\ Haec tantum, Fortuna, mihi concede precanti, \\ Vt tumulo scriptum per saecula longa legatur: \\ IHie Perdica iacet secumque Cupido peremtus.’ \\ 290 \\ 
      \end{verse}
  
            \subsection*{809}
      \begin{verse}
      B. M . \\ mm X \\ \poemtitle{TBERIANI}B. III 264. \\ Amnis ibat inter herbas, valle fusus frigida, \\ Luce ridens calculorum, flore pictus herbido. \\ 
        \pagebreak 
    \begin{center} \textbf{C. 809, 321, 12} \end{center} \marginpar{[297]} Caerulas superne laurus et virecta myrtea \\ Leniter motabat aura blandiente sibilo. \\ Subtus autem molle gramen flore pulcro creverat; \\ Et croco solum rubebat et lucebat liliis. \\ Tum nemus fraglabat omne violarum spiritu. \\ Inter ista dona veris gemmeasque gratias \\ Omnium regina odorum vel colorun lucifer \\ o Auriflora praeminebat forma dionis, rosa. \\ Roscidum nemus rigebat inter uda ramina: \\ Fonte crebro mnrmurabant hinc et inde rivuli, \\ 1 Quae fluenta labibunda guttis ibant lucidis. \\ Antra muscus et virentes intus  \lbrack hederae \rbrack  vinxerant. \\ s Ilas per umbras omnis ales plus canora quam putes \\ Cantibus vernis strepebat et susurris dulcibus: \\ Hic loquentis murmur amnis concinebat frondibus, \\ Quis melos vocalis aurae musa zephyri moverat. \\ Sic euntem per virecta pulchra odora et musica \\ Ales amnis aura lucus flos et umbra iuverat. \\ Eiustem berii \\ Aureos subducit ignes sudus ora Lucifer \\ Pegasus hinniens transvolat aethram \\ 
        \pagebreak 
     \marginpar{[298]} \begin{center} \textbf{C. 809, 35. C. 810} \end{center}Tiberianus in Prometheo ait, deos singula sua homini \\ tribuisse \\ APlatonis hereditatemDiogenes cynicus invadens nihil ibi \\ plus aurea lingua invenit, ut Tiberianus in libro \\ de Socrate memorat. \\ Tiberianus inducit epistolam vento allatam ab anti s \\ podibus, quae habet: Superi inferis salutem. \\ 
      \end{verse}
  
            \subsection*{810}
      \begin{verse}
      B. M. \\ \poemtitle{De avicula}B. III 266. \\ Ales dum madidis gravata pennis \\ Vdos tardius explicat volatus, \\ Defecta in medio repente nisu \\ Capta est pondere deprimente plumae: \\ Cassato solito vigore pennae, \\ Quae vitam dederant, dedere letum. \\ Sic, quis ardua nunc tenebat alis, \\ Isdem protinus incidit ruinae. \\ Quid sublimia circuisse prodest? \\ Qui celsi steterant, iacent sub imis! \\ Exemplum capiant nimis tenendum, \\ Qui ventis tumidi volant secundis. \\ 
        \pagebreak 
     \marginpar{[299]} \begin{center} \textbf{C. 811 812, 19} \end{center}
      \end{verse}
  
            \subsection*{811}
      \begin{verse}
      B. M. B. \\ Hymnus et laus Bacechi \\ Salve magne pater, divum suavissime, salve, \\ Liber, et o nostris saepe colende modis. \\ Numine nostra tuo dignaris tecta subire. \\ Te colimus laeta fronte: benignus ades. \\ Qui modo sollicitis fueramus mentibus, eece \\ Adventu exhilaras nos, pater alme, tuo. \\ lactashominummentemetconvivialaetaTu \\ Efficis, ac sine te gaudia cuucta silent. \\ 
      \end{verse}
  
            \subsection*{12}
      \begin{verse}
      ergilio tribut \\ B. M. \\ Ad puerum \\ B. IV 160. \\ Parce puer, si forte tuas sonus improbus aures \\ Advenit infandum \lbrack que \rbrack  audens exposcere munus. \\ Nam meminisse potes, servata lege pudicos \\ Esse aliquos longumque diu tenuisse pudorem, \\ Sed postquam aurata delegit cuspide telum \\ Caecus amor tenuique ofendit vulnere pectus, \\ Tum pudor et sacri reverentia pectoris omnem \\ Labitur in noxam: dolet heu sic velle, stupetque \\ lammigeros motus, et tandem cogitur ipsi \\ 
        \pagebreak 
     \marginpar{[300]} \begin{center} \textbf{C. 812. 10—15. C. 81B. C. 831, 12} \end{center}Snccubuisse deo et genua inclinasse tyranno. \\ Quare age, care puer, cuius modo forma decorque \\ Ingeniumque ferax omni probitate sacroque \\ Pieridum cultu renitens et Palladis arte \\ Vexant pervigili semper mea pectora flamma: \\ Da precor anxilium atque ignem lenito furentem! \\ 
      \end{verse}
  
            \subsection*{813}
      \begin{verse}
      B. II 6. M. 757. \\ \poemtitle{VERGILIVS}B. IV 160. \\ \poemtitle{De Caesare}Iuppiter in caelis, Caesar regit omnia terris. \\ 31 s54 B. \\ B. V 396. \\ 
      \end{verse}
  
            \subsection*{831}
      \begin{verse}
      831 M. 878 \\ Quisquis ad ista moves fulgentia limina gressus, \\ Priscorum hic poteris venerandos cernere vultus, \\ 
        \pagebreak 
    \begin{center} \textbf{C. 81, 36. C. 832 833} \end{center} \marginpar{[301]} Hic pacis bellique viros, quos aurea quondam \\ Roma tulit caeloque pares dedit inclyta virtus. \\ Grandia si placeant tantorum gesta virorum, \\ Pasce tuos inspectu oculos et singula lustra. \\ 
      \end{verse}
  
            \subsection*{832}
      \begin{verse}
      Romulus \\ B. M. 711. \\ Hic nova qui celsae fundavit moenia Troiae, \\ Vrbem lomanam proprio de nomine dixit. \\ Infantem gelidi proiectum ad Thybridis undas \\ Vberibus fecunda piis Larentia pavit. \\ Ausus finitimas praedari fraude Sabinas, \\ Fortem fortis humi postravit Acrona duello. \\ 
      \end{verse}
  
            \subsection*{833}
      \begin{verse}
      B. M. 717. \\ Cui dedit hirsutus nomen venerabile cirrus, \\ Quintius hic ille est, rigidis animosus in armis. \\ Is quoque dum curvo sudans penderet aratro, \\ Ante boves meritum meruit dictator honorem. \\ Consulis obsessi partes defendit inertis, \\ Inde triumphalem conscendit agricola currum. \\ 
        \pagebreak 
    \begin{center} \textbf{C. 834 836} \end{center} \marginpar{[302]} 
      \end{verse}
  
            \subsection*{834}
      \begin{verse}
      M. urius Camillus \\ M. 713. \\ Qui fuit en patriae quondam spes ampla ruentis, \\ Hic Senonum propria domuit virtute furores; \\ Vicit et opposito quos clausit Marte aliscos, \\ Brachia fallaci religato in terga magistro. \\ Quicquid ubique truces bello valuere decenni, \\ Inclita, Veientes, accessit pompa triumpho. \\ 
      \end{verse}
  
            \subsection*{835}
      \begin{verse}
      T. Manlius Torquatus \\ B. . 714 \\ Inclita Torquatae dedit hic cognomina genti. \\ Vir ferus ante acies prostrati guttura Galli \\ Perfodiens gladio poscentis voce duellum \\ Abstulit aurati pretiosa monilia torquis. \\ Consulis et Decii hello collega Latino \\ Victoris nati maculavit caede secures. \\ 
      \end{verse}
  
            \subsection*{836}
      \begin{verse}
      . Decius \\ M. 715 \\ Hic est qui vitam patriae devovit amatae. \\ Dum furor oppositos agitaret ad arma Latinos \\ Saevaque crudelem cecinissent classica pugnam, \\ Inter tela aciesque virum cuneosque pedestres \\ Candida sacrata religatus tempora vitta \\ Ante aciem moriens hostilibus occidit armis. \\ 
        \pagebreak 
     \marginpar{[303]} \begin{center} \textbf{C. 837 839, 1—4} \end{center}
      \end{verse}
  
            \subsection*{837}
      \begin{verse}
      B. M. 716. \\ M. Curius Denatus \\ Quid iuvat imperio populos rexisse potenti \\ Fulvaque Mygdoniis ornasse palatia gemmis? \\ Quamquam civis inops, toto notissimus orbe \\ Hic fuit, egregio domuit qui Marte Sabinos. \\ Fregerit ipse licet fulgentis robora Pyrrhi, \\ Pauperiem lato Samnitum praetulit auro. \\ 
      \end{verse}
  
            \subsection*{838}
      \begin{verse}
      B. M. 718. \\ C. abricius \\ Contentus modico tectique habitator egeni \\ Ilic erat et renuit devicti munera Pyrrbi, \\ Sprevit et oblatos, Samnitum munera, servos, \\ Respuit immensi locupletia ponderis aera, \\ lorruit infamem scelerata fraude magistrum \\ Pocula pollicitum regi miscere veneno. \\ 
      \end{verse}
  
            \subsection*{839}
      \begin{verse}
      B. M. 719. \\ . Pabius Maximus. \\ Vir fuit iste ferox, qui torvus fronte verenda \\ Vir fuit egregius et bello clarus et armis. \\ Captivos modici quamquam pauperrimus agri \\ Exemit pretio Poenorum in vincula missos. \\ 
        \pagebreak 
     \marginpar{[304]} \begin{center} \textbf{C. 839, 5 . C. 840 842,. 1—2} \end{center}Is qui cunctando nisi Punica fregerit arma, \\ Nulla foret Latiis lomana potentia terris. \\ 
      \end{verse}
  
            \subsection*{840}
      \begin{verse}
      B. M. 720. \\ Claudius Nero \\ Armorum virtute potens Nero Claudius hic est. \\ Coniunctus Livio Picentis ad arva Metauri \\ Prostravit Libycas memorando Marte cohortes. \\ Fortnate tui, iuvenis metuende, furoris! \\ Ausus es ignari iacere ad tentoria fratris \\ Cervicem Libyci media inter tela tyranni. \\ 
      \end{verse}
  
            \subsection*{841}
      \begin{verse}
      B. M. 721. \\ . Marcellus \\ Tu primus Libycum Nolae sub moenibus hostem \\ Insidiis periture tuis, Marcelle, fugasti. \\ Cumque Syracosii quondam †negaretur hbonoris \\ Pompa tibi, Albano gessisti in monte triumphbum. \\ Praedonum deprensa manu venerandaque multis \\ Luctibus heu patrio caruerunt ossa sepulcro. \\ 
      \end{verse}
  
            \subsection*{842}
      \begin{verse}
      B. II 4. M. 722. \\ Scipio \\ Ille ego sum, patriam Poeno qui Marte cadentem \\ Sustinui rapuique feris ex hostibus urbes \\ 
        \pagebreak 
    \begin{center} \textbf{C. 84, 38. C. 843 844} \end{center} \marginpar{[305]} Hispanas, †lannonisque acies magnumque Syphacem \\ Perdomui et fractum totiens armisque repulsum \\ s Hannibalem, victorque ferox mihi regna subegi \\ Punica et excelsas altae Carthaginis arces. \\ 
      \end{verse}
  
            \subsection*{843}
      \begin{verse}
      B. M. 727. \\ C. Marius \\ Et genus et nomen merui virtute feroci \\ Rusticus Arpinas, bellorum maximus auctor. \\ Effera post Numidae quam fregimus arma Iugurthae, \\ Cimbrica praeclaros geminavit pompa triumphos. \\ Exegi civile nefas servilibus armis, \\ Et mea Sullanos fregerunt arma furores. \\ 
      \end{verse}
  
            \subsection*{844}
      \begin{verse}
      B. M. 729. \\ M. Caesius Seaena \\ Igne calens belli mediaque in caede cruentus, \\ Pompeiana phalanx patulis exire ruinis \\ Dum furit et properat claustrorum frangere turres, \\ Scaeva ego Caesarei defendi culmina valli. \\ Dum timet Oceanus praeclari Caesaris arma, \\ Textum pampineae gessi sublime coronae. \\ 
        \pagebreak 
    \begin{center} \textbf{C. 845 847} \end{center} \marginpar{[306]} 
      \end{verse}
  
            \subsection*{845}
      \begin{verse}
      B. M. 730. \\ n. Pompeius \\ Arma tuli quondam toto victricia mundo, \\ Qui pelago Cilicas et Pontica regna subegi. \\ Vis mea, quos profugus commoverat exul ad arma, \\ Bellorum virtute truces prostravit Hiberos. \\ At me post soceri civilia bella cruenti \\ Dextera Septimii Phariis laceravit in undis. \\ 
      \end{verse}
  
            \subsection*{846}
      \begin{verse}
      B. II 52. M. 747. \\ . Porcius Cato \\ Cerne hic ora sacri semper veneranda Catonis! \\ ibertate potens animoque invictus et armis \\ Avius incerto peragravit tramite Syrtes. \\ Libertatis enim dulcedine captus amatae, \\ Ne sua servitio premerentur colla tyranni, \\ Fortia crudeli penetravit pectora ferro. \\ 
      \end{verse}
  
            \subsection*{847}
      \begin{verse}
      B. II 6B. . 750. \\ Iulius Caesar \\ Ille hic magnanimus, qui claris arduus actis \\ Non timuit generum nec inertia signa senatus, \\ Ne sibi Gallorum raperetur pompa triumphi, \\ Intnlit invitus per civica viscera ferrum. \\ Vis invicta viri reparata classe Britannos \\ Vicit et hostiles Rheni conpescuit undas. \\ 
        \pagebreak 
    \begin{center} \textbf{C. 848 850, 1—7} \end{center} \marginpar{[307]} 
      \end{verse}
  
            \subsection*{848}
      \begin{verse}
      B. M. 751. \\  \lbrack Iulius Caesar \\ Caesar, ad imperium civili sanguine partum \\ Venisti, famulasque manus Fortuna tetendit. \\ Ipsa tibi cessit tellus et iussibus aequor \\ Paruit omne tuis. verum invidiosa potestas \\ Et suspectus honos: adimunt civilibus armis \\ Imperium vitamque tibi formidine regni. \\ 
      \end{verse}
  
            \subsection*{849}
      \begin{verse}
      B. M. 752. \\ Iulius Caesar \\ Germanos domui Caesar Gallosque potentes, \\ Cesserunt signis omnia regna meis. \\ Et generum vici natosque ferumque Catonem \\ Imperioque fuit subdita lRoma meo. \\ ln tantis opibus me fors inimica peremit, \\ Omnis et immiti gloria sorte ruit. \\ 
      \end{verse}
  
            \subsection*{850}
      \begin{verse}
      B. M. 753. \\ Epitephium Iulii Caesaris \\ En adsum Caesar. quis me praestantior armis? \\ Quis prior eloquio? quis me clementior alter \\ Egregia pietate fuit? quoscumque subegi, \\ Hos vici pietate magis quam fortibus armis. \\ Plus mecum Fortuna fuit, plus obtulit uni \\ Quam sibi di dederunt. gestis belloque refulsi, \\ Germanos domui populos Gallosque potentes \\ 
        \pagebreak 
    \begin{center} \textbf{C. 850, 3—12. C. 851 853} \end{center} \marginpar{[308]} Atque tripartitum tenui sub fascibus orbem. \\ Deleri haud potuit virtus, non inclyta fama; \\ Haud patricida meum potuit restringere nomen. \\ Qua rapitur Titan curru, qua vergitur axis, \\ Hac Eephyrus nostros referet per saecla triumphos. \\ 
      \end{verse}
  
            \subsection*{851}
      \begin{verse}
      B. M. 755. \\ Caesar Augustus \\ Quae mihi, sancte, dabit grandes expromere laudes \\ Musa tuas? iam pauca canam. tu Caesaris alti \\ Vltus es indignam memorando nomine mortem, \\ Actiaco et Pharias superasti in gurgite classes, \\ Tranquillumque tuis faciens virtutibus orbem \\ Clausisti reserata diu sua limina Iano. \\ 
      \end{verse}
  
            \subsection*{852}
      \begin{verse}
      B. M 756. \\ Angustus \\ Solus habes aquila corvos laniante triumvir \\ Imperium, victusque tuis Antonius armis \\ Leucadio et regina freto; lanumque bifrontem \\ Clausisti. at toto cum feceris otia mundo \\ Teque audes conferre deo, te Livia sortis \\ Dicitur humanae  \lbrack tandem \rbrack  monuisse veneno. \\ 
      \end{verse}
  
            \subsection*{853}
      \begin{verse}
      B. M. 759. \\ erius \\ Claudius egregie vixit privatus et insons \\ Imperiis, Auguste, tuis; simulataque virtus \\ 
        \pagebreak 
     \marginpar{[309]} \begin{center} \textbf{C. 854 855} \end{center}
      \end{verse}
  
            \subsection*{854}
      \begin{verse}
      B. M. 771. \\ Traianus \\ Caesareos toto referens hic orbe triumphos \\ Notus erat mundo quondam pietate gementi. \\ Inclytus extremos penetravit victor ad Indos, \\ Belligerosque Arabas et Colchos sub iuga misit, \\ Armenia Parthos pepulit Babylone subacta, \\ Et dedit Albanis regem, quos vicerat armis. \\ B. II 64. M. 754. \\ 
      \end{verse}
  
            \subsection*{855}
      \begin{verse}
      B. V 403. \\ Octavianu Augutu \\ In Macetum campis ultus iam Caesaris umbras \\ Sum patris, Augustus belloque armisque superbus, \\ Atque meos sensit fugiens Antonius enses. \\ Quantum ingens mundus, quantum Iovis alta potestas, \\ Tantus in orbe fui. terras pontumque subegi: \\ Vix caelum superis et sidera summa reliqui. \\ 
        \pagebreak 
     \marginpar{[310]} \begin{center} \textbf{C. 855 855, 1 3} \end{center}B. M. \\ 
      \end{verse}
  
            \subsection*{855}
      \begin{verse}
      B. V 404. \\ Epiramma yrrhi \\ Demetrium Antigoni, Poenos, elapsus Epiro \\ Concussi bellis, Siculos, Lacedaemona Pyrrhus. \\ Romanos studui Tarento avertere et ardor \\ Fervidus assidui belli mihi semper inhaesit. \\ 
      \end{verse}
  
            \subsection*{8552}
      \begin{verse}
      B. M. \\ B. V 404. \\ Epitaphinm Alexandri \\ Magnus Alexander sum, qui victricibus omnem \\ Munificus bellis incussi et stragibus orbem. \\ Ante meos nullus potuit tam fortiter annos \\ Tam late imperium et regnorum extendere tines. \\ 
      \end{verse}
  
            \subsection*{855a}
      \begin{verse}
      B. M. \\ B. V 404. \\ Spigramma Iulii Caesaris \\ Magni animi Caesar populos virtute subegi \\ Iulius Arctoos, potui qui solus bonores \\ Flectere Pompeii et solus Romana tenere \\ Imperia atque unum mundo praeponere regem. \\ 
      \end{verse}
  
            \subsection*{8559}
      \begin{verse}
      B. M. B. \\ Epitphium Caesaris \\ Volve tuos oculos: metuendum hunc aspice, lector, \\ Armorum bellique ducem bellatrix in omnes \\ Terrarum pelagique plagas victricia quondam \\ 
        \pagebreak 
    \begin{center} \textbf{C. 855d, 4 8. C. 856 8, 12} \end{center} \marginpar{[311]} Victor sina tulit, nullo superatus ab hoste. \\ Gallia bellatrix, opibus superba virisque \\ Passa sub hoc dudum Romanas principe leges, \\ Pontica quo cecidit victore potentia; quartum \\ Caesareis titulis dedit Africa terra triumphum. \\ B. II B. M. 696. \\ 
      \end{verse}
  
            \subsection*{856}
      \begin{verse}
      B. V 402. \\ \poemtitle{De Nino}Ninus ab Assyriis sum primus regibus ausus \\ Non modo finitimos populos regesque potentes \\ Imperio pressisse meo, sed viribus omnem \\ Paene Asiam domui terraeque marique timendus. \\ B. II 4. . 700. \\ 
      \end{verse}
  
            \subsection*{857}
      \begin{verse}
      B. V 02. \\ \poemtitle{De Semiramide}Te stola ne dubium teneat cultusque venusti \\ Corporis ambiguus: sum clara Semiramis, alto \\ Non minor ipsa viro, belli pacisque probatis \\ Artibus insignis, nato miserabilis uno. \\ B. II 5. M. 698. \\ 
      \end{verse}
  
            \subsection*{858}
      \begin{verse}
      B. V 402. \\ \poemtitle{De Cyvro}Quantum fata valent atque inmutabilis ordo \\ Astrorum, docui Cyrus: quem nulla domare \\ 
        \pagebreak 
     \marginpar{[312]} \begin{center} \textbf{C. 858, 34. C. 859 860} \end{center}Vis potuit, non Astyages Babylonve superba, \\ Massagetis eadem me fata dedere tremendis. \\ B. II . M. 699. \\ 
      \end{verse}
  
            \subsection*{859}
      \begin{verse}
      B. V 403. \\ \poemtitle{De Tamyri}Quantum ego claruerim Mavortis in agmine, Cyrus \\ Ipse ferus novit, qui dum mea regna profanus \\ Spargapise insidiis rapto popularier audet, \\ Quam Tamyris turbata valet, cognovit in utre. \\ B. II 8. M. 704. \\ 
      \end{verse}
  
            \subsection*{860}
      \begin{verse}
      B. V 403. \\ \poemtitle{De Myrina \rbrack }Inter Amaonidas, quas insula celsa Tritonis \\ Hespera progenuit, qui me nescire Myrinam \\ Dixerit, ignarum sese fateatur oportet \\ Eximiae laudis: Libyamque Asiamque subegi. \\ 
        \pagebreak 
    \begin{center} \textbf{C. 861 863, 1—3} \end{center} \marginpar{[313]} 
      \end{verse}
  
            \subsection*{861}
      \begin{verse}
      B. II I0. M. 705. \\ B. V 403. \\ \poemtitle{De Penthesilea}Graiugenas acies superans passimque fugatas \\ Penthesilea premens ulcisci nobilis umbram \\ Hectoris ipsa mei potui, ni dirus Achillis \\ Ille Neoptolemus me clam misisset ad Orcum. \\ 
      \end{verse}
  
            \subsection*{862}
      \begin{verse}
      B. II 7. M. 701. \\ B. V 03. \\ \poemtitle{De Aexandro Magno}Magnus Alexander bellisque horrendus et armis, \\ Qui terrore mei trepidantem nominis orbem \\ Vsque sub Eoum parvo cum milite Gange \\ Vndique concussi, nulli sum laude secundus. \\ 
      \end{verse}
  
            \subsection*{863}
      \begin{verse}
      B. II 9. M. 749. \\ B. V 403. \\ \poemtitle{De Iulio Caesare}Sum genus Aeneadum Gaius cognomine Caesar \\ Iulius, Assyrios, Cyrum, Macetumque ferocem \\ Qui iuvenem probitate animi, qui corporis acer \\ Viribus exsuperans fueram pater urbis et orbis. \\ M. . \\ 
      \end{verse}
  
            \subsection*{863a}
      \begin{verse}
      B. V 394. \\  \lbrack Iudicium Paridis \rbrack  \\ ‘res deae ad Paridem \\ Tres sumus  \lbrack ecce \rbrack  deae: forma se quaelibet effert. \\ Hoc in discidio volumus, Paris, arbiter esto: \\ Cui pomum dederis, titulum simul ipse referto. \\ 
        \pagebreak 
    \begin{center} \textbf{C. 863a, 4 — 19.} \end{center} \marginpar{[314]} Venus ad Paridem \\ Plectra sonora, ioci, lusus, lasciva voluptas, \\ Haec mea. si reliquis me praefers, ipsa puellam \\ Pro mercede dabo, qua non formosior ulla. \\ Iuno eidem \\ Sceptrorum sublimis honor fascesque tremendi \\ Divitiaeque mei imris. te iudice palmam \\ Si tulerim, regno per me donabere summo. \\ Pallas eidem \\ Quae caelum, quae terra tenet, quae pontus et orcus, o \\ Legibus astringo certs: nil me sine rectum; \\ Et, si me sequeris, non abstrahet invius error. \\ Iudicium Paridis \\ Grata mihi tua forma, Venus, tua munera grata. \\ Plus aliis mihi mente sedes: certaminis ecce \\ Pignus habe victrix auri spectabile malum. \\ Poeta \\ Hac in lite triplex hominum mellita poesis \\ Depinxit studium, quorum datur optio cunctis. \\ Falluntur tamen optando plerique: sequaces \\ Luxus habet multos, honor et sapientia paucos. \\ 
        \pagebreak 
    \begin{center} \textbf{C. 864 865} \end{center} \marginpar{[315]} B. V 67. .1036. \\ 
      \end{verse}
  
            \subsection*{864}
      \begin{verse}
      B. V 379. \\ \poemtitle{ \lbrack De quatuor anni tempestatibus \rbrack }Aestatis Mlaius Tauro primordia prodit. \\ Iunius aestivo Geminorum cardine surgit. \\ Iulius aestivas Cancro secat alter aristas. \\ Autumni caputAugustus parat ore Leonis. \\ Autumnas uvas September Virgine curat. \\ Libra sub autumno Octobri dat semina sulco. \\ Scorpius innectit tempus brumale Novembris. \\ Arcitenens hiemis . \\ . Capricorni sidere frigens. \\ Inducit Februo ver udus aquarius arvo. \\ Mars flores vernis nemori sub Piscibus edit. \\ Aprili vernante novans Aries micat annum. \\ B. V 194. M. I123. \\ 
      \end{verse}
  
            \subsection*{865}
      \begin{verse}
      B. V 405. \\ Porcius octophoro fertur resnupinus ad aedem \\ Iunonis Triviae priscaque templa deae \\ Et circum Phoebi. merito stupet inscia turba \\ Atque ait: an horum victima porcus erit? \\ 
        \pagebreak 
    \begin{center} \textbf{C. 866 867, 15} \end{center} \marginpar{[316]} B. V 212. M.1141. \\ 
      \end{verse}
  
            \subsection*{866}
      \begin{verse}
      B. V 216. \\ Quis deus hoc medium flammavit crinibus aurum, \\ Iussit et in dumis sentibus esse rosam? \\ Aspicis ut magni coeant in foedus amantes: \\ Martem spina refert, flos Veneris pretium est. \\ est,lascve,sagittis?Quidtibicummagnis,puer, \\ Hoc melius telo pungere corda potes! \\ Nec flammas quaeras neque †alti pectoris ignis: \\ Si tibi ver tantum praebeat ista, sat est. \\ Pllens herba viret: color est hic semper amantum. \\ Tam fugitiva rosa est quam fugitivus amor. \\ Nam quod floricomis gaudet lasciva metallis, \\ Aurum significat vilius esse rosa. \\ B. M. \\ 
      \end{verse}
  
            \subsection*{867}
      \begin{verse}
      B. V 217. \\ Fabula constituit toto notissima mundo \\ Gorgoneos vultus saxificumque nefas. \\ Hoc monstrum natura potens novitate veneni \\ Ex oculis nostris iusserat esse malum. \\ Ianc auro genitus lovis ales praesule diva \\ 
        \pagebreak 
    \begin{center} \textbf{C. 867, 6 10. C. 868 870} \end{center} \marginpar{[317]} Mactans aerato conspicit ingenio. \\ Deriguit mirata necem fatumque veneni \\ Vertit et in mortem decidit ipsa lapis. \\ Sic praesens absensque simul caecumque videndo \\ Ludit et ignaro raptor ab hoste redit. \\ 
      \end{verse}
  
            \subsection*{868}
      \begin{verse}
      de nunc c. \\ 
      \end{verse}
  
            \subsection*{869}
      \begin{verse}
      B. M. B. \\ Pulcinus milvo dum portaretur hoc’ inquit \\ \poemtitle{. 2}‘Iam cado, ne timeas; non tenet . \\ 
      \end{verse}
  
            \subsection*{870}
      \begin{verse}
      B. M. 276. \\ B . \\ Augustinus \\ Roma labore vigil fregit Carthaginis arces: \\ Desidia interiit Roma subinde cito. \\ 
        \pagebreak 
    \begin{center} \textbf{C. 871 873} \end{center} \marginpar{[318]} 
      \end{verse}
  
            \subsection*{871}
      \begin{verse}
      M. 880. \\ Appianus \\ Postquam militia et belli sudore vacabant \\ Romani et nusquam bella vel hostis erat, \\ Desidia et luxu robur Romana iuventus \\ Perdidit. hoc cecidit inclita Roma modo. \\ 
      \end{verse}
  
            \subsection*{872}
      \begin{verse}
      le nunc c. \\ B. V 174. M. 1104. \\ 
      \end{verse}
  
            \subsection*{873}
      \begin{verse}
      B. V 405. \\ In equum mirae pernicittis \\ Te cuperet Phoebus roseis aptare quadrigis, \\ Sed fieret brevior te properante dies. \\ 
      \end{verse}
  
            \subsection*{873a}
      \begin{verse}
      E. M. B. V 405. \\ \poemtitle{rVLLII}de civitate TParenti \\ Ambitur gemini sinuosa fauce profundi \\ Trbs, quae de parvo flumine nomen habet; \\ Quam mare, quam tellus ditant, set dispare fato: \\ Pisce fretum, terra germine grata placent. \\ Vitis oliva seges surgunt tellure feraci \\ Et mare purpureo murice dite rubet. \\ 
        \pagebreak 
    \begin{center} \textbf{C. 8732 873} \end{center} \marginpar{[319]} B. M. \\ 
      \end{verse}
  
            \subsection*{8732}
      \begin{verse}
      B. IV 438. \\ Baiarum dum forte capit sub mollibus umbris \\ Fessus Amor somnum murmure captus aquae, \\ lpsa facem accurrens gelida celavit in unda, \\ Vt veteres flammas vindicet, alma Venus. \\ Quum primum liquor ille aeternos concipit ignes; \\ lgne novo (quisnam crederet?) arsit aqua. \\ Flammivomis igitur fumant haec balnea lymphis, \\ Quodfacnula una omnes vincit Amoris aquas. \\ 
      \end{verse}
  
            \subsection*{873}
      \begin{verse}
      B. M. \\ B. IV 40. \\ Dicite, cum melius cadere ante Lucretia posset, \\ Cnur potius post vim maluit illa mori \\ Crimine se absolvit manus illa habitura coactae \\ Vltorem et patriae depositura iugum. \\ Quam bene contempto lacerat sua pectora ferro, \\ Dum pariter famae consulit et patriae! \\ 
      \end{verse}
  
            \subsection*{8739}
      \begin{verse}
      B. M. B. \\ Dum non laeta fuit, defensa est Hlios armis: \\ Militibus gravidum laeta recepit equum. \\ 
        \pagebreak 
    \begin{center} \textbf{C. 879} \end{center} \marginpar{[320]} 
      \end{verse}
  
            \subsection*{8732}
      \begin{verse}
      B.III 146.M. 1564. \\ B. V 408. \\ nOris,praesentibusexpleCumtemortalem \\ Deliciis animum. Post mortem nulla voluptas. \\ Namque ego sum pulvis, qui nuper tanta tenebam. \\ ‘Iaec habeo quae edi, quaeque exsaturata libido \\ Hausit; at illa iacent multa et praeclara relicta.’ \\ Iloc sapiens vitae mortalibus est documentum. \\ 
        \pagebreak 
    \poemtitle{CARMINA}
      \end{verse}
  
            \subsection*{}
      \begin{verse}
      \poemtitle{OAE LBRI EATVM TVPIS}\poemtitle{DESCRITI EXIIBE7}
        \pagebreak 
     \marginpar{[874]} ilenunc.c.20e \\ 
      \end{verse}
  
            \subsection*{871a}
      \begin{verse}
      B. M B. V 214. \\ Auct. ntlquls. \\ \poemtitle{DRACONTII}XIV 227. \\ Ad Trasimundum eomitem CGapuae \\ \poemtitle{De mensibus}Ianuarius \\ Purpura iuridicis sacros largitur honores \\ Et nova fastorum permutat nomina libris. \\ Februaurius \\ Sol hiemis glacies solvit iam vere nivesque; \\ Cortice turgidulo rumpunt in palmite gemmae. \\ Matius \\ Martia iura movet, signis fera bella minatur, \\ Excitet ut turmas, et truncat falce novellas. \\ prilis \\ Post Chaos expulsum rident primordia mundi. \\ Tempora pensantur noctis cum luce diei. \\ 
        \pagebreak 
     \marginpar{[874]} tenunc.ec.70e \\ 
      \end{verse}
  
            \subsection*{871a}
      \begin{verse}
      B. M B.V 214. \\ Auct. nntlqui. \\ \poemtitle{DRACONTII}XIV 227. \\ Ad Trasimundum comitem CGapuae \\ \poemtitle{De mensibus}Ianuarius \\ Purpura iuridicis sacros largitur honores \\ Et nova fastorum permutat nomina libris. \\ Februarius \\ Sol hiemis glacies solvit iam vere nivesque; \\ Cortice turgidulo rumpunt in palmite gemmae. \\ Martius \\ Martia iura movet, signis fera bella minatur, \\ Excitet ut turmas, et truncat falce novellas. \\ Aprilis \\ Post Chaos expulsum rident primordia mundi. \\ Tempora pensantur noctis cum luce diei. \\ 
        \pagebreak 
     \marginpar{[874]} TVle nunc. c. 72oe \\ 
      \end{verse}
  
            \subsection*{871a}
      \begin{verse}
      B. M B. V 214. \\ Auct. ntlquls. \\ \poemtitle{DRAOTII}XIV 227. \\ Prasimundum eomitem CGapuae \\ \poemtitle{A0}\poemtitle{De mensibus}Ianuarius \\ Purpura iuridicis sacros largitur honores \\ Et nova fastorum permutat nomina libris. \\ Februarius \\ Sol hiemis glacies solvit iam vere nivesque; \\ Cortice turgidulo rumpunt in palmite gemmae. \\ Martius \\ Martia iura movet, signis fera bella minatur, \\ Excitet ut turmas, et truncat falce novellas. \\ Aprilis \\ Post Chaos expulsum rident primordia mundi. \\ Tempora pensantur noctis cum luce diei. \\ 
        \pagebreak 
    \begin{center} \textbf{C. 8742 2} \end{center} \marginpar{[324]} Maius \\ Prata per innumeros vernant gemmata colores; \\ loribus ambrosiis caespes stellatur odorus. \\ Iunius \\ Messibus armatis crispae flavantur aristae: \\ Rusticus expensas et fluctus nauta reposcit. \\ Iulius \\ Humida dant siccas messes domicilia Lunae: \\ Fontanas exhaurit aquas, ut Nilus inundet. \\ Augustus \\ Atria solis habet, sed nomen Caesaris adfert. \\ Mitia poma dabit, siccas terit area fruges. \\ September \\ Aestuat autumnus partim variantibus uvis, \\ Agricolis spondens mercedem vina laborum. \\ Oetober \\ Promitur agricolis saltantibus ebrius imber; \\ Rusticitasque †decet gaudens plus sordida musto. 2 \\ N ovember \\ Pigra redux torpescit hiems; mitescit oliva, \\ Et frumenta capit quae fenore terra refundat. \\ December \\ Algida bruma nivans onerat iuga celsa pruinis \\ Et glaciale gelu nutrit sub matribus agnos. \\ 
        \pagebreak 
    \begin{center} \textbf{C. 874b. 875} \end{center} \marginpar{[325]} 
      \end{verse}
  
            \subsection*{87}
      \begin{verse}
      B. M. B.V. 215. \\ Auct. nntiqui. \\ \poemtitle{De origine rosarum dieitur:}xIV 228. \\ Alma Venus  \lbrack quondam \rbrack  dum Martis vitat amores \\ Et pedibus nudis florea prata premit, \\ Sacrilega placidas irrepsit spina per herbas \\ Et tenero plantas vulnere mox lacerat. \\ Funditur inde cruor; vestitur spina rubore; \\ Quae scelus admisit, munus odoris habet. \\ Sanguine cuncta rubent croceos dumeta per agros \\ Et sancit vepres astra imitata rosa. \\ Quid prodest, Cypris, Martem fuisse cruentum, \\ Cum tibi puniceo sanguine planta madet? \\ Sanguineis Cytherea genis, sic crimina punis, \\ Mordacem ut spinam flammea gemma tegat? \\ Sic decuit doluisse deam, sic numen amorum, \\ Vindicet ut blandis vulnera muneribus. \\ 
      \end{verse}
  
            \subsection*{875}
      \begin{verse}
      B. II 262. M. 1006. \\ B. \\ Coitat Vrsidius sibi dote iugare puellam; \\ Vt placeat domino, cogitat Vrsidius. \\ Cogitat Vrsidius heredem tollere parvum; \\ Vt placeat domino, cogitat Vrsidius. \\ Cogitat Vrsidius domino quacunque placere \\ Virgine vel puero. quam sapit Vrsidius! \\ 
        \pagebreak 
    \begin{center} \textbf{C. 876, 125} \end{center} \marginpar{[326]}  \marginpar{[40]} 
      \end{verse}
  
            \subsection*{876}
      \begin{verse}
      B. III 58. M. 279. \\ Avienu ed. \\ \poemtitle{. RVTI TETI AVIENI}Qua venit Ausonias austro duce Poenus ad oras, \\ Si iam forte tuus Libyca rate misit agellus \\ Punica mala tibi Tyrrhenum vecta per aequor, \\ Quaeso aliquid nostris gustatibus inde relaxes. \\ Sic tua cuncta ratis plenis secet aequora velis, \\ Spumanti cum longa trahit vestigia sulco, \\ Romuleique Phari fauces illaesa relinquat: \\ Sit licet illa ratis, quam miserit alta Corinthos, \\ Adriacos surgente noto qua prospicit aestus,. \\ Quamve suis opibus cumularit Hiberia dives, \\ Solverit aut Libyco quam laetus navita portu. \\ Sed forsan, quae sint, quae poscam, mala, requiras. \\ Illa precor mittas, spisso quibus arta cohaeret \\ Granorum fetura situ, castrisque sedentes \\ Vt quaedam turmae socio latus agmine quadrant \\ Multiplicemque trahunt per mutua vellera pallam, \\ Vnde ligant teneros examina flammea casses. \\ Tunc ne pressa gravi sub pondere grana liquescant, \\ Divisere domos et pondera partibus aequant. \\ laec ut, amice, petam, cogunt fastidia longis \\ Nata malis et quod penitus fellitus, amarans \\ Ora, sapor nil dulce meo sinit esse palato. \\ Horum igitur suco forsan fastidia solvens \\ Ad solitas revocer mensis redeuntibus escas. \\ Nec tantum miseri videar possessor agelli, \\ 
        \pagebreak 
    \begin{center} \textbf{C. 786, 26 31. C. 877 87 , 43} \end{center} \marginpar{[327]} Vt genus hoc arbos nullo mihi floreat horto: \\ Nascitur ex multis onerans sua brachia pomis, \\ Sed gravis austerum fert sucus ad ora saporem. \\ lla autem, Libycas quae se sustollit ad oras, \\ o Mitescit meliore solo caelique tepentis \\ Nutrimenta trahens suco se nectaris implet. \\ B. V 134. M. 1072. \\ 
      \end{verse}
  
            \subsection*{877}
      \begin{verse}
      B. \\ Caesaris ad valvas sedeo, sto nocte dieque, \\ Nec datur ingressus, quo mea fata loquar. \\ Ite, deae faciles, et nostro nomine saltem \\ Dicite divini praesidis ante pedes: \\ Si nequeo placidas affari Caesaris auris, \\ Saltem aliquis veniat, qui mihi dicat ‘abi’. \\ B. M. B. \\ 
      \end{verse}
  
            \subsection*{878}
      \begin{verse}
      Claudianus ed. \\ \poemtitle{MEROBAVDIS v CLAVDIANI E}Ibi. XIV p.19. \\ Eaus Christi \\ Proles vera Dei cunctisque antiquior amnis, \\ Nunc genitus, qui semper eras, lucisque repertor \\ Ante tuae matrisque parens, quem misit ab astris \\ 
        \pagebreak 
     \marginpar{[328]} \begin{center} \textbf{C. 878, 4—30} \end{center}Aequaevus genitor verhumque in semina fusum \\ Virgineos habitare sinus et corporis arti \\ Iussit inire vias parvaque in sede morari, \\ Quem sedes non ulla capit, qui lumine primo \\ Vidisti quicquid mundo nascente crearas, \\ Ipse opifex, opus ipse tui, dignatus iniquas \\ Aetatis sentire vices et corporis huius \\ Dissimiles perferre modos hominemque subire, \\ Vt possis monstrare deum, ne lubricus error \\ Et decepta diu vani sollertia mundi \\ Pectora tam multis sineret mortalia saeclis \\ Auctorem nescire suum: te conscia partus \\ Mater et attoniti pecudum sensere timores. \\ Te nova sollicito lustrantes sidera visu \\ In caelo videre prius, lumenque secuti \\ Invenere magi. tu noxia pectora solvis \\ Elapsasque animas in corpora functa reducis \\ Et vitam remeare iubes. tu lege recepti \\ Muneris ad Manes penetras mortisque latebras \\ Immortalis adis. nasci tibi non fuit uni \\ Principium finisque mori; sed nocte refusa \\ In caelum patremque redis rursusque perenni \\ Ordine purgatis adimis contagia terris. \\ Tu solus patrisque comes, tu spiritus insons \\ Et toties unus triplicique in nomine simplex. \\ Quid, nisi pro cunctis, aliud quis credere possit \\ Te potuisse mori, poteras qui reddere vitam? \\ 
        \pagebreak 
     \marginpar{[329]} \begin{center} \textbf{C. 879} \end{center}
      \end{verse}
  
            \subsection*{879}
      \begin{verse}
      B. M. B. \\ Claudin od. \\ Miraeula Christi \\ Birt . 412. \\ Angelus alloquitur Mariam, quo praescia verbo \\ Concipiat salva virginitate deum. \\ Dant tibi Chaldaei praenuntia munera reges: \\ Myrrham homo, rex aurum, suscipe tura deus. \\ Permutat lymphas in vina liquentia Christus, \\ . Quo primum facto se probat esse deum. \\ Quinque explent panes, pisces duo milia quinque, \\ Et deus ex parvo plus superesse iubet. \\ Editus ex utero caecus nova lumina sentit \\ Et stupet ignotum se meruisse diem. \\ Laarus e tumulo Christo inclamante resurgit \\ Et durae mortis lex resoluta perit. \\ Nutantem quatit unda Petrum, cui Christus in alto \\ Et dextra gressus firmat et ore tidem. \\ Exsanguis Christi contingit femina vestem. \\ Stat cruor in venis. fit medicina fides. \\ Iussus post mnltos graditur paralyticus annos, \\ (Mirandum) lecti portitor ipse sui. \\ 
        \pagebreak 
    \begin{center} \textbf{C. 880 88, 15} \end{center} \marginpar{[330]} 880. 881 \\ Tide nuc c. 49te \\ B. III 109. \\ 
      \end{verse}
  
            \subsection*{882}
      \begin{verse}
      97. B \\ ‘Optimus est’ Cleobulus ait ‘modus’ incola Lindi. \\ Ex Ephyra Periandre, doces cuncta emeditanda. \\ Tempus uosce’ inquit Miylenis Pittacus ortus. \\ Plures esse malos Bias autumat ille Prieneus, \\ Milesiusque Thales sponsori damna minatur. \\ ‘Nosce’ inquit ‘tete’ Chilon lLacedaemone cretus, ; \\ Cecropiusque Solon ‘ne quid nimis’ induperavit. \\ B. M. \\ 
      \end{verse}
  
            \subsection*{883}
      \begin{verse}
      B. III 203. \\ Et tetracem, lomae quem nunc vocitare taracem \\ Coeperunt. avium est multo stultissima. namque \\ Cum pedicas necti sibi contemplaverit adstans, \\ Immemor ipse sui tamen in dispendia currit. \\ Tu vero adductos laquei cum senseris orbes, \\ 
        \pagebreak 
    \begin{center} \textbf{C. 88B, 6 18. C. 884} \end{center}Appropera et praedam pennis crepitantibus aufer. \\ Nam celer oppressi fallacia vincula colli \\ Excutit et rauca subsannat voce magistri \\ Consilium et laeta fruitur iam pace solutus. \\ o Hic prope Peltvinnm radicibus Apennini \\ Nidificat, patulis qua se sol obicit agris, \\ Persimilis cineri dorsum, maculosaque terga \\ Inticiunt pullae cacabantis imagine notae \\ Tarpeiae est custos arcis non corpore maior \\ Nec qui te volucres docuit, Palamede, figuras. \\ Saepe ego nutantem sub iniquo pondere vidi \\ Maonomi puerum, portat cum prandia, circo \\ Quae consul praetorve novus construxit ovanti. \\ B. M. \\ 
      \end{verse}
  
            \subsection*{884}
      \begin{verse}
      B. III 204. \\ Cum nemus omne suo viridi spoliatur honore, \\ Fultus equi niveis silvas pete protinus altas \\ Exuviis: praeda est facilis et amoena scolopax. \\ Corpore non Paphiis avibus maiore videbis. \\ q \\ lla sub aggeribus primis, qua proluit humor, \\ Pascitur, exiguos sectans obsonia vermes. \\ At non illa oculis, quibus est obtusior, etsi \\ Sint nuimium grandes, sed acutis naribus instat: \\ Impresso in terram rostri mucrone sequaces \\ o Vermiculos trahit et vili dat praemia gulae \\ 
        \pagebreak 
    \begin{center} \textbf{C. 885 887} \end{center} \marginpar{[332]} B. III 26.M. 1701. \\ 
      \end{verse}
  
            \subsection*{885}
      \begin{verse}
      B V 06. \\ Hortis lHesperidum, Sabelle, cultis \\ Nostrae cultior hortus est puellae. \\ Mirari, o bone, desinas, Sabelle. \\ Hortorum deus ipse nam Priapus \\ Cunctis hunc fodit et rigat diebus. \\ B. I 14. M. 247. \\ 
      \end{verse}
  
            \subsection*{886}
      \begin{verse}
      B. V 404. \\ Acidis tumulns \\ Acidos haec cernis montaua cacumina busti \\ Aequor et ex imis fluminis ire iugis? \\ Ista Cyclopei durant monumenta furoris: \\ Hic amor, hic dolor est, candida nympha, tuus. \\ Sed bene, si periit, iacet hac sub mole sepultus, \\ Nomen et exultans unda perenne vehit. \\ Sic manet ille quidem neque mortuus esse feretur \\ Vitaque per liquidas caerula manat aquas. \\ 
      \end{verse}
  
            \subsection*{887}
      \begin{verse}
      B. I 3B. M. 590. \\ B. V 401. \\ Pani et Apollini \\ Pan tibi, Pboebe tibi, lenimen dulce laborum, \\ Otia dum terimus, carmina mille damus. \\ 
        \pagebreak 
     \marginpar{[333]} \begin{center} \textbf{C. 888 890} \end{center}
      \end{verse}
  
            \subsection*{888}
      \begin{verse}
      Vile nunc AMnthol. lt. prphica 871 CIL. X IlI 581. \\ B. III 10. M. 70. \\ 
      \end{verse}
  
            \subsection*{889}
      \begin{verse}
      B. V 405. \\ \poemtitle{De urbe eltria}Feltria perpetuo nivium damnata rigore \\ Terra vale posthac non adeunda mihi. \\ B.II 235. M.1544. \\ 
      \end{verse}
  
            \subsection*{890}
      \begin{verse}
      B. V 006. \\ \poemtitle{IVLII}Petroni carmen divino pondere currit, \\ Quo iuvenum mores arguit atque senum. \\ Quare illepraesa gaudet lasciva puella, \\ At quoque delicias frigida sentit anus. \\ Nam rbter diri scripsitque Neronis †amictu, \\ Arbiter arbitrio dictus et ipse suo. \\ 
        \pagebreak 
     \marginpar{[334]} \begin{center} \textbf{C. 891 893, 12} \end{center}B. I 22. M. 1545. \\ 
      \end{verse}
  
            \subsection*{891}
      \begin{verse}
      B. V 406. \\ Infantem Nymphae Bacchum, quo tempore ab igne \\ Prodiit, inventum sub cinere abluerant. \\ Ex illo Nymphis cum Baccbo gratia multa est, \\ Seiunctus quod sit ignis et urat adhuc. \\ B.III217. M.987. \\ 
      \end{verse}
  
            \subsection*{892}
      \begin{verse}
      B. V 406. \\ Formosissima Lai feminarum, \\ Dum noctis pretium tibi requiro, \\ Magnum continuo petis talentum: \\ Tanti non emo, Lai, paenitere! \\ 
      \end{verse}
  
            \subsection*{893}
      \begin{verse}
      B. M. B. \\ \poemtitle{SEVERI SANCrI IDESr ENOELECM}rhetoris De mortibus boum \\ Aeon. Bucolus. Tityrus. \\ Quidnamsolivagus,Bucole,tristiaAeg \\ Demissis graviter luminibus gemis? \\ 
        \pagebreak 
    \begin{center} \textbf{C. 893B, 3—30} \end{center} \marginpar{[335]} Cur manant lacrimis largifluis genae? \\ Fac, ut norit amans tui! \\ Aegon, quaeso, sinas alta silentia \\ 5 Buc. \\ Aegris me penitus condere sensibus. \\ Nam vulnus reserat, qui mala publicat, \\ Claudit, qui tacitum premit. \\ Contra est quam loqueris, recta nec autumas. \\ Aeg. \\ Nam divisa minus sarcina fit gravis, \\ Et quicquid tegitur, saevius incoquit. \\ Prodest sermo doloribus. \\ Scis,Aegon,gregibusquamfuerimpotens, \lbrack 3uc. \rbrack  \\ Vt totis pecudes fluminibus vagae \\ Complerent etiam concava vallium, \\ Campos et iuga montium: \\ Nunc lapsae penitus spes et opes meae, \\ Et longus peperit quae labor omnibus \\ Vitae temporibus, perdita biduo. \\ Cursus tam citus est malis! \\ Iaec iam dira lues serpere dicitur. \\ Aeg. \\ Pridem Pannonios, Hllyrios quoque \\ Et Belgas graviter stravit et impio \\ Cursu nos quoque nunc petit. \\ Sed tu, qui solitus nosse salubribus \\ Sucis perniciem pellere noxiam, \\ Cur non anticipans, quae metuenda sunt, \\ Admosti medicas manus? \\ Tanti nulla metus praevia signa sunt, \\ uc. \\ Sed quod corript, id morbus et opprimit: \\ 
        \pagebreak 
     \marginpar{[336]} \begin{center} \textbf{C. 89, 3163} \end{center}Nec languere sinit nec patitur moras. \\ Sic mors ante luem venit. \\ Plaustris subdideram fortia corpora \\ Lectorum studio, quo potui, boum, \\ Queis mentes geminae, consona tinnulo \\ Concentu crepitacula, \\ Aetas consimilis saetaque concolor, \\ Mansuetudo eadem, robur idem fuit \\ Et fatum: medio nam ruit aggere \\ Par victum parili nece. \\ Mllito penitus farra dabam solo: \\ Largis putris erat glaeba liquoribus, \\ Sulcos perfacilis stiva tetenderat, \\ Nusquam vomer inbaeserat: \\ Laevus bos subito labitur impetu, \\ Aestas quem domitum uiderat altera \\ Tristem continuo disiugo coniugem, \\ Nil iam plus metuens mali; \\ Dicto sed citius consequitur necem, \\ Semper qui fuerat  \lbrack sanus \rbrack  et integer; \\ Tunc longis quatiens ilia pulsibus \\ Victum deposuit caput. \\ Angor, discrucior, maereo, lugeo. \\ Aeg. \\ Damnis quippe tuis non secus ac meis \\ Pectus conficitur. sed tamen arbitror \\ Salvos esse greges tibi? \\ lluc tendo miser, quo gravor acrius. \\ Buc. \\ Nam solamen erat vel minimum mali, \\ Si fetura daret posterior mihi, \\ Quod praesens rapuit lues. \\ Sed quis vera putet, progeniem quoque \\ Extinctam pariter? vidi ego cernuam \\ Iuuicem gravidam, vidi auimas duas \\ 
        \pagebreak 
    \begin{center} \textbf{C. 893, 61—97} \end{center} \marginpar{[337]} Vno in corpore perditas! \\ IHic fontis renuens, gramiuis immemor \\ Errat succiduo bucula poplite; \\ Nec longum refugit, sed graviter ruit \\ Leti compede claudicans. \\ At parte ex alia, qui vitulus modo \\ Lascivas saliens texuerat vias, \\ Vt matrem subiit, mox sibi morbido \\ Pestem traxit ab ubere. \\ Mater tristifico vulnere saucia, \\ Vt vidit vituli condita lumina, \\ Mugitus iterans ac misere gemens \\ Lapsa est et voluit mori. \\ Tunc tanquam metuens, ne sitis aridas \\ Fauces opprimeret, sic quoque dum iacet, \\ Admovit  \lbrack moriens \rbrack  ubera mortuo. \\ Post mortem pietas viget. \\ Hinc taurus solidi vir gregis et pater, \\ Cervicis validae frontis et arduae, \\ Laetus dum sibimet plus nimio placet, \\ Prato concidit herbido. \\ Quam multis foliis silva cadentibus \\ Nudatur gelidis tacta aquilonibus, \\ Quam densis fluitant velleribus nives, \\ Tam crebrae pecudum neces. \\ Nunc totum tegitur funeribus solum; \\ Inflantur tumidis corpora ventribus, \\ Albent lividulis lumina nubibus, \\ Tenso crura rigent pede. \\ Iam circum volitant agmina tristium \\ Dirarumque avium, iamque canum greges p \\ Insistunt laceris visceribus frui, \\ Ieu cur non etiam meis? \\ Aeg. Quidnam, quaeso, quid est, quod vario modo \\ 
        \pagebreak 
     \marginpar{[338]} \begin{center} \textbf{C. 893, 98—128} \end{center}Fatum triste necis transilit alteros \\ Afligitque alios? en uibi Tiyrus \\ Salvo laetus agit grege! \\ 100 \\ Ipsum contueor. dic age, Tityre: \\ Buc. \\ Quis te subripuit cladibus his deus, \\ Vt pestis pecudum, quae populata sit \\ Vicinos, tibi nulla sit? \\ Signum, quod perhibent esse crucis dei, \\ T7. \\ 105 \\ Magnis qui colitur solus in urbibus, \\ Christus, perpetui gloria numinis, \\ Cuius filius unicus. \\ IHoc signum mediis frontibus additum \\ Cunctarum pecudum certa salus fuit. \\ 110 \\ Sic vero deus hoc nomine praepotens \\ Salvator vocitatus est. \\ Fugit continuo saeva lues greges, \\ Morbis nil licuit. si tamen hunc deum \\ Exorare velis, credere sufficit: \\ Votum sola fides iuvat. \\ Non ullis madida est ara cruoribus \\ Nec morbus pecudum caede repellitur, \\ Sed simplex animi purificatio \\ Optatis fruitur bonis. \\ laec si certa probas, Tityre, nil moror, \\ Buc. \\ Quin veris famuler religionibus. \\ Errorem veterem defugiam libens, \\ Nam fallax et inanis est. \\ T71. \\ Atquiiamproperatmensmeavisere \\ Summi templa dei. quin, age, ucole, \\ Non longam pariter congredimur viam, \\ Christi et numina noscimus? \\ 
        \pagebreak 
    \begin{center} \textbf{C. 89B, 129132. C. 894 897} \end{center} \marginpar{[339]} Et me consiliis iungite prosperis. \\ Aeg. \\ Nam cur addubitem, quin homini quoque \\ 130 \\ Signum prosit idem perpete saeculo, \\ Quo vis morbida vincitur? \\ B.III 234. M.999. \\ 
      \end{verse}
  
            \subsection*{894}
      \begin{verse}
      . \\ Praxitelis Venerem lapidosa per oscula multi sqq \\ B.III235.M.1000. \\ 
      \end{verse}
  
            \subsection*{895}
      \begin{verse}
      B . \\ Mitte, aquila, hanc praedam. mitte, inquam, si sapis, \\ inquam, sqq \\ 
      \end{verse}
  
            \subsection*{95}
      \begin{verse}
      bHdem \\ Quicquid formosum est, penetrat, trahit, arripit ad se sqq. \\ B. I 6. M. 1552. \\ 
      \end{verse}
  
            \subsection*{896}
      \begin{verse}
      B . \\ ‘Quaenamhaecforma’dei.‘curversaest?’fulgura \\ lucis sqq \\ B.III275. M.262 \\ 
      \end{verse}
  
            \subsection*{897}
      \begin{verse}
      B. V 391. \\ Vde nune c. 803. \\ 
        \pagebreak 
     \marginpar{[340]} \begin{center} \textbf{C. 898 901} \end{center}
      \end{verse}
  
            \subsection*{898}
      \begin{verse}
      Ede nune c. 79. \\ B. II 252 M. 849. \\ 
      \end{verse}
  
            \subsection*{899}
      \begin{verse}
      B. V 407. \\ \poemtitle{CORNELII CESI}r \\ Dictantes Medici quandoque et Apollinis artes \\ Musas lomano iussimus ore loqui. \\ Nec minus est nobis per pauca volumina famae, \\ Quam quos nulla satis bibliotheca capit. \\ B. II 171. M. 557 \\ 
      \end{verse}
  
            \subsection*{900}
      \begin{verse}
      B. V 07. \\ \poemtitle{MODESTI}\poemtitle{De Lucretia}Roma, tibi ambiguum mea mors sine teste dedisset, \\ Esset utrum ut corpus sic scelerata anima. \\ Proin animam testor scelus haud tetigisse, sed ecce, \\ Vt solum foedum est corpus, et id fodio, \\ Qua cruor inflictam labem lavet et per aperta \\ Pectora ab invisa sit fuga sede animae. \\ B. III 166. M. 945. \\ 
      \end{verse}
  
            \subsection*{901}
      \begin{verse}
      B. V 407. \\ rr \\ \poemtitle{TVLLII MARCI}Callidus Afer eris semper, lomane disertus, \\ Semper Galle piger, semper IHibere celer. \\ 
        \pagebreak 
     \marginpar{[341]} \begin{center} \textbf{C. 902 904} \end{center}B III 160.M.923 \\ 
      \end{verse}
  
            \subsection*{902}
      \begin{verse}
      . \\ Cor non laudo hominis: nam perfidum et exitiale. \\ Non laudo os: namque est vaniloquum et fatuum. \\ Non oculos: etenim sunt lusci oculi atque patrantes. \\ Non barbam: rufa est. non cilia: haud cilia. \\ Non mihi laudanturque manus: hae namque rapaces. \\ Et non laudatur mentula: merdacea est. \\ Non nates: patulae. genua et non laudo: maligna. \\ Et non laudo pedes: pes geminusque fugax. \\ Non frontem: effrons est. non ventrem gurgulionis. \\ 0 Lumbos: elumbos. renem: ubicumque malus. \\ Non nomen Calvitoris: nam est omine nome \\ Tetrum, ut cor multis exitiale bonis. \\ Ecquid laudo igitur de hoc corpore? laudo capillos, \\ Tam foeda a calva qui modo profugerint. \\ 
      \end{verse}
  
            \subsection*{903}
      \begin{verse}
      Vite uoe Lnt. lat. cpir. 1522 C. II 1122. \\ B. 1I 236. M. 544. \\ 
      \end{verse}
  
            \subsection*{904}
      \begin{verse}
      B V 407. \\ \poemtitle{C. AVRELII ROMVLI}Cecropias noctes, doctorum exempla virorum, \\ Donat habere mihi nobilis Eustochius. \\ Vivat et aeternum laetus bona tempora ducat, \\ Qui sic dilecto tanta docenda dedit. \\ 
        \pagebreak 
     \marginpar{[342]} \begin{center} \textbf{C. 905 913} \end{center}B. M. 875. \\ 
      \end{verse}
  
            \subsection*{905}
      \begin{verse}
      B. V. 363. \\ Dente perit Lycabas, serpens pede, Nigra veneno, \\ Flumine avis, calamo †quoredimite lepus. \\ 906 908 \\ de nuc e. 7 \\ 
      \end{verse}
  
            \subsection*{909}
      \begin{verse}
      epir.885CI.BI1330aidenuncLthol.lat. \\ 
      \end{verse}
  
            \subsection*{910}
      \begin{verse}
      ide nunc c. \\ 
      \end{verse}
  
            \subsection*{911}
      \begin{verse}
      Vde nunc c. \\ B.III271. M. 1007. \\ 
      \end{verse}
  
            \subsection*{912}
      \begin{verse}
      B. V 408. \\ Virginis insano lulianus captus amore \\ Femina fit cultu dissimulatque virum \\ Et sic indutus muliebriter intrat ad illam. \\ Res patet, abscindit membra pudenda pater. \\ Femina virque prius nec vir nec femina nunc est: \\ Fit neutrum, credi femina dum voluit. \\ 
      \end{verse}
  
            \subsection*{913}
      \begin{verse}
      Me nnunc c. \\ 
        \pagebreak 
    \begin{center} \textbf{C. 914 917} \end{center} \marginpar{[343]} 
      \end{verse}
  
            \subsection*{914}
      \begin{verse}
      B. M. B. \\ Non fuit Arsacidum tanti expugnare Seleucen \\ Italaque Vltori signa referre Hovi, \\ Vt desiderio nostri curaque Lycoris \\ Heu iaceat menses paene sepulta novem \\ Pingit et Euphratis currentes mollius undas \\ Victricesque aquilas sub duce Ventidio, \\ Qui nunc Crassorum manes direptaque signa \\ Vindicat Augusti Caesaris auspiciis sqq \\ 
      \end{verse}
  
            \subsection*{915}
      \begin{verse}
      B. M. B. \\ Occurris cum mane mibi, ni purior ipsa sqq \\ B. III 172. \\ 
      \end{verse}
  
            \subsection*{916}
      \begin{verse}
      M. I565. B. \\ deliciumquemeumsqq.matrisamor \\ B. III 238. \\ 
      \end{verse}
  
            \subsection*{917}
      \begin{verse}
      M. 1003. B. \\ Subrides si, virgo, faces iocularis ocellis sqq \\ 
        \pagebreak 
    \begin{center} \textbf{C. 918, 1—10} \end{center} \marginpar{[344]} 918 921 \\ \poemtitle{VESTRICII SPVRNNAE}\poemtitle{De eontemptu saeculi}B. M. \\ I 91s) \\ B. V 408. \\ Dulces Vestricii iocos, \\ Seras Socmaticae relliquias domus, \\ Ne laudes nimium, Mari. \\ Contemnit placits  \lbrack nobilibus viris, \\ Soli qui spientiae \\ Post florem tepidum nec stabilem grdum \\ Aetatis senium dicat \\ Mentis compositae, qulis nb ardui \\ Ad se versa laboribus. \\ Quos non dat patriae. seposuit eibi \\ 
        \pagebreak 
     \marginpar{[345]} \begin{center} \textbf{C. 91, 1126. C. 919, 110} \end{center}Annos, oba lucro gravi. \\  \lbrack Iam nulla \rbrack  ambitio tegmine candida \\ Illudat gravidae spei! \\ Nos sero pelagus vicimus invium: \\ Quicquid viximus. interit. \\ Aestas quem decies septima dividit, \\ An leves memoret iocos \\ Atque aptos cithare conciliet modos, \\ Surdis aurieulis strepen? \\ Quisquis decrepiti corporis es reus. \\ Sat sese eloquii probat \\ Si ser \lbrack vat \rbrack  plaecidi iur silentii \\ Et patocinium otii. \\ HIoc cani grvitas verticis abstitit. \\ Non ut sponte su fugax. \\ Sed multi numeris carminis . . \\ II 919) \\ B. V 109. \\ Fave, sanct deum sata, \\ Nullis, Pauperies. numinibus minor, \\ Tecum ei spins tibi \\ Vitae mgnificis hospes honoribus, \\ Absolvens numerum tue \\ In te letitiae, sordida, cum quies \\ Raucis nuda umultibus \\ Ambit te patria rforcilis in domo. \\ Nulli tvendibilis plausibus, \\ Contemptrix queruli mgnnimis fori; \\ 
        \pagebreak 
     \marginpar{[346]} \begin{center} \textbf{C. 919,. 1121. C. 920} \end{center}Nl non sola potens, ubi \\ FTurtivis procerum suppliciis procul \\ Regns in propio sinu. \\ Pelix, quem teneris mater ab unguibus \\ Et regin rpis simul! \\ Non illum  \lbrack tumidi \rbrack  fascibus arduum \\ Versat nobilitas mala \\ Currum facilem fluctibus, ut sis \\ Orbem sideribu otet. \\ llnm splendida nox et deco improbe \\ Caecus pmecipiiant . . \\ III 920) \\ B. V 410. \\ Postquam fixa solo semel \\ Spernit fluctivagos ncora navitas \\ In saevum pelague eequi,. \\ Quam vitat rabido pemniciem mai. \\ In suo eperit sinu. \\ Herentem tumidis . . . dentibus \\ Aerugo popria exedit. \\ Ni te desidia sancta quies levet, \\ Turbas dum popnli fugis. \\ Privatis quaties fata tumultibus, \\ In ie ludere pervicax. \\ Nos somnus vigilans.  \lbrack quo plaeide \rbrack  fruor. \\ Tortis libemt anguibus \\ At presso gmcilis Cura manet pede. \\ 
        \pagebreak 
    \begin{center} \textbf{C. 921 923, 1—3} \end{center}
      \end{verse}
  
            \subsection*{}
      \begin{verse}
      \poemtitle{IV 92}B. V. 410. \\ Ingti nebulae t desidie caput \\ Circumstant trepidum; sos nimia in pobos \\ Incestis facilis annuit ausibus; \\ Sta contm assiduo pede. \\ Multum turba tenax . . . fidei \\ Vltma fata fuit non docili fugae \\ .premio.desider. \\ 
      \end{verse}
  
            \subsection*{922}
      \begin{verse}
      B. III 99. M. 225. \\ \poemtitle{AP2VLEI}B. V 411. \\ Principium vitae . . obitus meditatio est. \\ on vult emendari peccare nesciens. \\ Immoderata ira fructus est insniae. \\ Eiusdem \\ Pecuninm amico credens fert damnum duplex: \\ Agentum \lbrack que \rbrack  et sodalem perdidit simul. \\ B. I 157. M. 678 \\ 
      \end{verse}
  
            \subsection*{923}
      \begin{verse}
      B V 411. \\ Cerne Arbem Myham temerati a crimine lecti \\ Patria praecipiti linquere regn fug \\ Pone sequens genito vi vindicis imminet ensis \\ 
        \pagebreak 
     \marginpar{[348]} \begin{center} \textbf{C. 923, 4 6. C. 924 925. 1 3} \end{center}Et premit incestae iam pede Poea pedem, \\ Donec mutata in tfontem proiecta figur est \\ Sie quoque non certo lymphn liquore fugit. \\ B. I 10. M. 642. \\ 
      \end{verse}
  
            \subsection*{924}
      \begin{verse}
      B. V 412. \\ Cum patrem e patrios ferret cervice penates \\ Atque adeo ingentis fata noverca Asiae \\ Romanosque deos Europaeosque triumphos \\ Totiu et mundi t exsuviis spolia. \\ Tos Anchisindes potert peiisse per ignes. \\ Iaemonia ut virtus ract era ante rota. \\ At cum flamma furen tetigit senis or parentis, \\ Diesiliit nullis noxia turbinibus. \\ FTata opei restant, pietas si ola favebit \\ Invictum in mediis ignibus obsequium est. \\ 
      \end{verse}
  
            \subsection*{925}
      \begin{verse}
      B . I p. 738. \\ \poemtitle{LAETI AVIANI}M. 55. B. V 425. \\ Versu in praesens opus de Mereurii nuptii \\ Qui dubias nrtes per mystica dict subibis, \\ Mercurii doctos volvere disce toros. \\ Ille brevi ductu scandet sublime catbedram, \\ 
        \pagebreak 
    \begin{center} \textbf{C. 925, 4 6. C. 926 928, 1—3} \end{center} \marginpar{[349]} Qui ychnos discet, docte Cnpella, tuos. \\ Canitiem auctoris uge avitasque magistri: \\ Qui dicet libris, hinc cito proficiet. \\ B. III 98. M. 919. \\ 
      \end{verse}
  
            \subsection*{926}
      \begin{verse}
      IB. V 412. \\ Desidiam nolis. sed nec labor impobus omni \\ Arridet. dulce est inter utrumque nihil. \\ Ignavum scabies, luctantem iniuria pedit. \\ lle sui, alteius perdidit iste decus. \\ 
      \end{verse}
  
            \subsection*{927}
      \begin{verse}
      B. M. B. V 412. \\ Hinc est quod populus aurum quasi numen adomrans \\ Audet in ignoum sponte venire nefas, \\ Speque lucri totiens excedere ius et honestum \\ Sustinet, ut gtis nunc iuvet esse eum., \\ Ius ruit. odo pei, sceeri placet ora manusque \\ Vendere, qumque inopem, tam pudet esse probum. \\ B. M. B \\ C 8 Eece. Vint. \\ 
      \end{verse}
  
            \subsection*{928}
      \begin{verse}
      XXX 348. \\ Lux festa sacis vult litari pginis. \\ Remove profanos codices. \\ Iymno saeranda luminis primordia, \\ 
        \pagebreak 
     \marginpar{[350]} \begin{center} \textbf{C. 928,. — 36} \end{center}Quae Chistus impemt coli. \\ In ore Cbristus nectar, in lingua favus, \\ Ambrosia viva in gutture, \\ Lotcus beata in pectore, qua nescias \\ Abire gustata semel; \\ Mel in medullis, lux seena pupulis, \\ In auribus vitae sonus. \\ Elin vel plectrum suavis eloquentiae \\ Hoc nomine audito redit. \\ Cusu vagante, lubrico infortunio, \\ Tentationibus Mali, \\ Siti et fame, calore et algu mortuis \\ Malgm pmebetur potens. \\ Penetrle mentis dira desperatio \\ Peccaminum ob molem quatit,. \\ Saevit paterna in viscere imo prviias, \\ Libido succendit faces. \\ Anri cupido molle pectus incitat \\ Per fas nefasque cogier, \\ Taetentis animus haurit ingluviem gulae \\ Venere atque Baccho perditam, \\ Cmuentus ultor raptat iracundiam \\ Mucrone pompto stinguere. \\ Inferre mandat terror infortunium \\ Cavere quod possis male, \\ Vis impotens sui eicit patientiam \\ Et arm quaerit perdita, \\ Fidem sinistra qussat obstinatio \\ Felix errorum non diu: \\ Hoe ad slutem nomen auditum venit \\ Ioc omne tollit pharmacum. \\ Obsessa membra spiritusque daemone \\ Vexatus impurissimo \\ 
        \pagebreak 
    \begin{center} \textbf{C. 28, 3775} \end{center} \marginpar{[351]} Hoc, hoc medelam sotiuntur nomine \\ Redeunique sursum ab inferis \\ Cantata diro carmine, et bustis sono \\ Devota go corpor \\ Vim colligantis peridam excutiunt luis \\ Et sana vertuntur domum. \\ Compago quem soluta membratim iubet \\ Lecto edere debilem. \\ Quem caecitas, crux omnium miserrima, \\ Addixit alieno pedi, \\ Vtroque claudum quem venire poplite \\ Videt universa civitas, \\ Suis redire gressibus domum queunt \\ Nomen celebrntes dei. \\ Salve, o Apollo vere, Paean inclite. \\ Pulsor drconis inferi! \\ Duleis tui pharera testimonii. \\ Quod quattuor constat viris. \\ Sngitta melle tinctilis prophetico,. \\ Pinnata patrum oraculis; \\ Acus patemnae forte virtutis sonans, \\ Miraculis nervus potens, \\ Strvere veterem morte serpentem sua. \\ Io triumphe nobilis! \\ Salve beat saeculi victoria, \\ Parens beati temporis! \\ Salve, quod omnes caelici, medii, inferi \\ Nomen genu flexo audiunt. \\ Salve unu unus unus in trino deus, \\ Slve una in uno trinitas. \\ Haec lux Eoo cum levata cardine \\ Iter diurnum suscipit, \\ Iec, cum occidente sol subit curru fretum, \\ Benedictio me consecrat, \\ Crucifixe victor, expitor criminum \\ In morte vita prepotens, \\ Fac cum supremo sevocabor tempore,. \\ Ab obruto malis chao \\ Traducat ista me tibi benedictio, \\ 
        \pagebreak 
    \begin{center} \textbf{C. 328, 76 78. C. 929 931, 12} \end{center} \marginpar{[352]} Quod utile, interim dato: \\ Neu se inquieta mentis excruciet mora \\ Fetente vincta crcere. \\ 
      \end{verse}
  
            \subsection*{929}
      \begin{verse}
      B.III 274.M. 1011. \\ In senecetutem. \\ . \\ Vili es nulli, cunctis ingmta,. senectus: \\ Te Stygio peperit cana Megera deo. \\ Nil deo firmum est. quod non tua robora frngant: \\ Arma stilos cartas sax metall deos. \\ Carmina vivaci membranis illita succo \\ Annorum serie debilitata cadunt. \\ Ipsa mihi pugnas que nectere mille solebt, \\ Aequanles inter maxima dicta suas, \\ Numqum sueta nisi iugulnto cedere ab hoste \\ Inque imis mortes quaerere visceribus, \\ Virgineis ambita choris, damta puellis. \\ Qumque hostes etiam caram habuere sui, \\ Ill capu roseum lorenti sndice cinct, \\ Lnnguida caeruleo mentul victa situ est. \\ 
      \end{verse}
  
            \subsection*{930}
      \begin{verse}
      ide nnc c. 806. \\ B. III 106. M. 920. \\ 
      \end{verse}
  
            \subsection*{931}
      \begin{verse}
      IB V 14. \\ Tempore qui leto sortem ridemus mrm, \\ Certa velut stabiles non dea veret equos. \\ 
        \pagebreak 
    \begin{center} \textbf{C. 931. 331} \end{center} \marginpar{[353]} Cur iidem adverso laetam venamur in aesu? \\ Ah mea tam bene quod cognita causa mihi est! \\ O qui carorum cnput, o carissime quondm, \\ An potes et miseri nunc meminisse tui? \\ Et equidem in laetis nemo non promptus amicus, \\ psn homini advesis umbr inimica sua est. \\ Tantalus infelix, dieunt,. eonviva deorm \\ Nunc quoque apud Manes vietima scr Iovi est. \\ Vsque adeo poenns non delent ultima fata: \\ Qui ceeidit divis, sugere nemo valet. \\ Scilicet et nobis,. qui functo copoe nuper \\ Dur per antiqui busta fugamur avi,. \\ Pefida felices ostendit regula divos \\ Et voluit sanctis partem aliquam esse locis, \\ Non alio exemplo nisi ut eruta turbine diro \\ Desereret media nos rota fracta via. \\ Ah mihi preterito referant si numina vultus \\ (Non etenim veterem te, nova plag, quero), \\ V t facili amisos adblandiar ore favores,. \\ Tura ferens dignis crmina laeta diie \\ Vt ravis hibemno torens de monte volutus \\ Obvia non mgna arbuta verit ope,. \\ 2 Saepe domos etiam. snepe addita moenin aptat, \\ Curxens precipiti per iug long via: \\ Sic semel atque uno nos abstulit improbus ictu. \\ Mentem etiam et Musis pector vota, furor. \\ Altus emam certe Sortis furor omia vincit; \\ Obvius ut quisque est, ‘tu mihi’ clmo ‘veni’. \\ Improbus est, quiquis non hee pectacul cemnens \\ 
        \pagebreak 
     \marginpar{[354]} \begin{center} \textbf{C. 931, 3265} \end{center}Fortuna insiabili se quoque stae putat. \\ En ego nec caros habeo miserandus amicos, \\ A quibus auxilio verbave remve petam. \\ Vnicus est, qui me non plane spernit egentem \\ Cetera contemto supplice turba fugit. \\ Quid faciam infelix que non nisi in omnibus uni est, \\ Sollicitem longo carine saepe fidem? \\ Fracta fides cebro est precibus lugentis amici. \\ Hac tamen amissa num mge tristis ero? \\ Vos superi et, quorum es merito mihi numinis instar, \\ Quem numquam poshac spero videre, parens, \\ Vnius in gemio spes nost abiecta sodalis \\ Tabet et absentis perdita veba sonnt. \\ Ille iuvet forsan, sed veboa immensa locutis \\ Facta brevi constan non repetenda modo. \\ Scilicet in somnis quae poscunt pluima divos \\ In frtrum auxiliis experietur inops. \\ Nos ubi praeteritis componimus orsa secuti \\ Tempois, his etiam credere fata vetant. \\ Interea rpit hor nee est reparabilis horm: \\ Nos lenta placuit tabe perire deo, \\ Vt pereunt miseri, quos vitae fructus ademtus \\ Deseruit, nulla post eparndus ope. \\ Tempoe non illo Priamum periisse putabis, \\ Quo iacuit Teucro littore truncus ines; \\ Verum ut Thessalicis circum su moenia funus \\ Hectoreum apuit fortior hostis equis. \\ Non ea post ductos victo ex Oriente triumphos \\ Exstiuxit Magnum. quae tulit, hora, caput: \\ Illa illum exstinxit, campis conressa Phbilippis \\ Caesarea populum quae ruit hor manu. \\ Scilieet hic vitae finis, qui finis honorum est. \\ Cumque sua pereunt prosperitate viri. \\ Sic quod heri vixi, qui iam nox altem pmesto est, \\ 
        \pagebreak 
     \marginpar{[355]} \begin{center} \textbf{C. 931, 66—99} \end{center}Non illud tempus. quod modo duco, vocem. \\ Quaeque dies ibi vita sua est; cum transiit illa, \\ Heternae non est lux hodiemna memor. \\ Quare, quem fructus vivendi, vita relinquit \\ Nempe quid hoc demto. mors nisi maesta, venit \\ Vndique quae laetis si se tistissima monstrat, \\ Quid miseris censes nolle parare mali2 \\ Saepe ferunt venti, saepe est firmissima navis; \\ Mutarunt venti flamina: navis obit. \\ 7 Dum cavet hos scopulos, alios incurree nauta \\ Sentit et est damno serior rte suo. \\ Structa quid in magno sunt propugnacula saxo? \\ Iumano maius sepe labore ruun: \\ Callidus internis se spiritus ingerit antris; \\ Cum posit rupes sic gravis arce iacet. \\ Tuta nec in solido rerum Fortuna favore est; \\ Cum minime credas, impulit illa rotam. \\ Tum quoque, cum nitido ridet plcidissima vultu, \\ Tupis in adveso pectore fucus inest. \\ Elevat incautum, sperantem mxima nutu \\ Proiicit, ascendit cetema turba tmen. \\ Haec ego qui possum miseris miser ipse referre, \\ Non poteram dietis olim adbibere fidem, \\ Stulta quia et votis tam credula turba beta es, \\ 0 Rideat ut miseri mgna monenti opem. \\ Saepe mihi dixit Paeligni Musa poete. \\ Nunc ego mutta quod viee dieo tibi. \\ Credite, qui fidum Fortunn videtur asylum: \\ Non erit hnec vobis. non fuit illa mihi. \\ Pendet ab axe suo vitae variabilis ordo, \\ Nullan sui certa est hora nee ulla fuit, \\ Cumque nihil possit longi fug mobilis aevi, \\ Se amen hac semper mobilitate fugit. \\ Sed vitam incertam sequitur misernbile letum. \\ 
        \pagebreak 
     \marginpar{[356]} \begin{center} \textbf{931, 100—135} \end{center}HIumanas inter sola ea certa via est. \\ 100 \\ Stultus et indoctus qui, cum non nesciat, illo \\ Eximi ab innumeris tempom lassa mlis,. \\ Hoc tamen extremo timet et fugit undique nisu, \\ Quod sequitur fortem quod timidumque pemit. \\ Vitca tamen dulcis.’ quid si magis altem restat \\ 105 \\ Mors gmvis.’ an scimus non graviore pemi? \\ Altior at miseris cura est, dolor ossibus haeret. \\ Verba cadunt pulcxe. pectora maeror edit. \\ Sic ego qui mediecas aliis dare molio herbas, \\ Nullinus credam vulner nosr manu. \\ 110 \\ Vna sed in cunctis data sunt soltia curis. \\ Sustineant lpso quae mihi membm pede. \\ Nulla, licet tenuis linquat pimmina vitae, \\ Absque intervallo sors truculenta fuit, \\ Nempe quia est miseris aliquod quoque numen amicum \\ Qui patitur, nullis spernitur ille deis. \\ Venit ubi ad summum,. non pogressura ruina es. \\ Sed stetit et retro est viribus act suis. \\ ps quoque e bustis extollier ossa videmus. \\ Cum patia coluit relligione nepos. \\ 10 \\ Dirunta muali tormento saxa feruntu: \\ Erigit his alio moenia Marte labor. \\ Infelix quisquis nocet infelieibus umquam. \\ Hie meruit nullo tristia fine pati. \\ Omne quod hic cemnis, tenuis modo corpois umbr est; \\ Vix macie exesis artubus ossa traho. \\ Rarus aperta movet procul internodia poples. \\ Cousedit genubus tractus abo ore tremor. \\ Sodibu et multo similis squalore sepulti \\ Vix inter vivos larva videnda vagor. \\ 130 \\ Non tamen hoc hostes, non hoc mala numina posint, \\ Vtc valeat ponum nox pepulisse caput. \\ Biseis ut rpta est, amo aegrum inclusit Achillem, \\ 135 \\ 
        \pagebreak 
    \begin{center} \textbf{C. 931, 136—14. C. 932 933} \end{center} \marginpar{[357]} Son metus Hectoree Memnoniaeve facis. \\ lle sed exstincto rediit vigor acer amico, \\ Reddidit ut iustus deposit arma dolor. \\ Me quoque Fors mihi estitue, moribundaque membra \\ 0 Viibus ie suis splendida Fata dabunt. \\ Tum trucis invidiae furiis ulticibus horens \\ Involvam positis arma virumque malis. \\ Intere, ut libitum est, absentibus utee divis, \\ Livor, et in nosro gaudi quaere rogo! \\ B. V 16. M. 1027. \\ 
      \end{verse}
  
            \subsection*{932}
      \begin{verse}
      B. V 419. \\ Sera dedit Pboebo fugiente ecrepuscula Vesper. \\ Conticuit mediam rerum confectio noctem. \\ Exoriens tenebri dilucula restituit sol. \\ Tempora sol revehit, nox cedit tempor delens. \\ Nox accensetu soli ceu luna diei. \\ 
      \end{verse}
  
            \subsection*{933}
      \begin{verse}
      \poemtitle{SYMMACIVS}B. II 137. M. 2565. \\ de Boethio \\ B. V 419. \\ Fortunae et virtutis opus, Severine Boethi, \\ E patria pulsus non tua pe scelera. \\ Tandem ignotus habes qui te colat, ut tua virtus, \\ Vt tua fortuna promeruitque sophos. \\ Post obitum dant fata locum, post fat supestes \\ Voris propriae te quoque fama colit. \\ 
        \pagebreak 
    \begin{center} \textbf{C. 934, 1—30} \end{center} \marginpar{[358]} B. II 17. M. 697. \\ 
      \end{verse}
  
            \subsection*{934}
      \begin{verse}
      B. V 419. \\ Mille pos annos quater que centum \\ Graeeciae vindex capit arma mnndi \\ Et superborum gavis arma egum \\ Diripit audax. \\ lle non mgno genitore magnus \\ Indiae reges Mediaeque, Partbos. \\ Bactr cum Poro Ecbatanosque parvo \\ Militce vicit \\ Vincere ingenes animis et ausis, \\ Non docens genles numeo et metllo, \\ Vi cele, virtute valens, pofundo \\ Peetore doclus. \\ Hoc magister te docuit ministrum, \\ Nam volens parere, egit deinde. \\ Poenitet tempus abiisse tanum \\ Asque Cmena. \\ Scibit in pectus bona dicta rector, \\ Praecipit doctor pietate tota. \\ Sic Alexnder simul et magister \\ Omnia vincunt. \\ Ast ubi totum tenuit sub uno \\ Obem hbens ceptro, sibi victus ipsi \\ Corruit. vicit Bbylon triumpho \\ Te muliebri. \\ Sie Clitum ad Baechi necat  \lbrack udus \rbrack  aram,. \\ Sie deum temnit iuvenis protervus, \\ Sic duos et iam decem hbens reliquit \\ Sceptr per annos. \\ Ter decem vixit iuvenis per annos \\ Additi tantum tibus aut duobus. \\ 
        \pagebreak 
    \begin{center} \textbf{C. 934. 3136. C. 935 936} \end{center} \marginpar{[359]} Vivus exarsi, moriens flagravit \\ Obis in illo. \\ Non satis mundus fuit unus illi, \\ Nec satis quiquam fuit unus illi. \\ Maximus pinceps sine fine Baccho \\ inis in ipso est. \\ B. V I 9. M. 1142. \\ 
      \end{verse}
  
            \subsection*{935}
      \begin{verse}
      B. V 420. \\ Vt plaeidus noetu tibi Morpheus adsit oportet \\ V faeiat laetum sobria vit diem. \\ Qui laesere diem, laesere tyvrannida somni. \\ Hic furias. quo se vindicet, ultor habet. \\ Casta placent somno; mala sunt insomnia prgso, \\ Ebia lux foedis cum fuit acta iocis. \\ B. I 137. M. 664. \\ 
      \end{verse}
  
            \subsection*{936}
      \begin{verse}
      B. V 420. \\ Quos paribus nutrix edem pavisse papillis, \\ Pectore quos uno genitrix gestasse probatu, \\ Diseidiis indiscissis in muua saepe \\ Vulnera non una perituri clade ruerunt. \\ Quos post longaevos discordia deserit nnnos,. \\ Vt post commissas iunxerunt foedera dextrase, \\ Constantis dederunt documenta frequenler amoris. \\ Thebanum nullo linquit discerimine Tydeus; \\ Tydea nullo umquam Polynices Marte elinquit. \\ 
        \pagebreak 
     \marginpar{[360]} \begin{center} \textbf{C. 937 939} \end{center}B. I 2. M. 560. \\ 
      \end{verse}
  
            \subsection*{937}
      \begin{verse}
      B. V 21. \\ Mars gvis amorum, Pluto moderato Avermni estv, \\ Neptunus mais imperio dominatur, in astris \\ Imperium Iovis est; regnat vacuo aere Iuno. \\ At Venu in teis et ubique, Cupidine cinct. \\ B. II 26B. M. 852. \\ 
      \end{verse}
  
            \subsection*{938}
      \begin{verse}
      B. V 421. \\ V Venus in teis, in aquis dat iura Cupido,. \\ Sc Musae memores servant per saecula libros; \\ Sic comes axmorum crescit pe carmina Fama \\ Nec sinit aeterni occumbere scripta coronis. \\ Pallas anmat Musas atque ornat laude Camenas. \\ Belloum vivit placatis gloria Musis. \\ Et Veneri constant iunctaa cum Pallade Muse \\ Impeio. sic cunct egit divina Volupts. \\ B. I 52. M. 603. \\ 
      \end{verse}
  
            \subsection*{939}
      \begin{verse}
      B. V 421. \\ Matronam magni vehit ardens pavo Tonantis. \\ Ad Veneris curmum iuncta columb cvgno est. . \\ Pllnda bubo vehit, sed eam rota nulla figurat. \\ Anguibus alma Ceres Persephoneque venit. \\ Delia cum Luna esi geminn provecta iuvenca. \\ Venatrix cervas virgo Diana tenet. \\ 
        \pagebreak 
    \begin{center} \textbf{C. 940 941t, 15} \end{center} \marginpar{[361]} B. I 126. . 655. \\ 
      \end{verse}
  
            \subsection*{940}
      \begin{verse}
      B. V 422. \\ Thessalici poceres iunetis virtutibus Ago \\ Magnificm aedificant fatiloquamque mratem. \\ Hac. ope ventorum,. Iunone et Pallade faustis, \\ Auratum referunt trans freta longa pecu. \\ Omnis in hac picta est virtutis imagine forma. \\ Innc labor acer opue grnde parare inbet; \\ Cumque opus effeetum est, ventis piger indiget usus: \\ li tibi FPortunae dant rapiuntque rotm. \\ Iuno notat nummos, spientia Pallade constat \\ Hi ducibus tandem gloria parta venit. \\ Niteris incassum nec te aurea pmeda sequetur, \\ Hic nisi te ternus dux supe nstra vehat. \\ Quid facias iitur Fortunae gloria servit, \\ Divitiis servit, servit et illa libris. \\ Ora igitur divos. ut sis felixque bonusque; \\ Vtc doctus, Musas irequietus am. \\ . VI 89. M. 283. \\ 
      \end{verse}
  
            \subsection*{941}
      \begin{verse}
      B. V 422. \\ Patrcio cel Patrito nescio ci lscript \\ Vere novo forebat humus, saus aethere sudo \\ Imbre maritatum vegetabat spiritus orbem. \\ Ips quoque aetherea deducta propgine lamma \\ Viseeribus sufusn cavis nova gemina largo \\ Vrgebat gremio reparans elementa calore. \\ 
        \pagebreak 
    \begin{center} \textbf{C. 941, 6—40} \end{center} \marginpar{[362]} Latonae geminum numen, Cythereins ignis. \\ Iuppiter ipse parens et Maiae mobile pignus \\ Temperie unanimi, secluso figore tristi \\ Saturni veteris, mundi per apert nitebnt \\ Cum Venus Idaliis comitata sororibus exit, \\ Thessalicos visum Lares, ubi florid Tempe \\ Perpetuis faciles conservant cultibu hortos. \\ It Natura comes lactenti feta papilla,. \\ Vnde venit vitale decus; prope Gratia blando \\ Intuitu invegit floxem naseentibus hebis. \\ Ante deam tenui deeuit veste Volupts, \\ Otentans revocansque nitentia membra tegendo, \\ Purpureas croceo suras evincta cothurno, \\ Speque sua mior nullaque imitanda figura. \\ lnnda mnu implexam tenet hanc ducitque canendo \\ Aethereas Siren iterabile carmen ad auras. \\ Ad iuga blanda sodet niveas moderata columbas,. \\ Non satiand bonis, divae soror alma. Cupido \\ Aliger obsequio stipat puer agmen Amorum \\ Claudit agens ehoreas pictis exercitus armis. \\ Arrident late oto revolantia mundo \\ Sidera, blanditu dominam venerata sereno. \\ Ipsa levi residens curru, mitissima divum, \\ Ventilat afflatu caelum Eephyrisque remissis \\ Mulcet ngros lenique astris adremigt nura. \\ Protinus ut liquidum Phoebi iubar oe recussit \\ Progressamque deam docuere elementa favendo, \\ Lydia qui cedente reliquert arv sorore, \\ Obvi pampineis Lber rapit agmina thyvrsis. \\ Diti Paetoli supemat Peneius amnis \\ Munera, clnrus aquis nitidum stagnantibus aurum. \\ Exsnltant Dryadum facilee deliria Fuuni \\ Ec Satyri,. sub utroque deo promptissima pubes. \\ Comiger hos stimulis implet pue, aethem clangor \\ Verberat et erotalis responsant tyvmpan pulsu. \\ 
        \pagebreak 
    \begin{center} \textbf{C. 941. 174} \end{center} \marginpar{[363]} Ecce pater pando ecubans Silenus asello, \\ Cui lacer a summo pendebat cantharus armo \\ (Vina per os hitaeque fluunt compendia barbae). \\ E numero comitum Veneis vestiat et olim \\ Captus umore petit festino Chloida voto. \\ Nympha etro cedens dum spes alit inque furentem \\ Blanda micans oculis refugit pede, libemr lusu \\ Tuba favet totoque fremit petulantia coetu. \\ Hic ubito volitans spasas rumore per umbrs \\ Fama movet mentes incertaque murmur portat. \\ Orta dehinc lago narratur fabula mou: \\ Non videt auctoem,. sed sentit quisque refertque \\ Aque audisse puta, nec primus in gmine toto est: \\ Mane sub Eoo, dum divae cstr moventu, \\ Elapsum pennis et inobservata ferentem \\ Per liquidas Iymenaeum auras vestigia lomae \\ Advertisse pedem subitoque redire tumultu. \\ Ipse ademrt pompamque tarahens victosque iugali \\ Quos inconsulta eoniunxerat arte Diona, \\ Auspicium iuvenem atque aequaevae pectus Aelle. \\ llos prima patrum generosae stirpis alumnos \\ Nobilita tota pridem celebraverat urbe \\ Et species morumque opulentia compta nitore; \\ Nee semel Acitenens tentara spicula castis \\ Indere pectoribus, matris molimine magno. \\ Oi florebant studii Helicone potito,. \\ Nee chorus Aonidum nec sanctae Palladis ardo \\ Nec pater ipse ingi cuiquam maiore favebat \\ Ingenio. par cura animo, labor otia nescit \\ Improbus atque altis urit praecordia flammis. \\ Gloria, ab excelsa Laus intemerabilis arce \\ Monstrat iter, quo celsa petunt fstigi rerum \\ Semideae mentes, puro stirps prosata caelo. \\ At teneros aevi nec adhuc puerile sonantes \\ 
        \pagebreak 
    \begin{center} \textbf{C. 941, 7592. C. 942, 17} \end{center} \marginpar{[364]} A primis fausto ociarnt omine cunis \\ Sollieiti longa de posteritate paentes. \\ Hinc puer elus iuga Calliopeius illis \\ Nexuerat domina laetamque ferebat ad aulam \\ Luctantes piens iuvenes, immane tropaeum. \\ Indignata tamen risit dea; nec tibi tantum \\ Saepe licere velis, nimium studiose pudoris,. \\ Maternos nimium. puer, mplexate rigores. \\ Protinus instaurnt pompe genialis honorem \\ Deflectunque viam pelgoque advertere certant. \\ Ales at e medio revolans Concordia coetu \\ Iungere nune dextras, nunc oscula pangere mandat \\ Primaque perpetuis mysteria tmdere curis. \\ Ipsa phaetratos urget dea Pronuba fmtres. \\ Sancta Fides, flectant choreas ad moenia Rome \\ Et patrios laeto repetant rumore Penates, \\ Vnde ante ora suorum et avitae in sedibus ulae \\ Testentur fixum foedus thalamumque coronent, \\ B.III 117. .922. \\ 
      \end{verse}
  
            \subsection*{942}
      \begin{verse}
      B. V 425. \\ Quod cernunt oculi, deus est: fons nempe deorum est. \\ Maiests caeli vertitur orbe suo. \\ Terra geri gremio sese caelique euoque \\ Et finem ingentis monstrat uterque globi. \\ quaesepelgotraditnaturavidendam,.At \\ Luminibus dicit ‘non ego finem habeo. \\ Omnia me circum, super, omnia fundit aquae lex: \\ 
        \pagebreak 
     \marginpar{[365]} \begin{center} \textbf{C. 925, 8 14. C. 943 944, 19} \end{center}Sic nusquum immensi terminus Oceani est. \\ Hic oculos igir rerum in primordia mittis, \\ Exspimn omnes hic numeri atque noiae: \\ Nascitur hinc quicquid moitur retroque recedit; \\ Huc redit, aetemo quicquid in obe perit. \\ Hoc perimit flammas elementum, alit evoca auge, \\ Omnis best snpiens que Thalete pocul.’ \\ 
      \end{verse}
  
            \subsection*{943}
      \begin{verse}
      B. M. B. \\ Collisi silicis chaybem vigor urget in ignem \\ E flammm inter se quasea metalla vomunt. \\ Durities vitae fortunaeque improba virtus \\ V itam emendante sci generre preces. \\ icniminviviscintillasemicatignis. \\ Et sua fert miti damna levanda deo, \\ Siequ0quenturaegenemantecontrarialueem, \\ Factoris pateant ut monumenta dei. \\ Sie Ratio sese peccantem cernit in usu, \\ Admonet et superi lux peritura poli. \\ 
      \end{verse}
  
            \subsection*{944}
      \begin{verse}
      B. M. B. \\ Portiter in vita mortem contemnere iactant, \\ Quos Fortuna suis pulveruleniat equis. \\ At quibus aba dies procedit tramite laeto, \\ Dulcius aetherea nil sibi luce ferunt. \\ Omnia sed passae nunc hoc, nunc tempore in illo, \\ Communi superos mens prece cuncta petit. \\ Quicquid in hospitio succedat corporis isto, \\ Sidere vel retro vota ferente cadat. \\ Vnius anmbitio non est expleta senectae, \\ 
        \pagebreak 
     \marginpar{[366]} \begin{center} \textbf{C. 944, 10—20} \end{center}Centum annos etiam prima iuventa cupit. \\ A cui canities sparsit iam sera capillos, \\ Deficiunt toto quem sua membra grdu, \\ Sena licet vitae vicennia compleat, optat \\ Sperat et aestates addere posse decem. \\ Has quoque si addiderit, semper sitit insnper unam; \\ Vnu in immensos, si quea, annus et. \\ Nemo satur vitae mensa discedit ab ista, \\ Incusant omnes fata retroque vident. \\ At quia nulla dies ars virtus prorogat istos, \\ Ioe patitur meriti nescia Vita sui. \\ 
        \pagebreak 
    \begin{center} \textbf{APPEDIX.} \end{center}
      \end{verse}
  
            \subsection*{945}
      \begin{verse}
      Epitaphium annibalis \\ Primus in Ausonios scissis ex Alpibus agros \\ Iannibal inrupi, bellorum qui artibus omnes \\ Excessi et diris attrivi cladibus ubem. \\ Poenorum potui solus tardare ruins. \\ Epitaphium Scipionis \\ Iannibalem et longos fregi Cartbaginis ausus \\ Scipio, Eomano medium qui circiter orbem \\ Adieci imperio libertatemque labentem \\ Seuitur: \\ Quae peto, Laurenti. sis memor ergo mei. . \\ 
      \end{verse}
  
            \subsection*{946}
      \begin{verse}
      \poemtitle{AVSPICII}episcopi ecclesiae Tullensis ad Arbogastem comitem \\ Treverorum. \\ Praecelso exspectabili his Arbogasti comiti \\ Auspicius qui dilio salutem dico plurimam. \\ 
        \pagebreak 
    \begin{center} \textbf{APPENDIX} \end{center} \marginpar{[368]} Magnas caelesti domino rependo corde grtias, \\ Quod te Tullensi proxime magnum in urbe vidimus. \\ Clarus etenim genere, clarus et vitae moribus, \\ Iustus, pudicus, sobrius, totus inlustris redderis. \\ Pater in cunctis nobilis fuit tibi Arigiusa \\ Cuius tu famam nobilem aut renovas aut superas. \\ Congratulandum tibi est, o Trevirorum civitas, \\ Quae tali viro regeris antiquis conparabilem. \\ De magno origo semine descendit tui nominis: \\ Certe virtutis eius est, ut Arbogstis legitur \\ Fuit in armis alacer ille antiquus: verum est: \\ Sed infidelis moritur et morte ecuncta perdidit. \\ Hiec autem noster strenuus, belligerosus, inclitus, \\ Et quod his cunctis maius est, eultor divini nominis. \\ Vnum repelle vitium, ne corda pura inquinet, \\ Quod esse sacris seribitur radix malorum omnium: \\ Cupiditatem scilicet, que in alumnos desaevit s. \\ 
      \end{verse}
  
            \subsection*{947}
      \begin{verse}
      \poemtitle{RVRICII}Saneto Ruricius cliens patrono, \\ Sedato, monitis parens paternis \\ Grates concinit et refert salutem. \\ Quem blandis precibus rogt timendo \\ Ne fors displieeat levis camena \\ Tanti iudicio minor magistri. \\ 
        \pagebreak 
    \begin{center} \textbf{APPENDIX} \end{center} \marginpar{[369]} los tu luminibus libens recurre, \\ Hos sanctis manibus frequens revolve, \\ los tu dum relegis, mei memento. \\ Me semper recolat canatque lingua \\ Et mens me teneat, sopor retentet, \\ Me semper recinat tuum labellum. \\ Hos tu visceribus piis reconde, \\ Hos tecto residens viamque carpens, \\ Hos inter caliees toro recumbens \\ Et parcas epulas cibosque duleces \\ Antro pectoris et medull cordis \\ Inclusos recita canente mente. \\ Sie nos et mutuos videre vultus \\ Et vivis tribuat referre verbis, \\ Quae nune intima pectoris fatignt, \\ Laritor deus omnium bonorum, \\ Christus cum patre sempiterno regnans, \\ Sancto spiritui dignantes hymnos. \\ 
      \end{verse}
  
            \subsection*{948}
      \begin{verse}
      A Esticio commeoratum \\ Virgilium vatem melius sua carmina laudant, \\ In freta dum fluvii current, dum montibus umbrae \\ Lustrabunt convexa, polus dum sidera pnscet. \\ Semper honos nomenque tuum laudesque manebunt. \\ 
      \end{verse}
  
            \subsection*{919}
      \begin{verse}
      \poemtitle{AVDACIS}Cur mihi fons orbis parvo sermone meavit? \\ An minus apta suis speravit ceorda fluentis, \\ 
        \pagebreak 
    \begin{center} \textbf{APPENDIX} \end{center} \marginpar{[370]} Cum pteat mens omnis aquis spectetque loquacem \\ Religionis opem gratos dat sensibus imbres, \\ Epectt quos plena fides Christi de stipite pendens. \\ 
      \end{verse}
  
            \subsection*{950}
      \begin{verse}
      Possidoniu \\ Hie specular renitens fert et crystallina mira \\ Quidam poeta \\ Et raucos timuit discernere dammn molossos \\ Virgilius \\ Sole recens orto numerus ruit omnis in urbem \\ Pstorum. reboant saltus silvaeque cicadis \\ uidam \\ Terramque inpresso syrate verrat \\ Pauli uaestoris \\ Tartaream in sedem sequiur nova nupt maritum . . . \\ Arbiter aurarum qui fluctibus imperat atris . . . \\ Oceanum rapidis linquens repetensque quadrigis . . \\ Ambrosii \\ Dumque ecolorti rutilat pluga caerula mundi \\ Quicquid lRomani valuerunt perdere mores. \\ 
        \pagebreak 
    \begin{center} \textbf{APPENDIX} \end{center} \marginpar{[371]} 
      \end{verse}
  
            \subsection*{8}
      \begin{verse}
      \poemtitle{De diverso usu hominum tres versu}Mille hominum species et mille discolor usus; \\ Vlle suum cuique est nee voto vivitur uno. \\ Dissimilis cunctis vox vultus vita voluntas. \\ e Spicula curvato pelluntur ferrea cornu. \\ Gramineo, formose, iaces sine coniuge lecto. \\ 1 Omne quidem nimium semper vitare memento. \\ Brnti ad Dianam \\ Diva potens nemorum, terror slvestribus apris sq. \\ Ad eum cum quo eenabat \\ Boletos solus sumens atque ostrea voras: \\ Boletum qualem Claudius edit, edas. \\ iyllae poetridis \\ Tune ille aeterni species pulcherrima regmi . . . \\ Denumerat tacitis tot crimina conscius ultor . . \\ Vivat ut aeterno bonus ac malus ardent ine . . \\ 
      \end{verse}
  
        \end{document} 